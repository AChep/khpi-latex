\documentclass[a4paper,14pt,oneside,final]{extarticle}
\usepackage[top=2cm, bottom=2cm, left=3cm, right=1cm]{geometry}
\usepackage{scrextend}

\usepackage[T2A,T1]{fontenc}
\usepackage[ukrainian,russian,english]{babel}
\usepackage{tempora}
\usepackage{fontspec}
\setmainfont{tempora}

% Зачем: Отключает использование изменяемых межсловных пробелов.
% Почему: Так не принято делать в текстах на русском языке.
\frenchspacing

\usepackage{indentfirst}
\setlength{\parindent}{1.25cm}
\renewcommand{\baselinestretch}{1.5}

% Header
\usepackage{fancyhdr}
\pagestyle{fancy}
\fancyhead{}
\fancyfoot{}
\fancyhead[R]{\small \selectfont \thepage}
\renewcommand{\headrulewidth}{0pt}

% Captions
\usepackage{chngcntr}
\counterwithin{figure}{section}
\counterwithin{table}{section}
\usepackage[tableposition=top]{caption}
\usepackage{subcaption}
\DeclareCaptionLabelFormat{gostfigure}{Рисунок #2}
\DeclareCaptionLabelFormat{gosttable}{Таблиця #2}
\DeclareCaptionLabelSeparator{gost}{~---~}
\captionsetup{labelsep=gost}
\captionsetup[figure]{labelformat=gostfigure}
\captionsetup[table]{labelformat=gosttable}
\renewcommand{\thesubfigure}{\asbuk{subfigure}}

% Sections
\usepackage[explicit]{titlesec}
\newcommand{\sectionbreak}{\clearpage}

\titleformat{\section}
  {\centering}{\thesection \quad}{0pt}{\MakeUppercase{#1}}
\titleformat{\subsection}[block]
  {\bfseries}{\thesubsection \quad #1}{0cm}{}

\titlespacing{\section} {0cm}{0cm}{21pt}
\titlespacing{\subsection} {\parindent}{21pt}{0cm}
\titlespacing{\subsubsection} {\parindent}{0cm}{0cm}

% Lists
\usepackage{enumitem}
\renewcommand\labelitemi{--}
\setlist[itemize]{noitemsep, topsep=0pt, wide}
\setlist[enumerate]{noitemsep, topsep=0pt, wide, label=\arabic*}
\setlist[description]{labelsep=0pt, noitemsep, topsep=0pt, leftmargin=2\parindent, labelindent=\parindent, labelwidth=\parindent, font=\normalfont}

% Toc
\usepackage{tocloft}
\tocloftpagestyle{fancy}
\renewcommand{\cfttoctitlefont}{}
\setlength{\cftbeforesecskip}{0pt}
\renewcommand{\cftsecfont}{}
\renewcommand{\cftsecpagefont}{}
\renewcommand{\cftsecleader}{\cftdotfill{\cftdotsep}}


\begin{document}
\Ukrainian

\section*{Законодавча політика Євросоюзу в питаннях охорони праці. Рамкова директива 89/391/ЄС <<Про введення заходів, що сприяють поліпшенню безпеки та гігієни праці працівників>>. Конвенції та Рекомендації МОП.}

\subsection*{Законодавча політика Євросоюзу в питаннях охорони праці}
Політика ЄС у сфері охорони праці пройшла складний і довгий шлях формування.
У перших договорах про створення Європейського Співтовариства соціальним питанням, у тому числі проблемам охорони праці, належної уваги приділено не було. 
Після набуття чинності Договором про Європейське Економічне Співтовариство (Римський договір, 1957 р.) було розпочато гармонізацію трудового законодавства щодо статусу робітників-мігрантів, що поставило на порядок денний питання щодо узгодження відповідного соціального законодавства держав-членів ЄЕС. 
Низка європейських інституцій і профспілок виступили з ініціативою щодо конкретизації повноважень Співтовариства і держав-членів у сфері соціальної політики. У результаті Європейською Комісією було реалізовано комплексні наукові дослідження щодо можливості гармонізації механізмів правового регулювання у сфері соціального страхування, на основі яких було підготовлено та винесено на розгляд держав-членів відповідні пропозиції рекомендаційного характеру. 
Слід однак наголосити, що вказаніпропозиції далеко не відразу знайшли підтримку серед урядів європейських країн, а до розгляду проблем соціальної політики у ЄЕС повернулися лише після врегулювання кризи 1964–1966 рр~\cite{Shashula2015}.

У результаті було прийняте компромісне рішення --- на основі пропозицій Європейської Комісії розроблялась Робоча програма з соціальної політики  Співтовариства, яка передбачала детальне дослідження видатків на соціальне забезпечення та соціальний захист і порівняльний аналіз основних правових понять у сфері національного регулювання соціального страхування~\cite{Shashula2015}.

Однак у 60-х рр. ХХ ст. на рівні Європейського Співтовариства так і не було сформульовано єдиної концепції щодо врегулювання соціальних аспектів європейської інтеграції. Відсутність такої концепції пов’язують із двома чинниками: надмірно чутливою реакцією населення держав-членів ЄС на зміни соціального законодавства та відмінностями у системах соціального страхування держав-членів і, як наслідок, різним рівнем соціального забезпечення та захисту населення~\cite{Shashula2015}.

В 70-х рр. Європейська Комісія у своєму політичному дискурсі починає вживати поняття <<Соціальний Союз>>, наголошуючи, що <<…соціальна політика Співтовариства є реалізацією першої фази будівництва Європейського Соціального Союзу…>>~\cite{EuCouncil}. 
Конкретизуючи об’єкти соціальної політики Співтовариства, Європейська Комісія наголошувала на необхідності правової регламентації відносин зайнятості, становища жінок та сезонних робітників, охорони здоров’я, статусу категорій осіб, котрі потребують соціальної підтримки, співробітництва національних установ у сфері праці та посилення соціального діалогу~\cite{Shashula2015}. 

Необхідність гармонізації законодавства країн-членів ЄС у сфері охорони праці як один із вагомих чинників формування спільної економічної політики пов’язують зі зростанням ролі транснаціональних корпорацій та виникаючих на цьому тлі проблем соціального забезпечення і захисту населення; а також із взаємним впливом соціального та трудового законодавства на виробництво товарів і послуг, на конкурентоспроможність продукції. 
У 70-х рр. увагу до цих проблем було привернуто у зв’язку з масовими звільненнями у ФРН та сусідніх державахчленах ЄС, де існувала різна правова регламентація відповідних заходів.
Наслідком цих подій стала гармонізація процедури масових звільнень шляхом затвердження на рівні Співтовариства директиви 75/129 про захист працівників при масових звільненнях.

З метою формування соціальної політики на рівні Європейського Співтовариства, а отже, її правового забезпечення особливо у сфері трудового права та соціального захисту, важливу роль відіграла Паризька зустріч керівників держав та урядів країн-членів ЄС у 1972 р. 
Підсумком Паризького саміту стало досягнення згоди щодо необхідності формування єдиної соціальної політики Європейського Співтовариства. Відповідне консенсусне рішення було відображене у декларації Ради ЄС, де зокрема зазначалося, що соціальна політика Співтовариства повинна спрямовуватись на виконання власних завдань і слугувати орієнтиром для соціальної політики держав-членів~\cite{Shashula2015}. 

Соціальні завдання Європейського Співтовариства було детально викладено у відповідних положеннях Маастрихтського договору (далі – ДЄС). 
Так, предметом соціальної політики Європейського Співтовариства стало сприяння співробітництву між державами-членами щодо поліпшення та узгодження умов життя і праці робітників (ст. 117), співробітництво із соціальних питань (ст. 118), затвердження мінімальних стандартів і покращання умов праці (ст. 118а), діалог між соціальними партнерами (ст. 118b), однакова оплата праці для чоловіків та жінок (ст. 119), оплачувана відпустка (ст. 120), створення Європейського Соціального Фонду (ст.ст. 123-125), що передбачало узгодження правового забезпечення відносин у сфері соціальної політики Європейського Співтовариства.

30 жовтня 1989 р. за пропозицією Європейської Комісії та після консультацій з Економічним і Соціальним Комітетом та Європейським Парламентом Рада ЄС приймає <<Хартію Співтовариства про основні соціальні права працівників>>~\cite{Khartia96}, яка стала першою спробою міждержавного співробітництва з питань соціального забезпечення і захисту населення на рівні Європейського Співтовариства.

Текст Маастрихтського договору, що набув чинності в 1993 р., включав у себе <<Протокол про соціальну політику>> разом з <<Договором про соціальну політику>>. 
Слід зазначити, що обидва документи є складовою європейського права, однак <<Протокол про соціальну політику>> віднесено до первинного права, а <<Договір про соціальну політику>> є міжнародним договором, підписаним одинадцятьма державами-членами, що зумовлює його особливий статус у системі права ЄС~\cite{Shashula2015}. 

<<Протокол про соціальну політику>> уповноважив одинадцять держав-членів ЄС, за винятком Великобританії, використовувати органи, механізми та процедури Європейського Співтовариства задля вирішення спільних проблем у сфері соціальної політики. 
Однак слід наголосити, що у випадку колізії між нормативноправовими актами Ради з імплементації Протоколу та основними засадами права ЄС перевагу матимуть норми права Співтовариства. 
<<Протокол про соціальну політику>> слугував додатковим джерелом активізації спільної соціальної політики ЄС (за винятком Великобританії), формування її правового забезпечення на рівні Європейського Співтовариства~\cite{Shashula2015}.

Схожі положення містить і <<Договір про соціальну політику>>.
У ст. 1 договору предметом співробітництва між державами-членами та Європейським Співтовариством було визначено підвищення рівня зайнятості, покращання умов життя та праці, надання належного соціального захисту, сприяння діалогу між адміністрацією підприємств та працівниками, боротьбу із соціальним відчуженням. 
Крім того, на Європейську Комісію покладалось зобов’язання щодо заохочення країн-членів до співробітництва у сфері соціальної політики (ст. 5), а самі держави-члени мали забезпечувати дотримання принципу заборони дискримінації у сфері оплати праці (ст. 6)~\cite{Forster2002}. 
На підставі Договору держави-члени уповноважувались на використання інституційної системи Європейського Співтовариства задля досягнення мети міждержавного співробітництва у сфері соціальної політики. 
Значну увагу в Договорі приділено також проблемам співпраці між адміністрацією та працівниками підприємств. 
Так, ст. 4 договору передбачає можливість договірного врегулювання відносин між адміністрацією та працівниками, зокрема шляхом укладання колективних угод на рівні ЄС~\cite{Shashula2015}. 

В політичних заявах керівництва ЄС стосовно напрямів майбутнього розвитку європейського законодавства наголошується на тому, що воно повинно стати менш складним та більш послідовним. 
Крім того, з огляду на значну роль, що відіграють підприємства малого та середнього бізнесу в розвитку економіки всіх країн-членів ЄС, створюючи більшу частину робочих місць, законодавство ЄС повинне сприяти зміцненню їхніх позицій, а не послаблювати за рахунок введення додаткових законодавчих обмежень. 

Відповідна тенденція розповсюджується й на законодавство у сфері охорони праці. 
Так, більшість зобов’язань щодо забезпечення належного рівня охорони праці покладається на соціальних партнерів (роботодавців та працівників).
Подальшого розвитку набуває практика т. з. впливу <<на відстані>>, тобто уряд лише формує загальні правила, залишаючи деталі на розсуд соціальних партнерів. 
Це означає, що всі сторони соціального діалогу (держава, роботодавці, працівники) залучаються у процес створення і впровадження нового законодавства~\cite{Shashula2015}. 

Загальноєвропейська політика охорони праці формується за рахунок затвердження відповідних нормативно-правових актів (директив ЄС), що мають рекомендаційний характер, та визначають засадничі принципи, яким повинно відповідати законодавство держав-членів ЄС. 
При цьому форми, методи та строки практичної реалізації відповідних положень залишаються на розсуд самих держав.
Такий <<м’який>> підхід дозволяє створити загальноєвропейський правовий простір, зберігши особливості та традиції кожної національної законодавчої системи~\cite{Shashula2015}.

Безумовно, проблеми охорони праці на європейському рівні не вирішуються за рахунок простого перенесення вимог європейських директив до положень національного законодавства. 
Реальне покращання умов та підвищення безпеки праці досягається шляхом розвитку механізмів соціального партнерства на всіх рівнях від загальноєвропейського до локального. 
Лише за активної участі всіх суб’єктів охорони праці: держави, роботодавця та працівника можливе налагодження дійсно ефективних систем охорони праці. 

\subsection*{Рамкова директива 89/391/ЄС <<Про введення заходів, що сприяють поліпшенню безпеки та гігієни праці працівників>>. Конвенції та Рекомендації МОП.}
Важливе місце у нормативно-правовому полі з охорони праці займають міжнародні договори та угоди.
Особливо велике значення серед міжнародних договорів, якими регулюються трудові відносини, мають конвенції Міжнародної Організації Праці~\cite{LectureKPI}.

Директиви, що приймаються в рамках Європейського Союзу і є законом для всіх його країн, відповідають конвенціям МОП. 
При розробці нових конвенцій, рекомендацій та інших документів МОП враховується~\cite{LectureKPI}:
\begin{itemize}
	\item спільні стандарти здоров‘я і безпеки сприяють економічній інтеграції, оскільки продукти не можуть вільно циркулювати всередині Союзу, якщо ціни на аналогічні вироби різняться в різних країнах-членах через різні витрати, які накладає безпека та гігієна праці на бізнес;
	\item скорочення людських, соціальних та економічних витрат, пов‘язаних з нещасними випадками та професійними захворюваннями, приведе до великої фінансової економії і викличе суттєве зростання якості життя у всьому Співтоваристві;
	\item запровадження найбільш ефективних методів роботи повинно принести з собою ріст продуктивності, зменшення експлуатаційних (поточних) витрат і покращення трудових стосунків;
	\item регулювання певних ризиків (таких, як ризики, що виникають при великих вибухах) повинно узгоджуватися на наднаціональному рівні в зв‘язку з масштабом ресурсних затрат і з тим, що будь-яка невідповідність в суті і використанні таких положень приводить до <<викривлень>> у конкуренції і впливає на ціни товарів.	
\end{itemize}

Міжнародна Організація Праці (МОП) є спеціалізованою установою Організації Об`єднаних Націй (ООН). 
Основним завданням МОП є встановлення та поширення принципів соціальної справедливості і трудових прав та прав людини, визнаних світовою спільнотою.
МОП була заснована у 1919 р. в результаті підписання Версальського договору, який, в свою чергу, дав початок існуванню Ліги Націй, і стала першою спеціалізованою установою ООН у 1946 р.
До МОП зараз входить 173 країни. 

Документи МОП встановлюють мінімальні стандарти основних трудових прав: свободу створення та діяльність асоціацій, колективні переговори, заборону примусової праці, рівність можливостей і відношення до працюючого та інші стандарти, які регулюють умови та інші аспекти праці. 
За умов додержання відповідної процедури, конвенції та рекомендації МОП стають частиною національного законодавства. 
Через їх універсальний характер конвенції та рекомендації МОП можуть застосовуватися суддями у судовій практиці.

Структурно МОП складається з:
\begin{itemize}
	\item \textbf{Міжнародна Конференція праці} --- вищий орган МОП і тому вона зветься також Всесвітнім Парламентом праці --- проводиться щороку у червні за участю представників всіх країн-членів.
	\item \textbf{Міжнародне Бюро праці} --- це постійний секретаріат організації, який розробляє Кодекси практичних заходів, здійснює моніторинг фінансових справ, розробляє порядок денний наступних Міжнародних Конференцій праці.
	\item \textbf{Адміністративна Рада} (включає 28 урядових представників, 14 представників роботодавців та 14 представників робітників) --- здійснює контроль за діяльністю Міжнародного Бюро праці та зв‘язок між ним і Міжнародною Конференцією праці.
\end{itemize}

З часу свого заснування МОП ухвалила понад 180 Конвенцій~\cite{LectureKPI}, до основних з яких відносятся: 81 (інспекція праці); Конвенції 29 і 105 (примусова праця), 87 (свобода об'єднань), 100 і 111 (рівна оплата чоловіків і жінок за рівноцінну роботу; дискримінація), 135 (представники працівників), 155 (Безпека, гігієна, охорона праці), 159 (професійна реабілітація та зайнятість інвалідів), 177 (надомна праця), 182 (найгірші форми дитячої праці) та ін.

Особливе місце серед Конвенцій МОП займає Конвенція №155 <<Про безпеку і гігієну праці та виробничу санітарію>>, яка закладає міжнародно-правову основу національної політики щодо створення всебічної і послідовної системи профілактики нещасних випадків на виробництві і професійних захворювань~\cite{LectureKPI}.

Рамкова директива 89/391/ЄС <<Про введення заходів, що сприяють поліпшенню безпеки та гігієни праці працівників>> визначає нововведення та інновації~\cite{LectureOdessa}:
\begin{itemize}
	\item загальне завдання поліпшення виробничого середовища й відповідні йому обов’язки, що покладаються на держави-члени ЄС та роботодавців;
	\item розширене поняття виробничого середовища, в якому поряд із фізичною безпекою розглядається й організація роботи та соціальні відносини на робочих місцях;
	\item навчання та обов’язки робітників;
	\item необхідність зниження навантаження на працівників і створення морально здорових умов праці, а не тільки запобігання фізичної шкоди;
	\item докладний перелік засобів, які необхідно використовувати для рішення завдань.
\end{itemize}

Тобто директива 89/391/ЄС декларує загальні принципи профілактики та основи охорони праці. 
Сфера дії директиви: робочі місця, виробниче обладнання, індивідуальні засоби захисту, обробка тяжких вантажів, тимчасові або пересувні робочі місця тощо.

На підставі цієї Директиви було прийнято низку окремих директив~\cite{LectureOdessa}: 
\begin{itemize}
	\item 89/654 --- про мінімум вимог до безпеки та гігієни робочих місць;
	\item 89/655 --- про мінімальні вимоги по забезпеченню безпеки та охорони здоров’я робітників на робочому місці при використанні робочого обладнання; 
	\item 89/656 --- при використанні робітниками персональних захисних засобів на робочому місці; 
	\item 90/269 --- про мінімум вимог до безпеки та гігієни праці при ручній обробці вантажів при наявності ризику, особливо травми хребта робітників
	\item 90/270 --- про мінімум вимог до безпеки та гігієни праці при роботі з екранними пристроями; 
	\item 92/57 --- про виконання мінімуму вимог безпеки та гігієни праці на тимчасових або пересувних будівельних площадках; 
	\item 92/85 --- про запровадження заходів, які сприяють покращанню безпеки та гігієни праці вагітних, а також породіль та працюючих жінок-годувальниць; 
	\item 2004/37 --- про захист робітників від ризику, пов’язаного з дією канцерогенів або мутагенів на робочому місці, а також ряд інших.
\end{itemize}

У МОП діє система контролю за застосуванням в країнах-членах Організації конвенцій і рекомендацій. 
Кожна держава зобов‘язана подавати доповіді про застосування на своїй території ратифікованих нею конвенцій, а також інформації про стан законодавства і практики з питань, що порушуються в окремих, не ратифікованих нею конвенціях.

Згідно діючих конвенцій МОП, при регулюванні трудових відносин соціальне страхування повинно стати обов’язковим і являти собою систему прав і гарантій, що спрямовані на матеріальну підтримку громадян, насамперед працюючих, і членів їх сімей у разі втрати ними з незалежних від них обставин (захворювання, нещасний випадок, безробіття, досягнення пенсійного віку тощо) заробітку, а також здійснення заходів, пов‘язаних з охороною здоров‘я застрахованих осіб. 
Соціальне страхування є важливим фактором соціального захисту населення.

\printbibliography[heading=bibintoc, title={Список джерел інформації}]


\end{document}
