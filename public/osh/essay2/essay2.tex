\documentclass[a4paper,14pt,oneside,final]{extarticle}
\usepackage[top=2cm, bottom=2cm, left=3cm, right=1cm]{geometry}
\usepackage{scrextend}

\usepackage[T2A,T1]{fontenc}
\usepackage[ukrainian,russian,english]{babel}
\usepackage{tempora}
\usepackage{fontspec}
\setmainfont{tempora}

% Зачем: Отключает использование изменяемых межсловных пробелов.
% Почему: Так не принято делать в текстах на русском языке.
\frenchspacing

\usepackage{indentfirst}
\setlength{\parindent}{1.25cm}
\renewcommand{\baselinestretch}{1.5}

% Header
\usepackage{fancyhdr}
\pagestyle{fancy}
\fancyhead{}
\fancyfoot{}
\fancyhead[R]{\small \selectfont \thepage}
\renewcommand{\headrulewidth}{0pt}

% Captions
\usepackage{chngcntr}
\counterwithin{figure}{section}
\counterwithin{table}{section}
\usepackage[tableposition=top]{caption}
\usepackage{subcaption}
\DeclareCaptionLabelFormat{gostfigure}{Рисунок #2}
\DeclareCaptionLabelFormat{gosttable}{Таблиця #2}
\DeclareCaptionLabelSeparator{gost}{~---~}
\captionsetup{labelsep=gost}
\captionsetup[figure]{labelformat=gostfigure}
\captionsetup[table]{labelformat=gosttable}
\renewcommand{\thesubfigure}{\asbuk{subfigure}}

% Sections
\usepackage[explicit]{titlesec}
\newcommand{\sectionbreak}{\clearpage}

\titleformat{\section}
  {\centering}{\thesection \quad}{0pt}{\MakeUppercase{#1}}
\titleformat{\subsection}[block]
  {\bfseries}{\thesubsection \quad #1}{0cm}{}

\titlespacing{\section} {0cm}{0cm}{21pt}
\titlespacing{\subsection} {\parindent}{21pt}{0cm}
\titlespacing{\subsubsection} {\parindent}{0cm}{0cm}

% Lists
\usepackage{enumitem}
\renewcommand\labelitemi{--}
\setlist[itemize]{noitemsep, topsep=0pt, wide}
\setlist[enumerate]{noitemsep, topsep=0pt, wide, label=\arabic*}
\setlist[description]{labelsep=0pt, noitemsep, topsep=0pt, leftmargin=2\parindent, labelindent=\parindent, labelwidth=\parindent, font=\normalfont}

% Toc
\usepackage{tocloft}
\tocloftpagestyle{fancy}
\renewcommand{\cfttoctitlefont}{}
\setlength{\cftbeforesecskip}{0pt}
\renewcommand{\cftsecfont}{}
\renewcommand{\cftsecpagefont}{}
\renewcommand{\cftsecleader}{\cftdotfill{\cftdotsep}}

\usepackage{float}
\usepackage{pgfplots}
\usepackage{graphicx}
\usepackage{multirow}
\usepackage{amssymb,amsfonts,amsmath,amsthm}
\usepackage{csquotes}

\usepackage{listings}
\lstset{basicstyle=\footnotesize\ttfamily,breaklines=true}
\lstset{language=Matlab}

\usepackage[
	backend=biber,
	sorting=none,
	language=auto,
	autolang=other
]{biblatex}
\DeclareFieldFormat{labelnumberwidth}{#1}

\lstdefinelanguage{Python}{
  keywords={and, break, class, continue, def, yield, del, elif, else, except, exec, finally, for, from, global, if, import, in, lambda, not, or, pass, print, raise, return, try, while, assert, with},
  keywordstyle=\color{NavyBlue}\bfseries,
  ndkeywords={True, False},
  ndkeywordstyle=\color{BurntOrange}\bfseries,
  emph={as},
  emphstyle={\color{OrangeRed}},
  identifierstyle=\color{black},
  sensitive=true,
  commentstyle=\color{gray}\ttfamily,
  comment=[l]{\#},
  morecomment=[s]{/*}{*/},
  stringstyle=\color{ForestGreen}\ttfamily,
  morestring=[b]',
  morestring=[s]{"""*}{*"""},
}


\addbibresource{essay2.bib}

\lhead{\small \selectfont Чепурний А.С., 2.КН201н.8а}

\begin{document}
\Ukrainian

\section*{Рекомендації щодо дії населення у надзвичайних ситуаціях}

При першій можливості покиньте разом із сім’єю небезпечну зону. 
У разі неможливості виїхати особисто, відправити дітей і родичів похилого віку до родичів, знайомих. Необхідно взяти із собою всі документи, коштовні речі і цінні папери~\cite{iFrank}.

Підготовку до можливого перебування у зоні надзвичайної ситуації доцільно починати завчасно. Необхідно підготувати "екстрену валізку" з речами, які можуть знадобитись при знаходженні у зоні надзвичайної ситуації або при евакуації у безпечні райони~\cite{iFrank}.

\subsection*{Дії населення під час терористичного акту}
Основні заходи щодо запобігання можливого терористичного акту~\cite{cherkass}:
\begin{itemize}
	\item не торкайте у вагоні поїзда, під'їзді або на вулиці нічийні пакети (сумки), не підпускайте до них інших. Повідомите про знахідку співробітникові міліції;
	\item у присутності терористів не виказуйте своє невдоволення, утримаєтеся від різких рухів, лементу й стогонів;
	\item при погрозі застосування терористами зброї лягайте на живіт, захищаючи голову руками, подалі від вікон, засклених дверей, проходів, сходів; у випадку поранення рухайтеся; використайте будь-яку можливість для порятунку; якщо відбувся вибух - вживайте заходів щодо недопущення пожежі та паніки, надайте домедичну допомогу постраждалим;
	\item намагайтеся запам'ятати прикмети підозрілих людей і повідомте їх прибулим співробітникам спецслужб.
\end{itemize}

Дії під час перестрілки~\cite{cherkass}:
\begin{itemize}
	\item якщо стрілянина застала вас на вулиці, відразу ж лягте й озирніться, виберіть найближче укриття й проберіться до нього, не піднімаючись у повний зріст. Укриттям можуть служити виступи будинків, пам'ятники, бетонні стовпи або бордюри, канави.

	Пам'ятайте, що автомобіль - не найкращий захист, тому що його метал тонкий, а пальне - вибухонебезпечне. За першої нагоди сховайтеся у під'їзді будинку, підземному переході, дочекайтеся закінчення перестрілки;
	
	\item вжийте заходи для порятунку дітей, за необхідності прикрийте їх своїм тілом. За можливості повідомте про інцидент співробітників міліції;
	
	\item якщо в ході перестрілки ви перебуваєте у будинку - укрийтеся у ванній кімнаті й лягте на підлогу, тому що перебувати у кімнаті небезпечно через можливість рикошету. Перебуваючи в укритті, стежте за можливим початком пожежі. Якщо пожежа почалася, а стрілянина не припинилася, залиште квартиру й укрийтеся в під'їзді, далі від вікон.
\end{itemize}

Дії у випадку захоплення літака (автобуса)~\cite{cherkass}: 
\begin{itemize}
	\item якщо Ви виявилися в захопленому літаку (автобусі), не привертайте до себе уваги терористів. Огляньте салон, визначте місця можливого укриття на випадок стрілянини;
	\item заспокойтеся, спробуйте відволіктися від того, що відбувається, читайте, розгадуйте кросворди; зніміть ювелірні прикраси;
	\item не дивіться в очі терористам, не пересувайтеся по салону та не відкривайте сумки без їхнього дозволу;
	\item не реагуйте на провокаційну або зухвалу поведінку;
	\item жінкам у міні-спідницях бажано прикрити ноги.
\end{itemize}

Якщо ви стали жертвою телефонного терориста~\cite{cherkass}:
\begin{itemize}
	\item покладіть слухавку поряд з телефоном; подзвоніть з іншого телефону (мобільного, від сусідів) на вузол зв'язку і скажіть причину дзвінка, своє прізвище, адресу та номер свого телефону. Диспетчер встановить номер того, хто дзвонив та скаже вам, звідки дзвонили - з квартири чи з автомату;
	\item напишіть заяву начальнику відділення міліції, на території якого ви проживаєте, для прийняття необхідних заходів;
	\item постарайтесь затягнути розмову та записати її на диктофон чи дайте послухати свідкам (сусідам);
	\item подзвоніть на вузол зв'язку з іншого телефону (аналогічно п. 1), а потім напишіть заяву в відділення міліції;
	\item міліція, за запитом на вузол зв'язку, отримає номер телефону, адресу, прізвище того, хто дзвонив і прийме необхідні міри;
	\item одночасно, з розмовою і записом на диктофон, друга людина дзвонить з іншого телефону на вузол зв'язку, а потім в міліцію за телефоном 102 для термінового затримання того, хто телефонував.
\end{itemize}

\subsection*{Дії населення під час пожежі}
Якщо пожежа застала вас у приміщенні~\cite{iFrank}:
\begin{itemize}
	\item ви прокинулись від шуму пожежі і запаху диму --- не сідайте у ліжку, а скотіться з нього та повзіть під хмарою диму до дверей, але не відчиняйте їх одразу;
	\item обережно доторкніться до дверей тильною стороною долоні,
	якщо двері не гарячі, то обережно відчиніть їх та швидко виходьте, а якщо
	гарячі --- ні в якому разі не відчиняйте їх;
	\item щільно закрийте двері, а всі щілини та отвори позатикайте
	тканиною, по можливості мокрою, щоб уникнути подальшого проникнення
	диму, та повертайтесь повзком углиб приміщення, приймайте заходи для
	порятунку;
	\item присядьте, глибоко вдихніть, розчиніть вікно, висуньтеся та
	кричіть: "Допоможіть, пожежа!", а якщо ви не в силі відчинити вікно ---
	розбийте скло твердим предметом та приверніть до себе увагу людей, які
	можуть викликати пожежно-рятувальну службу;
	\item якщо ви вибрались через двері – зачиніть їх і повзком
	пересувайтесь до виходу з приміщення (обов’язково зачиніть за собою всі
	двері);
	\item якщо ви знаходитесь у висотному будинку --- не біжіть донизу
	крізь полум’я, а скористайтеся можливістю вибратися на дах будівлі.
\end{itemize}

Користуватися ліфтом під час пожежі заборонено~\cite{iFrank}!

У всіх випадках намагайтеся викликати пожежно-рятувальну
службу за телефоном 101.

При рятуванні потерпілих з палаючих будинків необхідно~\cite{iFrank}:
\begin{itemize}
	\item перед входом в палаюче приміщення накритися з головою мокрим покривалом, ковдрою, плащем, пальтом;
	\item двері в приміщення відчиняти обережно, щоб запобігати спалаху вогню від швидкого притоку повітря;
	\item в сильно загазованому приміщенні пересуватися повзком або пригинаючись (небезпечно входити в зону задимлення, якщо видимість менше 10 метрів);
	\item для захисту від чадного газу дихати через вологу тканину (або спеціальні засоби захисту органів дихання від чадного газу);
	\item у першу чергу рятувати дітей, інвалідів та літніх людей;
	\item пам’ятайте, що маленькі діти від страху часто ховаються під ліжко, в шафу або забиваються в куток;
	\item якщо загорівся ваш одяг, не можна бігти, потрібно лягти на землю і перекочуватись, збиваючи полум’я;
	\item побачивши людину, на якій горить одяг, звалити її на землю та швидко накинути на неї пальто, плащ чи покривало (бажано зволожене) і щільно притиснути до тіла;
	\item якщо горить електричне обладнання або проводка, вимкнути рубильник, вимикач або запобіжники, тільки після цього починати гасити вогонь.
\end{itemize}

Якщо ви опинилися в осередку пожежі на відкритій місцевості~\cite{cherkass}:
\begin{itemize}
	\item не панікуйте та не тікайте від полум’я, що швидко наближається, у протилежний від вогню бік, а долайте крайку вогню проти вітру, закривши голову і обличчя одягом;
	\item з небезпечної зони, до якої наближається полум’я, виходьте швидко, перпендикулярно напряму розповсюдження вогню;
	\item якщо втекти від пожежі неможливо, то вийдіть на відкриту місцевість або галявину, ввійдіть у водойму або накрийтесь мокрим одягом і дихайте повітрям, що знаходиться низько над поверхнею землі – повітря тут менш задимлене, рот і ніс при цьому прикривайте одягом чи шматком будь-якої тканини;
	\item гасити полум’я невеликих низових пожеж можна, забиваючи полум’я гілками листяних порід дерев, заливаючи водою, закидаючи вологим ґрунтом та затоптуючи ногами;
	\item під час гасіння пожежі, не відходьте далеко від доріг та просік, не втрачайте з виду інших учасників гасіння пожежі, підтримуйте з ними зв’язок за допомогою голосу;
	\item будьте обережні в місцях горіння високих дерев, вони можуть завалитися та травмувати вас;
	\item особливо будьте обережні у місцях торф’яних пожеж, враховуйте, що там можуть створюватися глибокі прогари, тому пересувайтеся, за можливістю, перевіряючи палицею глибину шару, що вигорів;
	\item після виходу з осередку пожежі повідомте місцеву адміністрацію та пожежну службу про місце, розміри та характер пожежі.
\end{itemize}

\subsection{Дії населення в умовах надзвичайних ситуації воєнного характеру}
При першій можливості покиньте разом із сім’єю небезпечну зону. У разі неможливості виїхати особисто, відправити дітей і родичів похилого віку до родичів, знайомих. Необхідно взяти із собою всі документи, коштовні речі і цінні папери.

Підготовку до можливого перебування у зоні надзвичайної ситуації доцільно починати завчасно. Необхідно підготувати "екстрену валізку" з речами, які можуть знадобитись при знаходженні у зоні НС або при евакуації у безпечні райони.
 
Підготовка оселі:
\begin{itemize}	
	\item нанести захисні смуги зі скочу (паперу, тканини) на віконне скло для підвищення його стійкості до вибухової хвилі та зменшення кількості уламків і уникнення травмування у разі його пошкодження;
	\item по можливості обладнайте укриття у підвалі, захистіть його мішками з піском, передбачте наявність аварійного виходу;
	\item при наявності земельної ділянки обладнайте укриття на такій відстані від будинку, яка  більше його висоти;
	\item зробити вдома запаси питної та технічної води;
	\item зробити запас продуктів тривалого зберігання;
	\item додатково укомплектувати домашню аптечку засобами надання першої медичної допомоги;
	\item підготувати (закупити) засоби первинного пожежогасіння;
	\item підготувати ліхтарики (комплекти запасних елементів живлення), гасові лампи та свічки на випадок відключення енергопостачання;
	\item підготувати (закупити)  прилади (примус) для приготування їжі у разі відсутності газу і електропостачання;
	\item підготувати необхідні речі та документи на випадок термінової евакуації або переходу до захисних споруд цивільної оборони або інших сховищ (підвалів, погребів тощо);
	\item особистий транспорт завжди мати у справному стані і запасом палива для виїзду у небезпечний район;
	\item при наближенні зимового періоду необхідно продумати питання щодо обігріву оселі у випадку відключення централізованого опалення.
\end{itemize}

Необхідно:
\begin{itemize}
	\item зберігати особистий спокій, не реагувати на провокації;
	\item не сповіщати про свої майбутні дії (плани) малознайомих людей, а також знайомих з ненадійною репутацією;
	\item завжди мати при собі документ (паспорт) що засвідчує особу, відомості про групу крові своєї та близьких родичів, можливі проблеми зі здоров’ям (алергію на медичні препарати тощо);
	\item знати місце розташування захисних споруд цивільної оборони поблизу місця проживання, роботи, місцях частого відвідування (магазини, базар, дорога до роботи, медичні заклади тощо). Без необхідності старатися як найменше знаходитись поза місцем проживання, роботи та малознайомих місцях;
	\item при виході із приміщень, пересуванні сходинами багатоповерхівок або до споруди цивільної оборони (сховища) дотримуватись правила правої руки (як при русі автомобільного транспорту) з метою уникнення тисняви. Пропускати вперед та надавати допомогу жінкам, дітям, перестарілим людям та інвалідам, що значно скоротить терміни зайняття укриття;
	\item уникати місць скупчення людей;
	\item не вступати у суперечки з незнайомими людьми, уникати можливих провокацій;
	\item у разі отримання будь-якої інформації від органів державної влади про можливу небезпеку або заходи щодо підвищення безпеки передати її іншим людям (за місцем проживання, роботи тощо);
	\item при появі озброєних людей, військової техніки, заворушень негайно покидати цей район;
	\item посилювати увагу і за можливості, також залишити цей район, у разі появи засобів масової інформації сторони-агресора;
	\item у разі появи підозрілих людей (не орієнтуються на місцевості, розмовляють з акцентом, не характерна зовнішність, протиправні і провокативні дії, проведення незрозумілих робіт тощо) негайно інформувати органи правопорядку, місцевої влади, військових;
	\item у разі потрапляння у район обстрілу сховатись у найближчу захисну споруду цивільної оборони, сховище (укриття). У разі відсутності пристосованих сховищ, для укриття використовувати нерівності рельєфу, (канави, окопи, заглиблення від вибухів тощо). У разі раптового обстрілу та відсутності поблизу споруд цивільного захисту, сховища і укриття − лягти на землю головою в сторону, протилежну вибухам. Голову прикрити руками (за наявності, для прикриття голови використовувати валізу або інші речі). Не виходьте з укриття до кінця обстрілу;
	\item надавати першу допомогу іншим людям у разі їх поранення. Визвати швидку допомогу, представників ДСНС України, органів правопорядку за необхідності військових;
	\item у разі, якщо ви стали свідком поранення або смерті людей, протиправних до них дій (арешт, викрадення, побиття тощо) постаратися з’ясувати та зберегти як найбільше інформації про них та обставини події для надання допомоги, пошуку, встановлення особи тощо. Необхідно пам’ятати, що Ви самі або близькі Вам люди, також можуть опинитись у скрутному становищі і будуть потребувати допомоги.
\end{itemize}

Не рекомендується: 
\begin{itemize}
	\item підходити до вікон, якщо почуєте постріли;
	\item спостерігати за ходом бойових дій;
	\item стояти чи перебігати під обстрілом;
	\item конфліктувати з озброєними людьми;
	\item носити армійську форму або камуфльований одяг;
	\item демонструвати зброю або предмети, схожі на неї;
	\item підбирати покинуті зброю та боєприпаси.
\end{itemize}

\printbibliography[heading=bibintoc, title={Список джерел інформації}]


\end{document}
