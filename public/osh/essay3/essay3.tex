\documentclass[a4paper,14pt,oneside,final]{extarticle}
\usepackage[top=2cm, bottom=2cm, left=3cm, right=1cm]{geometry}
\usepackage{scrextend}

\usepackage[T2A,T1]{fontenc}
\usepackage[ukrainian,russian,english]{babel}
\usepackage{tempora}
\usepackage{fontspec}
\setmainfont{tempora}

% Зачем: Отключает использование изменяемых межсловных пробелов.
% Почему: Так не принято делать в текстах на русском языке.
\frenchspacing

\usepackage{indentfirst}
\setlength{\parindent}{1.25cm}
\renewcommand{\baselinestretch}{1.5}

% Header
\usepackage{fancyhdr}
\pagestyle{fancy}
\fancyhead{}
\fancyfoot{}
\fancyhead[R]{\small \selectfont \thepage}
\renewcommand{\headrulewidth}{0pt}

% Captions
\usepackage{chngcntr}
\counterwithin{figure}{section}
\counterwithin{table}{section}
\usepackage[tableposition=top]{caption}
\usepackage{subcaption}
\DeclareCaptionLabelFormat{gostfigure}{Рисунок #2}
\DeclareCaptionLabelFormat{gosttable}{Таблиця #2}
\DeclareCaptionLabelSeparator{gost}{~---~}
\captionsetup{labelsep=gost}
\captionsetup[figure]{labelformat=gostfigure}
\captionsetup[table]{labelformat=gosttable}
\renewcommand{\thesubfigure}{\asbuk{subfigure}}

% Sections
\usepackage[explicit]{titlesec}
\newcommand{\sectionbreak}{\clearpage}

\titleformat{\section}
  {\centering}{\thesection \quad}{0pt}{\MakeUppercase{#1}}
\titleformat{\subsection}[block]
  {\bfseries}{\thesubsection \quad #1}{0cm}{}

\titlespacing{\section} {0cm}{0cm}{21pt}
\titlespacing{\subsection} {\parindent}{21pt}{0cm}
\titlespacing{\subsubsection} {\parindent}{0cm}{0cm}

% Lists
\usepackage{enumitem}
\renewcommand\labelitemi{--}
\setlist[itemize]{noitemsep, topsep=0pt, wide}
\setlist[enumerate]{noitemsep, topsep=0pt, wide, label=\arabic*}
\setlist[description]{labelsep=0pt, noitemsep, topsep=0pt, leftmargin=2\parindent, labelindent=\parindent, labelwidth=\parindent, font=\normalfont}

% Toc
\usepackage{tocloft}
\tocloftpagestyle{fancy}
\renewcommand{\cfttoctitlefont}{}
\setlength{\cftbeforesecskip}{0pt}
\renewcommand{\cftsecfont}{}
\renewcommand{\cftsecpagefont}{}
\renewcommand{\cftsecleader}{\cftdotfill{\cftdotsep}}

\usepackage{float}
\usepackage{pgfplots}
\usepackage{graphicx}
\usepackage{multirow}
\usepackage{amssymb,amsfonts,amsmath,amsthm}
\usepackage{csquotes}

\usepackage{listings}
\lstset{basicstyle=\footnotesize\ttfamily,breaklines=true}
\lstset{language=Matlab}

\usepackage[
	backend=biber,
	sorting=none,
	language=auto,
	autolang=other
]{biblatex}
\DeclareFieldFormat{labelnumberwidth}{#1}

\lstdefinelanguage{Python}{
  keywords={and, break, class, continue, def, yield, del, elif, else, except, exec, finally, for, from, global, if, import, in, lambda, not, or, pass, print, raise, return, try, while, assert, with},
  keywordstyle=\color{NavyBlue}\bfseries,
  ndkeywords={True, False},
  ndkeywordstyle=\color{BurntOrange}\bfseries,
  emph={as},
  emphstyle={\color{OrangeRed}},
  identifierstyle=\color{black},
  sensitive=true,
  commentstyle=\color{gray}\ttfamily,
  comment=[l]{\#},
  morecomment=[s]{/*}{*/},
  stringstyle=\color{ForestGreen}\ttfamily,
  morestring=[b]',
  morestring=[s]{"""*}{*"""},
}


\addbibresource{essay3.bib}

\lhead{\small \selectfont Чепурний А.С., 2.КН201н.8а}

\begin{document}
\Ukrainian

\section*{Средства индивидуальной защиты}
\subsection*{Определение размера противогаза ГП-5}
Определить размер шлем-маски ГП-5 можно двумя способами:
\begin{enumerate}
	\item Размер определяют по данным двух измерений головы:
	\begin{itemize}
		\item по замкнутой линии, проходящей через макушку, подбородок и щеки (см).
		\item по линии, соединяющей отверстия ушей и проходящей через надбровные дуги. Результаты обоих измерений складываются, и по таблице определяется размер противогаза (см).
	\end{itemize} 
	
	Требуемый размер шлема-маски:
	\begin{itemize}
		\item до 92 --- 0;
		\item от 92 до 95.5 --- 1;
		\item от 95.5 до 99 --- 2;
		\item от 99 до 102.5 --- 3;
		\item более 102.5 --- 4;
	\end{itemize}
	\item Для определения размера шлема-маски ГП-5 достаточно мерной лентой измерить голову только по замкнутой линии, проходящей через макушку, подбородок и щеки.

	Требуемый размер шлема-маски:
	\begin{itemize} 
		\item до 63.5 --- 0;
		\item от 63.5 до 65.5 --- 1;
		\item от 66.0 до 68.0 --- 2;
		\item от 68.5 до 70.5 --- 3;
		\item более 71.0 –-- 4.
	\end{itemize}
\end{enumerate}

При втором способе для определения размера шлема-маски ГП-5 достаточно мерной лентой измерить голову только по замкнутой линии, проходящей через макушку, подбородок и щеки, и определить ее размер по таблице.

\subsection*{Принцип действия и устройство противогаза ГП5}
Принцип защитного действия противогаза ГП5 основан на том, что используемый для дыхания воздух предварительно очищается (фильтруется) от отравляющих, радиоактивных веществ и бактериальных (биологических) средств в противогазовой коробке. Для этого противогазовая коробка снаряжена специальным поглотителем и противодымным (аэрозольным) фильтром.

Противогазовая коробка имеет цилиндрическую форму. На крышке коробки имеется навинтованная горловина для присоединения коробки к лицевой части противогаза, а в дне коробки - круглое отверстие, через которое поступает вдыхаемый воздух.

Лицевая часть противогаза ГП5 обеспечивает поведение очищенного в противогазовой коробке воздуха к органам дыхания и защищает глаза и лицо от попадания на них отравляющих, радиоактивных веществ и бактериальных (биологических) средств. Лицевая часть состоит из резинового корпуса с обтекателями и очками, клапанной коробки (К-62) с клапанами вдоха и выдоха. Шлем-маски лицевых частей противогаза ГП5 выпускаются 5 размеров: 0, 1, 2, 3, 4. Размер обозначается цифрой на подбородочной части шлем-маски.

Клапанная коробка К-62 лицевой части служит для распределения потоков вдыхаемого и выдыхаемого воздуха. Внутри клапанной коробки устанавливаются вдыхательный и два выдыхательных клапана (основной и дополнительный). Выдыхательные клапаны наиболее ответственные и, вместе с тем, наиболее уязвимые детали клапанной коробки, так как при их неисправности (засорение, замерзание и т.п.) наружный зараженный воздух будет проникать в подмасочное пространство, минуя противогазовую коробку.

Противогазовая сумка снабжена плечевой тесьмой с передвижными пряжками для ношения противогаза через плечо и тесьмой для закрепления противогаза на туловище. Кроме того, сумка имеет один плоский и два объемных кармана. Плоский карман предназначен для размещения коробки с незапотевающими пленками, два объемных кармана - один для перевязочного пакета, другой для индивидуального противохимического пакета.

Комплект незапотевающих пленок предназначен для защиты очковых стекол от запотевания.

\subsection*{Недостатки}
Гражданский противогаз ГП-5 не обеспечивает защиту от аммиака и его производных, органических паров и газов с температурой кипения менее 65 °C (таких как: метан, этан, ацетилен, окись этилена, изобутан и др.), монооксида углерода, оксидов азота.

\end{document}
