\usepackage{tikz}

\counterwithout{figure}{section}
\counterwithout{table}{section}
\counterwithout{equation}{section}

\titleformat{\subsection}[block]
  {\bfseries\filcenter}{#1}{0cm}{}
\titlespacing{\subsection}{0cm}{21pt}{21pt}

\DeclareCaptionLabelFormat{gosttable}{Таблица #2}

\usepackage{float}
\usepackage{pgfplots}
\usepackage{graphicx}
\usepackage{multirow}
\usepackage{amssymb,amsfonts,amsmath,amsthm}

\usepackage{listings}
\lstset{basicstyle=\footnotesize\ttfamily,breaklines=true}
\lstset{language=Matlab}

\lstdefinelanguage{Python}{
  keywords={and, break, class, continue, def, yield, del, elif, else, except, exec, finally, for, from, global, if, import, in, lambda, not, or, pass, print, raise, return, try, while, assert, with},
  keywordstyle=\color{NavyBlue}\bfseries,
  ndkeywords={True, False},
  ndkeywordstyle=\color{BurntOrange}\bfseries,
  emph={as},
  emphstyle={\color{OrangeRed}},
  identifierstyle=\color{black},
  sensitive=true,
  commentstyle=\color{gray}\ttfamily,
  comment=[l]{\#},
  morecomment=[s]{/*}{*/},
  stringstyle=\color{ForestGreen}\ttfamily,
  morestring=[b]',
  morestring=[s]{"""*}{*"""},
}


\lhead{\small \selectfont Чепурний А.С., 2.КН201н.8а}

\begin{document}
\Ukrainian

\section{Практика: Стійкість об’єктів економіки до ударної хвилі (бензин)}

\begin{table}[H]
\caption{Вихідні дані для оцінки стійкості до впливу ударної хвилі при вибуху ємності з бензином -- вариант 5}
\begin{tabular}{|p{5cm}|p{8cm}|} 
	\hline
	Об’єм резервуару з бензином, $\textup{м}^3$ & 1000 \\ \hline
	Заповнення ємності бензином, \% & 70 \\ \hline
	Масова частка бензину в паровій фазі, \% & 2.5 \\ \hline
	Відстань & 200 \\ \hline
	Споруда цеху & Промисловий із металевим каркасом \\ \hline
	Устаткування цеху & Контрольно-вимірювальні прилади (не захищені), верстати важкі і середні, трубопроводи на залізобетонних естакадах, трубопроводи наземні, наземні кабельні лінії \\ \hline
\end{tabular}
\end{table}

\subsection{Визначення за вихідними даними максимально можливого надлишкового тиску, який очікується на об’єкті}
Визначимо об’єм пару бензину $V_{steam}$ у резервуарі, $\textup{м}^3$:
\[
	V_{steam} = 1000 - 1000 \cdot \frac{70}{100} = 300 \textup{м}^3
\] 

Визначимо об’єм бензину $V_{steam_g}$ та масу бензину $Q_{steam_g}$ в пароподібному стані, т:
\begin{align*}
	V_{steam_g} &= 300 \cdot \frac{2.5}{100} = 7.5 \textup{м}^3, \\
	Q_{steam_g} &= 7.5 \cdot 0.75 = 5.625 \textup{т}
\end{align*}

Визначаємо максимальний можливий надлишковий тиск $\Delta P_{\textup{ф}_{max}}$ в районі цеху на перехресті ординат - відстань $200$ м. і маса бензину в паровій фазі $5.625$ т., кПа. 
\[
	\Delta P_{\textup{ф}_{max}} = 32 \textup{кПа}
\]

Отже, при руйнуванні ємкості і вибуху парів бензину надлишковий тиск в районі цеху може скласти 32 кПа.

\subsection{Визначення стійкості елементів об’єкту (цеху) та об’єкту в цілому до дії ударної хвил}
\begin{table}[H]
\caption{Результати з оцінки стійкості роботи цеха до дії ударної хвилі}
\begin{tabular}{|p{5cm}|p{3cm}|p{3cm}|p{3cm}|} 
	\hline
	Елементи об’єкту & Слабкі руйнування & Середні руйнування & Сильні руйнування \\ \hline
	Промисловий із металевим каркасом & 20-40 & 40-50 & 50-60 \\ \hline
	Контрольно-вимірювальні прилади (не захищені) & 20-40 & 40-50 & 50-60 \\ \hline
	Верстати важкі і середні & 15-25 & 25-35 & 35-45 \\ \hline
	Трубопроводи на залізобетонних естакадах & 20-30 & 30-40 & 40-50 \\ \hline
	Трубопроводи наземні & 20-50 & 50-130 & 130 і більше \\ \hline
	Наземні кабельні лінії & 10-30 & 30-50 & 50-60 \\ \hline
\end{tabular}
\end{table}

Висновки і пропозиції:
\begin{enumerate}
	\item Оскільки, знайдена межа стійкості цеху $P_{\textup{ф}_{lim}} < P_{\textup{ф}_{max}}$, адже 25 кПа < 32 кПа, то цех нестійкий до ударної хвилі.
	\item Найслабкішим елементом до дії ударної хвилі є середні верстати цеху ($P_\textup{ф} = 25$ кПа), тому при $P_\textup{ф} = 32$ кПа випуск продукції не може бути налагоджений.
	\item Межа стійкості більшості елементів 30 кПа, а максимальний надлишковий тиск складає 32 кПа, тому доцільно підвищувати межу стійкості до 32 кПа.
\end{enumerate}

Отже, при максимальному надлишковому тиску $P_{\textup{ф}_{max}} = 32$ кПа , що очікується на об’єкті, цех нестійкий у роботі. Межу стійкості об’єкта необхідно підвищувати до $P_{\textup{ф}_{max}}$.

\subsection*{Рекомендації}
Для підвищення стійкості цеху до 32 кПа необхідно підвищити стійкість верстатів; електрокабельну мережу і трубопроводи прокласти під землею; 

\end{document}
