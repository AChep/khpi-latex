\usepackage{tikz}

\counterwithout{figure}{section}
\counterwithout{table}{section}
\counterwithout{equation}{section}

\titleformat{\subsection}[block]
  {\bfseries\filcenter}{#1}{0cm}{}
\titlespacing{\subsection}{0cm}{21pt}{21pt}

\DeclareCaptionLabelFormat{gosttable}{Таблица #2}

\usepackage{float}
\usepackage{pgfplots}
\usepackage{graphicx}
\usepackage{multirow}
\usepackage{amssymb,amsfonts,amsmath,amsthm}

\usepackage{listings}
\lstset{basicstyle=\footnotesize\ttfamily,breaklines=true}
\lstset{language=Matlab}

\lstdefinelanguage{Python}{
  keywords={and, break, class, continue, def, yield, del, elif, else, except, exec, finally, for, from, global, if, import, in, lambda, not, or, pass, print, raise, return, try, while, assert, with},
  keywordstyle=\color{NavyBlue}\bfseries,
  ndkeywords={True, False},
  ndkeywordstyle=\color{BurntOrange}\bfseries,
  emph={as},
  emphstyle={\color{OrangeRed}},
  identifierstyle=\color{black},
  sensitive=true,
  commentstyle=\color{gray}\ttfamily,
  comment=[l]{\#},
  morecomment=[s]{/*}{*/},
  stringstyle=\color{ForestGreen}\ttfamily,
  morestring=[b]',
  morestring=[s]{"""*}{*"""},
}


\lhead{\small \selectfont Чепурний А.С., 2.КН201н.8а}

\begin{document}
\Ukrainian

\section{Практика: Стійкість об’єктів економіки до ударної хвилі (пропан)}

\begin{table}[H]
\caption{Вихідні дані для оцінки стійкості до впливу ударної хвилі при вибуху ємності з пропаном -- вариант 5}
\begin{tabular}{|p{5cm}|p{8cm}|} 
	\hline
	Маса ємності з пропаном, т & 400 \\ \hline
	Відстань & 800 \\ \hline
	Споруда цеху & Зі збірного залізобетону \\ \hline
	Устаткування цеху & Трансформатори, масляні вимикачі, повітряна лінія високої напруги \\ \hline
\end{tabular}
\end{table}

\subsection{Визначення за вихідними даними максимально можливого надлишкового тиску, який очікується на об’єкті}
Визначаємо радіус $r_1$ зони детаніційної хвилі, м:
\[
	r_1 = 17.5 \sqrt[3]{400} = 128.9411 \textup{м}
\] 

Визначаємо радіус $r_2$ зони дії продуктів вибуху, м:
\[
	r_2 = 1.7 \cdot 128.9411 = 219.1998 \textup{м}
\] 

Визначимо відносну величину $\Psi$:
\[
	\Psi = 0.24 \cdot \frac{800}{128.9411} = 1.4890
\] 

Визначимо максимально можливий надлишковий тиск $P_{\textup{ф}_{max}}$ у районі об’єкта (зоні дії повітряної ударної хвилі), кПа.

Так як $\Psi<2$, то скористаємося формулою:
\[
	P_{\textup{ф}_{max}} = \frac{700}{3 (\sqrt{1+29.8 \Psi^3} - 1)} = 26.0158 \textup{кПа}
\]

Отже, при руйнуванні ємкості і вибуху зрідженого пропану надлишковий тиск в районі цеху може скласти 26 кПа.

\subsection{Визначення стійкості елементів об’єкту (цеху) та об’єкту в цілому до дії ударної хвил}
\begin{table}[H]
\caption{Результати з оцінки стійкості роботи цеха до дії ударної хвилі}
\begin{tabular}{|p{5cm}|p{3cm}|p{3cm}|p{3cm}|} 
	\hline
	Елементи об’єкту & Слабкі руйнування & Середні руйнування & Сильні руйнування \\ \hline
	Будинки зі збірного залізобетону & 20-40 & 40-50 & 50-60 \\ \hline
	Трансформатори та генератори & 30-40 & 40-60 & 60 і більше \\ \hline
	Масляні вимикачі & 10-20 & 20-30 & 30 і більше \\ \hline
	Повітряні лінії високої напруги & 25-30 & 30-50 & 50-70 \\ \hline
\end{tabular}
\end{table}

Висновки і пропозиції:
\begin{enumerate}
	\item Оскільки, знайдена межа стійкості цеху $P_{\textup{ф}_{lim}} < P_{\textup{ф}_{max}}$, адже 20 кПа < 26 кПа, то цех нестійкий до ударної хвилі.
	\item Найслабкішим елементом до дії ударної хвилі є масляні вимикачі цеху ($P_\textup{ф} = 20$ кПа), тому при $P_\textup{ф} = 26$ кПа випуск продукції не може бути налагоджений.
\end{enumerate}

Отже, при максимальному надлишковому тиску $P_{\textup{ф}_{max}} = 26$ кПа , що очікується на об’єкті, цех нестійкий у роботі. Межу стійкості об’єкта необхідно підвищувати до $P_{\textup{ф}_{max}}$.

\subsection*{Рекомендації}
Для підвищення стійкості цеху до 26 кПа необхідно підвищити стійкість масляних вимикачів -- додати ізоляції.

\end{document}

