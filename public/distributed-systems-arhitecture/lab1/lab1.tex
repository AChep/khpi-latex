\usepackage{tikz}

\counterwithout{figure}{section}
\counterwithout{table}{section}
\counterwithout{equation}{section}

\titleformat{\subsection}[block]
  {\bfseries\filcenter}{#1}{0cm}{}
\titlespacing{\subsection}{0cm}{21pt}{21pt}

\DeclareCaptionLabelFormat{gosttable}{Таблица #2}

\usepackage{float}
\usepackage{pgfplots}
\usepackage{graphicx}
\usepackage{multirow}
\usepackage{amssymb,amsfonts,amsmath,amsthm}

\usepackage{listings}
\lstset{basicstyle=\footnotesize\ttfamily,breaklines=true}
\lstset{language=Matlab}


\newcommand{\labnumber}{1} % first lab
\usepackage{tikz}

\counterwithout{figure}{section}
\counterwithout{table}{section}
\counterwithout{equation}{section}

\titleformat{\subsection}[block]
  {\bfseries\filcenter}{#1}{0cm}{}
\titlespacing{\subsection}{0cm}{21pt}{21pt}

\DeclareCaptionLabelFormat{gosttable}{Таблица #2}

\newcommand{\khpistudentgroup}{2.КН201н.8а}
\newcommand{\khpistudentname}{Чепурний~А.~С.}

\newcommand{\khpidepartment}{Програмна інженерія та інформаційні технології управління}
\newcommand{\khpititlewhat}{
	Розрахунково-графічне завдання \\
	з предмету <<Фреймворки та платформи>>
}
\newcommand{\khpititlewho}{
	Виконав: \\
	\hspace*{\parindent} ст. групи \khpistudentgroup \\
	\hspace*{\parindent} \khpistudentname \\
	Перевірила: \\
	\hspace*{\parindent} к. т. н., вик. каф. ПІІТУ \\
	\hspace*{\parindent} Добряк~В.~С. \\
}


\graphicspath{{figures/}}

\begin{document}
\Ukrainian

\begin{titlepage}

\begin{center}
	МІНІСТЕРСТВО ОСВІТИ І НАУКИ УКРАЇНИ \\
	НАЦІОНАЛЬНИЙ ТЕХНІЧНИЙ УНІВЕРСИТЕТ \\
	«ХАРКІВСЬКИЙ ПОЛІТЕХНІЧНИЙ ІНСТИТУТ» \\
	Кафедра <<\khpidepartment>> \\
\end{center}

\vspace{6cm}

\begin{center}
	\khpititlewhat
\end{center}

\vspace{3cm}

\begin{addmargin}[10cm]{0cm}
	\khpititlewho
\end{addmargin}

\vspace{\fill}

\begin{center}
	Харків \the\year
\end{center}

\end{titlepage}

\addtocounter{page}{1}

\section*{Grid-системи та їх ефективність}
\subsubsection*{Ціль роботи}
Ознайомитися з роботой Grid-систем на прикладі системи World Community Grid.

\subsection*{Хід роботи}
World Community Grid --- це глобальна спілка користувачів, які надають вільні ресурси своїх комп'ютерів для вирішення складних завдань.

Пропонується великий вибір досліджень з боротьби проти раку, СНІДу, грипу та інших захворювань, з яких учасник може зробити вибір. 
Проект обчислюється не лише добровольцями але і партнерськими організаціями з багатьох країн. 
На вересень 2018 року нараховується 756 000 зареєстрованих користувачів, процесорних обчислень більш, ніж на 500 тисяч років.

При запуску мобільного застосунку BOINC надається можливість обрати проекти над якими буде працювати програма (рисунок~\ref{fig:scr_projects}).

\begin{figure}[h]
    \centering
    \includegraphics[width=0.25\textwidth]{scr_projects}
    \caption{Доступні проекти}
    \label{fig:scr_projects}
\end{figure}

Після встановлення програма оцінює продуктивність системи (рисунок~\ref{fig:scr_benchmark}).

\begin{figure}[h]
    \centering
    \includegraphics[width=0.25\textwidth]{scr_benchmark}
    \caption{Оцінювання продуктивності системи}
    \label{fig:scr_benchmark}
\end{figure}

Після налаштування програма починає роботу і у вікні можна побачити задачі що виконується та інформацію про них (рисунок~\ref{fig:scr_tasks}).

\begin{figure}[h]
    \centering
    \includegraphics[width=0.25\textwidth]{scr_tasks}
    \caption{Поточні задачі}
    \label{fig:scr_tasks}
\end{figure}

Розробники завіряють, що програма не заважає роботі системи і використовує лише не зайняті ресурси. Доступні опції застосунку показано на рисуноку~\ref{fig:scr_settings}.

\begin{figure}[h]
    \centering
    \includegraphics[width=0.25\textwidth]{scr_settings}
    \caption{Доступні опції}
    \label{fig:scr_settings}
\end{figure}

\subsection*{Висновки} 
У ході виконання лабораторної роботи було отримано досвід роботи з World Community Grid.

\end{document}
