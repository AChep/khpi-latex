\usepackage{tikz}

\counterwithout{figure}{section}
\counterwithout{table}{section}
\counterwithout{equation}{section}

\titleformat{\subsection}[block]
  {\bfseries\filcenter}{#1}{0cm}{}
\titlespacing{\subsection}{0cm}{21pt}{21pt}

\DeclareCaptionLabelFormat{gosttable}{Таблица #2}

\usepackage{float}
\usepackage{pgfplots}
\usepackage{graphicx}
\usepackage{multirow}
\usepackage{amssymb,amsfonts,amsmath,amsthm}

\usepackage{listings}
\lstset{basicstyle=\footnotesize\ttfamily,breaklines=true}
\lstset{language=Matlab}

\lstdefinelanguage{Python}{
  keywords={and, break, class, continue, def, yield, del, elif, else, except, exec, finally, for, from, global, if, import, in, lambda, not, or, pass, print, raise, return, try, while, assert, with},
  keywordstyle=\color{NavyBlue}\bfseries,
  ndkeywords={True, False},
  ndkeywordstyle=\color{BurntOrange}\bfseries,
  emph={as},
  emphstyle={\color{OrangeRed}},
  identifierstyle=\color{black},
  sensitive=true,
  commentstyle=\color{gray}\ttfamily,
  comment=[l]{\#},
  morecomment=[s]{/*}{*/},
  stringstyle=\color{ForestGreen}\ttfamily,
  morestring=[b]',
  morestring=[s]{"""*}{*"""},
}


\newcommand{\labnumber}{2}
\usepackage{tikz}

\counterwithout{figure}{section}
\counterwithout{table}{section}
\counterwithout{equation}{section}

\titleformat{\subsection}[block]
  {\bfseries\filcenter}{#1}{0cm}{}
\titlespacing{\subsection}{0cm}{21pt}{21pt}

\DeclareCaptionLabelFormat{gosttable}{Таблица #2}

\newcommand{\khpistudentgroup}{2.КН201н.8а}
\newcommand{\khpistudentname}{Чепурний~А.~С.}

\newcommand{\khpidepartment}{Програмна інженерія та інформаційні технології управління}
\newcommand{\khpititlewhat}{
	Розрахунково-графічне завдання \\
	з предмету <<Фреймворки та платформи>>
}
\newcommand{\khpititlewho}{
	Виконав: \\
	\hspace*{\parindent} ст. групи \khpistudentgroup \\
	\hspace*{\parindent} \khpistudentname \\
	Перевірила: \\
	\hspace*{\parindent} к. т. н., вик. каф. ПІІТУ \\
	\hspace*{\parindent} Добряк~В.~С. \\
}


\graphicspath{{figures/}}

\begin{document}
\Ukrainian

\begin{titlepage}

\begin{center}
	МІНІСТЕРСТВО ОСВІТИ І НАУКИ УКРАЇНИ \\
	НАЦІОНАЛЬНИЙ ТЕХНІЧНИЙ УНІВЕРСИТЕТ \\
	«ХАРКІВСЬКИЙ ПОЛІТЕХНІЧНИЙ ІНСТИТУТ» \\
	Кафедра <<\khpidepartment>> \\
\end{center}

\vspace{6cm}

\begin{center}
	\khpititlewhat
\end{center}

\vspace{3cm}

\begin{addmargin}[10cm]{0cm}
	\khpititlewho
\end{addmargin}

\vspace{\fill}

\begin{center}
	Харків \the\year
\end{center}

\end{titlepage}

\addtocounter{page}{1}

\section*{Ресурсне планування проектних робіт}
\subsubsection*{Мета роботи}
\begin{itemize}
	\item набути навичок ресурсного планування проектів.
	\item вивчити принципи призначення ресурсів на роботи, розподілу їх навантаження, вирівнювання переобтяжених ресурсів та аналізу результатів вирівнювання.
\end{itemize}

\subsubsection*{Завдання}
\begin{enumerate}
  	\item Скласти список трудових ресурсів.
  	\item Скласти список матеріальних ресурсів.
  	\item Визначити типи завдань.
  	\item Призначити трудові ресурси на завдання.
  	\item Заповнити відомості про призначення ресурсів.
  	\item Призначити матеріальні ресурси на завдання.
  	\item Визначити переобтяжені ресурси та вирівняти їх завантаження засобами MS Project.
\end{enumerate}

\textit{Варіант 10.}

\subsection*{Хід роботи}
Для ручного вирівнювання перевантаженого ресурсу <<Керівник проекту>> обом конфліктуючим роботам <<Технічне проектування>> та <<Робоче проектування>> було надано профіль <<Завантаження в початку>> та <<Завантаження в кінці>> відповідно. Також на роботу <<Робоче проектування>> було зменшено кількість годин.  

Результати роботи, а саме: Ресурси з перевищенням доступності, Використання ресурсів, Лист ресурсів, Діаграми Ганта з ресурсами після автоматичного та ручного вимірювань було згенеровано автоматично, та розташовано в окремих файлах.

\subsection*{Висновки}
В процесі виконання лабораторної роботи, було створено
перелік матеріальних та трудових ресурсів, здійснено їх призначення на задачі та виконано вирівнювання перевантажених ресурсів.

\end{document}
