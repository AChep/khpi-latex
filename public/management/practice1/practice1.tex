\usepackage{tikz}

\counterwithout{figure}{section}
\counterwithout{table}{section}
\counterwithout{equation}{section}

\titleformat{\subsection}[block]
  {\bfseries\filcenter}{#1}{0cm}{}
\titlespacing{\subsection}{0cm}{21pt}{21pt}

\DeclareCaptionLabelFormat{gosttable}{Таблица #2}

\usepackage{float}
\usepackage{pgfplots}
\usepackage{graphicx}
\usepackage{multirow}
\usepackage{amssymb,amsfonts,amsmath,amsthm}

\usepackage{listings}
\lstset{basicstyle=\footnotesize\ttfamily,breaklines=true}
\lstset{language=Matlab}


\usepackage{tikz}

\counterwithout{figure}{section}
\counterwithout{table}{section}
\counterwithout{equation}{section}

\titleformat{\subsection}[block]
  {\bfseries\filcenter}{#1}{0cm}{}
\titlespacing{\subsection}{0cm}{21pt}{21pt}

\DeclareCaptionLabelFormat{gosttable}{Таблица #2}

\newcommand{\khpistudentgroup}{2.КН201н.8а}
\newcommand{\khpistudentname}{Чепурний~А.~С.}

\newcommand{\khpidepartment}{Програмна інженерія та інформаційні технології управління}
\newcommand{\khpititlewhat}{
	Розрахунково-графічне завдання \\
	з предмету <<Фреймворки та платформи>>
}
\newcommand{\khpititlewho}{
	Виконав: \\
	\hspace*{\parindent} ст. групи \khpistudentgroup \\
	\hspace*{\parindent} \khpistudentname \\
	Перевірила: \\
	\hspace*{\parindent} к. т. н., вик. каф. ПІІТУ \\
	\hspace*{\parindent} Добряк~В.~С. \\
}


\lhead{\small \selectfont \khpistudentgroup}
\chead{\small \selectfont \khpistudentname}

\graphicspath{{figures/}}

\begin{document}
\Ukrainian

\section*{Практика №1}
\subsection*{Завдання №1}
\subsection*{Завдання №2}
Мережний графік зображено на рисунку~\ref{fig:1_network}. Список повних шляхів та іх повних резервів:
\begin{table}[H]        
  \small
  \begin{tabular}{l|c}
    Шлях & Повний резерв \\ \hline
    \texttt{A-D-I} & 11 \\
    \texttt{A-D-H-J-K} & 3 \\
    \texttt{C-H-J-K} & 3 \\
    \texttt{B-E-H-J-K} & 0 \\
    \texttt{B-F-J-K} & 7 \\
    \texttt{B-G-K} & 10 \\
  \end{tabular}
\end{table}

\begin{figure}[h]
    \centering
        \includegraphics[width=\textwidth]{1_network}
    \caption{Мережний графік №1}
    \label{fig:1_network}
\end{figure}

\clearpage

\subsection*{Завдання №3}
Діаграму Ганта зображено на рисунку~\ref{fig:1_gantt}.

\begin{figure}[h]
    \centering
        \includegraphics[width=\textwidth]{1_gantt}
    \caption{Діаграма Ганта №1}
    \label{fig:1_gantt}
\end{figure}

\subsection*{Завдання №4}
Діаграму Ганта зображено на рисунку~\ref{fig:2_gantt}.

\begin{figure}[h]
    \centering
        \includegraphics[width=\textwidth]{2_gantt}
    \caption{Діаграма Ганта №2}
    \label{fig:2_gantt}
\end{figure}

\end{document}
