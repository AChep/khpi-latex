\usepackage{tikz}

\counterwithout{figure}{section}
\counterwithout{table}{section}
\counterwithout{equation}{section}

\titleformat{\subsection}[block]
  {\bfseries\filcenter}{#1}{0cm}{}
\titlespacing{\subsection}{0cm}{21pt}{21pt}

\DeclareCaptionLabelFormat{gosttable}{Таблица #2}

\usepackage{float}
\usepackage{pgfplots}
\usepackage{graphicx}
\usepackage{multirow}
\usepackage{amssymb,amsfonts,amsmath,amsthm}

\usepackage{listings}
\lstset{basicstyle=\footnotesize\ttfamily,breaklines=true}
\lstset{language=Matlab}

\lstdefinelanguage{CLIPS}{
  keywords={defrule, deftemplate},
  keywordstyle=\color{NavyBlue}\bfseries,
  emph={retract, modify, printout, assert, not},
  emphstyle={\color{OrangeRed}},
  identifierstyle=\color{black},
  sensitive=true,
  commentstyle=\color{gray}\ttfamily,
  comment=[l]{;;;},
}


\newcommand{\labnumber}{5} % second lab
\usepackage{tikz}

\counterwithout{figure}{section}
\counterwithout{table}{section}
\counterwithout{equation}{section}

\titleformat{\subsection}[block]
  {\bfseries\filcenter}{#1}{0cm}{}
\titlespacing{\subsection}{0cm}{21pt}{21pt}

\DeclareCaptionLabelFormat{gosttable}{Таблица #2}

\newcommand{\khpistudentgroup}{2.КН201н.8а}
\newcommand{\khpistudentname}{Чепурний~А.~С.}

\newcommand{\khpidepartment}{Програмна інженерія та інформаційні технології управління}
\newcommand{\khpititlewhat}{
	Розрахунково-графічне завдання \\
	з предмету <<Фреймворки та платформи>>
}
\newcommand{\khpititlewho}{
	Виконав: \\
	\hspace*{\parindent} ст. групи \khpistudentgroup \\
	\hspace*{\parindent} \khpistudentname \\
	Перевірила: \\
	\hspace*{\parindent} к. т. н., вик. каф. ПІІТУ \\
	\hspace*{\parindent} Добряк~В.~С. \\
}


\lstset{language=CLIPS}
\graphicspath{{figures/}}

\begin{document}
\Ukrainian

\begin{titlepage}

\begin{center}
	МІНІСТЕРСТВО ОСВІТИ І НАУКИ УКРАЇНИ \\
	НАЦІОНАЛЬНИЙ ТЕХНІЧНИЙ УНІВЕРСИТЕТ \\
	«ХАРКІВСЬКИЙ ПОЛІТЕХНІЧНИЙ ІНСТИТУТ» \\
	Кафедра <<\khpidepartment>> \\
\end{center}

\vspace{6cm}

\begin{center}
	\khpititlewhat
\end{center}

\vspace{3cm}

\begin{addmargin}[10cm]{0cm}
	\khpititlewho
\end{addmargin}

\vspace{\fill}

\begin{center}
	Харків \the\year
\end{center}

\end{titlepage}

\addtocounter{page}{1}

\section*{Використання об'єктно-орієнтованого розширення CLIPS при створенні експертних систем}
\subsection*{Мета}
\begin{enumerate}
	\item Навчитися вирішувати типові завдання штучного інтелекту.
	\item Оволодіти методами об'єктно-орієнтованого розширення CLIPS.
\end{enumerate}

\subsection*{Завдання}
\begin{enumerate}
	\item Для обраної завдання штучного інтелекту описати розробляються класи і їх ієрархію.
	\item Розробити і налагодити методи цих класів.
	\item Продемонструвати працездатність експертної системи при пошуку кінцевого рішення з різних початкових станів фактів.
\end{enumerate}

\begin{center}
	Задача штучного інтелекту: мавпа та банани
\end{center}

\subsection*{Хід роботи}
Діаграма класів розробленої програми представлено на рисунку~\ref{fig:uml_class}.

\begin{figure}[H]
	\centering
	    \includegraphics{uml_class}
	\caption{UML-діаграма класів}
	\label{fig:uml_class}
\end{figure}

Були розроблені правила для таких сценаріїв:
\texttt{use-box-to-hold},
\texttt{climb-box-to-hold},
\texttt{grab-object-from-box},
\texttt{climb-to-hold},
\texttt{walk-to-hold},
\texttt{drop-to-hold},
\texttt{grab-object},
\texttt{drop-object},
\texttt{hold-object-to-move},
\texttt{move-object-to-place},
\texttt{drop-object-once-moved},
\texttt{already-moved-object},
\texttt{already-at-place},
\texttt{get-on-floor-to-walk},
\texttt{walk-holding-nothing},
\texttt{walk-holding-object},
\texttt{jump-onto-floor},
\texttt{walk-to-place-to-climb},
\texttt{drop-to-climb},
\texttt{climb-indirectly},
\texttt{climb-directly},
\texttt{already-on-object}.

Початковий стан системи було задано як:
\begin{lstlisting}
(defrule startup ""
  =>
  (assert (monkey (location door) (on-top-of floor) (holding blank)))
  (assert (thing (name box) (location window)))
  (assert (thing (name bananas) (location center-of-the-room) (on-top-of ceiling)))
  (assert (goal-is-to (action hold) (arguments bananas))))
\end{lstlisting}

Результат виконання програми представлено на рисунку~\ref{fig:result}.

\begin{figure}[H]
	\centering
	    \includegraphics{result}
	\caption{Результат виконання програми}
	\label{fig:result}
\end{figure}

\subsection*{Висновки}
В процесі виконання лабораторної роботи, було написано програму на мові CLIPS для вирішення задачі <<Мавпа та банани>>.

\subsection*{Додаток}
Вихідний код програми:
\lstinputlisting{code/main.clp} 

\end{document}
