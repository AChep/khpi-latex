\usepackage{tikz}

\counterwithout{figure}{section}
\counterwithout{table}{section}
\counterwithout{equation}{section}

\titleformat{\subsection}[block]
  {\bfseries\filcenter}{#1}{0cm}{}
\titlespacing{\subsection}{0cm}{21pt}{21pt}

\DeclareCaptionLabelFormat{gosttable}{Таблица #2}

\usepackage{float}
\usepackage{pgfplots}
\usepackage{graphicx}
\usepackage{multirow}
\usepackage{amssymb,amsfonts,amsmath,amsthm}

\usepackage{listings}
\lstset{basicstyle=\footnotesize\ttfamily,breaklines=true}
\lstset{language=Matlab}

\lstdefinelanguage{Python}{
  keywords={and, break, class, continue, def, yield, del, elif, else, except, exec, finally, for, from, global, if, import, in, lambda, not, or, pass, print, raise, return, try, while, assert, with},
  keywordstyle=\color{NavyBlue}\bfseries,
  ndkeywords={True, False},
  ndkeywordstyle=\color{BurntOrange}\bfseries,
  emph={as},
  emphstyle={\color{OrangeRed}},
  identifierstyle=\color{black},
  sensitive=true,
  commentstyle=\color{gray}\ttfamily,
  comment=[l]{\#},
  morecomment=[s]{/*}{*/},
  stringstyle=\color{ForestGreen}\ttfamily,
  morestring=[b]',
  morestring=[s]{"""*}{*"""},
}


\newcommand{\tasknumber}{2} % first

\usepackage{tikz}

\counterwithout{figure}{section}
\counterwithout{table}{section}
\counterwithout{equation}{section}

\titleformat{\subsection}[block]
  {\bfseries\filcenter}{#1}{0cm}{}
\titlespacing{\subsection}{0cm}{21pt}{21pt}

\DeclareCaptionLabelFormat{gosttable}{Таблица #2}

\input{../constants_task}
\usepackage{lscape}
\usepackage{appendix}

\graphicspath{{figures/}}

\begin{document}
\Ukrainian

\begin{titlepage}

\begin{center}
	МІНІСТЕРСТВО ОСВІТИ І НАУКИ УКРАЇНИ \\
	НАЦІОНАЛЬНИЙ ТЕХНІЧНИЙ УНІВЕРСИТЕТ \\
	«ХАРКІВСЬКИЙ ПОЛІТЕХНІЧНИЙ ІНСТИТУТ» \\
	Кафедра <<\khpidepartment>> \\
\end{center}

\vspace{6cm}

\begin{center}
	\khpititlewhat
\end{center}

\vspace{3cm}

\begin{addmargin}[10cm]{0cm}
	\khpititlewho
\end{addmargin}

\vspace{\fill}

\begin{center}
	Харків \the\year
\end{center}

\end{titlepage}

\addtocounter{page}{1}

\subsection{Інтелектуальна карта}
Інтелектуальна карта для моделей представлення знань зображена на рисунку~\ref{fig:intellig}.

\subsection{Концептуальна карта}
Концептуальна карта для моделей представлення знань зображена на рисунку~\ref{fig:concept}.

\subsection*{Висновки}
Інтелектуальні карти --- це тип представлення знань, де інформація представляється у виді радіанної структури. 
Інтелектуальні карти допомагають систематизувати дані, можуть бути використані для класифікації об’єктів або відображення ієрархічної структури даних.

Концептуальні карти --- це тип представлення знань, де інформація подається у виді об’єктів предметної області та зв’язків між ними. 
Концептуальні карти допомагають систематизувати дані, можуть бути використані для представлення знань про предметну область з великою кількістю зв’язків між об’єктами (поняттями). 

\begin{appendices}
    % custom commands 
    \newcommand\appendixsection[1]{
        \addtocounter{section}{1}
        \clearpage
        \section*{Додаток \thesection. #1}
        \addcontentsline{toc}{section}{Додаток \thesection. #1}
    }
    \appendixsection{Вихідний код}
Вихідний код файлу \texttt{main.py}:
\lstinputlisting{code/main.py} 

Вихідний код файлу \texttt{gradients.py}:
\lstinputlisting{code/gradients.py} 

    \begin{landscape}
\appendixsection{Концептуальна карта моделей представлення знань}

\begin{figure}[H]
    \centering
        \includegraphics[width=0.7\linewidth]{concept}
    \caption{Концептуальна карта моделей представлення знань}
    \label{fig:concept}
\end{figure}

\end{landscape}

 \end{appendices}

\end{document}
