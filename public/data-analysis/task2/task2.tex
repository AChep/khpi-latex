\documentclass[a4paper,14pt,oneside,final]{extarticle}
\usepackage[top=2cm, bottom=2cm, left=3cm, right=1cm]{geometry}
\usepackage{scrextend}

\usepackage[T2A,T1]{fontenc}
\usepackage[ukrainian,russian,english]{babel}
\usepackage{tempora}
\usepackage{fontspec}
\setmainfont{tempora}

% Зачем: Отключает использование изменяемых межсловных пробелов.
% Почему: Так не принято делать в текстах на русском языке.
\frenchspacing

\usepackage{indentfirst}
\setlength{\parindent}{1.25cm}
\renewcommand{\baselinestretch}{1.5}

% Header
\usepackage{fancyhdr}
\pagestyle{fancy}
\fancyhead{}
\fancyfoot{}
\fancyhead[R]{\small \selectfont \thepage}
\renewcommand{\headrulewidth}{0pt}

% Captions
\usepackage{chngcntr}
\counterwithin{figure}{section}
\counterwithin{table}{section}
\usepackage[tableposition=top]{caption}
\usepackage{subcaption}
\DeclareCaptionLabelFormat{gostfigure}{Рисунок #2}
\DeclareCaptionLabelFormat{gosttable}{Таблиця #2}
\DeclareCaptionLabelSeparator{gost}{~---~}
\captionsetup{labelsep=gost}
\captionsetup[figure]{labelformat=gostfigure}
\captionsetup[table]{labelformat=gosttable}
\renewcommand{\thesubfigure}{\asbuk{subfigure}}

% Sections
\usepackage[explicit]{titlesec}
\newcommand{\sectionbreak}{\clearpage}

\titleformat{\section}
  {\centering}{\thesection \quad}{0pt}{\MakeUppercase{#1}}
\titleformat{\subsection}[block]
  {\bfseries}{\thesubsection \quad #1}{0cm}{}

\titlespacing{\section} {0cm}{0cm}{21pt}
\titlespacing{\subsection} {\parindent}{21pt}{0cm}
\titlespacing{\subsubsection} {\parindent}{0cm}{0cm}

% Lists
\usepackage{enumitem}
\renewcommand\labelitemi{--}
\setlist[itemize]{noitemsep, topsep=0pt, wide}
\setlist[enumerate]{noitemsep, topsep=0pt, wide, label=\arabic*}
\setlist[description]{labelsep=0pt, noitemsep, topsep=0pt, leftmargin=2\parindent, labelindent=\parindent, labelwidth=\parindent, font=\normalfont}

% Toc
\usepackage{tocloft}
\tocloftpagestyle{fancy}
\renewcommand{\cfttoctitlefont}{}
\setlength{\cftbeforesecskip}{0pt}
\renewcommand{\cftsecfont}{}
\renewcommand{\cftsecpagefont}{}
\renewcommand{\cftsecleader}{\cftdotfill{\cftdotsep}}

\usepackage{float}
\usepackage{pgfplots}
\usepackage{graphicx}
\usepackage{multirow}
\usepackage{amssymb,amsfonts,amsmath,amsthm}
\usepackage{csquotes}

\usepackage{listings}
\lstset{basicstyle=\footnotesize\ttfamily,breaklines=true}
\lstset{language=Matlab}

\usepackage[
	backend=biber,
	sorting=none,
	language=auto,
	autolang=other
]{biblatex}
\DeclareFieldFormat{labelnumberwidth}{#1}

\lstdefinelanguage{Python}{
  keywords={and, break, class, continue, def, yield, del, elif, else, except, exec, finally, for, from, global, if, import, in, lambda, not, or, pass, print, raise, return, try, while, assert, with},
  keywordstyle=\color{NavyBlue}\bfseries,
  ndkeywords={True, False},
  ndkeywordstyle=\color{BurntOrange}\bfseries,
  emph={as},
  emphstyle={\color{OrangeRed}},
  identifierstyle=\color{black},
  sensitive=true,
  commentstyle=\color{gray}\ttfamily,
  comment=[l]{\#},
  morecomment=[s]{/*}{*/},
  stringstyle=\color{ForestGreen}\ttfamily,
  morestring=[b]',
  morestring=[s]{"""*}{*"""},
}


\newcommand{\tasknumber}{2} % first

\documentclass[a4paper,14pt,oneside,final]{extarticle}
\usepackage[top=2cm, bottom=2cm, left=3cm, right=1cm]{geometry}
\usepackage{scrextend}

\usepackage[T2A,T1]{fontenc}
\usepackage[ukrainian,russian,english]{babel}
\usepackage{tempora}
\usepackage{fontspec}
\setmainfont{tempora}

% Зачем: Отключает использование изменяемых межсловных пробелов.
% Почему: Так не принято делать в текстах на русском языке.
\frenchspacing

\usepackage{indentfirst}
\setlength{\parindent}{1.25cm}
\renewcommand{\baselinestretch}{1.5}

% Header
\usepackage{fancyhdr}
\pagestyle{fancy}
\fancyhead{}
\fancyfoot{}
\fancyhead[R]{\small \selectfont \thepage}
\renewcommand{\headrulewidth}{0pt}

% Captions
\usepackage{chngcntr}
\counterwithin{figure}{section}
\counterwithin{table}{section}
\usepackage[tableposition=top]{caption}
\usepackage{subcaption}
\DeclareCaptionLabelFormat{gostfigure}{Рисунок #2}
\DeclareCaptionLabelFormat{gosttable}{Таблиця #2}
\DeclareCaptionLabelSeparator{gost}{~---~}
\captionsetup{labelsep=gost}
\captionsetup[figure]{labelformat=gostfigure}
\captionsetup[table]{labelformat=gosttable}
\renewcommand{\thesubfigure}{\asbuk{subfigure}}

% Sections
\usepackage[explicit]{titlesec}
\newcommand{\sectionbreak}{\clearpage}

\titleformat{\section}
  {\centering}{\thesection \quad}{0pt}{\MakeUppercase{#1}}
\titleformat{\subsection}[block]
  {\bfseries}{\thesubsection \quad #1}{0cm}{}

\titlespacing{\section} {0cm}{0cm}{21pt}
\titlespacing{\subsection} {\parindent}{21pt}{0cm}
\titlespacing{\subsubsection} {\parindent}{0cm}{0cm}

% Lists
\usepackage{enumitem}
\renewcommand\labelitemi{--}
\setlist[itemize]{noitemsep, topsep=0pt, wide}
\setlist[enumerate]{noitemsep, topsep=0pt, wide, label=\arabic*}
\setlist[description]{labelsep=0pt, noitemsep, topsep=0pt, leftmargin=2\parindent, labelindent=\parindent, labelwidth=\parindent, font=\normalfont}

% Toc
\usepackage{tocloft}
\tocloftpagestyle{fancy}
\renewcommand{\cfttoctitlefont}{}
\setlength{\cftbeforesecskip}{0pt}
\renewcommand{\cftsecfont}{}
\renewcommand{\cftsecpagefont}{}
\renewcommand{\cftsecleader}{\cftdotfill{\cftdotsep}}

\newcommand{\khpistudentgroup}{2.КН201н.8а}
\newcommand{\khpistudentname}{Чепурний~А.~С.}

\newcommand{\khpidepartment}{Програмна інженерія та інформаційні технології управління}
\newcommand{\khpititlewhat}{
	Розрахунково-графічне завдання №\tasknumber \\
	з предмету <<Основи проектування інтелектуальних систем>>
}
\newcommand{\khpititlewho}{
	Виконав: \\
	\hspace*{\parindent} ст. групи \khpistudentgroup \\
	\hspace*{\parindent} \khpistudentname \\
	Перевірила: \\
	\hspace*{\parindent} ст. в. каф. ПІІТУ \\
	\hspace*{\parindent} Єршова~С.~І. \\
}

\usepackage{lscape}
\usepackage{appendix}

\graphicspath{{figures/}}

\begin{document}
\Ukrainian

\begin{titlepage}

\begin{center}
	МІНІСТЕРСТВО ОСВІТИ І НАУКИ УКРАЇНИ \\
	НАЦІОНАЛЬНИЙ ТЕХНІЧНИЙ УНІВЕРСИТЕТ \\
	«ХАРКІВСЬКИЙ ПОЛІТЕХНІЧНИЙ ІНСТИТУТ» \\[0.5cm]
	Кафедра <<\khpidepartment>> \\
\end{center}

\vspace{6cm}

\begin{center}
	\khpititlewhat
\end{center}

\vspace{3cm}

\begin{addmargin}[10cm]{0cm}
	\khpititlewho
\end{addmargin}

\vspace{\fill}

\begin{center}
	Харків \the\year
\end{center}

\end{titlepage}

\addtocounter{page}{1}

\subsection{Інтелектуальна карта}
Інтелектуальна карта для моделей представлення знань зображена на рисунку~\ref{fig:intellig}.

\subsection{Концептуальна карта}
Концептуальна карта для моделей представлення знань зображена на рисунку~\ref{fig:concept}.

\subsection*{Висновки}
Інтелектуальні карти --- це тип представлення знань, де інформація представляється у виді радіанної структури. 
Інтелектуальні карти допомагають систематизувати дані, можуть бути використані для класифікації об’єктів або відображення ієрархічної структури даних.

Концептуальні карти --- це тип представлення знань, де інформація подається у виді об’єктів предметної області та зв’язків між ними. 
Концептуальні карти допомагають систематизувати дані, можуть бути використані для представлення знань про предметну область з великою кількістю зв’язків між об’єктами (поняттями). 

\begin{appendices}
    % custom commands 
    \newcommand\appendixsection[1]{
        \addtocounter{section}{1}
        \clearpage
        \section*{Додаток \thesection. #1}
        \addcontentsline{toc}{section}{Додаток \thesection. #1}
    }
    \appendixsection{Порівняння платформ для розробки мультагентних систем}
{
	\small
	\tabulinesep=1.2mm
	\begin{longtabu} to \textwidth {|X[2,l]|X[2,l]|X[3,l]|X[3,l]|}
  		\caption{Порівняння платформ для розробки \acrshort{mas}~\cite{Kravari2015}}
  		\label{tab:mas_platform_comparsion} \\
		\hline
		& \textbf{Agent Factory} & \textbf{\acrshort{jade}} & \textbf{AnyLogic} \\\hline\endfirsthead
  		\caption*{Продовження таблиці \thetable{}}\\
		\hline
		& \textbf{Agent Factory} & \textbf{\acrshort{jade}} & \textbf{AnyLogic} \\\hline\endhead
		% Platform properties
		Организація & University College Dublin & Telecom Italia (TILAB) & Компанія AnyLogic \\\hline
		Основна доменна область & Агенти для загального призначення & Розподілені мультиагентні системи & Розподілені мільтиагентні симуляції \\\hline
		Ліцензія & LGPL & LGPL & Комерційна та академічна ліцензії \\
		\hline
		% Usability overview
		Простота використання & Простий / Нестача функціоналу у \acrshort{gui} & Зручний, простий \acrshort{gui}, багато звичних функцій & Середній / Багатий функціонал \acrshort{gui} \\\hline
		Легкість вивчення платформи & Середня & Легка (багато прикладів) & Легка \\\hline
		Масштабованість & Добра & Висока & Висока \\\hline
		Сумісність зі стандартами & \acrshort{fipa} & \acrshort{fipa}, \acrshort{corba} & \acrshort{gis}, 3D-можливості \\
		\hline
		% Operating ability of each agent platform
		Продуктивність & Добра & Висока (дуже швидка взаємодія між агентами) & Висока \\\hline
		Надійність & Середня & Висока & Висока \\\hline
		Мови програмування & Java, \acrshort{afapl}, AgentSpeak & Java & Java, \acrshort{uml}-RT (\acrshort{uml} для реального часу) \\\hline
		Операційні системи & Будь-яка з \acrshort{jvm} & Будь-яка з \acrshort{jvm} & Будь-яка з \acrshort{jvm} \\
		\hline
		% Pragmatics overview
		Рівень підтримки & Добрий (документація, розсилка пошти, форум) & Високий (\acrshort{faq}, розсилка пошти, список дефектів, \acrshort{api}, документація) & Високий (документація) \\\hline
		Популярність & Низька & Висока (найпопулярніша платформа) & Середня \\\hline
		Зрілість технології & Стабільний випуск, статус розробки (неактивний) & Стабільний випуск, статус розробки (активний) & Стабільний випуск, статус розробки (активний) \\\hline
		Вартість & Безкоштовна & Безкоштовна & AnyLogic Advanced \$6,199, Professional \$15,800, University Researcher License \$3,500, Educational Licenses \$485 \\
		\hline
		% Security management overview
		Кінцевий рівень безпеки & Підпис і шифрування & Підпис і шифрування, підтримка HTTPS & Аутентифікація \\\hline
		Справедливість \textit{(fairness)} & Ні & Так & Так \\\hline
		Безпека платформи & Середня & Сильна (аутентифікація, Jaas \acrshort{api}) & Сильна (закрита система) \\
		\hline
	\end{longtabu}
}
    \begin{landscape}
\appendixsection{Концептуальна карта моделей представлення знань}

\begin{figure}[H]
    \centering
        \includegraphics[width=0.7\linewidth]{concept}
    \caption{Концептуальна карта моделей представлення знань}
    \label{fig:concept}
\end{figure}

\end{landscape}

 \end{appendices}

\end{document}
