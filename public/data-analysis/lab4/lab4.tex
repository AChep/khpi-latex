\usepackage{tikz}

\counterwithout{figure}{section}
\counterwithout{table}{section}
\counterwithout{equation}{section}

\titleformat{\subsection}[block]
  {\bfseries\filcenter}{#1}{0cm}{}
\titlespacing{\subsection}{0cm}{21pt}{21pt}

\DeclareCaptionLabelFormat{gosttable}{Таблица #2}

\usepackage{float}
\usepackage{pgfplots}
\usepackage{graphicx}
\usepackage{multirow}
\usepackage{amssymb,amsfonts,amsmath,amsthm}

\usepackage{listings}
\lstset{basicstyle=\footnotesize\ttfamily,breaklines=true}
\lstset{language=Matlab}

\lstdefinelanguage{CLIPS}{
  keywords={defrule, deftemplate},
  keywordstyle=\color{NavyBlue}\bfseries,
  emph={retract, modify, printout, assert, not},
  emphstyle={\color{OrangeRed}},
  identifierstyle=\color{black},
  sensitive=true,
  commentstyle=\color{gray}\ttfamily,
  comment=[l]{;;;},
}


\newcommand{\labnumber}{4} % fourth lab
\usepackage{tikz}

\counterwithout{figure}{section}
\counterwithout{table}{section}
\counterwithout{equation}{section}

\titleformat{\subsection}[block]
  {\bfseries\filcenter}{#1}{0cm}{}
\titlespacing{\subsection}{0cm}{21pt}{21pt}

\DeclareCaptionLabelFormat{gosttable}{Таблица #2}

\newcommand{\khpistudentgroup}{2.КН201н.8а}
\newcommand{\khpistudentname}{Чепурний~А.~С.}

\newcommand{\khpidepartment}{Програмна інженерія та інформаційні технології управління}
\newcommand{\khpititlewhat}{
	Розрахунково-графічне завдання \\
	з предмету <<Фреймворки та платформи>>
}
\newcommand{\khpititlewho}{
	Виконав: \\
	\hspace*{\parindent} ст. групи \khpistudentgroup \\
	\hspace*{\parindent} \khpistudentname \\
	Перевірила: \\
	\hspace*{\parindent} к. т. н., вик. каф. ПІІТУ \\
	\hspace*{\parindent} Добряк~В.~С. \\
}


\lstset{language=CLIPS}
\graphicspath{{figures/}}

\begin{document}
\Ukrainian

\begin{titlepage}

\begin{center}
	МІНІСТЕРСТВО ОСВІТИ І НАУКИ УКРАЇНИ \\
	НАЦІОНАЛЬНИЙ ТЕХНІЧНИЙ УНІВЕРСИТЕТ \\
	«ХАРКІВСЬКИЙ ПОЛІТЕХНІЧНИЙ ІНСТИТУТ» \\
	Кафедра <<\khpidepartment>> \\
\end{center}

\vspace{6cm}

\begin{center}
	\khpititlewhat
\end{center}

\vspace{3cm}

\begin{addmargin}[10cm]{0cm}
	\khpititlewho
\end{addmargin}

\vspace{\fill}

\begin{center}
	Харків \the\year
\end{center}

\end{titlepage}

\addtocounter{page}{1}

\section*{Розробка прототипу діагностичної експертної системи}
\subsection*{Мета}
\begin{enumerate}
	\item Сформувати для ПО поле знань, список фактів, а також правил для роботи з ними.
	\item Оволодіти базовими конструкціями мови представлення знань CLIPS, такими як deftemplate, deffacts, defrule, deffunction, defglobal.
	\item Освоїти принципи пошуку рішення в експертних системах, заснованих на правилах виду <<ЯКЩО-ТО>>, формування послідовності активації правил при виведенні результату.
\end{enumerate}

\subsection*{Завдання}
\begin{enumerate}
	\item Описати словесно факти і правила для розроблюваного прототипу, уявити можливу ієрархію понять. 
	\item Перекласти факти і правила в синтаксис мови CLIPS.
	\item Продемонструвати працездатність прототипу на конкретних прикладах.
\end{enumerate}

\begin{center}
Управління страховими договорами
\end{center}

\subsection*{Хід роботи}
Описати словесно факти і правила для розроблюваного прототипу, уявити можливу ієрархію понять.

Факти предметної області:
\begin{enumerate}
	\item \textbf{income}: low, normal, high; 
	\item \textbf{age}: young, mature, elderly;
	\item \textbf{status}: single, plural.
\end{enumerate}

Правила предметної області:
\begin{enumerate}
	\item \textbf{income} high якщо >20000, normal якщо >10000, інакше low; 
	\item \textbf{age} elderly якщо >60, mature якщо >45, інакше young;
	\item \textbf{status} plural якщо є родина, інакше single.
	\item Правило результату приведено у вигляді таблиці: \\
		\begin{tabular}{l|lll}
			\textbf{Do I need an insurance?} & \textbf{income} & \textbf{age} & \textbf{status} \\\hline 
			\textbf{No} & low & young & single \\ 
			\textbf{No} & normal & young & single \\ 
			\textbf{Yes} & high & young & single \\ 
			\textbf{No} & low & mature & single \\ 
			\textbf{Yes} & normal & mature & single \\ 
			\textbf{Yes} & high & mature & single \\ 
			\textbf{Yes} & low & elderly & single \\ 
			\textbf{Yes} & normal & elderly & single \\ 
			\textbf{Yes} & high & elderly & single \\ 
			\textbf{No} & low & young & plural \\ 
			\textbf{Yes} & normal & young & plural \\ 
			\textbf{Yes} & high & young & plural \\ 
			\textbf{Yes} & low & mature & plural \\ 
			\textbf{Yes} & normal & mature & plural \\ 
			\textbf{Yes} & high & mature & plural \\ 
			\textbf{Yes} & low & elderly & plural \\ 
			\textbf{Yes} & normal & elderly & plural \\ 
			\textbf{Yes} & high & elderly & plural \\ 
		\end{tabular}
\end{enumerate}

Демонстрацію працездатності прототипа на кенкретних прикладах представлено на рисунках~\ref{fig:test_1},~\ref{fig:test_2},~\ref{fig:test_3},~\ref{fig:test_4}.

\begin{figure}[H]
	\centering
	    \includegraphics{test_1}
	\caption{Результат виконання програми №1}
	\label{fig:test_1}
\end{figure} 

\begin{figure}[H]
	\centering
	    \includegraphics{test_2}
	\caption{Результат виконання програми №2}
	\label{fig:test_2}
\end{figure} 

\begin{figure}[H]
	\centering
	    \includegraphics{test_3}
	\caption{Результат виконання програми №3}
	\label{fig:test_3}
\end{figure} 

\begin{figure}[H]
	\centering
	    \includegraphics{test_4}
	\caption{Результат виконання програми №4}
	\label{fig:test_4}
\end{figure} 

\subsection*{Висновки}
У ході виконання лабораторної роботи я сформував для ПО поле знань, список фактів, а також правил для роботи з ними. 
Також я оволоділ базовими конструкціями мови представлення знань CLIPS, такими як deftemplate, deffacts, defrule, deffunction, defglobal. 
Я ознайомився з принципами пошуку рішення в експертних системах, заснованих на правилах виду "ЯКЩО-ТО", формування послідовності активації правил при виведенні результату.

\subsection*{Додаток}
Вихідний код програми:
\lstinputlisting{code/main.clp} 

\end{document}
