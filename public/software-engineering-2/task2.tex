% !TEX program = xelatex
\documentclass[a4paper,14pt,oneside,final]{extarticle}
\usepackage{scrextend}

\usepackage[T2A,T1]{fontenc}
\usepackage[ukrainian,russian,english]{babel}
\usepackage{tempora}
\usepackage{fontspec}
\setmainfont{tempora}

% Зачем: Отключает использование изменяемых межсловных пробелов.
% Почему: Так не принято делать в текстах на русском языке.
\frenchspacing

\usepackage{float}
\usepackage{pgfplots}
\usepackage{graphicx}
\usepackage{multirow}
\usepackage{amssymb,amsfonts,amsmath,amsthm}
\usepackage{csquotes}

\usepackage{listings}
\lstset{basicstyle=\footnotesize\ttfamily,breaklines=true}
\lstset{language=Matlab}

\usepackage[
	backend=biber,
	sorting=none,
	language=auto,
	autolang=other
]{biblatex}
\DeclareFieldFormat{labelnumberwidth}{#1}


\usepackage{subcaption}

\counterwithout{subsection}{section}
\counterwithout{subsubsection}{section}
\counterwithin{subsubsection}{subsection}

\begin{document}
\Russian

\begin{flushright}
2.КН201н.8а \\[0.4em]
{\large Чепурной~А.~С.} \\[0.8em]
\end{flushright}

\begin{enumerate}
    \item Определения и примеры:
    \begin{enumerate}
        \item Программный продукт --- программное средство, предназначенное для поставки, передачи, продажи пользователю.
        \item Функциональные требования описывают функции, которые должно выполнять разрабатываемое ПО.
        \item Нефункциональные требования --- требования, определяющие свойства, которые система должна демонстрировать, или ограничения, которые она должна соблюдать, не относящиеся к поведению системы. \textit{Например, производительность, удобство сопровождения, расширяемость, надежность, факторы эксплуатации.}
    \end{enumerate}
    \item Информация о проектной работе.
    \begin{enumerate}
        \item В каких проектах участвуете в настоящее время при обучении по дуальной форме на 5-м курсе: \\
        Работа в компании CHI Software (аутстаф, не партнер кафедры) на позиции Android Developer.
        \item Насколько тема д/р связана с тематикой проектов предыдущего пункта: \\
        Связана в определенной степени: частично совпадают используемые технологии (Kotlin, Kotlin Coroutines). 
    \end{enumerate}
\end{enumerate}

\end{document}