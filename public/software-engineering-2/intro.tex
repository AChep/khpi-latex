% !TEX program = xelatex
\documentclass[a4paper,14pt,oneside,final]{extarticle}
\usepackage{scrextend}

\usepackage[T2A,T1]{fontenc}
\usepackage[ukrainian,russian,english]{babel}
\usepackage{tempora}
\usepackage{fontspec}
\setmainfont{tempora}

% Зачем: Отключает использование изменяемых межсловных пробелов.
% Почему: Так не принято делать в текстах на русском языке.
\frenchspacing

\usepackage{float}
\usepackage{pgfplots}
\usepackage{graphicx}
\usepackage{multirow}
\usepackage{amssymb,amsfonts,amsmath,amsthm}
\usepackage{csquotes}

\usepackage{listings}
\lstset{basicstyle=\footnotesize\ttfamily,breaklines=true}
\lstset{language=Matlab}

\usepackage[
	backend=biber,
	sorting=none,
	language=auto,
	autolang=other
]{biblatex}
\DeclareFieldFormat{labelnumberwidth}{#1}


\usepackage{subcaption}

\counterwithout{subsection}{section}
\counterwithout{subsubsection}{section}
\counterwithin{subsubsection}{subsection}

\begin{document}
\Russian

\begin{flushright}
2.КН201н.8а \\[0.4em]
{\large Чепурной~А.~С.} \\[0.8em]
\end{flushright}

\begin{enumerate}
    \item Информация о дипломной работе.
    \begin{enumerate}
        \item Название темы, ФИО, уч. степень, звание, должность рук. д/р: \\
        \textbf{Тема:} <<Разработка программной системы для анализа стойкости логистической системы дистрибьюции>>. \\
        \textbf{Руководитель:} Годлевский Михаил Дмитриевич, профессор, доктор технических наук, зав. кафедры.
        \item Краткое описание ПрО (цель, объект, предмет и ) + основные задачи выполнения д/р (макс. 1 стр. А4): \\
        \textbf{Проблема:} Необходимость моделирования логистической системы для проверки ее оптимальности. \\
        \textbf{Цель:}​ Разработать програмный продукт для анализа стойкости логистической системы дистрибьюции. \\
        \textbf{Объект:} ​Процесс принятия решений логистической системой дистрьбьюции. \\
        \textbf{Предмет:} ​Мультиагентная модель логистической системы дистрьбьюции. \\
        \textbf{Краткое описание:} \\
        В ходе магистерской диссертации планируется рассмотреть процесс разработки программной системы для моделирования мультиагентной логистической системы. 
        В частности, будут рассмотрены возможные архитектурные решения таких систем, описаны их недостатки и преимущества. 

        Перед разработкой программной системы будуд описаны методы и алгоритмы для управления логистическими системами дистрибьюции, описаны цели и жизненные циклы ее агентов.
        
        Будет разработана программая система и представлено сравнение ее с существующеми аналогами.
        \item Название темы к/р (НДР) 1-го (прошлого) семестра и состояние дел: \\
        \textbf{Тема:} <<Разработка программной системы для анализа стойкости логистической системы дистрибьюции>> (1-й раздел д/р). \\
        \textbf{Статус:} Сдана (балл: 90).
        \item Название темы к/р (НДР) 2-го (текущего) семестра, прим. \% выполнения (дать собств. оценку по сост. на 20.04.19): \\
        \textbf{Тема:} <<Разработка программной системы для анализа стойкости логистической системы дистрибьюции>> (2-й раздел д/р). \\
        \textbf{Статус:} В разработке (процент выполенения: 40\%).
        \item Какие задачи исследования процессов разработки и эксплуатации ПО д.б. решены в д/р (по собственной оценке): \\
        Задача построения архитектуры программной системы для моделирования агентных систем. 
        \item Участие в написании (совместно с руководителем) научных статей по тематике д/р: \\
        \textbf{Статус:} В разработке (процент выполенения: 70\%). \\
    \end{enumerate}
    \item Информация о проектной работе.
    \begin{enumerate}
        \item В каких проектах участвуете в настоящее время при обучении по дуальной форме на 5-м курсе: \\
        Работа в компании CHI Software (аутстаф, не партнер кафедры) на позиции Android Developer.
        \item Насколько тема д/р связана с тематикой проектов предыдущего пункта: \\
        Связана в определенной степени: частично совпадают используемые технологии (Kotlin, Kotlin Coroutines). 
    \end{enumerate}
\end{enumerate}

\end{document}