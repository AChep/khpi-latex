\usepackage{tikz}

\counterwithout{figure}{section}
\counterwithout{table}{section}
\counterwithout{equation}{section}

\titleformat{\subsection}[block]
  {\bfseries\filcenter}{#1}{0cm}{}
\titlespacing{\subsection}{0cm}{21pt}{21pt}

\DeclareCaptionLabelFormat{gosttable}{Таблица #2}

\usepackage{float}
\usepackage{pgfplots}
\usepackage{graphicx}
\usepackage{multirow}
\usepackage{amssymb,amsfonts,amsmath,amsthm}

\usepackage{listings}
\lstset{basicstyle=\footnotesize\ttfamily,breaklines=true}
\lstset{language=Matlab}


\newcommand{\labnumber}{3} % third lab
\usepackage{tikz}

\counterwithout{figure}{section}
\counterwithout{table}{section}
\counterwithout{equation}{section}

\titleformat{\subsection}[block]
  {\bfseries\filcenter}{#1}{0cm}{}
\titlespacing{\subsection}{0cm}{21pt}{21pt}

\DeclareCaptionLabelFormat{gosttable}{Таблица #2}

\newcommand{\khpistudentgroup}{2.КН201н.8а}
\newcommand{\khpistudentname}{Чепурний~А.~С.}

\newcommand{\khpidepartment}{Програмна інженерія та інформаційні технології управління}
\newcommand{\khpititlewhat}{
	Розрахунково-графічне завдання \\
	з предмету <<Фреймворки та платформи>>
}
\newcommand{\khpititlewho}{
	Виконав: \\
	\hspace*{\parindent} ст. групи \khpistudentgroup \\
	\hspace*{\parindent} \khpistudentname \\
	Перевірила: \\
	\hspace*{\parindent} к. т. н., вик. каф. ПІІТУ \\
	\hspace*{\parindent} Добряк~В.~С. \\
}


\graphicspath{{figures/}}

\begin{document}
\Ukrainian

\begin{titlepage}

\begin{center}
	МІНІСТЕРСТВО ОСВІТИ І НАУКИ УКРАЇНИ \\
	НАЦІОНАЛЬНИЙ ТЕХНІЧНИЙ УНІВЕРСИТЕТ \\
	«ХАРКІВСЬКИЙ ПОЛІТЕХНІЧНИЙ ІНСТИТУТ» \\
	Кафедра <<\khpidepartment>> \\
\end{center}

\vspace{6cm}

\begin{center}
	\khpititlewhat
\end{center}

\vspace{3cm}

\begin{addmargin}[10cm]{0cm}
	\khpititlewho
\end{addmargin}

\vspace{\fill}

\begin{center}
	Харків \the\year
\end{center}

\end{titlepage}

\addtocounter{page}{1}

\section*{Проектування програмного забезпечення для моделювання неперервних детермінованих систем}
\subsubsection*{Мета роботи}
Отримати практичні навички архітектурного проектування
програмного забезпечення.
\subsubsection*{Хід роботи}
\begin{enumerate}
\item Визначити стратегію проектування для реалізації програмного забезпечення згідно індивідуального завдання.
\item Розробити моделі для статичного представлення архітектури програмного забезпечення.
\item Розробити моделі для динамічного представлення архітектури програмного забезпечення.
\item Задокументувати розроблені моделі, використовуючи Visual Paradigm for UML.
\item Оформити звіт, який повинен містити обґрунтування вибору архітектури програмного забезпечення, розроблені за допомогою Visual Paradigm for UML моделі структурного та динамічного представлення компонентів системи.
\end{enumerate}

\end{document}
