\usepackage{tikz}

\counterwithout{figure}{section}
\counterwithout{table}{section}
\counterwithout{equation}{section}

\titleformat{\subsection}[block]
  {\bfseries\filcenter}{#1}{0cm}{}
\titlespacing{\subsection}{0cm}{21pt}{21pt}

\DeclareCaptionLabelFormat{gosttable}{Таблица #2}

\usepackage{float}
\usepackage{pgfplots}
\usepackage{graphicx}
\usepackage{multirow}
\usepackage{amssymb,amsfonts,amsmath,amsthm}

\usepackage{listings}
\lstset{basicstyle=\footnotesize\ttfamily,breaklines=true}
\lstset{language=Matlab}


\newcommand{\labnumber}{6} % sixth lab
\usepackage{tikz}

\counterwithout{figure}{section}
\counterwithout{table}{section}
\counterwithout{equation}{section}

\titleformat{\subsection}[block]
  {\bfseries\filcenter}{#1}{0cm}{}
\titlespacing{\subsection}{0cm}{21pt}{21pt}

\DeclareCaptionLabelFormat{gosttable}{Таблица #2}

\newcommand{\khpistudentgroup}{2.КН201н.8а}
\newcommand{\khpistudentname}{Чепурний~А.~С.}

\newcommand{\khpidepartment}{Програмна інженерія та інформаційні технології управління}
\newcommand{\khpititlewhat}{
	Розрахунково-графічне завдання \\
	з предмету <<Фреймворки та платформи>>
}
\newcommand{\khpititlewho}{
	Виконав: \\
	\hspace*{\parindent} ст. групи \khpistudentgroup \\
	\hspace*{\parindent} \khpistudentname \\
	Перевірила: \\
	\hspace*{\parindent} к. т. н., вик. каф. ПІІТУ \\
	\hspace*{\parindent} Добряк~В.~С. \\
}


\graphicspath{{figures/}}

\begin{document}
\Ukrainian

\begin{titlepage}

\begin{center}
	МІНІСТЕРСТВО ОСВІТИ І НАУКИ УКРАЇНИ \\
	НАЦІОНАЛЬНИЙ ТЕХНІЧНИЙ УНІВЕРСИТЕТ \\
	«ХАРКІВСЬКИЙ ПОЛІТЕХНІЧНИЙ ІНСТИТУТ» \\
	Кафедра <<\khpidepartment>> \\
\end{center}

\vspace{6cm}

\begin{center}
	\khpititlewhat
\end{center}

\vspace{3cm}

\begin{addmargin}[10cm]{0cm}
	\khpititlewho
\end{addmargin}

\vspace{\fill}

\begin{center}
	Харків \the\year
\end{center}

\end{titlepage}

\addtocounter{page}{1}

\section*{Моделювання та дослідження неперервних детермінованих систем}
\subsubsection*{Мета роботи}
Ознайомитися з задачами стратегічного й тактичного планування експериментів з моделями. 
Отримати практичні навички планування, проведення та документування експериментів. 
\subsubsection*{Хід роботи}
\begin{enumerate}
\item Скласти план експерименту з моделлю, розробленою згідно з індивідуальним завданням.
\item Провести експеримент із розробленою моделлю.
\item Оформити звіт, який повинен містити:
\begin{itemize}
\item план проведення експерименту;
\item задокументовані результати експерименту. 
\end{itemize}
\end{enumerate}

\subsection{План проведення експерименту}
Рівняння системи що моделюється, являє собою синусоїду. 
Рівняння даної системи приймає наступний вигляд:
\[
x = A \cdot \sin (\sqrt{\frac{k}{m}} \cdot t),
\]
\begin{description}
\item[де] $A$ --- амплітуда коливання;
\item $k$ --- коефіцієнт жорсткості пружини;
\item $t$ --- момент часу;
\item $m$ --- вага вантажу.
\end{description}

Проведемо дослідження моделі відносно параметрів $A$, $k$ та~$m$. 

\subsection{Результати експерименту}
Отримані результати дослідження представлено на рисунку~\ref{fig:results}.

\begin{figure}[H]
    \centering
    \begin{subfigure}[t]{0.45\linewidth}
    	\includegraphics[width=1\linewidth]{experiment1}
    	\caption{Експеримент №1: $m = 2.4$, $k = 4$, $t = 4.8$, $A = 1$}
    \end{subfigure}
    ~
    \begin{subfigure}[t]{0.45\linewidth}
    	\includegraphics[width=1\linewidth]{experiment2}
    	\caption{Експеримент №2: $m = 9$, $k = 4$, $t = 4.8$, $A = 1$}
    \end{subfigure}
    ~
    \begin{subfigure}[t]{0.45\linewidth}
    	\includegraphics[width=1\linewidth]{experiment3}
    	\caption{Експеримент №3: $m = 9$, $k = 15$, $t = 4.8$, $A = 1$}
    \end{subfigure}
    ~
    \begin{subfigure}[t]{0.45\linewidth}
    	\includegraphics[width=1\linewidth]{experiment4}
    	\caption{Експеримент №4: $m = 2.4$, $k = 4$, $t = 4.8$, $A = 0.2$}
    \end{subfigure}
    \caption{Результати проведення експериментів}
    \label{fig:results}
\end{figure}

Проаналізувавши отримані данні можна зробити висновки, що чим більше початкова відстань вантажу від точки рівноваги системи, тим більша амплітуда синусоїди. 
Циклічна частота коливань дорівнює $\sqrt{\cfrac{k}{m}}$ та залежить тільки від маси вантажу та жорсткості пружини --- чим пружина жорсткіше або вантаж легше, тим менша частота коливань.

\subsection*{Висновки}
У ході виконання лабораторної роботи було проведено планування експерименту для моделі за завданням.
Були отримані практичні навички планування, проведення та документування експериментів.

\end{document}
