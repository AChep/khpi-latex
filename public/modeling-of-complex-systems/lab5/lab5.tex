\usepackage{tikz}

\counterwithout{figure}{section}
\counterwithout{table}{section}
\counterwithout{equation}{section}

\titleformat{\subsection}[block]
  {\bfseries\filcenter}{#1}{0cm}{}
\titlespacing{\subsection}{0cm}{21pt}{21pt}

\DeclareCaptionLabelFormat{gosttable}{Таблица #2}

\usepackage{float}
\usepackage{pgfplots}
\usepackage{graphicx}
\usepackage{multirow}
\usepackage{amssymb,amsfonts,amsmath,amsthm}

\usepackage{listings}
\lstset{basicstyle=\footnotesize\ttfamily,breaklines=true}
\lstset{language=Matlab}

\input{../../../commons_kotlin}

\newcommand{\labnumber}{5} % fifth lab
\usepackage{tikz}

\counterwithout{figure}{section}
\counterwithout{table}{section}
\counterwithout{equation}{section}

\titleformat{\subsection}[block]
  {\bfseries\filcenter}{#1}{0cm}{}
\titlespacing{\subsection}{0cm}{21pt}{21pt}

\DeclareCaptionLabelFormat{gosttable}{Таблица #2}

\newcommand{\khpistudentgroup}{2.КН201н.8а}
\newcommand{\khpistudentname}{Чепурний~А.~С.}

\newcommand{\khpidepartment}{Програмна інженерія та інформаційні технології управління}
\newcommand{\khpititlewhat}{
	Розрахунково-графічне завдання \\
	з предмету <<Фреймворки та платформи>>
}
\newcommand{\khpititlewho}{
	Виконав: \\
	\hspace*{\parindent} ст. групи \khpistudentgroup \\
	\hspace*{\parindent} \khpistudentname \\
	Перевірила: \\
	\hspace*{\parindent} к. т. н., вик. каф. ПІІТУ \\
	\hspace*{\parindent} Добряк~В.~С. \\
}


\lstset{language=Kotlin}
\graphicspath{{figures/}}

\begin{document}
\Ukrainian

\begin{titlepage}

\begin{center}
	МІНІСТЕРСТВО ОСВІТИ І НАУКИ УКРАЇНИ \\
	НАЦІОНАЛЬНИЙ ТЕХНІЧНИЙ УНІВЕРСИТЕТ \\
	«ХАРКІВСЬКИЙ ПОЛІТЕХНІЧНИЙ ІНСТИТУТ» \\
	Кафедра <<\khpidepartment>> \\
\end{center}

\vspace{6cm}

\begin{center}
	\khpititlewhat
\end{center}

\vspace{3cm}

\begin{addmargin}[10cm]{0cm}
	\khpititlewho
\end{addmargin}

\vspace{\fill}

\begin{center}
	Харків \the\year
\end{center}

\end{titlepage}

\addtocounter{page}{1}

\section*{Верифікація та валідація програмного забезпечення для моделювання неперервних детермінованих систем}
\subsubsection*{Мета роботи}
Ознайомитися із принципами верифікації та валідації моделей та програмних рішень. 
Отримати практичні навички тестування програмного забезпечення.
\subsubsection*{Хід роботи}
\begin{enumerate}
\item Провести тестування розроблених програмних рішень та довести їхню відповідність встановленим вимогам.
\item Перевірити відповідність розробленої моделі загальним принципам V\&V\&T моделей.
\item Розробити та реалізувати тестовий приклад.
\item Оформити звіт, який повинен містити:
\begin{itemize}
\item опис обраного підходу тестування розроблених програмних рішень, видів тестування, які було проведено, та спеціальних програмних продуктів, які було використано при тестуванні;
\item опис методик тестування та критеріїв прийняття результатів тестування;
\item опис тестового прикладу та обґрунтування валідності розробленої моделі;
\item результати використання можливостей нотації Visual Paradigm for UML для документування V\&V\&T.
\end{itemize}
\end{enumerate}

\subsection{Верифікація}
Верифікація --- це процес оцінки системи або її компонентів з метою визначення того, чи задовольняють результати поточного етапу розробки умови, сформовані на початку цього етапу. 
Тобто, чи виконуються завдання, цілі та
строки по розробці продукту.

Верифікація були здійснена шляхом виконання модульного тестування, у якому були порівняні розрахунки прогнозу з реальними експериментами.

Лістинг коду верифікації:
\lstinputlisting{code/kunit.kt}

\subsection{Валідація}
Валідація --- це визначення відповідності створюваного ПЗ очікуванням і потребам користувача, вимогам до системи.

Однією з найважливіших особливостей даного ПЗ є робота з БД, отже необхідно проводити валідацію даних для коректного зберігання даних та уникнення конфліктів.

Лістинг коду валідації:
\lstinputlisting{code/validation.kt}

\subsection*{Висновки}
У ході виконання лабораторної роботи була обрана стратегія для тестування програмного застосунку, та був проведений ряд модульних тестів для програмного застосунку, що розроблюється. 

\end{document}
