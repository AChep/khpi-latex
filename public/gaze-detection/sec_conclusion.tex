\section*{Висновки}
\addcontentsline{toc}{section}{Висновки}
Сучасні алгоритми <<комп'юторного зору>> дозволяють досить швидко створювати вражаючі програми які реалізують зоровий інтерфейс.
Використання нейронних мереж спрощує задачу переведення точок обличчя у положення погляду не екрані.

В ході написання роботи дану задачу було розкрито повною мірою на прикладі проектування та реалізації програмної системи для визначення погляду користувача з видеоряду.    

В процесі дослідження були сформульовані вимоги до програмної системи, та обрана й спроектована її архітектура.
Обгрунтований вибір інструментальних засобів розробки.
