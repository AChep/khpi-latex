\section{Методи виявлення і розпізнавання зіниць людей}
\subsection{Методи виявлення облич людей}
У даному пункті представлений огляд існуючих методів виявлення облич людей. 
В огляд включалися тільки ті методи, які, на думку автора, є найбільш успішними й широко використовуються в сучасних системах виявлення облич людей.

\subsubsection{Метод Віоли-Джонса}
На сьогодні метод Віоли-Джонса є
найбільш популярним методом для пошуку області обличчя на зображеннях через його високу швидкість і ефективності~\cite{Violaa,Violab}.

Цей метод зробив можливим реалізацію виділення облич на зображеннях у реальному часі в практичних додатках, таких як, наприклад, цифрові камери й програмне забезпечення для керування фотоальбомами~\cite{Violab}.

Детектор облич Віоли-Джонса базується на трьох основних принципах:
\begin{itemize}
	\item інтегральне подання зображень за ознаками Хаара, яке дозволяє дуже швидко обчислювати необхідні ознаки;
	\item метод побудови класифікатора на основі алгоритму адаптивного бустинга (AdaBoost);
	\item метод комбінування класифікаторів у каскадну структуру.
\end{itemize}

Ці ідеї дозволяють швидко здійснювати
пошук обличчя на зображенні в режимі реального часу.

Переваги:
\begin{itemize}
	\item висока швидкість виявлення об'єктів;
	\item висока ймовірність точного виявлення обличчя (понад 90\%) для фронтальних зображень і спостережень об'єкта під невеликим кутом, приблизно до 30°;
	\item низька ймовірність помилкового виявлення обличчя.
\end{itemize}

Недоліки:
\begin{itemize}
	\item дуже великий час навчання;
	\item при великому куті нахилу голови ймовірність виявлення обличчя різко падає;
	\item чутливість до умов освітлення. 
\end{itemize}

\subsubsection{Метод власних облич}
Метод власних облич використовує
аналіз головних компонентів для зменшення розмірності даних без істотної втрати інформації~\cite{Turk1991}.

Простір власних облич утворюється за допомогою застосування методу головних компонентів до навчальної множини зображень. 
Потім навчальні зображення проектують на простір власних облич. 
Далі тестове зображення проектується на новий простір і обчислюється відстань між спроектованим тестовим зображенням і зображеннями з навчального набору. 
Розпізнаним приймається найближче навчальне зображення.

Також метод головних компонентів застосовується лише для виявлення обличчя на зображенні. 
Для облич значення компонент у власному просторі мають більші значення, а в доповненні власного простору --- близькі до нуля. 
За цим фактом можна виявити, чи є вхідне зображення обличчям.

Переваги:
\begin{itemize}
	\item при дотриманні ідеалізованих умов точність розпізнавання з використанням даного методу може досягати значення понад 90\%;
	\item зберігання і пошук зображень у великих базах даних, реконструкція зображень.
\end{itemize}

Недоліки:
\begin{itemize}
	\item обчислення набору власних векторів вирізняється високою трудомісткістю;
	\item зображення повинні бути отримані в близьких умовах освітленості, однаковому ракурсі (вирішується додаванням у навчальну вибірку зображень у різних ракурсах);
	\item повинна бути проведена якісна попередня обробка, що приводить зображення до стандартних умов;
	\item відсутність таких перешкод, як окуляри або бороди.
\end{itemize}

\subsubsection{Порівняння шаблонів}
Основа цього методу полягає у виділенні областей обличчя на зображенні і наступному порівнянні цих областей для двох різних зображень.
Кожна область, що збіглася, збільшує міру подібності зображень. 
Для порівняння областей використовуються найпростіші алгоритми як, наприклад, попіксельне порівняння~\cite{PerveenN2013}.

Недолік цього методу полягає в тому, що він вимагає багато ресурсів як для зберігання ділянок, так і для їхнього порівняння. 
Через те, що використовується найпростіший алгоритм порівняння, зображення повинні бути зняті в строго встановлених умовах: не допускається помітних змін ракурсу, освітлення, емоційних виразів та ін.

Точність розпізнавання з використанням даного методу становить близько 80\%, що є гарним результатом.

\subsection{Методи розпізнавання контурів облич}
\subsubsection{Активні моделі зовнішнього вигляду}
Активні моделі зовнішнього вигляду --- це статистичні моделі зображень, які шляхом різного роду деформацій можуть бути підігнані під реальне зображення.
Активна модель зовнішнього вигляду містить два типи параметрів: параметри, пов'язані з формою, і параметри, пов'язані зі стохастичною моделлю пікселів зображення або текстурою.
Перед використанням модель повинна бути навчена на безлічі заздалегідь розмічених зображень. Розмітка зображень ставиться вручну~\cite{Rawlinson2010,Edwards1998}. 

\subsubsection{Активні моделі форми}
Активні моделі форми враховують статистичні зв'язки у взаємному розташуванні антропометріческіз точок.
На кожному зображенні вибірки експерт розмічає розташування антропометричних точок.
Для того, щоб привести координати на всіх зображеннях до єдиної системи зазвичай виконується узагальнений аналіз, в результаті якого всі точки приводяться до одного масштабу і центруються.
Далі для всього набору образів обчислюється середня форма і матриця коваріації.
На основі матриці коваріації обчислюються власні вектора, які потім сортуються в порядку убування відповідних їм власних значень.
Локалізація ASM моделі на новому, що не входить в навчальну вибірку зображенні здійснюється в процесі рішення оптимізаційної задачі~\cite{Choa}.    

\subsection{Методи розпізнавання зіниць}
\subsubsection{Метод Віоли-Джонса}
Метод Віоли-Джонса також використовується для розпізнавання зіниць очей~\cite{Violaa}.

Переваги:
\begin{itemize}
	\item висока швидкість виявлення об'єктів;
	\item низька ймовірність помилкового виявлення зіниці очей.
\end{itemize}

Недоліки:
\begin{itemize}
	\item низька точність розпізнавання центру зіниці ока;
	\item дуже великий час навчання;
	\item чутливість до умов освітлення. 
\end{itemize}

\subsubsection{Метод середніх градієнтів}
Детектор центра зрачков методом средних градиентов~\cite{MeansOfGrads2011a} использует производные первого и второго порядка в красном спектре изображения для нахождения возможных центров зрачков:
\begin{align}
c^* &= \arg \max_c \{\cfrac{1}{N} \sum^N_{i=1} (d^T_ig_i)^2\}, \\
d_i &= \cfrac{x_i - c}{||x_i-c||_2}, \quad
\forall i : ||g_i||_2 = 1.
\end{align}
\begin{description}
	\item[де] $c$ --- можливий центр зіниці;
	\item $g_i$ --- градієнт у точці $x_i$;
	\item $d_i$ --- вектор зміщення.
\end{description}

Переваги:
\begin{itemize}
	\item висока швидкість виявлення об'єктів;
	\item висока точність розпізнавання центру зіниці ока;  
\end{itemize}

Недоліки:
\begin{itemize}
	\item чутливість до умов освітлення. 
\end{itemize}

\subsection{Висновки}
Було вирішено використовувати метод Віоли-Джонса для пошуку облич, активні моделі форми для пошуку контурів та метод середніх градієнтів для знаходження центру зіниць.

Головними крітерями вибору були висока точність виявлення та висока швидкість роботи.

Для виявлення підморгувань було обрано метод, оснований на різниці відстаней меж віками. Цей метод є простим у реализації та показуе високі результати~\cite{Blink2014}.
