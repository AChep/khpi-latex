\section*{Вступ}
\addcontentsline{toc}{section}{Вступ}

Основну частину інформації про навколишній світ людина сприймає за допомогою зору. 

Системи відстеження напрямку погляду є перспективним і достатньо новим напрямком.
Сфери застосування таких систем~\cite{eyeControllPresentAndFuture}:
\begin{itemize}
	\item оцінка зручності графічних інтерфейсів (додатки, веб-сторінки), дослідження в області психології і нейробіології;
	\item діагностика медичних захворювань: характерні рухи очей можуть свідчити про хвороби людини;
	\item в безпеці: ідентифікація людини за характером руху очей;
 	\item в інтерактивних системах, які дозволяють здійснювати управління різними пристроями за допомогою очей: управління і фокус камери поглядом оператора, показ додаткової інформації в шоломах пілотів. Також, для паралізованих людей такий спосіб взаємодії часто є єдиним можливим.
\end{itemize}

Об'єктом дослідження є визначення напрямку погляду.

Предметом дослідження є нейронна мережа для  відстеження напрямку погляда з відеопотоку.

Метою і завданням дослідження є дослідження застосування нейронних мереж для визначення напрямку погляду.
Для досягнення цієї мети були сформульовані і вирішені такі завдання:
\begin{itemize}
	\item опис методів визначення контурів обличчя, очей та зіниць людини;
	\item проектування архітектури програмної системи;
	\item проектування нейронної мережі;
	\item опис способу отримання даних для навчання нейронної мережі;
	\item тестування розробленої системи.
\end{itemize}
