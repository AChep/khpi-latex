\begin{titlepage}
	\vspace*{\fill} % center the frame vertically
	
	\begin{framed}
		\begin{center}
			МІНІСТЕРСТВО ОСВІТИ І НАУКИ УКРАЇНИ \\
			НАЦІОНАЛЬНИЙ ТЕХНІЧНИЙ УНІВЕРСИТЕТ \\
			<<ХАРКІВСЬКИЙ ПОЛІТЕХНІЧНИЙ ІНСТИТУТ>> \\
			Кафедра програмної інженерії та інформаційних технологій управління
		\end{center}
	
		\begin{center}
			\MakeUppercase{Науково-дослідна робота} \\ 
			<<\thetitle>>
		\end{center}
	
		\noindent	
		Керівник роботи: \\
		\hspace*{\parindent} проф. каф. ПІІТУ, д.т.н. \hfill Годлевський~М.~Д. \\
		Виконавець: \\
		\hspace*{\parindent} ст. групи КН-Н218а \hfill \theauthor
	
		\begin{center}
			Харків \the\year
		\end{center}
	\end{framed}

	\vspace*{\fill} % center the frame vertically
\end{titlepage}

{
\newcommand{\fillemptyline}{\uline{\hspace*{\fill}}}
\newcommand{\fillline}[2][]{\uline{#1\hspace*{\fill}#2\hspace*{\fill}\hphantom{#1}}}

\newcommand{\suline}[1]{\uline{\hspace{12pt}#1\hspace{12pt}}}
\newcommand{\undercaption}[1]{{\centering\footnotesize#1\\\noindent}}

\begin{titlepage}
	\begin{center}
		МІНІСТЕРСТВО ОСВІТИ І НАУКИ УКРАЇНИ \\
		НАЦІОНАЛЬНИЙ ТЕХНІЧНИЙ УНІВЕРСИТЕТ \\
		<<ХАРКІВСЬКИЙ ПОЛІТЕХНІЧНИЙ ІНСТИТУТ>> \\
		Кафедра програмної інженерії та інформаційних технологій управління
	\end{center}
	\vspace*{\fill}
	\begin{center}
		\MakeUppercase{\large\bfseries Науково-дослідна робота}
	\end{center}
	\noindent
	\fillline[з]{Програмне забезпечення інтелектуальних систем} \\
	\undercaption{(назва дисципліни)}
	\fillline[на тему:]{Розробка інформаційної технології аналізу стійкості} \\
	\fillline{функціонування логістичної системи дистрибуції} \\
	\fillline{при стратегічному плануванні}
	
	\vspace*{\fill}

	\begin{addmargin}[7cm]{0cm}
		\small
		Студента \suline{6} курсу \suline{КН-Н218а} групи \hspace*{\fill} \\
		спеціальності \fillline{121 Інженерія програмного забезпечення} \\ 
		\fillline{Чепурного~А.~С.} \\
		\undercaption{(прізвище та ініціали)}
		Керівник \fillline{проф. каф. ПІІТУ, д.т.н. Годлевський~М.~Д.} \\
		\undercaption{(посада, вчене звання, науковий ступінь, прізвище та ініціали)}
		Національна шкала \fillemptyline \\
		Кількість балів	\fillemptyline Оцінка ECTS \fillemptyline	
	\end{addmargin}

	\begin{flushright}
		\small
		\newcommand{\member}[1]{
			& \hspace{4cm} & & #1 \\ \cline{2-2} \cline{4-4} 
			& {\footnotesize (підпис)} & & {\footnotesize (прізвище та ініціали)}  \\
		}
		\begin{tabular}{cccc}
			Члени комісії 
			\member{Годлевський М. Д.}
			\member{Чередніченко О. Ю.}
			\member{Шматко О. В.}
		\end{tabular}
	\end{flushright}
	
	\vspace*{\fill}

	\begin{center}
		м. Харків --- \the\year~рік
	\end{center}
\end{titlepage}

\begin{titlepage}
	\begin{center}
		МІНІСТЕРСТВО ОСВІТИ І НАУКИ УКРАЇНИ \\
		НАЦІОНАЛЬНИЙ ТЕХНІЧНИЙ УНІВЕРСИТЕТ \\
		<<ХАРКІВСЬКИЙ ПОЛІТЕХНІЧНИЙ ІНСТИТУТ>> \\
		Кафедра програмної інженерії та інформаційних технологій управління
	\end{center}
	\noindent
	Студент \suline{\theauthor} \hfill Група \suline{КН-Н218а}

	\vspace*{\fill}

	\begin{center}
		\MakeUppercase{Завдання} \\
		на науково-дослідну роботу \\
		з курсу <<Програмне забезпечення інтелектуальних систем>>
	\end{center}
	\noindent
	\textbf{Тема:} <<\thetitle>>
	
	\vspace*{\fill}

	\begin{addmargin}[0cm]{1.5cm} 
		\textbf{Короткий зміст роботи} \\
		\textit{а) реферативна частина} \\
		\uline{
		Опис розподільчої логістичної системи та проблем моделювання і управління такими системами. 
		Аналіз існуючих моделей логістичніх систем.
		Постановка задачі дослідження.
		} \\
		\textit{б) теоретична частина} \\
		\uline{
		Опис і порівняння моделей та алгоритмів для управління розподільчими логістичними системами.
		} \\
		\textit{в) експеріментальна частина} \\
		\uline{
		Розробка метрик оцінки моделі розподільчої системи.
		}
	\end{addmargin}
	
	\vspace*{\fill}

	\noindent
	Дата видачі завдання: 01.09.19 \hfill Термін захисту: 24.01.20 \\
	Керівник курсової роботи: \hfill /проф. каф. ПІІТУ Годлевский~М.~Д./
\end{titlepage}
}

\begin{titlepage}
\begin{center}
	\MakeUppercase{Відгук} \\
	на науково-дослідну роботу \\
	<<\thetitle>>
\end{center}
\end{titlepage}

\begin{titlepage}
\section*{Реферат}
Пояснювальна записка до НДР: \pageref{LastPage}~с., \totalfigures~рис., \totaltables~табл., 10 дж. \bigbreak
\textit{Ключові слова}: \MakeUppercase{мультиагентна система, розподільча логістична система, моделювання, агентне моделювання}. \bigbreak

Об'єктом дослідження є процес прийняття рішень розподільчою логістичною системою. 

Предметом дослідження є мультиагентна модель розподільчої логістичною системи.

Метою та завданням дослідження є аналіз предметної області та опис моделей та алгоритмів для управління розподільчими логістичними системами.

Для досягнення поставленої мети були розглянуті основні проблеми моделювання логістичних систем, описані методи та алгоритми управління логістичними системами.

\end{titlepage}

\begin{titlepage}
\section*{Abstract}
Explanatory note to the thesis: \pageref{LastPage}~pages, \totalfigures~fig., \totaltables~tab., 20 sources, 1 appendix. \bigbreak 
\textit{Keywords}: \MakeUppercase{multiagent system, distribution logistic system, modeling, agent modeling}. \bigbreak 

The object of the paper is the decision-making process of distribution logistic system.

The subject of the paper is the multiagent model of  distribution logistic system.

The goal of the research is to develop a multiagent system to study distribution logistic system.

To archive the goal, the main problems of modeling logistic systems were reviewed, described methods and algorithms of managing logistic systems.
Formed the requrements to the software.
Based on the requrements and comparing software platforms for developing multiagent systems were choosen program technologies for implementing the system. 

The software was developed and tested.

The system can be used by logistic companies to improve their service or by teachers and students to study the logistic systems.

\end{titlepage}
