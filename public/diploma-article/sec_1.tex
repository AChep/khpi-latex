%\section{Аналіз існуючих моделей і алгоритмів управління логістичними системами. Постановка задачі}
\section{Аналіз предметної області}
\subsection{Розподільча логістична система як об'єкт дослідження}
% https://essuir.sumdu.edu.ua/bitstream/123456789/38038/1/Bilovodska_Kyslyi_Olefirenko_Solyanyk.pdf
Розподільча логістика --- це частина загальної логістичної системи, яка забезпечує найбільш ефективну організацію розподілу продукції, охоплюючи систему товароруху і виконуючи логістичні операції транспортування, складування, упакування та ін.~\cite{Kusluy2010}.

Розподільча логістика спрямована на комплексне планування, управління та фізичне опрацювання потоку готових виробів у супроводі необхідного інформаційного, фінансового та сервісного потоку від моменту здачі-приймання товарів з виробництва до замовника (споживача) з метою оптимізації витратних та часових характеристик зазначеної частини матеріального і нематеріального потоків.
Головна мета розподільчої логістики --- організація розподільчої діяльності відповідно до замовлень клієнтів з мінімальними загальними витратами~\cite{Kusluy2010}.

Принципова відмінність розподільчої логістики від традиційного розуміння збуту полягає насамперед у системному взаємозв'язку процесу розподілу з процесами виробництва і закупівель під час управління матеріальними потоками, а також системному взаємозв'язку всіх функцій всередині самого розподілу.

Матеріальний потік у сфері розподілу має форму готової продукції.
Залежно від суб'єкту економічних відносин, який бере участь у доведенні ресурсів до споживача, потік готової продукції можна подати як товарний потік або як вантажний потік (на транспорті).

Розподільча логістика будується на загальних логістичних принципах~\cite{Anikin1999}:
\begin{itemize}
	\item координація всіх процесів товароруху, починаючи від кінцевих операцій товаровиробника та закінчуючи сервісом споживача;
	\item інтеграція всіх функцій управління процесами розподілу готової продукції та послуг, починаючи від визначення мети та закінчуючи контролем;
	\item адаптація комерційного, канального та фізичного розподілу до постійно змінних вимог ринку та потреб споживача;
	\item координація всіх процесів товароруху, починаючи від кінцевих операцій товаровиробника та закінчуючи сервісом споживача;
	\item системність як управління розподілом в його цілісності та взаємозалежності всіх елементів збутової діяльності;
	\item комплексність, тобто вирішення всієї сукупності проблем, пов’язаних із задоволенням платоспроможного попиту покупців;
	\item оптимальність стосовно як елементів системи, так і режиму її функціонування;
	\item раціональність як в організаційній структурі, так і в організації управління.
\end{itemize}

Склад завдань розподільчої логістики на мікро- та на макрорівні різний~(таблиця~\ref{tab:logistic_functions}). 

\begin{table}[H]
	\caption{Завдання розподільчої логістики на мікро- та макрорівнях}
	\label{tab:logistic_functions}
	\begin{tabular}{@{}|p{0.53\linewidth}|p{0.4\linewidth}|@{}}
	 	\hline
		Мікрорівень & Макрорівень \\ \hline
		\begin{itemize}[leftmargin=*]
			\item оптимізація формування портфеля замовлень;
			\item укладання договорів із замовниками на постачання продукції;
			\item забезпечення ритмічності та дотримання планомірності реалізації продукції;
			\item вивчення і задоволення потреб у логістичному сервісі;
			\item раціоналізація параметрів, структури і просування динамічних матеріальних потоків;
			\item оптимізація параметрів і умов зберігання запасів товарного характеру;
			\item формування і вдосконалення системи інформаційного забезпечення.
		\end{itemize}
		&
		\begin{itemize}[leftmargin=*]
			\item вибір схеми розподілу матеріального потоку;
			\item визначення оптимальної кількості розподільчих центрів на території, яка обслуговується;
			\item визначення оптимального місця розташування розподільчого центру на території, яка обслуговується, та ін.
		\end{itemize} \\ \hline
	\end{tabular}
\end{table}

\subsection{Проблеми моделювання і управління розподільчими логістичними системами}
Основною проблемою, характерною для об'єкта дослідження, яка породжує безліч інших проблем, є його ієрархічність і розподіленість. 
В таких системах процеси розосереджені по окремих підсистемах і знаходяться на різних рівнях ієрархії. 
Для таких систем вирішується комплекс взаємопов'язаних задач в режимі багатосторонньої взаємодії між менеджерами-аналітиками, що відповідають за окремі локальні завдання і \acrshort{computer}. 
Основними ознаками розподіленості будь-якої логістичної системи можна вважати:
\begin{itemize}
	\item наявність механізму розбиття даної системи на окремі взаємопов'язані підсистеми;
	\item окремі складові системи географічно відокремлені;
	\item відносна автономність окремих підсистем;
	\item спільне завдання всієї системи розглядається у вигляді набору окремих локальних підзадач;
	\item паралельність і асинхронність рішення окремих локальних задач різними виконавцями.
\end{itemize}

Першою проблемою, яку необхідно вирішувати, є розбиття кожної системи на окремі локальні підсистеми. 
Можна сказати, що формалізація цих двох завдань здійснюється на основі декомпозиції і агрегування. 
Декомпозиція полягає в розчленуванні вихідної задачі на ряд відносно незалежних підзадач, а агрегування --- в заміні окремих груп змінних, що характеризують ефективність функціонування системи, змінними-агрегатами. 
При цьому висувається вимога повної (достатньої) еквівалентності задач. 
Агрегування параметрів і змінних здійснюється в ході руху вгору по ієрархії. 
Це пов'язано з великою розмірністю завдання і неможливістю прийняття рішень на основі варіювання всіх параметрів і змінних. 
Основні ідеї, які реалізуються при синтезі моделі на основі декомпозиції і агрегування полягають у наступному:
\begin{itemize}
\item нехтуючи слабкими зв'язками між окремими підсистемами, зробити декомпозицію;
\item використовуючи трохи відмінності між ними, зробити агрегування;
\item використовуючи сильні відмінності, виділити <<вузькі місця>>, відкинувши на основі апріорних оцінок несуттєві обмеження.
\end{itemize}

\subsection{Моделі логістичних систем}
Моделювання в загальному вигляді являє собою один з основних методів пізнання, є формою відображення дійсності і полягає у з'ясуванні або відтворенні тих чи інших властивостей реальних об'єктів, процесів, явищ за допомогою абстрактного опису у вигляді зображення, плану, карти, сукупності рівнянь, алгоритмі і програм.

Одним з найбільш ефективних методів дослідження складних систем розподільчої логістики є імітаційне моделювання~\cite{Kobelev2003}.

Імітаційне моделювання --- експериментальний метод дослідження реальної системи за її імітаційною моделлю, який поєднує особливості експериментального підходу і специфічні умови використання обчислювальної техніки~\cite{Emelyanov2002}.

Серед переваг імітаційного моделювання відзначають~\cite{Emelyanov2002}: 
\begin{enumerate}
	\item Відображення динамічних процесів і поведінкових аспектів зовнішнього середовища.
	\item Можливість виявлення закономірностей, динамічних тенденцій розвитку і функціонування складної системи в умовах неповної та неточної інформації.
	\item Опис взаємодії та поведінки безлічі активних агентів в соціальних системах.
	\item Реалізацію принципів об'єктно-орієнтованого проектування і застосування високотехнологічних рішень при побудові комп'ютерних моделей та ін.
\end{enumerate}

Головною проблемою при побудові будь імітаційної моделі є необхідність побудови комплексних математичних моделей і розробки програмного коду імітаційної моделі. 

У імітаційному моделюванні виділяють такі основні підходи:
\begin{itemize}
	\item системна динаміка;
	\item дискретне моделювання;
	\item агентне моделювання.
\end{itemize}

\subsubsection{Системна динаміка}
Як методологія системна динаміка була запропонована в 1961 році Дж. Форрестером в якості інструменту дослідження інформаційних зворотних зв'язків у виробничо-господарської діяльності. 
Процеси, що відбуваються в реальному світі, в системній динаміці представляються в термінах накопичувачів і потоків між ними.
Системнодинамічна модель описує поведінку системи та її структуру як безліч взаємодіючих зворотних зв'язків і затримок. 
Математично така модель виглядає як система диференціальних рівнянь. 
Результатом моделювання в системній динаміці є виявлення глобальних залежностей і причинно-наслідкових зв'язків у досліджуваній системі~\cite{Shamrin2016}. 

\subsubsection{Дискретне моделювання}
Основний об'єкт в системі дискретного моделювання --- пасивний транзакт, який може певним чином представляти собою працівників, деталі, сировину, документи, сигнали і т. п.
Переміщаючись по моделі, транзакти стають в черги до одноканальним і багатоканальним пристроям, захоплюють і звільняють їх, розщеплюються, знищуються і т. д.
Відмінною особливістю даного підходу є час просування по моделі: або від події до події, або через дискретні проміжки часу. 
Дискретне моделювання застосовується, якщо можливо припустити, що змінні в системі змінюються миттєво в певні проміжки часу. 
Даний підхід імітаційного моделювання є одним з найпоширеніших і застосовується для дослідження соціально-економічних, технічних, логістичних та інших процесів.
На основі дискретного підходу реалізовано найбільше
число систем імітаційного моделювання~\cite{Shamrin2016}. 

\subsubsection{Агентне моделювання}
Агентське моделювання з'явилося в 90-х роках і використовується для дослідження децентралізованих систем, динаміка функціонування яких визначається не глобальними правилами і законами (як в інших парадигмах моделювання), а коли ці глобальні правила і закони є результатом індивідуальної активності членів групи.

Агентно-орієнтована система може складатися з одного агента (наприклад, програмний секретар~\cite{Maes1995}), проте весь потенціал розкривається з використанням мультиагентної системи~\cite{Waters1989}.
Під агентом розуміється система, яка має такі властивості~\cite{Jennings1998,Wooldridge1995}:
\begin{enumerate}[label={\arabic*)}]
	\item автономність: агенти мають внутрішній стан (який недоступний іншим агентам) та приймають рішення на основі своїх даних, без прямого втручання людини;
	\item реактивність: агенти розміщуються в навколишньому середовищі (яке може бути фізичним світом, множиною інших агентів, інтернетом і т.д.), здатні спостерігати і своєчасно реагувати на зміни;
	\item проактивність: агенти не тільки реагують на зміни в зовнішньому середовищі, вони здатні виявляти ініціативу для досягнення своєї мети;
	\item соціальність: агенти взаємодіють з іншими агентами (і, можливо, людиною) через спеціальний інтерфейс для досягнення їх цілей.
\end{enumerate}

Мета агентських моделей --- отримати уявлення про ці глобальні правила, загальну поведінку системи, виходячи з припущень про індивідуальну, приватну поведінку її окремих активних об'єктів і взаємодію цих об'єктів в системі.
У разі моделювання логістичних систем, що містять великі кількості активних об'єктів (людей, машин, підприємств чи навіть проектів, активів, товарів і т. п.), які об'єднує наявність елементів індивідуальної поведінки, агентське моделювання є підходом більш універсальним і потужним, оскільки дозволяє врахувати будь-які складні структури та їх поведінку~\cite{Shamrin2016}. 

\subsection{Глосарій проекту}
Глосарій проекту --- це розвернутий словник у вигляді таблиці, який складається з термінів що характеризують дану предметну область.

Глосарій проекту:
\begin{enumerate}
    \item Агент \textit{(agent)} --- це сутність, що спостерігає за навколишнім середовищем і діє у ньому, при цьому його поведінка раціональна в тому розумінні, що він здатен до розуміння і його дії завжди спрямовані на досягнення якої-небудь мети~\cite{Jennings1998}.
    \item Транспорт \textit{(transport)} --- сукупність засобів, для переміщення людей, вантажів, сигналів та інформації з одного місця в інше.
    \item Склад \textit{(warehouse)} --- це складна технічна споруда, яка складається із взаємопов'язаних елементів, що має певну структуру та виконує ряд функцій з перетворення матеріальних потоків, а також накопичення, переробки та розподілу вантажів між споживачами~\cite{Kusluy2010}. 
	\item Запаси гарантійні \textit{(insurance stocks)} призначені для безперервного постачання споживачів у випадку непередбачених обставин: відхилення в періодичності та величині партій поставок від запланованих, зміни інтенсивності споживання, затримки поставок та ін. і є постійною величиною, що залежить від умов виконання конкретних поставок~\cite{Kusluy2010}. 
	\item Ланцюг логістичний \textit{(logistics chain)}  --- це складна система, що формується впорядкованою і взаємодіючою сукупністю фізичних чи юридичних осіб на ринку виробництва і постачання матеріальних ресурсів, виробництва та розподілу продукції, які виконують логістичні операції, спрямовані на доведення матеріального потоку від однієї логістичної системи до іншої та до кінцевого споживача~\cite{Kusluy2010}.  
    \item Логістика \textit{(logistics)} --- системоохоплюючий механізм, який можна трактувати як досягнення компромісу (узгодження) між виконанням зобов’язань і необхідними для цього витратами в сфері виробництва, транспортно-складського забезпечення, у процесі отримання потрібних товарів або послуг у потрібному місці, у потрібний час, у необхідній кількості з мінімальними загальними витратами при високій якості обслуговування споживача~\cite{Kusluy2010}.
	\item Логістика розподільча \textit{(distribution logistics)} --- галузь логістики, яка забезпечує найбільш ефективну організацію розподілу продукції, охоплюючи систему товароруху і виконуючи логістичні операції транспортування, складування, упакування та ін.~\cite{Kusluy2010}.
	\item Рівень сервісу \textit{(service level)} --- кількісна характеристика відповідності фактичних значень показників якості і кількості логістичних послуг оптимальним або теоретично можливим значенням~\cite{Kusluy2010}.
    \item Операції логістичні \textit{(logistic operations)} --- відособлена сукупність дій, скерована на перетворення матеріального та супутніх йому потоків~\cite{Kusluy2010}.
    \item Сервіс логістичний \textit{(logistic service)} --- це сукупність функцій і видів діяльності всіх підсистем підприємства, що забезпечують зв’язок <<підприємство-споживач>> для кожного матеріального та інформаційного потоку за показниками номенклатури, якості, кількості, ціни, місця і часу постачання продукції відповідно до вимог ринку~\cite{Kusluy2010}.
\end{enumerate}

\subsection{Постановка задачі}
Завданням переддипломної роботи є розробка мультиагентної системи для дослідження різних конфігурацій розподільчої логістичної системи.

Розроблена система повинна мати простий графічний інтерфейс та бути зручною для використання. 

Даний програмний продукт може бути використаний логістичними компаніями з метою покращення сервісу, викладачами та студентами для дослідження логістичних систем.

Для розробки даної системи необхідно вирішити наступні задачі:
\begin{itemize}
	\item визначення основних понять та аналіз проблем моделювання логістичних систем;
	\item огляд методів моделювання;
	\item декомпозіция агентів логистічної системи;
	\item опис мети агентів;
	\item опис логістичних моделей агентів;
	\item опис метрики оцінки моделі.
\end{itemize}
