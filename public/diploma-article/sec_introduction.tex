\section*{Вступ}
\addcontentsline{toc}{section}{Вступ}

%Керуючись життєвим досвідом і науковими знаннями, людина будує моделі --- від паперових корабликів до картини світу. 
%Чим вони багатші і чим точніше ми можемо ними оперувати, тим краще наша свідомість --- наша <<найважливіша модель>>, відповідає реальності і знаходить способи її зміни.

% NaUKMAkn_2013_151_16.pdf
% http://www.economy.in.ua/pdf/1_2016/9.pdf
На сучасному етапі розвитку складні логістичні системи вимушені працювати в умовах високої невизначеності, що суттєво ускладнює управління ними. 
В процесі прийняття управлінських рішень виникає проблема прогнозування поведінки системи та зовнішнього середовища. 
Результати прогнозів необхідно постійно коригувати по ходу розвитку подій, що дозволяє пристосовуватися до змін оточення та гнучко реагувати на негативні впливи. 

У нагоді тут стає агентне моделювання, яке сягає своїм історичним корінням складних адаптивних систем і принципу побудови систем знизу вгору.
Агентне моделювання дозволяє здійснити множину прогнозів за різними сценаріями залежно від формування різноманітних ситуацій практично необмеженої складності. 

Основними елементами агентного моделювання є агенти, стосунки між ними і простір, в якому відбувається взаємодія. 
Агенти моделюються індивідуально. 
Вони можуть мати неповну інформацію, здійснювати помилки, адаптуватися до ситуації, проявляти ініціативу. 
В основу агентного моделювання закладені такі принципи, як різноманітність, взаємозв’язок і міра взаємодії. 
Тип взаємодій різних агентів може відрізнятися і носити ймовірнісний характер. 
Результатом динамічної взаємодії може бути певний рівноважний стан системи, а може бути і нова якість, яку неможливо передбачати з аналізу окремих складових системи.

Об'єктом дослідження є процес моделювання логистічної системи. 

Предметом дослідження є моделі та інструментальні засоби для розробки системи для оцінки рівня сервісу логістичних систем.

Теоретико-методологічною основою роботи є агентне моделювання, системний аналіз, а також базова теорія логістики.

Метою дослідження є розробка та дослідження моделей та програмна реализація прототипа системи для визначення рівня сервісу логістічної системи.
Для досягнення поставленої мети в переддипломній роботі були сформульовані та вирішені наступні задачі:
\begin{itemize}
	% logistics
	\item опис динаміки розподільчої логістичної системи;
	\item дослідження проблеми імітаціонного моделювання розподільчої логістичної системи;
	% agents
	\item огляд методів моделювання;
	\item декомпозіция агентів логистічної системи;
	\item огляд засобів створення мультіагентних систем; 
	\item опис мети агентів;
	\item опис логістичних моделей агентів;
	\item опис метрики оцінки моделі.
\end{itemize}
