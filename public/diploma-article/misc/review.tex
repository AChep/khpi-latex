\documentclass[a4paper,14pt,oneside,final]{extarticle}
\usepackage[top=2cm, bottom=2cm, left=3cm, right=1cm]{geometry}
\usepackage{scrextend}

\usepackage[T2A,T1]{fontenc}
\usepackage[ukrainian,russian,english]{babel}
\usepackage{tempora}
\usepackage{fontspec}
\setmainfont{tempora}

% Зачем: Отключает использование изменяемых межсловных пробелов.
% Почему: Так не принято делать в текстах на русском языке.
\frenchspacing

\usepackage{indentfirst}
\setlength{\parindent}{1.25cm}
\renewcommand{\baselinestretch}{1.5}

% Header
\usepackage{fancyhdr}
\pagestyle{fancy}
\fancyhead{}
\fancyfoot{}
\fancyhead[R]{\small \selectfont \thepage}
\renewcommand{\headrulewidth}{0pt}

% Captions
\usepackage{chngcntr}
\counterwithin{figure}{section}
\counterwithin{table}{section}
\usepackage[tableposition=top]{caption}
\usepackage{subcaption}
\DeclareCaptionLabelFormat{gostfigure}{Рисунок #2}
\DeclareCaptionLabelFormat{gosttable}{Таблиця #2}
\DeclareCaptionLabelSeparator{gost}{~---~}
\captionsetup{labelsep=gost}
\captionsetup[figure]{labelformat=gostfigure}
\captionsetup[table]{labelformat=gosttable}
\renewcommand{\thesubfigure}{\asbuk{subfigure}}

% Sections
\usepackage[explicit]{titlesec}
\newcommand{\sectionbreak}{\clearpage}

\titleformat{\section}
  {\centering}{\thesection \quad}{0pt}{\MakeUppercase{#1}}
\titleformat{\subsection}[block]
  {\bfseries}{\thesubsection \quad #1}{0cm}{}

\titlespacing{\section} {0cm}{0cm}{21pt}
\titlespacing{\subsection} {\parindent}{21pt}{0cm}
\titlespacing{\subsubsection} {\parindent}{0cm}{0cm}

% Lists
\usepackage{enumitem}
\renewcommand\labelitemi{--}
\setlist[itemize]{noitemsep, topsep=0pt, wide}
\setlist[enumerate]{noitemsep, topsep=0pt, wide, label=\arabic*}
\setlist[description]{labelsep=0pt, noitemsep, topsep=0pt, leftmargin=2\parindent, labelindent=\parindent, labelwidth=\parindent, font=\normalfont}

% Toc
\usepackage{tocloft}
\tocloftpagestyle{fancy}
\renewcommand{\cfttoctitlefont}{}
\setlength{\cftbeforesecskip}{0pt}
\renewcommand{\cftsecfont}{}
\renewcommand{\cftsecpagefont}{}
\renewcommand{\cftsecleader}{\cftdotfill{\cftdotsep}}

\usepackage{float}
\usepackage{pgfplots}
\usepackage{graphicx}
\usepackage{multirow}
\usepackage{amssymb,amsfonts,amsmath,amsthm}
\usepackage{csquotes}

\usepackage{listings}
\lstset{basicstyle=\footnotesize\ttfamily,breaklines=true}
\lstset{language=Matlab}

\usepackage[
	backend=biber,
	sorting=none,
	language=auto,
	autolang=other
]{biblatex}
\DeclareFieldFormat{labelnumberwidth}{#1}


\usepackage{lastpage}
\usepackage{calc}
\usepackage{soul}
\usepackage{pbox}
\usepackage{ulem}
\usepackage{titling}
\usepackage{framed}
\usepackage{tabu}
\usepackage{lscape}
\usepackage{appendix}
\usepackage{pdflscape}
\usepackage{longtable}
\usepackage[figure,table]{totalcount}

\title{Огляд існуючих програмних систем для аналізу логістичної системи дистрибуції при стратегічному плануванні}
\author{Чепурний А. С.}
 
\addbibresource{bibliography.bib}

\begin{document}
\Ukrainian

\begin{titlepage}
\begin{center}
	\small
		\textbf{\theauthor} -- 2.КН201н.8а
\end{center}

\vspace*{\fill}
\begin{center}
	\Large
		\thetitle
\end{center}
\vspace*{\fill}

\end{titlepage}

\addtocounter{page}{1}
\renewcommand\contentsname{\hspace*{\fill}\bfseries\MakeUppercase{Зміст}\hspace*{\fill}}
\tableofcontents


\section*{Вступ}
\addcontentsline{toc}{section}{Вступ}
Логістичні системи є венами бізнесу, вони притаманні як виробничим так і сервісним підприємствам~\cite{Wanga}.

Вимоги до управління логістичними системами постійно зростають~\cite{Wanga,Croom2000} і компаніям доводиться адаптуватися до них. 

На даний момент існує безліч програмних систем для моделювання та управління логістичними системами.
Такі системи включають в себе не тільки оцінку стійкості функціонування логістичної системи, а інші компоненти для побудови і оптимізації логістичних систем, управління складами, транспортуванням, попитом, співробітництвом з постачальниками~\cite{Das2007,Hill2007}.   

Об'єктом дослідження програмні системи для аналізу логистичних систем.

Предметом дослідження є порівняння програмних систем для аналізу логистичних систем.

Метою і завданням дослідження є опис існуючих програмних програмних систем для аналізу логистичних систем, опис іх переваг та недоліків.

\section{Огляд програм для управління логістичними системами}
\subsection{Oracle E-Business Suite}
Сімейство застосунків Oracle E-Business Suite (R12) інтегрує і автоматизує всі ключові процеси логістичної системи. Компанії можуть передбачити вимоги ринку і ризики, адаптувати рішення до змін ринку.

Система надає уніфіковану модель всієї логістичної системи~\cite{Wanga}.

В даний момент Oracle активно працює над цією лінійкою продуктів~\cite{Wanga}.

\subsection{SAP}
SAM позиціонується як високоякісна платформа для бізнесу.
Сервіси SAM включають в себе допомогу експертів, методології та програмні інструменти~\cite{Wanga}.

SAM надає повний набір функцій для побудови адаптивної логістичної системи, включає в себе інструменти для планування (стратегічне, тактичне та операційне).
Використовуючи SAM компанія може оптимізувати планування попиту, запасів товару, конфігурацію логістичної мережі, і інше~\cite{Wanga}.

\subsection{Infor}
Infor розроблює низку продуктів у таких сферах: виробництво, логістика, фінанси, управління проектами та інше~\cite{Wanga}.

Модуль логістики Infor побудований на системі Open SOA і надає широкі можливості по кастомізації.
Цей модель є складним в початковій конфігурації і тому націлений на середній і великий бізнес.
Для маленьких фірм Infor розробив окрему лінійку продуктів, які покривають базові потреби компаній і мають невелику вартість~\cite{Wanga}. 

\section{Порівняння програм для управління логістичними системами}
Для проведення аналізу якості буде використовуватися програмна система Expert System.

\subsection{Критерії порівняння}
Для проведення аналізу біли обрані такі критерії якості:
\begin{itemize}
	\item переносимість: час, кількість платформ, підтримка незалежних розробників;
	\item надійність і стабільність: підтримка різних типов моделей, підтримка створення власних моделей та шаблонів; 
	\item ефективність: відповідність функцій системи цілям бізнеса, швидкість роботи;
	\item легкість інтеграції та розширення: доступність демо, необхідна спеціалізація користувачів, лінія підтримки;
	\item простота інтерфейсу;
	\item ціноутворення.
\end{itemize}

Результати порівняння представлені у таблиці~\ref{tab:results}~\cite{Wanga}.

\begin{table}[H]
	\caption{Результати порівняння}
	\label{tab:results}
	\begin{tabular}{l|c|c|c}
		Критерій & SAP & Infor & Oracle \\\hline
		Переносимість & $0.968$ & $0.343$ & $1.000$ \\
		Рівень підтримки & $0.950$ & $0.601$ & $1.000$ \\
		Легкість розширення & $0.991$ & $0.717$ & $1.000$ \\
		Ефективність & $1.000$ & $0.620$ & $0.899$ \\
		Надійність & $0.991$ & $0.692$ & $1.000$ \\
		Простота інтерфейсу & $0.824$ & $0.934$ & $1.000$ \\
		Ціноутворення & $0.931$ & $0.610$ & $1.000$ \\\hline
		$\sum$ & $6.665$ & $4.407$ & $6.899$ \\
	\end{tabular}
\end{table}

\subsection{Сценарії порівняння}
\subsubsection{Онлайн стартап}
Невелика компанія по продажу товарів, з декількома партнерами, без фіксованого місця розташування, обмежені знання про галузь~\cite{Wanga}.

Основуваючись на опису компанії були обрані такі основні критерії:
\begin{itemize}
	\item переносимість, бо вони потребують програмне рішення, яке може працювати з їх інтернет платформою та в різних операційних системах;
	\item сервіс підтримки, бо вони не мають достатніх знань про галузь логістики;
	\item ціноутворення.
\end{itemize}

Критерії порівняння представлені у таблиці~\ref{tab:criterias_startup}. Результати порівняння представлені у таблиці~\ref{tab:results_startup}.

\begin{table}[H]
	\caption{Ваги критеріїв порівняння для онлайн стартапу}
	\label{tab:criterias_startup}
	\begin{tabular}{l|c}
		Критерій & Значимість \\\hline
		Ціноутворення & $0.300$ \\
		Переносимість & $0.239$ \\
		Рівень підтримки & $0.183$ \\
		Легкість розширення & $0.176$ \\
		Ефективність & $0.138$ \\
		Надійність & $0.103$ \\
		Простота інтерфейсу & $0.067$ \\
	\end{tabular}
\end{table}

\begin{table}[H]
	\caption{Результати порівняння для онлайн стартапу}
	\label{tab:results_startup}
	\begin{tabular}{l|c|c|c}
		Критерій & SAP & Infor & Oracle \\\hline
		Переносимість & $0.231$ & $0.081$ & $0.239$ \\
		Рівень підтримки & $0.173$ & $0.109$ & $0.183$ \\
		Легкість розширення & $0.174$ & $0.126$ & $0.176$ \\
		Ефективність & $0.138$ & $0.085$ & $0.124$ \\
		Надійність & $0.102$ & $0.071$ & $0.103$ \\
		Простота інтерфейсу & $0.055$ & $0.062$ & $0.067$ \\
		Ціноутворення & $0.279$ & $0.183$ & $0.300$ \\\hline
		$\sum$ & $1.154$ & $0.720$ & $1.192$ \\
	\end{tabular}
\end{table}

\section*{Висновки}
\addcontentsline{toc}{section}{Висновки}
В програмної індустрії системам для аналізу логістичних систем надається все більше уваги.
Ця увага призводить до більших інвестицій в цю область і більшої конкуренції серед програм, що призводить до збільшення їх якості.

У даній роботі були оцінено якість таких систем.
Для загальної ситуації найкращою системою виявилася система від Oracle --- Oracle E-Business Suite R12.

Був складений і перевірений один сценарій для невеликої стартап компанії.
Для цього сценарію краще програмою також виявилася система від Oracle.

Однак, оскільки проблеми, критерії та потреби варіюються від компанії до компанії, то не можна стверджувати що одна програма краща за іншу у всіх ситуаціях.
Щоб вибрати кращу з них необхідно спочатку виділити та описати всі ключові фактори компанії.

\printbibliography[heading=bibintoc, title={Список джерел інформації}]


\end{document}