\section*{Висновки}
\addcontentsline{toc}{section}{Висновки}
Сучасні інструменти імітаційного моделювання дозволяють ефективно застосовувати його
не тільки в наукових дослідженнях, а й як засоби для побудови систем підтримки прийняття рішень у бізнесі. 

Агентне моделювання дозволяє змоделювати систему максимально наближену до реальності, зробити значний крок у розумінні та управлінні сукупністю складних процесів.

Основним завданням даної роботи була розробка мультиагентної системи для дослідження розподільчої логістичної системи.
Для досягнення поставленої мети роботи виконано наступні завдання:
\begin{enumerate}
    \item Здійснено огляд проблем моделювання та управління розподільчими логістичними системами. Описано та порівняно моделі логістичних систем, в результаті чого було обрано агентне моделювання як найбільш гнучкий та простий метод.
    \item Проведена декомпозіция логістичної системи на агентів: склад, постачальник, роздрібний торговець; сформульовані цілі кожного з агентів.
    \item Надано порівняльній аналіз архитектурніх рішень для створення мультиагентних систем.
	\item Описано методи та алгоритми для управління логістичними системами.
	\item Сформульовано метрику оцінки розробленої моделі.  
\end{enumerate}

