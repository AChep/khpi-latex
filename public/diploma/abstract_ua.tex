\section*{Реферат}
Пояснювальна записка до дипломної роботи: \pageref{LastPage}~с., \totalfigures~рис., \totaltables~табл., \total{citnum}~дж., 1 додаток. \bigbreak
\textit{Ключові слова}: \MakeUppercase{мультиагентна система, розподільча логістична система, моделювання, агентне моделювання}. \bigbreak

Об’єктом дослідження є процес моделювання логістичної системи дистрибуції за допомогою мультиагентних систем.

Предметом дослідження є моделі та інструментальні засоби для розробки системи оцінки рівня сервісу логістичних систем.

Метою та завданням дослідження є оцінка рівня сервісу логістичних систем дистрибуції товарів масового вжитку.

Для досягнення поставленої мети були розглянуті основні проблеми моделювання логістичних систем, описані методи та алгоритми управління логістичними системами. Сформульовані вимоги до програмної системи, представлена модель її архітектури. На основі порівняння програмних платформ для розробки мультиагентних систем та вимог були обрані програмні технології для реалізації системи.

Була реалізована та протестована програмна система. 

Подальше використання отриманих результатів пов'язане з визначенням стійкості рівня сервісу до різноманітних надзвичайних ситуацій. Отримані результати є основою для формування організаційної структури управління логістичною системою дистрибуції.

Дана система може бути використана логістичними компаніями з метою покращення сервісу, викладачами та студентами для дослідження логістичних систем. 
