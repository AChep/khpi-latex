%\section{Аналіз існуючих моделей і алгоритмів управління логістичними системами. Постановка задачі}
\section{Аналіз предметної області}
\subsection{Розподільча логістична система як об'єкт дослідження}
% https://essuir.sumdu.edu.ua/bitstream/123456789/38038/1/Bilovodska_Kyslyi_Olefirenko_Solyanyk.pdf
Розподільча логістика --- це частина загальної логістичної системи, яка забезпечує найбільш ефективну організацію розподілу продукції, охоплюючи систему товароруху і виконуючи логістичні операції транспортування, складування, упакування та ін.~\cite{Kusluy2010}.

Розподільча логістика спрямована на комплексне планування, управління та фізичне опрацювання потоку готових виробів у супроводі необхідного інформаційного, фінансового та сервісного потоку від моменту здачі-приймання товарів з виробництва до замовника (споживача) з метою оптимізації витратних та часових характеристик зазначеної частини матеріального і нематеріального потоків.
Головна мета розподільчої логістики --- організація розподільчої діяльності відповідно до замовлень клієнтів з мінімальними загальними витратами~\cite{Kusluy2010}.

Принципова відмінність розподільчої логістики від традиційного розуміння збуту полягає насамперед у системному взаємозв'язку процесу розподілу з процесами виробництва і закупівель під час управління матеріальними потоками, а також системному взаємозв'язку всіх функцій всередині самого розподілу.

Матеріальний потік у сфері розподілу має форму готової продукції.
Залежно від суб'єкту економічних відносин, який бере участь у доведенні ресурсів до споживача, потік готової продукції можна подати як товарний потік або як вантажний потік (на транспорті).

Розподільча логістика будується на загальних логістичних принципах~\cite{Anikin1999}:
\begin{itemize}
	\item координація всіх процесів товароруху, починаючи від кінцевих операцій товаровиробника та закінчуючи сервісом споживача;
	\item інтеграція всіх функцій управління процесами розподілу готової продукції та послуг, починаючи від визначення мети та закінчуючи контролем;
	\item адаптація комерційного, канального та фізичного розподілу до постійно змінних вимог ринку та потреб споживача;
	\item координація всіх процесів товароруху, починаючи від кінцевих операцій товаровиробника та закінчуючи сервісом споживача;
	\item системність як управління розподілом в його цілісності та взаємозалежності всіх елементів збутової діяльності;
	\item комплексність, тобто вирішення всієї сукупності проблем, пов’язаних із задоволенням платоспроможного попиту покупців;
	\item оптимальність стосовно як елементів системи, так і режиму її функціонування;
	\item раціональність як в організаційній структурі, так і в організації управління.
\end{itemize}

Склад завдань розподільчої логістики на мікро- та на макрорівні різний~(таблиця~\ref{tab:logistic_functions}). 

\begin{table}[H]
	\caption{Завдання розподільчої логістики на мікро- та макрорівнях}
	\label{tab:logistic_functions}
	\begin{tabular}{@{}|p{0.53\linewidth}|p{0.4\linewidth}|@{}}
	 	\hline
		Мікрорівень & Макрорівень \\ \hline
		\begin{itemize}[leftmargin=*]
			\item оптимізація формування портфеля замовлень;
			\item укладання договорів із замовниками на постачання продукції;
			\item забезпечення ритмічності та дотримання планомірності реалізації продукції;
			\item вивчення і задоволення потреб у логістичному сервісі;
			\item раціоналізація параметрів, структури і просування динамічних матеріальних потоків;
			\item оптимізація параметрів і умов зберігання запасів товарного характеру;
			\item формування і вдосконалення системи інформаційного забезпечення.
		\end{itemize}
		&
		\begin{itemize}[leftmargin=*]
			\item вибір схеми розподілу матеріального потоку;
			\item визначення оптимальної кількості розподільчих центрів на території, яка обслуговується;
			\item визначення оптимального місця розташування розподільчого центру на території, яка обслуговується, та ін.
		\end{itemize} \\ \hline
	\end{tabular}
\end{table}

\subsection{Проблеми моделювання і управління розподільчими логістичними системами}
Основною проблемою, характерною для об'єкта дослідження, яка породжує безліч інших проблем, є його ієрархічність і розподіленість. 
В таких системах процеси розосереджені по окремих підсистемах і знаходяться на різних рівнях ієрархії. 
Для таких систем вирішується комплекс взаємопов'язаних задач в режимі багатосторонньої взаємодії між менеджерами-аналітиками, що відповідають за окремі локальні завдання. 
Їх знання, компетенція, функції та відповідальність розосереджені по окремих етапах і робочих місць, які пов'язані як <<по вертикалі>>, так і <<по горизонталі>>.
Основними ознаками розподіленості будь-якої логістичної системи можна вважати:
\begin{itemize}
	\item наявність механізму розбиття даної системи на окремі взаємопов'язані підсистеми;
	\item окремі складові системи географічно відокремлені;
	\item відносна автономність окремих підсистем;
	\item спільне завдання всієї системи розглядається у вигляді набору окремих локальних підзадач;
	\item паралельність і асинхронність рішення окремих локальних задач різними виконавцями.
\end{itemize}

Першою проблемою, яку необхідно вирішувати, є розбиття кожної системи на окремі локальні підсистеми. 
Можна сказати, що формалізація цих двох завдань здійснюється на основі декомпозиції і агрегування. 
Декомпозиція полягає в розчленуванні вихідної задачі на ряд відносно незалежних підзадач, а агрегування --- в заміні окремих груп змінних, що характеризують ефективність функціонування системи, змінними-агрегатами. 
При цьому висувається вимога повної (достатньої) еквівалентності задач. 
Агрегування параметрів і змінних здійснюється в ході руху вгору по ієрархії. 
Це пов'язано з великою розмірністю завдання і неможливістю прийняття рішень на основі варіювання всіх параметрів і змінних. 
Основні ідеї, які реалізуються при синтезі моделі на основі декомпозиції і агрегування полягають у наступному:
\begin{itemize}
	\item нехтуючи слабкими зв'язками між окремими підсистемами, зробити декомпозицію;
	\item використовуючи трохи відмінності між ними, зробити агрегування;
	\item використовуючи сильні відмінності, виділити <<вузькі місця>>, відкинувши на основі апріорних оцінок несуттєві обмеження.
\end{itemize}

\subsection{Моделі та алгоритми для управління розподільчими логістичними системами}
\subsubsection{Розрахунок витрат на запаси}
Загальні витрати, пов'язані з запасами, представляють собою суму витрат на закупівлю, поповнення запасу і утримання запасів~\cite{Sterligova2008}:
\begin{equation} \label{eq:t}
T=C\cdot S+\cfrac{S}{C}\cdot A+(Z_s+\cfrac{Q}{2}\cdot I)
,
\end{equation}
\begin{description}
	\item[де] $T$ --- загальні витрати, пов'язані з запасом, г.~о.;
	\item $C$ --- закупівельні ціна одиниці товару, г.~о.;
	\item $Q$ --- розмір замовлення, одиниць;
	\item $S$ --- обсяг потреби в запасі, одиниць;
	\item $A$ --- витрати на виконання одного замовлення, г.~о.;
	\item $Z_s$ --- розмір страхового запасу, одиниць;
	\item $I$ --- витрати на утримання одиниці запасу, г.~о.
\end{description}

\subsubsection{Розрахунок оптимального розміру поповнення запасу}
В основі оптимізації рівня запасу лежить розрахунок розміру замовлення, який може забезпечити оптимальний рівень запасу при обслуговуванні на заданому рівні.
Критерієм оптимізації при цьому є, як правило, мінімум загальних витрат, пов'язаних з запасами.

Формула Вільсона --- найбільш відомий і широко вживаний метод розрахунку розміру замовлення~\cite{Sterligova2008}:
\begin{equation}
Q^*=\cfrac{dT}{dQ}=\sqrt{\cfrac{2\cdot A\cdot S}{I}}
,
\end{equation}
\begin{description}
	\item[де] $T$ --- загальні витрати~\eqref{eq:t}, пов'язані з запасом, г.~о.;
	\item $Q$ --- розмір замовлення, одиниць;
	\item $S$ --- обсяг потреби в запасі, одиниць;
	\item $A$ --- витрати на виконання одного замовлення, г.~о.;
	\item $I$ --- витрати на утримання одиниці запасу, г.~о.;
	\item $Q^*$ --- оптимальний розмір замовлення, одиниць.
\end{description}

\subsubsection{Модель з фіксованим розміром замовлень}
Методика управління запасами на основі фіксації розміру замовлення~(рисунок~\ref{fig:model_fs:dynamic}) полягає в тому, що замовлення на поповнення запасу робляться в момент зниження запасу до визначеного порогового рівня запасу, що дорівнює оптимальному розміру замовлення~\cite{Sterligova2008}. 

\begin{figure}[H]
  \centering
\begin{tikzpicture}
  \begin{axis}[ 
    xlabel={Час},
    ylabel={Запас},
    xmin=0,xmax=10,ymin=0,ymax=1
  ] 
	\addplot
		coordinates {
			(0,0.8) [0]
			(3,0.2) [1]
			(3,0.9) [2]
			(7,0.25) [3]
			(7,0.95) [4]
			(9,0.1) [5]
			(9,0.8) [6]
			(10,0.5) [7]
		};
	\draw[<->] (axis cs:2.8,0.2) -- node[left]{\footnotesize $Q$} (axis cs:2.8,0.9);

    \draw [densely dotted] (axis cs:0,0.4) -- node[below]{\footnotesize Пороговий запас} (axis cs:10,0.4);
    \draw [densely dotted] (axis cs:0,0.2) -- node[below]{\footnotesize Страховий запас} (axis cs:10,0.2);
  \end{axis}
\end{tikzpicture}
  \captionsetup{justification=centering}
  \caption{Динаміка запасу в моделі управління з фіксованим розміром замовлень}
  \label{fig:model_fs:dynamic}
\end{figure}

Максимальний бажаний запас може бути розрахований таким чином~\cite{Sterligova2008}:
\begin{equation} \label{eq:model_fs:mws}
MWS=Q^*+Z_s
,
\end{equation}
\begin{description}
	\item[де] $MWS$ --- максимальний бажаний запас, одиниць;
	\item $Q^*$ --- оптимальний розмір замовлення;
	\item $Z_s$ --- обсяг страхового запасу, одиниць.
\end{description}

Розмір страхового запасу може бути розрахований різними методами.
Метод прямого рахунку~\cite{Sterligova2008}:
\begin{equation} \label{eq:model_fs:zs1}
Z_s=C_d-t_{od}
,
\end{equation}
\begin{description}
	\item[де] $C_d$ --- очікуване денне споживання, одиниць;
	\item $t_{od}$ --- час затримки постачання, дні.
\end{description}

Страховий запас визначається як різниця між максимальним споживанням під час виконання замовлення і очікуваним споживання під час виконання замовлення~\cite{Sterligova2008}:
\begin{equation} \label{eq:model_fs:zs2}
Z_s=MC-EC
,
\end{equation}
\begin{description}
	\item[де] $MC$ --- максимальне споживання за час виконання замовлення, одиниць;
	\item $EC$ --- очікуване споживання за час виконання замовлення, одиниць.
\end{description}

У свою чергу максимальне споживання за час виконання замовлення розраховується по формулі~\cite{Sterligova2008}:
\begin{equation}
MC=C_d\cdot(t_d+t_{od})
,
\end{equation}
\begin{description}
	\item[де] $t_d$ --- час виконання замовлення, дні.
\end{description}

Очікуване денне споживання $C_d$ розраховується виходячи з очікуваної потреби в запасі за весь період~\cite{Sterligova2008}:
\begin{equation}
S_d=\cfrac{S}{N}
,
\end{equation}
\begin{description}
	\item[де] $S_d$ --- очікуване денне споживання, одиниць;
	\item $S$ --- обсяг потреби в запасі, одиниць;
	\item $N$ --- число робочих днів у плановому періоді.
\end{description}

Очікуване споживання за час виконання замовлення $EC$ розраховується як добуток очікуваного денного споживання на час виконання замовлення~\cite{Sterligova2008}:
\begin{equation}
EC=S_d\cdot t_d
,
\end{equation}
\begin{description}
	\item[де] $EC$ --- очікуване споживання за час виконання замовлення, одиниць.
\end{description}

Страхових запас $Z_s$ може також бути розрахований за іншими формулами, які мають статистичний, імовірнісний або емпіричній характер~\cite{Sterligova2008}.

\subsubsection{Модель з фіксованим інтервалом часу між замовленнями}
У моделі з фіксованим інтервалом часу між замовленнями  \textit{(fixed order interval model)} замовлення робляться в строго певні моменти часу, які знаходяться один від одного на рівні інтервали (рисунок~\ref{fig:model_fi:dynamic}).

\begin{figure}[H]
  \centering
\begin{tikzpicture}
  \begin{axis}[ 
    xlabel={Час},
    ylabel={Запас},
    xmin=0,xmax=10,ymin=0,ymax=1
  ] 
	\addplot
		coordinates {
			(0,0.8) [0]
			(3,0.2) [1]
			(3,0.8) [2]
			(7,0.25) [3]
			(7,0.9) [4]
			(9,0.1) [5]
			(9,0.8) [6]
			(10,0.5) [7]
		};
	\draw[<->] (axis cs:2.8,0.2) -- node[left]{\footnotesize $Q_i$} (axis cs:2.8,0.8);
   
    \draw [dashed] (axis cs:1.5,0) -- node[left]{\footnotesize Зам.} (axis cs:1.5,1);
    \draw [dashed] (axis cs:4.5,0) -- node[left]{\footnotesize Зам.} (axis cs:4.5,1);
    \draw [dashed] (axis cs:7.5,0) -- node[left]{\footnotesize Замовлення} (axis cs:7.5,1);

    \draw [densely dotted] (axis cs:0,0.4) -- node[below]{\footnotesize Пороговий запас} (axis cs:10,0.4);
    \draw [densely dotted] (axis cs:0,0.2) -- node[below]{\footnotesize Страховий запас} (axis cs:10,0.2);
  \end{axis}
\end{tikzpicture}
  \captionsetup{justification=centering}
  \caption{Динаміка запасу в моделі з фіксованим інтервалом часу між замовленнями}
  \label{fig:model_fi:dynamic}
\end{figure}

Фіксований інтервал часу між замовленнями повинен мати оптимальний розмір. 
Оптимізація рівня запасу зв'язується з оптимізацією розміру замовлення на заповнення запасу. 
Таким чином, визначати оптимальний інтервал часу між замовленнями слід на основі оптимального розміру замовлення. 
Оптимальний розмір замовлення дозволяє мінімізувати сукупні витрати на утримання та поповнення запасу, а також досягти найкращого поєднання таких факторів, як використовувана площа складських приміщень, витрати на зберігання запасу і вартість замовлення~\cite{Sterligova2008}.

Формула для розрахунку інтервалу між замовленнями~\cite{Sterligova2008}:
\begin{equation} \label{eq:model_fi:time}
t_d=\cfrac{N\cdot Q^*}{S}
,
\end{equation}
\begin{description}
	\item[де] $t_d$ --- інтервал часу між замовленнями, дні;
	\item $N$ --- число робочих днів у плановому періоді, дні;
	\item $Q^*$ --- оптимальний розмір замовлення;
	\item $S$ --- обсяг потреби в запасі, одиниць.
\end{description}

Отриманий за допомогою формули~\eqref{eq:model_fi:time} інтервал часу між замовленнями не є обов'язковим.
Він може бути скоригований на основі експертних оцінок.

Максимальний бажаний запас визначається для відстеження доцільності завантаження площ складу з точки зору критерію мінімізації сукупних логістичних витрат.

Максимальний бажаний запас, як видно з рисунку~\ref{fig:model_fi:dynamic}, може бути розрахований таким чином~\cite{Sterligova2008}:
\begin{equation} \label{eq:model_fi:mws}
MWS=EC_t+Z_s
,
\end{equation}
\begin{description}
	\item[де] $MWS$ --- максимальний бажаний запас, одиниць;
	\item $EC_t$ --- очікуване споживання за інтервал часу між замовленнями;
	\item $Z_s$ --- обсяг страхового запасу, одиниць.
\end{description}

З урахуванням формули~\eqref{eq:model_fi:mws} розмір замовлення може бути розрахований за формулами~\cite{Sterligova2008}:
\begin{equation} \label{eq:order}
Q_i=EC_t+Z_s-Z_{Ti}-Z_t
,
\end{equation}
\begin{equation} \label{eq:order2}
Q_i=MWS-Z_{Ti}+EC-Z_{ti}
,
\end{equation}
\begin{description}
	\item[де] $Q_i$ --- розмір замовлення $i$, одиниць;
	\item $EC$ --- очікуване споживання за час виконання замовлення, одиниць;
	\item $Z_t$ --- обсяг запасу у дорозі, одиниць.
	\item $Z_{ti}$ --- обсяг запасу у дорозі, не отриманого до моменту видачі замовлення $i$, одиниць;
	\item $Z_{Ti}$ --- обсяг поточного запасу при видачі замовлення $i$, одиниць;
\end{description}

Страховий запас (формули~\ref{eq:model_fs:zs1},~\ref{eq:model_fs:zs2}) дозволяє задовольняти потребу в запасі на час передбачуваної затримки постачання.
При цьому під можливою затримкою постачання мається на увазі максимальна можлива затримка.

\subsection{Цілі агентів}
\subsubsection{Постачальник}
\begin{enumerate}
	\item Зменшити вартість зберігання запасу, замовляючи ретельно розраховану кількість товарів.
	\item Доставка товарів без затримок.
\end{enumerate}

\subsubsection{Роздрібний торговець}
\begin{enumerate}
	\item Зменшити вартість зберігання запасу, замовляючи ретельно розраховану кількість товарів згідно до прогнозу попиту.
	\item Коли рівень запасу знижується до критичного, то роздрібний торговець повинен надіслати замовлення до найближчих інших точок, щоб екстрено збільшити рівень запасу.
	\item Коли рівень запасу занадто великий, то він може надіслати частину товару до інших точок.
\end{enumerate}

\subsection{Постановка задачі}
Завданням переддипломної роботи є розробка прототипу  мультиагентної системи для моделювання розподільчої логістичної системи.
Задачами є:
\begin{itemize}
	% logistics
	\item опис задач розподільчої логістичної системи;
	\item дослідження проблеми імітаціонного моделювання розподільчої логістичної системи;
	% agents
	\item опис агентів системи;
	\item проектування мультиагентної системи;
	\item реалізація прототипу мультиагентної системи;
	\item розробка плану тестування.
\end{itemize}
