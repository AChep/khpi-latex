\begin{titlepage}
	\vspace*{\fill} % center the frame vertically
	
	\begin{framed}
		\begin{center}
			МІНІСТЕРСТВО ОСВІТИ І НАУКИ УКРАЇНИ \\
			НАЦІОНАЛЬНИЙ ТЕХНІЧНИЙ УНІВЕРСИТЕТ \\
			<<ХАРКІВСЬКИЙ ПОЛІТЕХНІЧНИЙ ІНСТИТУТ>> \\
			Кафедра програмної інженерії та інформаційних технологій управління
		\end{center}
	
		\begin{center}
			\MakeUppercase{Курсова робота} \\ 
			<<\thetitle>>
		\end{center}
	
		\noindent	
		Керівник роботи: \\
		\hspace*{\parindent} проф. каф. ПІІТУ, д.т.н. \hfill Годлевський~М.~Д. \\
		Виконавець: \\
		\hspace*{\parindent} ст. групи КН-34г \hfill \theauthor
	
		\begin{center}
			Харків \the\year
		\end{center}
	\end{framed}

	\vspace*{\fill} % center the frame vertically
\end{titlepage}

{
\newcommand{\fillemptyline}{\uline{\hspace*{\fill}}}
\newcommand{\fillline}[2][]{\uline{#1\hspace*{\fill}#2\hspace*{\fill}\hphantom{#1}}}

\newcommand{\suline}[1]{\uline{\hspace{12pt}#1\hspace{12pt}}}
\newcommand{\undercaption}[1]{{\centering\footnotesize#1\\\noindent}}

\begin{titlepage}
	\begin{center}
		МІНІСТЕРСТВО ОСВІТИ І НАУКИ УКРАЇНИ \\
		НАЦІОНАЛЬНИЙ ТЕХНІЧНИЙ УНІВЕРСИТЕТ \\
		<<ХАРКІВСЬКИЙ ПОЛІТЕХНІЧНИЙ ІНСТИТУТ>> \\
		Кафедра програмної інженерії та інформаційних технологій управління
	\end{center}
	\vspace*{\fill}
	\begin{center}
		\MakeUppercase{\large\bfseries Курсовий проект} \\
		\MakeUppercase{\large\bfseries (робота)}
	\end{center}
	\noindent
	\fillline[з]{Якість та тестування програмного забезпечення} \\
	\undercaption{(назва дисципліни)}
	\fillline[на тему:]{Розробка мультиагентної системи для дослідження} \\
	\fillline{розподільчої логістичної системи}
	
	\vspace*{\fill}

	\begin{addmargin}[7cm]{0cm}
		\small
		Студента \suline{4} курсу \suline{КН-34г} групи \hspace*{\fill} \\
		спеціальності \fillline{121 Інженерія програмного забезпечення} \\ 
		\fillline{Чепурного~А.~С.} \\
		\undercaption{(прізвище та ініціали)}
		Керівник \fillline{проф. каф. ПІІТУ, д.т.н. Годлевський~М.~Д.} \\
		\undercaption{(посада, вчене звання, науковий ступінь, прізвище та ініціали)}
		Національна шкала \fillemptyline \\
		Кількість балів	\fillemptyline Оцінка ECTS \fillemptyline	
	\end{addmargin}

	\begin{flushright}
		\small
		\newcommand{\member}[1]{
			& \hspace{4cm} & & #1 \\ \cline{2-2} \cline{4-4} 
			& {\footnotesize (підпис)} & & {\footnotesize (прізвище та ініціали)}  \\
		}
		\begin{tabular}{cccc}
			Члени комісії 
			\member{Годлевський М. Д.}
			\member{Чередніченко О. Ю.}
			\member{Шматко О. В.}
		\end{tabular}
	\end{flushright}
	
	\vspace*{\fill}

	\begin{center}
		м. Харків --- \the\year~рік
	\end{center}
\end{titlepage}

\begin{titlepage}
	\begin{center}
		МІНІСТЕРСТВО ОСВІТИ І НАУКИ УКРАЇНИ \\
		НАЦІОНАЛЬНИЙ ТЕХНІЧНИЙ УНІВЕРСИТЕТ \\
		<<ХАРКІВСЬКИЙ ПОЛІТЕХНІЧНИЙ ІНСТИТУТ>> \\
		Кафедра програмної інженерії та інформаційних технологій управління
	\end{center}
	\noindent
	Студент \suline{\theauthor} \hfill Група \suline{КН-34г}

	\vspace*{\fill}

	\begin{center}
		\MakeUppercase{Завдання} \\
		на курсову роботу \\
		з курсу <<Якість та тестування програмного забезпечення>>
	\end{center}
	\noindent
	\textbf{Тема:} <<\thetitle>>
	
	\vspace*{\fill}

	\begin{addmargin}[0cm]{1.5cm} 
		\textbf{Короткий зміст роботи} \\
		\textit{а) реферативна частина} \\
		\uline{
		Опис розподільчої логістичної системи та проблем моделювання і управління такими системами. 
		Аналіз існуючих моделей логістичніх систем. 
		Огляд існуючих технологій побудови мультиагентної системи. 
		Постановка задачі дослідження.
		} \\
		\textit{б) теоретична частина} \\
		\uline{
		Опис моделей та алгоритмів для управлінн розподільчими логістичними системами.
		} \\
		\textit{в) експериментальна частина} \\ 
		\uline{
		Розробка вимог до програмної системи.
		Проектування архітектури системи.
		Обгрунтування вибору платформи розробки та інструментальних засобів.
		}
	\end{addmargin}
	
	\vspace*{\fill}

	\noindent
	Дата видачі завдання: 09.10.17 \hfill Термін захисту: 26.12.17 \\
	Керівник курсової роботи: \hfill /проф. каф. ПІІТУ Годлевский~М.~Д./
\end{titlepage}
}

\begin{titlepage}
\begin{center}
	\MakeUppercase{Відгук} \\
	на курсову роботу \\
	<<\thetitle>>
\end{center}
\end{titlepage}

\begin{titlepage}
\section*{Реферат}
Пояснювальна записка до дипломної роботи: \pageref{LastPage}~с., \totalfigures~рис., \totaltables~табл., \total{citnum}~дж., 1 додаток. \bigbreak
\textit{Ключові слова}: \MakeUppercase{мультиагентна система, розподільча логістична система, моделювання, агентне моделювання}. \bigbreak

Об’єктом дослідження є процес моделювання логістичної системи дистрибуції за допомогою мультиагентних систем.

Предметом дослідження є моделі та інструментальні засоби для розробки системи оцінки рівня сервісу логістичних систем.

Метою та завданням дослідження є оцінка рівня сервісу логістичних систем дистрибуції товарів масового вжитку.

Для досягнення поставленої мети були розглянуті основні проблеми моделювання логістичних систем, описані методи та алгоритми управління логістичними системами. Сформульовані вимоги до програмної системи, представлена модель її архітектури. На основі порівняння програмних платформ для розробки мультиагентних систем та вимог були обрані програмні технології для реалізації системи.

Була реалізована та протестована програмна система. 

Подальше використання отриманих результатів пов'язане з визначенням стійкості рівня сервісу до різноманітних надзвичайних ситуацій. Отримані результати є основою для формування організаційної структури управління логістичною системою дистрибуції.

Дана система може бути використана логістичними компаніями з метою покращення сервісу, викладачами та студентами для дослідження логістичних систем. 

\end{titlepage}

\begin{titlepage}
\Russian
\section*{Реферат}
Пояснительная записка к ДР: \pageref{LastPage}~с., \totalfigures~рис., \totaltables~табл., 20 ист., 1 приложение. \bigbreak
\textit{Ключевые слова}: \MakeUppercase{мультиагентная система, логистическая система распределения, моделирование, агентное моделирование}. \bigbreak

Объектом исследования является процесс принятия решений логистической системой распределения.

Предметом исследования является мультиагентная модель  логистической системой распределения. 

Целью и заданием исследования является разработка мультиагентной системы для исследования логистической системы распределения.

Для достижения поставленной цели были рассмотрены основные проблемы моделирования логистических систем, описаны методы и алгоритмы управления логистическими системами.
Сформулированы требования к программной системе, представлена модель ее архитектуры.
На основе сравнения программных платформ для разработки мультиагентных систем и требований были выбраны программные технологии для реализации системы. 

Была реализована и протестирована програмная система.

Данная система может быть использована логистическими компаниями с целью улучшения сервиса, преподавателями и студентами для исследования логистических систем.

\Ukrainian

\end{titlepage}

\begin{titlepage}
\section*{Abstract}
Explanatory note to the research: \pageref{LastPage}~pages, \totalfigures~fig., \totaltables~tab., 10 sources. \bigbreak 
\textit{Keywords}: \MakeUppercase{multiagent system, distribution logistic system, modeling, agent modeling}. \bigbreak 

The object of the paper is the decision-making process of distribution logistic system.

The subject of the paper is the multiagent model of  distribution logistic system.

The goal of the research is to analyze the domain, describe models and algorithms of managing logistic systems.

To archive the goal, the main problems of modeling logistic systems were reviewed, described methods and algorithms of managing logistic systems.

\end{titlepage}
