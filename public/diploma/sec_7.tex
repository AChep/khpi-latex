\section{Цивільний захист}
Цивільний захист --- це функція держави, спрямована на захист населення, території, навколишнього природного середовища та майна від надзвичайних ситуацій шляхом запобігання таких ситуацій, ліквідації їх наслідків та надання допомоги постраждалим в мирний час та в особливий період.

У даному розділі дипломного проекту (роботи) розглядається питання: <<Основні шляхи підвищення стійкості роботи об'єктів господарювання в умовах \acrshort{es}>>.

Виникнення надзвичайної ситуації в країні військового чи мирного часу дуже часто призводить до значних економічних втрат, скорочення промислової діяльності,  дефіциту продуктів харчування, сільськогосподарської продукції, погіршується стійкість господарства країни, а як наслідок --- її життєдіяльність в цілому.
Підвищення стійкості роботи об'єктів господарювання дозволить укріпити економічний стан країни, приведе до покращення непростих при \acrshort{es} умов життя населення, прискорить процес боротьби з наслідками \acrshort{es}.  

Під стійкістю господарства країни розуміють здатність забезпечити виробництво промислової продукції, необхідної для підтримки життєдіяльності
держави і успішного ведення дій по захисту її незалежності та недоторканності кордонів, роботу енергетики, транспорту, зв'язку, торгівлі, сільськогосподарське виробництво~\cite{Civ2}.

Під стійкістю роботи об'єкта господарювання розуміють його здатність за умов дії надзвичайних ситуацій виробляти продукцію в запланованих обсязі та номенклатурі, а при одержанні слабких чи середніх руйнувань відновлювати своє виробництво в мінімальні терміни.

На стійкість роботи промислового об'єкта впливають такі фактори~\cite{Civ3}:
\begin{itemize}
	\item захищеність робітників та службовців від уражальних факторів у \acrshort{es};
	\item здатність інженерно-технічного комплексу об'єкта (будівель, споруд, обладнання та комунально-енергетичних мереж) протистояти руйнівній дії уражальних факторів аварій, катастроф, стихійного лиха та сучасної зброї;
	\item надійність постачання об'єкта електроенергією, водою, паливом, комплектуючими та сировиною;
	\item підготовленість об'єкта до проведення аварійно-рятувальних та відновлюваних робіт;
	\item оперативність управління виробництвом та здійсненням заходів ЦЗ у \acrshort{es}.
\end{itemize}

Задля забезпечення стійкості об'єкта проводиться комплекс інженерно-технічних, технологічних, організаційних заходів.

До інженерно-технічних заходів належать роботи, що спрямовані на запобігання виникненню надзвичайних ситуацій, забезпечення захисту населення і територій від них та небезпеки, що може виникнути під час воєнних (бойових) дій або внаслідок таких дій, а також створення умов для забезпечення сталого функціонування суб'єктів господарювання і територій в особливий період~\cite{Civ1}.

Технологічні заходи забезпечують підвищення стійкості об'єкта спрощенням технологічного процесу виробництва кінцевої продукції та виключенням або обмеженням розвитку аварій.

Організаційні заходи передбачають розробку ефективних дій керівного складу, служб та формувань ЦЗ, спрямованих на захист виробничого персоналу, проведення рятувальних та інших невідкладних робіт та відновлення виробництва.

При проведенні цих заходів необхідно враховувати конкретні умови об'єкта народного господарства. Проте є загальні організаційні інженерно-технічні заходи, які мають проводитись на всіх об'єктах~\cite{Civ4}:
\begin{enumerate}
	\item Забезпечення захисту людей та їх життєдіяльності. Створення на об'єкті надійної системи оповіщення про загрозу нападу противника, радіоактивне забруднення, хімічне і біологічне зараження, загрозу стихійного лиха і виробничої аварії. Організація розвідки і спостереження за радіоактивним забрудненням, хімічним і біологічним зараженням; гідрометеорологічне спостереження за рівнем води, напрямком і швидкістю вітру, рухом і поширенням хмари радіоактивного забруднення. Створення фонду захисних споруд ЦО, запасів засобів індивідуального захисту і забезпечення своєчасної видачі їх населенню. Завчасна підготовка до масової санітарної обробки населення і знезаражування одягу, організація взаємодії з установами охорони здоров'я для медичного обслуговування населення у надзвичайних ситуаціях.

	      Підготовка до евакуації населення, розміщеного в зонах можливих руйнувань і катастрофічного затоплення. Завчасна підготовка місць евакуації, організація прийому евакуйованого населення на територію населених пунктів. Постачання населення продуктами харчування, питною водою, предметами першої необхідності; комунальне побутове обслуговування населення з урахуванням проведення евакуаційних заходів, забезпечення захисту продовольчих запасів.

	      Навчання населення способам захисту , надання першої допомоги, практичним діям в умовах надзвичайних ситуацій, морально-психологічна підготовка населення для виживання. Забезпечення чіткої інформації про обстановку та правила дій і поведінки населення в надзвичайних ситуаціях мирного і воєнного часу.
	\item Захист цінного й унікального устаткування. Захистити цінне і унікальне устаткування можна завдяки проведенню інженерно-технічних заходів, щоб зменшити небезпеку пошкодження і руйнування цінного й унікального устаткування, станків з програмним керуванням, шліфувальних, токарних, розточних, зубофрезерних, пресових станків, автоматичних конвеєрних ліній та іншого устаткування.

	      Варіантами такого захисту є розміщення зазначеного устаткування в заглиблених приміщеннях а також використання спеціальних захисних пристосувань, закріплення станків на фундаментах, застосування контрфорсів для підвищення стійкості проти перекидання обладнання
	\item Підвищення стійкості мереж комунального господарства. Для забезпечення стійкості роботи об'єктів повинні проводитись інженерно-технічні заходи на мережах комунального господарства з метою захисту джерел тепла із заглибленням у ґрунт комунікацій. Котельні слід розміщувати в спеціальному окремо розміщеному приміщенні.

	      Якщо об'єкт одержує тепло з міської теплоцентралі, необхідно провести заходи для забезпечення стійкості трубопроводів і розподільних пристроїв, підведених до об'єкта. Теплова мережа має будуватися за кільцевою системою з прокладанням труб у спеціальних каналах зі з'єднанням паралельних ділянок.

	      Система каналізації має будуватись окремо: одна для дощових, друга для промислових і господарських вод. На об'єкті має бути не менше двох виводів з підключенням до міських каналізаційних колекторів, а також виводи і колодязі з аварійними засувками на об'єктових колекторах з інтервалом 50 м на території, що не завалюється, для аварійного скидання неочищеної води в найближчі штучні та природні заглиблення.
	\item Забезпечення стійкості роботи паливно-енергетичного комплексу і водопостачання. Створення резерву енергетичних потужностей за рахунок автономних пересувних електростанцій, а також місцевих джерел електроенергії. Підготовка автономних електростанцій до роботи за спеціальним режимом (графіком) для забезпечення технологічних процесів виробництва, для яких неможливі тривалі перерви в електропостачанні.

	      З метою попередження аварій на електричних мережах необхідно установити автоматичну систему відключення при виникненні перенапруги. Повітряні лінії електропостачання замінити на підземно-кабельні.

	      Для підвищення стійкості забезпечення водою слід провести такі заходи. Необхідно створити основні і резервні джерела водопостачання. Як резервне джерело краще мати артезіанську свердловину, яку необхідно підключити до системи водопостачання. Крім того, воду можна брати з близько розміщеної природної водойми або спорудити штучну водойму чи резервуари з обладнанням пристроїв для збору і перекачування води.

	      Всі ділянки водопостачання повинні бути заглиблені в ґрунт з обладнанням пожежних гідрантів і пристроїв для відключення пошкоджених ділянок. Локальні мережі водопостачання окремих великих підприємств варто з'єднати із загальноміською системою водопостачання в єдине кільце.
	\item Стійкість роботи автотранспортної та іншої техніки, технологічного обладнання і механізмів. Організація своєчасного оповіщення гаража, технологічного парку, їх керівників, водіїв, механізаторів про загрозу надзвичайної ситуації. Підготовка автотранспортної техніки до проведення робіт в умовах радіоактивного забруднення, хімічного біологічного зараження і світломаскування.

	      Пристосування і використання всіх видів транспортних засобів для евакуації населення і перевезення потерпілих. Розробка заходів з метою пристосування автотранспортної, іншої техніки для виконання завдань ЦЗ.

	      Розробка пристосувань і технологічних процесів для відбору потужностей тракторів і автомобілів з метою приведення в дію електрогенераторів і технологічного обладнання, насосів для подачі води до місця споживання зі свердловин, відкритих водойм і шахтних колодязів. Підготовка всієї техніки для проведення рятувальних та інших невідкладних робіт у надзвичайних умовах мирного і воєнного часу.
	\item Забезпечення стійкого постачання об'єкта. Для забезпечення виробництва продукції необхідні електроенергія, паливо, мастила, засоби захисту рослин, міндобрива, профілактичні й лікувальні препарати ветеринарної медицини, запасні частини, сировина та інші матеріально-технічні засоби.

	      Запас резервних матеріалів необхідно розраховувати на такі строки роботи підприємства, за які можливе відновлення регулярного постачання. Передбачити, на випадок перебоїв в постачанні підприємствами-суміжниками, створення місцевих матеріалів, сировини для виготовлення комплектуючих виробів і інструментів силами свого підприємства.

	\item Забезпечення збереження й відновлення будівель і споруд. Оцінка можливих ступенів руйнування будівель і споруд господарства, населеного пункту. Визначення обсягу невідкладних ремонтних робіт, потреби в будівельних матеріалах. Розрахунок сил і засобів для проведення невідкладних ремонтних та інших робіт, а також знезаражування приміщень, виробничих ділянок і території.

	      Створення і підготовка спеціальних формувань для ремонтно-відновних, будівельних та інших робіт на об'єкті. При будівництві нових будівель і захисних споруд врахувати вимоги ЦЗ. Розробка комплексу протипожежних заходів, які виключали б можливість виникнення масових пожеж.

	\item Забезпечення надійності системи управління і зв'язку. Організація захищеного пункту управління, оснащення його засобами зв'язку, які б дали можливість швидко доводити сигнали ЦЗ до всіх виробничих підрозділів і населення у місцях проживання. Розробка документів, які регламентують чіткі дії персоналу для забезпечення сталої роботи об'єкта в надзвичайних умовах. Підготовка необхідного резерву кадрів спеціалістів, механізаторів і керівних працівників для зміни тим, які будуть мобілізовані.

	      Планування збору даних про обстановку, передачу команд і розпоряджень в умовах впливу на об'єкт уражаючих факторів. Організація використання радіозасобів, телефонного зв'язку, посильних для зв'язку з віддаленими населеними пунктами, виробничими підрозділами, а також з колонами евакуйованого населення, що перебувають у дорозі, і відповідальними особами, які супроводжують під час евакуації. Забезпечення дублювання ліній і каналів зв'язку.
\end{enumerate}

Таким чином, наведені вище шляхи підвищення стійкості роботи об'єктів господарювання в надзвичайних ситуаціях можуть знизити економічні втрати, допоможуть прискорити процес боротьби з наслідками небезпечних природних явищ і катастроф.
Загальні організаційні інженерно-технічні заходи найбільш ефективні як регулярні заходи.
