\selectlanguage{english}
\section*{Abstract}
Explanatory note to the thesis: \pageref{LastPage}~pages, \totalfigures~fig., \totaltables~tab., \total{citnum}~sources, 1 appendix. \bigbreak 
\textit{Keywords}: \MakeUppercase{multiagent system, distribution logistic system, modeling, agent modeling}. \bigbreak 

The object of the paper is the modeling process of a logistics distribution system using multi-agent approach.

The subject of the paper is models and tools for developing a system for modeling the service level of logistics systems.

The goal of the paper is to model the service level of logistics systems for the distribution of consumer goods.

To achieve this goal, the main problems of modeling logistics systems were reviewed, described methods and algorithms of managing logistic systems. The requirements to the software system were formulated, the model of its architecture were presented. Based on a comparison of multi-agent software platforms and requirements, software technologies were selected.

The software was implemented and tested.

Further use of the results is associated with determining the sustainability of the level of service in various emergencies. The results obtained are the basis for the formation of the organizational structure of the management of the logistics distribution system.

The system can be used by logistic companies to improve their service or by teachers and students to study the logistic systems.

\selectlanguage{ukrainian}