\documentclass[a4paper,14pt,oneside,final]{extarticle}
\usepackage[top=2cm, bottom=2cm, left=3cm, right=1cm]{geometry}
\usepackage{scrextend}

\usepackage[T2A,T1]{fontenc}
\usepackage[ukrainian,russian,english]{babel}
\usepackage{tempora}
\usepackage{fontspec}
\setmainfont{tempora}

% Зачем: Отключает использование изменяемых межсловных пробелов.
% Почему: Так не принято делать в текстах на русском языке.
\frenchspacing

\usepackage{indentfirst}
\setlength{\parindent}{1.25cm}
\renewcommand{\baselinestretch}{1.5}

% Header
\usepackage{fancyhdr}
\pagestyle{fancy}
\fancyhead{}
\fancyfoot{}
\fancyhead[R]{\small \selectfont \thepage}
\renewcommand{\headrulewidth}{0pt}

% Captions
\usepackage{chngcntr}
\counterwithin{figure}{section}
\counterwithin{table}{section}
\usepackage[tableposition=top]{caption}
\usepackage{subcaption}
\DeclareCaptionLabelFormat{gostfigure}{Рисунок #2}
\DeclareCaptionLabelFormat{gosttable}{Таблиця #2}
\DeclareCaptionLabelSeparator{gost}{~---~}
\captionsetup{labelsep=gost}
\captionsetup[figure]{labelformat=gostfigure}
\captionsetup[table]{labelformat=gosttable}
\renewcommand{\thesubfigure}{\asbuk{subfigure}}

% Sections
\usepackage[explicit]{titlesec}
\newcommand{\sectionbreak}{\clearpage}

\titleformat{\section}
  {\centering}{\thesection \quad}{0pt}{\MakeUppercase{#1}}
\titleformat{\subsection}[block]
  {\bfseries}{\thesubsection \quad #1}{0cm}{}

\titlespacing{\section} {0cm}{0cm}{21pt}
\titlespacing{\subsection} {\parindent}{21pt}{0cm}
\titlespacing{\subsubsection} {\parindent}{0cm}{0cm}

% Lists
\usepackage{enumitem}
\renewcommand\labelitemi{--}
\setlist[itemize]{noitemsep, topsep=0pt, wide}
\setlist[enumerate]{noitemsep, topsep=0pt, wide, label=\arabic*}
\setlist[description]{labelsep=0pt, noitemsep, topsep=0pt, leftmargin=2\parindent, labelindent=\parindent, labelwidth=\parindent, font=\normalfont}

% Toc
\usepackage{tocloft}
\tocloftpagestyle{fancy}
\renewcommand{\cfttoctitlefont}{}
\setlength{\cftbeforesecskip}{0pt}
\renewcommand{\cftsecfont}{}
\renewcommand{\cftsecpagefont}{}
\renewcommand{\cftsecleader}{\cftdotfill{\cftdotsep}}

\usepackage{float}
\usepackage{pgfplots}
\usepackage{graphicx}
\usepackage{multirow}
\usepackage{amssymb,amsfonts,amsmath,amsthm}
\usepackage{csquotes}

\usepackage{listings}
\lstset{basicstyle=\footnotesize\ttfamily,breaklines=true}
\lstset{language=Matlab}

\usepackage[
	backend=biber,
	sorting=none,
	language=auto,
	autolang=other
]{biblatex}
\DeclareFieldFormat{labelnumberwidth}{#1}


\usepackage{titling}
\usepackage{multicol}

\newcommand{\khpistudentgroup}{КН-34г}
\newcommand{\khpistudentname}{Чепурний~А.~С.}

\newcommand{\khpidepartment}{Програмна інженерія та інформаційні технології управління}
\newcommand{\khpititlewhat}{
	Лабораторна робота №\labnumber \\
	з предмету <<Моделювання систем>>
}
\newcommand{\khpititlewho}{
	Виконав: \\
	\hspace*{\parindent} ст. групи \khpistudentgroup \\
	\hspace*{\parindent} \khpistudentname \\
	Перевірила: \\
	\hspace*{\parindent} ст. в. каф. ПІІТУ \\
	\hspace*{\parindent} Єршова~С.~І. \\
	\hspace*{\parindent} ас. каф. ПІІТУ \\
	\hspace*{\parindent} Литвинова~Ю.~С. \\
}



\begin{document}
\Ukrainian

\begin{center}
\small
	\textbf{\theauthor} -- 2.КН201н.8а
\end{center}

\begin{center}
\MakeUppercase{Розробка інформаційної технології аналізу стійкості функціонування логістичної системи дистрибуції при стратегічному плануванні}
\end{center}

На сучасному етапі розвитку складні логістичні системи вимушені працювати в умовах високої невизначеності, що суттєво ускладнює управління ними. 
В процесі прийняття управлінських рішень виникає проблема прогнозування поведінки системи та зовнішнього середовища. 
Результати прогнозів необхідно постійно коригувати по ходу розвитку подій, що дозволяє пристосовуватися до змін оточення та гнучко реагувати на негативні впливи. 

У нагоді тут стає агентне моделювання, яке сягає своїм історичним корінням складних адаптивних систем і принципу побудови систем знизу вгору.
Агентне моделювання дозволяє здійснити множину прогнозів за різними сценаріями залежно від формування різноманітних ситуацій практично необмеженої складності. 

Основними елементами агентного моделювання є агенти, стосунки між ними і простір, в якому відбувається взаємодія. 
Агенти моделюються індивідуально. 
Вони можуть мати неповну інформацію, здійснювати помилки, адаптуватися до ситуації, проявляти ініціативу. 
В основу агентного моделювання закладені такі принципи, як різноманітність, взаємозв’язок і міра взаємодії. 
Тип взаємодій різних агентів може відрізнятися і носити ймовірнісний характер. 
Результатом динамічної взаємодії може бути певний рівноважний стан системи, а може бути і нова якість, яку неможливо передбачати з аналізу окремих складових системи.

\textbf{Об'єктом дослідження} є процес прийняття рішень розподільчою логістичною системою. 

\textbf{Предметом дослідження} є мультиагентна модель розподільчої логістичної системи. 

Теоретико-методологічною основою роботи є агентне моделювання, системний аналіз, а також базова теорія логістики.

\textbf{Метою і завданням дослідження} є розробка мультиагентної системи для дослідження розподільчої логістичної системи.

В ході магістерської роботи будуть розглянуті фреймворки для реалізації мультиагентних систем, а саме: JADE (Java Agent DEvelopment Framework) та SPADE (Smart Python Agent Development Environment); розглянуті моделі для моделювання динаміки логістичної системи.  

Розподільча логістична система буде представлена у вигяді агентів, для кожного формалізована ціль та методи досягнення цілі. 

На основі описаних агентів буде розроблена та протестована агентна система, описані особливості, переваги та недоліки її роботи. 

\textbf{Задачами дослідження} є:
\begin{enumerate}
	\item Опис проблеми імітаціонного моделювання розподільчої логістичної системи.
	\item Опис динаміки розподільчої логістичної системи, опис моделей моделювання.
	\item Порівняння та обрання фреймворку для реалізації мультиагентної системи.
	\item Розробка специфікацій та реалізація мультиагентної системи для моделювання розподільчої логістичної системи. 
	\item Порівняння результатів моделювання з результатами існуючих програмних систем.
\end{enumerate}

\end{document}
