\documentclass[a4paper,14pt,oneside,final]{extarticle}
\usepackage[top=2cm, bottom=2cm, left=3cm, right=1cm]{geometry}
\usepackage{scrextend}

\usepackage[T2A,T1]{fontenc}
\usepackage[ukrainian,russian,english]{babel}
\usepackage{tempora}
\usepackage{fontspec}
\setmainfont{tempora}

% Зачем: Отключает использование изменяемых межсловных пробелов.
% Почему: Так не принято делать в текстах на русском языке.
\frenchspacing

\usepackage{indentfirst}
\setlength{\parindent}{1.25cm}
\renewcommand{\baselinestretch}{1.5}

% Header
\usepackage{fancyhdr}
\pagestyle{fancy}
\fancyhead{}
\fancyfoot{}
\fancyhead[R]{\small \selectfont \thepage}
\renewcommand{\headrulewidth}{0pt}

% Captions
\usepackage{chngcntr}
\counterwithin{figure}{section}
\counterwithin{table}{section}
\usepackage[tableposition=top]{caption}
\usepackage{subcaption}
\DeclareCaptionLabelFormat{gostfigure}{Рисунок #2}
\DeclareCaptionLabelFormat{gosttable}{Таблиця #2}
\DeclareCaptionLabelSeparator{gost}{~---~}
\captionsetup{labelsep=gost}
\captionsetup[figure]{labelformat=gostfigure}
\captionsetup[table]{labelformat=gosttable}
\renewcommand{\thesubfigure}{\asbuk{subfigure}}

% Sections
\usepackage[explicit]{titlesec}
\newcommand{\sectionbreak}{\clearpage}

\titleformat{\section}
  {\centering}{\thesection \quad}{0pt}{\MakeUppercase{#1}}
\titleformat{\subsection}[block]
  {\bfseries}{\thesubsection \quad #1}{0cm}{}

\titlespacing{\section} {0cm}{0cm}{21pt}
\titlespacing{\subsection} {\parindent}{21pt}{0cm}
\titlespacing{\subsubsection} {\parindent}{0cm}{0cm}

% Lists
\usepackage{enumitem}
\renewcommand\labelitemi{--}
\setlist[itemize]{noitemsep, topsep=0pt, wide}
\setlist[enumerate]{noitemsep, topsep=0pt, wide, label=\arabic*}
\setlist[description]{labelsep=0pt, noitemsep, topsep=0pt, leftmargin=2\parindent, labelindent=\parindent, labelwidth=\parindent, font=\normalfont}

% Toc
\usepackage{tocloft}
\tocloftpagestyle{fancy}
\renewcommand{\cfttoctitlefont}{}
\setlength{\cftbeforesecskip}{0pt}
\renewcommand{\cftsecfont}{}
\renewcommand{\cftsecpagefont}{}
\renewcommand{\cftsecleader}{\cftdotfill{\cftdotsep}}

\usepackage{float}
\usepackage{pgfplots}
\usepackage{graphicx}
\usepackage{multirow}
\usepackage{amssymb,amsfonts,amsmath,amsthm}
\usepackage{csquotes}

\usepackage{listings}
\lstset{basicstyle=\footnotesize\ttfamily,breaklines=true}
\lstset{language=Matlab}

\usepackage[
	backend=biber,
	sorting=none,
	language=auto,
	autolang=other
]{biblatex}
\DeclareFieldFormat{labelnumberwidth}{#1}


\usepackage{titling}

\newcommand{\khpistudentgroup}{КН-34г}
\newcommand{\khpistudentname}{Чепурний~А.~С.}

\newcommand{\khpidepartment}{Програмна інженерія та інформаційні технології управління}
\newcommand{\khpititlewhat}{
	Лабораторна робота №\labnumber \\
	з предмету <<Моделювання систем>>
}
\newcommand{\khpititlewho}{
	Виконав: \\
	\hspace*{\parindent} ст. групи \khpistudentgroup \\
	\hspace*{\parindent} \khpistudentname \\
	Перевірила: \\
	\hspace*{\parindent} ст. в. каф. ПІІТУ \\
	\hspace*{\parindent} Єршова~С.~І. \\
	\hspace*{\parindent} ас. каф. ПІІТУ \\
	\hspace*{\parindent} Литвинова~Ю.~С. \\
}



\begin{document}
\Ukrainian

\begin{flushright}
	\textbf{\theauthor}\\
	2.КН201н.8а
\end{flushright}

\subsection*{Тема}
Розробка інформаційної технології аналізу стійкості функціонування логістичної системи дистрибуції при стратегічному плануванні.

\subsection*{Пояснення до теми}
В ході магістерської дисертації планується розглянути процес прогнозування поведінки (зокрема, аналіз стійкості функціонування) розподільчої системи з використанням агентних систем та порівняти результат моделювання та особливості розробки з іншими методами моделювання.

Будуть розглянуті фреймворки для реалізації мультиагентних систем, а саме: JADE (Java Agent DEvelopment Framework) та SPADE (Smart Python Agent Development Environment); розглянуті моделі для моделювання динаміки логістичної системи.  

Розподільча логістична система буде представлена у вигяді агентів, для кожного формалізована ціль та методи досягнення цілі. 

На основі описаних агентів буде розроблена та протестована агентна система, описані особливості, переваги та недоліки її роботи. 

\subsection*{Проблема}
Прогнозування поведінки розподільчої логістичної системи.  

\subsection*{Ціль}
Дослідження особливостей використання агентних систем для моделювання розподільчої логістичної системи. 

\subsection*{Задачі}
\begin{enumerate}
	\item Формалізація проблеми імітаціонного моделювання розподільчої логістичної системи.
	\item Опис динаміки розподільчої логістичної системи, опис моделей моделювання.
	\item Порівняння та обрання фреймворку для реалізації мультиагентної системи.
	\item Розробка специфікацій та реалізація мультиагентної системи для моделювання розподільчої логістичної системи. 
	\item Порівняння результатів моделювання з результатами існуючих моделей.
\end{enumerate}

\subsection*{Мотивація вибору теми}
% NaUKMAkn_2013_151_16.pdf
% http://www.economy.in.ua/pdf/1_2016/9.pdf
На сучасному етапі розвитку складні логістичні системи вимушені працювати в умовах високої невизначеності, що суттєво ускладнює управління ними. 
В процесі прийняття управлінських рішень виникає проблема прогнозування поведінки системи та зовнішнього середовища. 
Результати прогнозів необхідно постійно коригувати по ходу розвитку подій, що дозволяє пристосовуватися до змін оточення та гнучко реагувати на негативні впливи. 

У нагоді тут стає агентне моделювання, яке сягає своїм історичним корінням складних адаптивних систем і принципу побудови систем знизу вгору.
Агентне моделювання дозволяє здійснити множину прогнозів за різними сценаріями залежно від формування різноманітних ситуацій практично необмеженої складності. 

Основними елементами агентного моделювання є агенти, стосунки між ними і простір, в якому відбувається взаємодія. 
Агенти моделюються індивідуально. 
Вони можуть мати неповну інформацію, здійснювати помилки, адаптуватися до ситуації, проявляти ініціативу. 
В основу агентного моделювання закладені такі принципи, як різноманітність, взаємозв’язок і міра взаємодії. 
Тип взаємодій різних агентів може відрізнятися і носити ймовірнісний характер. 
Результатом динамічної взаємодії може бути певний рівноважний стан системи, а може бути і нова якість, яку неможливо передбачати з аналізу окремих складових системи.

Об'єктом дослідження є процес прийняття рішень розподільчою логістичною системою. 

Предметом дослідження є мультиагентна модель розподільчої логістичної системи. 

Теоретико-методологічною основою роботи є агентне моделювання, системний аналіз, а також базова теорія логістики.

Метою і завданням дослідження є розробка мультиагентної системи для дослідження розподільчої логістичної системи.
Для досягнення поставленої мети в переддипломній роботі були сформульовані та вирішені наступні задачі:
\begin{itemize}
	% logistics
	\item опис динаміки розподільчої логістичної системи;
	\item дослідження проблеми імітаціонного моделювання розподільчої логістичної системи, опис різних моделей;
	% agents
	\item порівняння фреймворкій для реалізації мультиагентних систем, обрання та обґрунтування вибору;
	\item розробка специфікації мультиагентної системи;
	\item реалізація мультиагентної системи;
	\item тестування мультиагентної систему;
	\item викладення пропозиції щодо перспектив розвитку та удосконалення розробленої мультиагентної системи.
\end{itemize}



%Для сучасних IT компаній найважливішим фактором успіху та виживання в умовах високої конкуренції є успішне виконання проектів згідно з вимогами замовника або кінцевих користувачів. Отже, підвищення ефективності виконання проектів для IT компаній являється найважливішим та найбільш актуальним питанням.

%Існує багато факторів успішної проектної діяльності, а саме: формування команди виконавців, розподіл ролей в команді, розподіл завдань між виконавцями, забезпечення комунікації між членами виконавчої групи, тощо. Одним з найефективнішим методом підвищення ефективності виконання проектів являється впровадження системи управління проектами. Даний клас систем дозволяє керувати інформацією про проектну діяльність за одним чи декількома процесами та автоматизувати виконання задач, пов’язаних із обраними для автоматизації процесами.

%Процеси формування команди та розподілу в ній ролей являються фундаментальним фактором успішності проекту, а отже й одним з найсуттєвіших факторів його успішного виконання. Для автоматизації цих процесів та підвищення ефективності їх виконання потребується використання систем управління командами, які входять до кластеру систем управління проектами. Такі системи зберігають інформацію про команди, її членів та дозволяють керувати процесами створення нових команд, розподілу ролей в командах, розпуску команд, тощо.

%Формування команди завжди здійснюється з урахуванням задач, з яких складається виконання проекту, та співробітників, які можуть бути задіяні у проекті. Для формування ефективної команди, здатної якісно виконати проект, треба провести оцінку навичок потенційних виконавців та на основі проведеної оцінки сформувати оптимальний склад команди.

%Важливим етапом проектування систем, які зберігають багато різнорідних даних та оперують різноманітною інформацію, до яких відносяться й системи управління проектами, є побудова моделі представлення знань. Побудова такої моделі допомагає формалізувати усі дані, зв’язки між ними, їх об’єднання в інформаційні блоки та приріст інформації (отримання нових даних на основі наявних). Виконання даного етапу дозволяє коректно спроектувати модель бази даних та запити до неї.

%Існує декілька методів представлення знань, кожен з яких дозволяє по різному відобразити дані, зв’язки між ними та інформаційними блоками та способи отримання знань. Різні методи показують модель представлення знань з різних боків та підходять до вирішення різних задач. Вибір найбільш оптимального методу представлення знань являються суттєвим для побудови коректної для даної програмної системи моделі представлення знань.

\end{document}
