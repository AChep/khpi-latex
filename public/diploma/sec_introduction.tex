\section*{Вступ}
\addcontentsline{toc}{section}{Вступ}

%Керуючись життєвим досвідом і науковими знаннями, людина будує моделі --- від паперових корабликів до картини світу. 
%Чим вони багатші і чим точніше ми можемо ними оперувати, тим краще наша свідомість --- наша <<найважливіша модель>>, відповідає реальності і знаходить способи її зміни.

% NaUKMAkn_2013_151_16.pdf
% http://www.economy.in.ua/pdf/1_2016/9.pdf
На сучасному етапі розвитку складні логістичні системи вимушені працювати в умовах високої невизначеності, що суттєво ускладнює управління ними. 
В процесі прийняття управлінських рішень виникає проблема прогнозування поведінки системи та зовнішнього середовища. 
Результати прогнозів необхідно постійно коригувати по ходу розвитку подій, що дозволяє пристосовуватися до змін оточення та гнучко реагувати на негативні впливи. 

У нагоді тут стає агентне моделювання, яке сягає своїм історичним корінням складних адаптивних систем і принципу побудови систем знизу вгору.
Мультиагентний підхід до має наступні переваги перед традиційними аналітичними методами:
\begin{enumerate}
	\item Можливість додавання нового функціоналу (агентів) без
	необхідності модифікувати решту агентів, яка дозволяє
	застосовувати ітераційні методи управління проектом моделювання
	та значно скорочує час побудови системи.
	\item Легкість врахування випадкових чинників, що особливо важливо
	при моделюванні бізнесу в кризових умовах.
	\item Використання однієї моделі для аналізу функціонування
	підприємства в різних умовах ринку та незалежність агентів моделі
	один від одного.
	\item Простота побудови різноманітних імітаційних експериментів та
	перевірки їх результатів.
	\item Використання агентних імітаційних моделей знижує витрати на
	відділ аналітики підприємства, адже значно прискорює процес
	пошуку оптимальних стратегій бізнесу.
\end{enumerate}

Основними елементами агентного моделювання є агенти, стосунки між ними і простір, в якому відбувається взаємодія. 
Агенти моделюються індивідуально. 
Вони можуть мати неповну інформацію, здійснювати помилки, адаптуватися до ситуації, проявляти ініціативу. 
В основу агентного моделювання закладені такі принципи, як різноманітність, взаємозв’язок і міра взаємодії. 
Тип взаємодій різних агентів може відрізнятися і носити ймовірнісний характер. 
Результатом динамічної взаємодії може бути певний рівноважний стан системи, а може бути і нова якість, яку неможливо передбачати з аналізу окремих складових системи.

Об'єктом дослідження є процес агентного моделювання логістичної системи. 

Предметом дослідження є моделі та інструментальні засоби розробки системи для оцінки рівня сервісу логістичних систем дистрибуції.

Теоретико-методологічною основою роботи є агентне моделювання, системний аналіз, а також базова теорія логістики.

Метою дослідження є розробка та дослідження моделей та програмна реалізація інформаційної системи для визначення рівня сервісу логістичної системи.
Для досягнення поставленої мети в дипломній роботі були сформульовані та вирішені наступні задачі:
\begin{itemize}
	\item провести аналіз предметної області;
	\item провести аналіз математичного забезпечення задачі;
	\item розглянути основні принципи і допущення конфігурації логістичний мережі дистрибуції;
	\item описати агентів програмної системи;
	\item описати вимоги до програмної системи;
	\item на основі аналізу вимог запропонувати варіант цільової архітектури;
	\item на основі результатів аналізу затвердити кінцевий варіант цільової архітектури для реалізації у програмній компоненті моделювання мережевого сервісу логістичної системи дистрибуції;
	\item розробити програмну компоненту на базі прийнятого архітектурного рішення;
	\item провести тестування розробленої програмної системи;
	\item провести експеримент та проаналізувати його результати.
\end{itemize}
