\selectlanguage{russian}
\selectlanguage{russian}
\section*{Реферат}
Пояснительная записка к дипломной работе: \pageref{LastPage}~с., \totalfigures~рис., \totaltables~табл., \total{citnum}~ист., 1 приложение. \bigbreak
\textit{Ключевые слова}: \MakeUppercase{мультиагентная система, логистическая система распределения, моделирование, агентное моделирование}. \bigbreak

Объектом исследования является процесс моделирования логистической системы дистрибуции с помощью мультиагентных систем.

Предметом исследования являются модели и инструментальные средства для разработки системы оценки уровня сервиса логистических систем.

Целью и задачей исследования является оценка уровня сервиса логистических систем дистрибуции товаров массового потребления.

Для достижения поставленной цели были рассмотрены основные проблемы моделирования логистических систем, описаны методы и алгоритмы управления логистическими системами. Сформулированы требования к программной системе, представлена модель ее архитектуры. На основе сравнения программных платформ для разработки мультиагентных систем и требований были выбраны программные технологии для реализации системы.

Была реализована и протестирована программная система, проведен анализ результатов моделирования.

Дальнейшее использование полученных результатов связано с определением устойчивости уровня сервиса в различных чрезвычайных ситуаций. Полученные результаты являются основой для формирования организационной структуры управления логистической системой дистрибуции.

Данная система может быть использована логистическими компаниями с целью улучшения сервиса, преподавателями и студентами для исследования логистических систем.

\selectlanguage{ukrainian}
