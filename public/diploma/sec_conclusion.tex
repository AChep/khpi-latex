\section*{Висновки}
\addcontentsline{toc}{section}{Висновки}
Сучасні інструменти імітаційного моделювання дозволяють ефективно застосовувати його
не тільки в наукових дослідженнях, а й як засоби для побудови систем підтримки прийняття рішень у бізнесі. 

Агентне моделювання дозволяє змоделювати систему максимально наближену до реальності, зробити значний крок у розумінні та управлінні сукупністю складних соціальних процесів.

Основним завданням даної роботи була розробка прототипу мультиагентної системи для дослідження розподільчої логістичної системи.

В ході написання роботи дану задачу було розкрито повною мірою на прикладі проектування та реалізації прототипу агентної моделі логістичної системи.

Перед реалізацією практичної частини були описані методи та алгоритми для управління логістичними системами. 

В процесі дослідження були сформульовані вимоги до програмної системи та спроектована її архітектура. 
Обґрунтований вибір інструментальних засобів розробки.
