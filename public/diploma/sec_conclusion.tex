\section*{Висновки}
\addcontentsline{toc}{section}{Висновки}
Сучасні інструменти імітаційного моделювання дозволяють ефективно застосовувати його
не тільки в наукових дослідженнях, а й як засоби для побудови систем підтримки прийняття рішень у бізнесі. 

Агентне моделювання дозволяє змоделювати систему максимально наближену до реальності, зробити значний крок у розумінні та управлінні сукупністю складних процесів.

Основним завданням даної роботи була розробка мультиагентної системи для дослідження розподільчої логістичної системи.

В ході написання роботи дану задачу було розкрито повною мірою на прикладі проектування та реалізації програмної системи для моделювання та аналізу розподільчих логістичних систем.

Перед реалізацією практичної частини були описані методи та алгоритми для управління логістичними системами. 

В процесі дослідження були сформульовані вимоги до програмної системи, та була обрана і спроектована її архітектура. 
Обґрунтований вибір інструментальних засобів розробки.
