\section*{Висновки}
\addcontentsline{toc}{section}{Висновки}
Основним завданням даної роботи було дослідження використання мультиагентних систем в задачі моделювання логістичних систем.

В ході розробки дипломної роботи дану задачу було розкрито повною мірою на прикладі проектування та реалізації агентних моделей логістичних систем та проведення експериментів над ними. 

Перед реалізацією практичної частини було проаналізовано різні парадигми реалізації мультиагентних систем та обґрунтовано вибір середовища моделювання.

Було проаналізовано поточну реалізацію системи конфігурування логістичної мережі дистриб’юції товарів масового використання, розглянуто архітектуру розробленої програмної компоненти та математичні моделі, що лягли в її основу, запропоновано  варіант еталонної архітектури для реалізації у програмній компоненті. Відповідно до архітектури було розроблено програмну компоненту конфігурування логістичної системи логістичної системи дистрибуції товарів масового вжитку.

Основним завданням даної роботи була розробка мультиагентної системи для дослідження розподільчої логістичної системи.
Для досягнення поставленої мети роботи виконано наступні завдання:
\begin{enumerate}
    \item Здійснено огляд проблем моделювання та управління розподільчими логістичними системами. Описано та порівняно моделі логістичних систем, в результаті чого було обрано агентне моделювання як найбільш гнучкий та простий метод.
    \item Надано порівняльній аналіз архитектурніх рішень для створення мультиагентних систем.
    \item Описані основні принципи і допущення конфігурації логістичний мережі дистрибуції.
    \item Проведена декомпозіция логістичної системи на агентів; сформульовані цілі кожного з агентів.
	\item Сформовані вимоги до програмної системи.
	\item Запропоновано архітектурне рішення для системи.
	\item Розроблено програмна компонента на базі прийнятого архітектурного рішення.
	\item Проведено тестування розробленої програмної системи.
	\item Проведено експеримент та проаналізовані його результати.
\end{enumerate}

Подальше використання отриманих результатів пов'язане з визначенням стійкості рівня сервісу до різноманітних надзвичайних ситуацій. Отримані результати є основою для формування організаційної структури управління логістичною системою дистрибуції.
