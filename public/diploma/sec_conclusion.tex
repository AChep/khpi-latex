\section*{Висновки}
\addcontentsline{toc}{section}{Висновки}
Основним завданням даної курсової роботи була розробка  мультиагентної системи для дослідження розподільчої логістичної системи.

В ході розробки роботи дану задачу було розкрито повною мірою на прикладі проектування та реалізації агентної моделі логістичної системи та проведення оптимізаційних експериментів над ними. 
Було встановлено, що мультиагентний підхід перевершує традиційні методи в ефективності, адже дозволяє врахувати значно більшу кількість факторів та спрощує проведення оптимізаційних експериментів.

Перед реалізацією практичної частини було проаналізовано різні парадигми реалізації мультиагентних систем та обґрунтовано вибір програмної платформи \acrshort{jade}.
Описані методи та алгоритми для управління логістичними системами. 

В процесі дослідження були сформульовані вимоги до програмної системи та спроектована її архітектура. 
Обґрунтований вибір інструментальних засобів розробки.
