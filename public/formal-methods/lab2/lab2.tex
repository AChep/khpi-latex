\usepackage{tikz}

\counterwithout{figure}{section}
\counterwithout{table}{section}
\counterwithout{equation}{section}

\titleformat{\subsection}[block]
  {\bfseries\filcenter}{#1}{0cm}{}
\titlespacing{\subsection}{0cm}{21pt}{21pt}

\DeclareCaptionLabelFormat{gosttable}{Таблица #2}

\usepackage{float}
\usepackage{pgfplots}
\usepackage{graphicx}
\usepackage{multirow}
\usepackage{amssymb,amsfonts,amsmath,amsthm}

\usepackage{listings}
\lstset{basicstyle=\footnotesize\ttfamily,breaklines=true}
\lstset{language=Matlab}

\lstdefinelanguage{Python}{
  keywords={and, break, class, continue, def, yield, del, elif, else, except, exec, finally, for, from, global, if, import, in, lambda, not, or, pass, print, raise, return, try, while, assert, with},
  keywordstyle=\color{NavyBlue}\bfseries,
  ndkeywords={True, False},
  ndkeywordstyle=\color{BurntOrange}\bfseries,
  emph={as},
  emphstyle={\color{OrangeRed}},
  identifierstyle=\color{black},
  sensitive=true,
  commentstyle=\color{gray}\ttfamily,
  comment=[l]{\#},
  morecomment=[s]{/*}{*/},
  stringstyle=\color{ForestGreen}\ttfamily,
  morestring=[b]',
  morestring=[s]{"""*}{*"""},
}


\newcommand{\labnumber}{2} % second lab
\usepackage{tikz}

\counterwithout{figure}{section}
\counterwithout{table}{section}
\counterwithout{equation}{section}

\titleformat{\subsection}[block]
  {\bfseries\filcenter}{#1}{0cm}{}
\titlespacing{\subsection}{0cm}{21pt}{21pt}

\DeclareCaptionLabelFormat{gosttable}{Таблица #2}

\newcommand{\khpistudentgroup}{2.КН201н.8а}
\newcommand{\khpistudentname}{Чепурний~А.~С.}

\newcommand{\khpidepartment}{Програмна інженерія та інформаційні технології управління}
\newcommand{\khpititlewhat}{
	Розрахунково-графічне завдання \\
	з предмету <<Фреймворки та платформи>>
}
\newcommand{\khpititlewho}{
	Виконав: \\
	\hspace*{\parindent} ст. групи \khpistudentgroup \\
	\hspace*{\parindent} \khpistudentname \\
	Перевірила: \\
	\hspace*{\parindent} к. т. н., вик. каф. ПІІТУ \\
	\hspace*{\parindent} Добряк~В.~С. \\
}


\graphicspath{{figures/}}

\begin{document}
\Ukrainian

\begin{titlepage}

\begin{center}
	МІНІСТЕРСТВО ОСВІТИ І НАУКИ УКРАЇНИ \\
	НАЦІОНАЛЬНИЙ ТЕХНІЧНИЙ УНІВЕРСИТЕТ \\
	«ХАРКІВСЬКИЙ ПОЛІТЕХНІЧНИЙ ІНСТИТУТ» \\
	Кафедра <<\khpidepartment>> \\
\end{center}

\vspace{6cm}

\begin{center}
	\khpititlewhat
\end{center}

\vspace{3cm}

\begin{addmargin}[10cm]{0cm}
	\khpititlewho
\end{addmargin}

\vspace{\fill}

\begin{center}
	Харків \the\year
\end{center}

\end{titlepage}

\addtocounter{page}{1}

\section*{Побудова та аналіз однофакторних регресійних моделей}
\subsubsection*{Мета}
Навчитися будувати адекватні регресійні моделі.
\subsubsection*{Завдання для виконання}
\begin{enumerate}
    \item Здійснити концептуальну постановку задачі за специфікацією регресійних моделей, які відображають залежність обсягу надання послуг від чисельності персоналу (парні варіанти) і розміру виробничої площі перукарні (непарні варіанти).
    \item Побудувати адекватну регресійну модель. 
    \item Перевірити модель на адекватність:
    \begin{itemize}
        \item перевірити величину стандартної помилки оцінки; 
        \item оцінити стійкість параметрів регресійної моделі: оцінити стійкість параметру $b_0$ регресійної моделі за Ст'юдентом; оцінити стійкість параметру $b_1$ регресійної моделі за Фішером; та оцінити стійкість коефіцієнти кореляції за Ст'юдентом;
 d            \item проаналізувати залишки моделі.
    \end{itemize}
    \item Визначити інтервали довіри для параметрів регресійної моделі.
    \item Розрахувати прогнозні значення, знайти інтервали довіри.
    \item Підготувати аналітичну довідку для прийняття рішення керівництвом ЧП <<Поступ>>.
\end{enumerate}

\subsection*{Хід роботи}
Предметною областю є управління медичними страховими договорами приєднання.

\begin{figure}[H]
    \centering
        \includegraphics{idef_classification}
    \caption{Класифікація договорів страхування}
    \label{fig:idef_classification}
\end{figure}

\subsection*{Висновки}
У ході виконання лабораторної роботи було виконано аналіз предметної області, що пов'язана із управлінням страховими договорами. Була розроблена онтологія предметної області у вигляді діаграми взаємозв'язку, що відображає зв'язки між сутностями предметної області та діаграми станів. Також була розроблена фізична модель даних системи управління страховими договорами.

\end{document}
