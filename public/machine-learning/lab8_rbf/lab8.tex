\usepackage{tikz}

\counterwithout{figure}{section}
\counterwithout{table}{section}
\counterwithout{equation}{section}

\titleformat{\subsection}[block]
  {\bfseries\filcenter}{#1}{0cm}{}
\titlespacing{\subsection}{0cm}{21pt}{21pt}

\DeclareCaptionLabelFormat{gosttable}{Таблица #2}

\usepackage{float}
\usepackage{pgfplots}
\usepackage{graphicx}
\usepackage{multirow}
\usepackage{amssymb,amsfonts,amsmath,amsthm}

\usepackage{listings}
\lstset{basicstyle=\footnotesize\ttfamily,breaklines=true}
\lstset{language=Matlab}

\lstdefinelanguage{Python}{
  keywords={and, break, class, continue, def, yield, del, elif, else, except, exec, finally, for, from, global, if, import, in, lambda, not, or, pass, print, raise, return, try, while, assert, with},
  keywordstyle=\color{NavyBlue}\bfseries,
  ndkeywords={True, False},
  ndkeywordstyle=\color{BurntOrange}\bfseries,
  emph={as},
  emphstyle={\color{OrangeRed}},
  identifierstyle=\color{black},
  sensitive=true,
  commentstyle=\color{gray}\ttfamily,
  comment=[l]{\#},
  morecomment=[s]{/*}{*/},
  stringstyle=\color{ForestGreen}\ttfamily,
  morestring=[b]',
  morestring=[s]{"""*}{*"""},
}


\newcommand{\labnumber}{8}
\usepackage{tikz}

\counterwithout{figure}{section}
\counterwithout{table}{section}
\counterwithout{equation}{section}

\titleformat{\subsection}[block]
  {\bfseries\filcenter}{#1}{0cm}{}
\titlespacing{\subsection}{0cm}{21pt}{21pt}

\DeclareCaptionLabelFormat{gosttable}{Таблица #2}

\newcommand{\khpistudentgroup}{2.КН201н.8а}
\newcommand{\khpistudentname}{Чепурний~А.~С.}

\newcommand{\khpidepartment}{Програмна інженерія та інформаційні технології управління}
\newcommand{\khpititlewhat}{
	Розрахунково-графічне завдання \\
	з предмету <<Фреймворки та платформи>>
}
\newcommand{\khpititlewho}{
	Виконав: \\
	\hspace*{\parindent} ст. групи \khpistudentgroup \\
	\hspace*{\parindent} \khpistudentname \\
	Перевірила: \\
	\hspace*{\parindent} к. т. н., вик. каф. ПІІТУ \\
	\hspace*{\parindent} Добряк~В.~С. \\
}


\graphicspath{{figures/}}

\begin{document}
\Russian

\begin{titlepage}

\begin{center}
	МІНІСТЕРСТВО ОСВІТИ І НАУКИ УКРАЇНИ \\
	НАЦІОНАЛЬНИЙ ТЕХНІЧНИЙ УНІВЕРСИТЕТ \\
	«ХАРКІВСЬКИЙ ПОЛІТЕХНІЧНИЙ ІНСТИТУТ» \\
	Кафедра <<\khpidepartment>> \\
\end{center}

\vspace{6cm}

\begin{center}
	\khpititlewhat
\end{center}

\vspace{3cm}

\begin{addmargin}[10cm]{0cm}
	\khpititlewho
\end{addmargin}

\vspace{\fill}

\begin{center}
	Харків \the\year
\end{center}

\end{titlepage}

\addtocounter{page}{1}

\section*{Радиально-базисная нейронная сеть}
\subsubsection*{Цель работы}
Изучение основ RBF.
\subsubsection*{Постановка задачи}
Реализовать решения задачи аппроксимации данных использованием искусственной нейронной сети на основе RBF. 

\subsection*{Ход выполнения работы}
Исходный код класса \texttt{RBFN}:  
\lstinputlisting{code/RBFN.py} 

Исходный код программы для аппроксимации синусоиды с использованием класса \texttt{RBFN}:  
\lstinputlisting{code/main.py} 

Результат выполнения данной программы представлен на рисунке~\ref{fig:main}.

\begin{figure}[H]
    \centering
        \includegraphics[width=\textwidth]{main}
    \caption{Аппроксимация синусоиды}
    \label{fig:main}
\end{figure}

\subsection*{Выводы}
В результате выполнения данной лабораторной работы были получены практические навыки использования радиально-базисных нейронных сетей для аппроксимации данных.

\end{document}
