\usepackage{tikz}

\counterwithout{figure}{section}
\counterwithout{table}{section}
\counterwithout{equation}{section}

\titleformat{\subsection}[block]
  {\bfseries\filcenter}{#1}{0cm}{}
\titlespacing{\subsection}{0cm}{21pt}{21pt}

\DeclareCaptionLabelFormat{gosttable}{Таблица #2}

\usepackage{float}
\usepackage{pgfplots}
\usepackage{graphicx}
\usepackage{multirow}
\usepackage{amssymb,amsfonts,amsmath,amsthm}

\usepackage{listings}
\lstset{basicstyle=\footnotesize\ttfamily,breaklines=true}
\lstset{language=Matlab}

\lstdefinelanguage{Python}{
  keywords={and, break, class, continue, def, yield, del, elif, else, except, exec, finally, for, from, global, if, import, in, lambda, not, or, pass, print, raise, return, try, while, assert, with},
  keywordstyle=\color{NavyBlue}\bfseries,
  ndkeywords={True, False},
  ndkeywordstyle=\color{BurntOrange}\bfseries,
  emph={as},
  emphstyle={\color{OrangeRed}},
  identifierstyle=\color{black},
  sensitive=true,
  commentstyle=\color{gray}\ttfamily,
  comment=[l]{\#},
  morecomment=[s]{/*}{*/},
  stringstyle=\color{ForestGreen}\ttfamily,
  morestring=[b]',
  morestring=[s]{"""*}{*"""},
}


\newcommand{\labnumber}{3} % third lab
\usepackage{tikz}

\counterwithout{figure}{section}
\counterwithout{table}{section}
\counterwithout{equation}{section}

\titleformat{\subsection}[block]
  {\bfseries\filcenter}{#1}{0cm}{}
\titlespacing{\subsection}{0cm}{21pt}{21pt}

\DeclareCaptionLabelFormat{gosttable}{Таблица #2}

\newcommand{\khpistudentgroup}{2.КН201н.8а}
\newcommand{\khpistudentname}{Чепурний~А.~С.}

\newcommand{\khpidepartment}{Програмна інженерія та інформаційні технології управління}
\newcommand{\khpititlewhat}{
	Розрахунково-графічне завдання \\
	з предмету <<Фреймворки та платформи>>
}
\newcommand{\khpititlewho}{
	Виконав: \\
	\hspace*{\parindent} ст. групи \khpistudentgroup \\
	\hspace*{\parindent} \khpistudentname \\
	Перевірила: \\
	\hspace*{\parindent} к. т. н., вик. каф. ПІІТУ \\
	\hspace*{\parindent} Добряк~В.~С. \\
}


\graphicspath{{figures/}}

\begin{document}
\Russian

\begin{titlepage}

\begin{center}
	МІНІСТЕРСТВО ОСВІТИ І НАУКИ УКРАЇНИ \\
	НАЦІОНАЛЬНИЙ ТЕХНІЧНИЙ УНІВЕРСИТЕТ \\
	«ХАРКІВСЬКИЙ ПОЛІТЕХНІЧНИЙ ІНСТИТУТ» \\
	Кафедра <<\khpidepartment>> \\
\end{center}

\vspace{6cm}

\begin{center}
	\khpititlewhat
\end{center}

\vspace{3cm}

\begin{addmargin}[10cm]{0cm}
	\khpititlewho
\end{addmargin}

\vspace{\fill}

\begin{center}
	Харків \the\year
\end{center}

\end{titlepage}

\addtocounter{page}{1}

\section*{Кластеризация}
\subsubsection*{Цель работы}
Получить практические навыки кластеризации с учителем и без учителя
\subsubsection*{Постановка задачи}
Сделать кластеризацию по данным Титаника и соотнести результаты с признаком <<Выживаемость>>.

\subsection*{Подготовка данных}
Из исходных данных были удалены колонки \texttt{row.names}, \texttt{home.dest}, \texttt{name}, \texttt{room}, \texttt{boat}, \texttt{ticket} так как они, субъективно, не могут влиять на признак выживаемости и будут мешать кластеризации.

\subsection*{Кластеризация данных с учителем}
Исходный код кластеризации данных с учителем:  
\lstinputlisting{code/main_1.py} 

Результатом выполнения данной программы есть точность предсказывания признака выживаемости в зависимости от факторов:

\begin{table}[H]
  \begin{tabular}{l|l}
    Класс билета & 0.6173913043478261 \\
	Выживаемость & 1.0 \\
	Возраст пассажира & 0.6434782608695652 \\
	Пол пассажира: & 0.7956521739130434 \\\hline
	\textbf{По всем факторам,} \\ \textbf{кроме выживаемости} & \textbf{0.7739130434782608} \\
  \end{tabular}
  \label{tab:main_1_results}
\end{table}

Ключевым фактором выживаемости является пол пассажира.

\subsection*{Кластеризация данных без учителя}
Исходный код кластеризации данных без учителя:  
\lstinputlisting{code/main_2.py} 

Результатом выполнения данной программы есть точность предсказывания признака выживаемости в зависимости от факторов:

\begin{table}[H]
  \begin{tabular}{l|l}
    Класс билета & 0.6066350710900474 \\
	Выживаемость & 1.0 \\
	Возраст пассажира & 0.5102685624012638 \\
	Пол пассажира: & 0.8009478672985783 \\\hline
	\textbf{По всем факторам,} \\ \textbf{кроме выживаемости} & \textbf{0.6382306477093207} \\
  \end{tabular}
  \label{tab:main_2_results}
\end{table}

\subsection*{Выводы}
В результате выполнения данной лабораторной работы были получены практические навыки кластеризации с использовании языка Python.

Данные о выживаемости пассажиров на Титанике являются неполными, что мешает получить качественную модель для кластеризации.

Результаты, полученные методами кластеризации с и без учителя показывают похожие результаты и являются логичными.

\end{document}
