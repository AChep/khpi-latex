\usepackage{tikz}

\counterwithout{figure}{section}
\counterwithout{table}{section}
\counterwithout{equation}{section}

\titleformat{\subsection}[block]
  {\bfseries\filcenter}{#1}{0cm}{}
\titlespacing{\subsection}{0cm}{21pt}{21pt}

\DeclareCaptionLabelFormat{gosttable}{Таблица #2}

\usepackage{float}
\usepackage{pgfplots}
\usepackage{graphicx}
\usepackage{multirow}
\usepackage{amssymb,amsfonts,amsmath,amsthm}

\usepackage{listings}
\lstset{basicstyle=\footnotesize\ttfamily,breaklines=true}
\lstset{language=Matlab}

\lstdefinelanguage{Python}{
  keywords={and, break, class, continue, def, yield, del, elif, else, except, exec, finally, for, from, global, if, import, in, lambda, not, or, pass, print, raise, return, try, while, assert, with},
  keywordstyle=\color{NavyBlue}\bfseries,
  ndkeywords={True, False},
  ndkeywordstyle=\color{BurntOrange}\bfseries,
  emph={as},
  emphstyle={\color{OrangeRed}},
  identifierstyle=\color{black},
  sensitive=true,
  commentstyle=\color{gray}\ttfamily,
  comment=[l]{\#},
  morecomment=[s]{/*}{*/},
  stringstyle=\color{ForestGreen}\ttfamily,
  morestring=[b]',
  morestring=[s]{"""*}{*"""},
}


\newcommand{\labnumber}{7} % third lab
\usepackage{tikz}

\counterwithout{figure}{section}
\counterwithout{table}{section}
\counterwithout{equation}{section}

\titleformat{\subsection}[block]
  {\bfseries\filcenter}{#1}{0cm}{}
\titlespacing{\subsection}{0cm}{21pt}{21pt}

\DeclareCaptionLabelFormat{gosttable}{Таблица #2}

\newcommand{\khpistudentgroup}{2.КН201н.8а}
\newcommand{\khpistudentname}{Чепурний~А.~С.}

\newcommand{\khpidepartment}{Програмна інженерія та інформаційні технології управління}
\newcommand{\khpititlewhat}{
	Розрахунково-графічне завдання \\
	з предмету <<Фреймворки та платформи>>
}
\newcommand{\khpititlewho}{
	Виконав: \\
	\hspace*{\parindent} ст. групи \khpistudentgroup \\
	\hspace*{\parindent} \khpistudentname \\
	Перевірила: \\
	\hspace*{\parindent} к. т. н., вик. каф. ПІІТУ \\
	\hspace*{\parindent} Добряк~В.~С. \\
}


\graphicspath{{figures/}}

\begin{document}
\Russian

\begin{titlepage}

\begin{center}
	МІНІСТЕРСТВО ОСВІТИ І НАУКИ УКРАЇНИ \\
	НАЦІОНАЛЬНИЙ ТЕХНІЧНИЙ УНІВЕРСИТЕТ \\
	«ХАРКІВСЬКИЙ ПОЛІТЕХНІЧНИЙ ІНСТИТУТ» \\
	Кафедра <<\khpidepartment>> \\
\end{center}

\vspace{6cm}

\begin{center}
	\khpititlewhat
\end{center}

\vspace{3cm}

\begin{addmargin}[10cm]{0cm}
	\khpititlewho
\end{addmargin}

\vspace{\fill}

\begin{center}
	Харків \the\year
\end{center}

\end{titlepage}

\addtocounter{page}{1}

\section*{Основы визуализации и анализа данных}
\subsubsection*{Цель работы}
Научиться использовать возможности matplotlib и sklearn.

\subsection*{Подготовка данных}
Из исходных данных были удалены колонки \texttt{row.names}, \texttt{home.dest}, \texttt{name}, \texttt{room}, \texttt{boat}, \texttt{ticket} так как они, субъективно, не могут влиять на признак выживаемости и будут мешать обучению.

\subsection*{Построить столбиковую диаграмму по возрасту и соответствующему их количеству}
Исходный код:  
\lstinputlisting{code/main_1.py} 

Результат выполнения данной программы представлен на рисунке~\ref{fig:main_1}.

\begin{figure}[H]
    \centering
        \includegraphics[width=\textwidth]{main_1}
    \caption{Столбиковая диаграмма по возрасту и соответствующему их количеству}
    \label{fig:main_1}
\end{figure}

\subsection*{Построить парную диаграмму между количеством пассажиров разделенных по половому признаку и количеству пассажиров разделенных по возрасту}
Исходный код:  
\lstinputlisting{code/main_2.py} 

Результат выполнения данной программы представлен на рисунке~\ref{fig:main_2}.

\begin{figure}[H]
    \centering
        \includegraphics[width=0.7\textwidth]{main_2}
    \caption{Парная диаграмма между количеством пассажиров разделенных по половому признаку и количеству пассажиров разделенных по возрасту}
    \label{fig:main_2}
\end{figure}

\subsection*{Получить лучшую регрессионную зависимость выживаемости от прочих показателей данных Титаника}
Исходный код:  
\lstinputlisting{code/main_3.py} 

Результат выполнения данной программы:

\begin{table}[H]
  \begin{tabular}{l|l|l}
    \textbf{Accuracy} & \textbf{R2 score} & \textbf{Intercept} \\\hline
	linear & 0.329776 & 0.900708 \\
	poly & 0.359776 & 0.899618 \\
	rbf: & 0.362801 & 0.617190 \\
  \end{tabular}
  \label{tab:main_3_results}
\end{table}

\subsection*{Построить классификационную модель для выживаемости на основе деревьев решений}
Исходный код:  
\lstinputlisting{code/main_4.py} 

Результат выполнения данной программы:

\begin{table}[H]
  \begin{tabular}{l}
    \textbf{Accuracy} \\\hline
	0.804124 \\
  \end{tabular}
  \label{tab:main_4_results}
\end{table}

\subsection*{Выводы}
В результате выполнения данной лабораторной работы были получены практические навыки основ визуализации и анализа данных.

Классификационная модель для виживаемости на основе деревьев решений показала намного более точный результат, чем регрессионная модель.

\end{document}
