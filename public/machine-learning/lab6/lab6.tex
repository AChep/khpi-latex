\usepackage{tikz}

\counterwithout{figure}{section}
\counterwithout{table}{section}
\counterwithout{equation}{section}

\titleformat{\subsection}[block]
  {\bfseries\filcenter}{#1}{0cm}{}
\titlespacing{\subsection}{0cm}{21pt}{21pt}

\DeclareCaptionLabelFormat{gosttable}{Таблица #2}

\usepackage{float}
\usepackage{pgfplots}
\usepackage{graphicx}
\usepackage{multirow}
\usepackage{amssymb,amsfonts,amsmath,amsthm}

\usepackage{listings}
\lstset{basicstyle=\footnotesize\ttfamily,breaklines=true}
\lstset{language=Matlab}

\lstdefinelanguage{Python}{
  keywords={and, break, class, continue, def, yield, del, elif, else, except, exec, finally, for, from, global, if, import, in, lambda, not, or, pass, print, raise, return, try, while, assert, with},
  keywordstyle=\color{NavyBlue}\bfseries,
  ndkeywords={True, False},
  ndkeywordstyle=\color{BurntOrange}\bfseries,
  emph={as},
  emphstyle={\color{OrangeRed}},
  identifierstyle=\color{black},
  sensitive=true,
  commentstyle=\color{gray}\ttfamily,
  comment=[l]{\#},
  morecomment=[s]{/*}{*/},
  stringstyle=\color{ForestGreen}\ttfamily,
  morestring=[b]',
  morestring=[s]{"""*}{*"""},
}


\newcommand{\labnumber}{6}
\usepackage{tikz}

\counterwithout{figure}{section}
\counterwithout{table}{section}
\counterwithout{equation}{section}

\titleformat{\subsection}[block]
  {\bfseries\filcenter}{#1}{0cm}{}
\titlespacing{\subsection}{0cm}{21pt}{21pt}

\DeclareCaptionLabelFormat{gosttable}{Таблица #2}

\newcommand{\khpistudentgroup}{2.КН201н.8а}
\newcommand{\khpistudentname}{Чепурний~А.~С.}

\newcommand{\khpidepartment}{Програмна інженерія та інформаційні технології управління}
\newcommand{\khpititlewhat}{
	Розрахунково-графічне завдання \\
	з предмету <<Фреймворки та платформи>>
}
\newcommand{\khpititlewho}{
	Виконав: \\
	\hspace*{\parindent} ст. групи \khpistudentgroup \\
	\hspace*{\parindent} \khpistudentname \\
	Перевірила: \\
	\hspace*{\parindent} к. т. н., вик. каф. ПІІТУ \\
	\hspace*{\parindent} Добряк~В.~С. \\
}


\graphicspath{{figures/}}

\begin{document}
\Russian

\begin{titlepage}

\begin{center}
	МІНІСТЕРСТВО ОСВІТИ І НАУКИ УКРАЇНИ \\
	НАЦІОНАЛЬНИЙ ТЕХНІЧНИЙ УНІВЕРСИТЕТ \\
	«ХАРКІВСЬКИЙ ПОЛІТЕХНІЧНИЙ ІНСТИТУТ» \\
	Кафедра <<\khpidepartment>> \\
\end{center}

\vspace{6cm}

\begin{center}
	\khpititlewhat
\end{center}

\vspace{3cm}

\begin{addmargin}[10cm]{0cm}
	\khpititlewho
\end{addmargin}

\vspace{\fill}

\begin{center}
	Харків \the\year
\end{center}

\end{titlepage}

\addtocounter{page}{1}

\section*{Анализ временных рядов}
\subsubsection*{Цель работы}
Построение модели прогнозирования временных рядов. 
\subsubsection*{Постановка задачи}
Написать программу для прогнозирования курса доллара к гривне.

\subsection*{Ход работы}
В качестве данных временного ряда были взяты значения курса доллара к гривне в период с 2016 года по декабрь 2018 года.
Усредненные по неделям данные представлены на рисунке~\ref{fig:series_week}.

\begin{figure}[H]
    \centering
        \includegraphics[width=0.7\textwidth]{series_week}
    \caption{Временной ряд}
    \label{fig:series_week}
\end{figure}

Для того, чтобы привести данный временной ряд к стационарному виду, была применена логарифмическая трансформация и сдвиг на первую разницу (рисунок~\ref{fig:series_week_log_diff}).

\begin{figure}[H]
    \centering
        \includegraphics[width=0.7\textwidth]{series_week_log_diff}
    \caption{Стационарный временной ряд}
    \label{fig:series_week_log_diff}
\end{figure}

Проведен тест Дики -- Фуллера (таблица~\ref{tab:df_test_results}), который показал, что данный ряд является стационарным.

\begin{table}[H]
  \caption{Результат теста Дики -- Фуллера}
  \begin{tabular}{l|l}
    Test Statistic &                -5.837626e+00 \\
    p-value &                        3.843316e-07 \\
    Lags Used &                     2.000000e+00 \\
    Number of Observations Used &    1.500000e+02 \\
    Critical Value (1\%) &          -3.474715e+00 \\
    Critical Value (5\%) &          -2.881009e+00 \\
    Critical Value (10\%) &          -2.577151e+00 \\
  \end{tabular}
  \label{tab:df_test_results}
\end{table}

Для того, чтобы расчитать параметры \texttt{p} и \texttt{q} модели ARIMA были построены графики автокорреляционной функции (рисунок~\ref{fig:series_week_log_diff_acf}).

\begin{figure}[H]
    \centering
    \begin{subfigure}[t]{0.5\linewidth}
        \centering
        \includegraphics{series_week_log_diff_acf}
        \caption{Автокорреляционная функция}
    \end{subfigure}%
    ~ 
    \begin{subfigure}[t]{0.5\linewidth}
        \centering
        \includegraphics{series_week_log_diff_pacf}
        \caption{Частичная автокорреляционная функция}
    \end{subfigure}
    \caption{Автокорреляционные функции}
    \label{fig:series_week_log_diff_acf}
\end{figure}

Параметрами модели были выбраны $\texttt{t}=3$, $\texttt{d}=1$, $\texttt{q}=1$.

Исходный код программы для предсказания значений курса доллара по отношению к гривне: 
\lstinputlisting{code/main.py}

На рисунке представлены результаты прогнозирования временного ряда. Ошибка $RMSE = 0.696$ является большой, однако модель показывает достаточно хорошие результаты с учетом нестабильности графика.

\begin{figure}[H]
    \centering
        \includegraphics[width=0.7\textwidth]{series_week_ARIMA}
    \caption{Спрогнозированный и исходный временной ряд}
    \label{fig:series_week_ARIMA}
\end{figure}

\subsection*{Выводы}
В результате выполнения данной лабораторной работы были получены практические навыки анализа и прогнозирования временных рядов.
Была разработана и проверена модель ARIMA для прогнозирования курса доллара к гривне.
Данная модель, несмотря на легкость разработки, показала достаточно хорошие результаты.  

\end{document}
