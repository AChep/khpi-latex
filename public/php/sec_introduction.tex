\section*{Вступ}
\addcontentsline{toc}{section}{Вступ}
В даний час Інтернет стає все більш розвиненим середовищем для здійснення комунікацій зі споживачами, він стає зручним і досить дешевим <<торговим майданчиком>>. 

Інтернет магазин актуальний для дослідження, бо у сучасному інформаційному суспільстві покупки через Інтернет стають популярнішими з кожним днем. 
Магазин становиться візитною карткою компанії. 

Об'єктом дослідження є процес розробки інтернет-магазину з використанням PHP та \acrshort{sql}. 

Предметом дослідження є інтернет-магазин <<\thesitename>>. 

Метою і завданням дослідження є розробка веб-ресурсу магазину <<\thesitename>> з використанням PHP та \acrshort{sql}.
Для досягнення поставленої мети в курсовій роботі були сформульовані та вирішені наступні задачі:
\begin{itemize}
	\item розроблення специфікації системи;
	\item реалізація систему;
	\item верифікація розроблену систему;
	\item тестування системи;
	\item викладення пропозицій щодо перспектив розвитку та удосконалення розробленої системи.
\end{itemize}
