\begin{titlepage}

\section*{Завдання}
\subsection*{Вимоги до сайту}
\begin{enumerate}
	\item Сайт передбачає орієнтацію на двох користувачів: замовника і адміністратора. 
	Замовник може обирати товари та формувати замовлення (вказуючи свої номер телефону та e-mail). 
	Після цього замовлення надходить адміністратору, який здійснює його обробку. 
	Робота адміністратора з сайтом здійснюється через адміністраторську панель.

	\item Адміністратор через адміністраторську панель може додавати в інтернет-магазин нові категорії товарів, нові товари, редагувати інформацію по старим товарам, видаляти записи, що стосуються товарів і категорій товарів.

	\item Панель адміністратора забезпечує доступ до записів у базі даних, які стосуються наявності на складі товару, а також до записів, які стосуються замовлень. 
	Отримуючи замовлення, адміністратор здійснює перевірку наявності товару на складі. 
	Передбачається, що адміністратор повідомляє замовнику про наявність товару за вказаним номером телефону. 
	Після того, як замовлення оплачене і відправлене, адміністратор робить відповідні відмітки і переміщує замовлення в архів. 

	\item За записами в архіві можливе формування звітів.
	Звіт повинен містити наступну інформацію:
	номер замовлення,
	дату замовлення і дату надсилання в архів,
	назви замовлених товарів (з визначенням категорії товару), їх кількість та ціну,
	загальну вартість замовлення.

	\item Формування звіту передбачає фільтрацію замовлень за датою замовлення, датою надсилання в архів, назвою товару, його категорією, кількістю товару, загальною вартістю замовлення. 
	Наприклад, адміністратор має можливість сформувати список замовлень, які надійшли в період з 1.02.2018 по 5.02.2018 на суму більше 2000 грн. 
	Або список замовлень, які включали в себе обраний товар у кількості більш ніж 5 одиниць тощо.
\end{enumerate}

\end{titlepage}
