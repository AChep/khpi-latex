\section{Результати застосування розробленої системи}
\subsection{Стислі відомості щодо розгортання системи}

Вимоги до серверної машини наведено у таблиці~\ref{tab:sw_requirements}. 

\begin{table}[h]
	\caption{Мінімальні вимоги до серверного обладнання}
	\label{tab:sw_requirements}
	\begin{tabular}{l|l}
		Процесор & 1000 МГц \\ \hline
		ОЗП & 512 МБ \\ \hline
		Об'єм пам'яті диску & 64 ГБ 
	\end{tabular}
\end{table}

Програмне забезпечення серверу може бути встановлено на операційні системи родини \textit{Linux}, \textit{Windows} або \textit{macOS}.
Рекомендовано використовувати \acrshort{ssd} у якості накопичувача.

Для розгортання системи необхідно встановити наступні програми:
\begin{itemize}
	\item Apache Server v2.4.29+;
	\item SQLite v3.22.0+;
	\item PHP v7.2.2+;
\end{itemize}
та встановити драйвер взаємодії PHP з SQLite.

\subsection{Опис роботи з сайтом}

\subsection{Результати тестування та рекомендації щодо удосконалення розробленої системи}
При тестуванні основною проблемою виявилася підтримка сайтом різних пристроїв та браузерів. 
При більш детальному вивченні стандартів HTML та CSS ця проблема була вирішена.

Одним із напрямків розвитку розробленої системи є співробітництво з іншими постачальниками, створення платформи для постачальників. 
