\usepackage{tikz}

\counterwithout{figure}{section}
\counterwithout{table}{section}
\counterwithout{equation}{section}

\titleformat{\subsection}[block]
  {\bfseries\filcenter}{#1}{0cm}{}
\titlespacing{\subsection}{0cm}{21pt}{21pt}

\DeclareCaptionLabelFormat{gosttable}{Таблица #2}

\usepackage{float}
\usepackage{pgfplots}
\usepackage{graphicx}
\usepackage{multirow}
\usepackage{amssymb,amsfonts,amsmath,amsthm}

\usepackage{listings}
\lstset{basicstyle=\footnotesize\ttfamily,breaklines=true}
\lstset{language=Matlab}


\newcommand{\labnumber}{2} % second lab
\usepackage{tikz}

\counterwithout{figure}{section}
\counterwithout{table}{section}
\counterwithout{equation}{section}

\titleformat{\subsection}[block]
  {\bfseries\filcenter}{#1}{0cm}{}
\titlespacing{\subsection}{0cm}{21pt}{21pt}

\DeclareCaptionLabelFormat{gosttable}{Таблица #2}

\newcommand{\khpistudentgroup}{2.КН201н.8а}
\newcommand{\khpistudentname}{Чепурний~А.~С.}

\newcommand{\khpidepartment}{Програмна інженерія та інформаційні технології управління}
\newcommand{\khpititlewhat}{
	Розрахунково-графічне завдання \\
	з предмету <<Фреймворки та платформи>>
}
\newcommand{\khpititlewho}{
	Виконав: \\
	\hspace*{\parindent} ст. групи \khpistudentgroup \\
	\hspace*{\parindent} \khpistudentname \\
	Перевірила: \\
	\hspace*{\parindent} к. т. н., вик. каф. ПІІТУ \\
	\hspace*{\parindent} Добряк~В.~С. \\
}


\graphicspath{{figures/}}

\begin{document}
\Ukrainian

\begin{titlepage}

\begin{center}
	МІНІСТЕРСТВО ОСВІТИ І НАУКИ УКРАЇНИ \\
	НАЦІОНАЛЬНИЙ ТЕХНІЧНИЙ УНІВЕРСИТЕТ \\
	«ХАРКІВСЬКИЙ ПОЛІТЕХНІЧНИЙ ІНСТИТУТ» \\
	Кафедра <<\khpidepartment>> \\
\end{center}

\vspace{6cm}

\begin{center}
	\khpititlewhat
\end{center}

\vspace{3cm}

\begin{addmargin}[10cm]{0cm}
	\khpititlewho
\end{addmargin}

\vspace{\fill}

\begin{center}
	Харків \the\year
\end{center}

\end{titlepage}

\addtocounter{page}{1}

\section*{Feature Oriented Domain Analysis}
\subsubsection*{Цель работы}
Знакомство с методом доменного моделирования FODA (Feature Oriented Domain Analysis) и инструметарием Feature IDE.

\subsection*{Ход работы}
\begin{enumerate}
    \item Контекстный анализ (Context Analysis).
    \item Моделирование домена (Domain Modeling).
    \item Моделирование архитектуры (Architectural Modeling).
\end{enumerate}

\subsection{Контекстный анализ}
Доменная область представляет собой задачу дистрибьюции логистической системы. 
Начальная структура логистической цели вводится пользователем. 
Система должна показать график уровня сервиса смоделированной логистической системы.

Этот домен не включает в себя качественное моделирование спроса людей и изменения их предпочтений.

Модель потоков данных предметной области показана на рисунке~\ref{fig:dfd}.

\begin{figure}[H]
    \centering
    \includegraphics[width=0.65\textwidth]{dfd}
    \caption{DFD модель}
    \label{fig:dfd}
\end{figure}

\subsection{Моделирование домена}
Для моделирования домена была использована FeatureIDE. 
Результат моделирования представлен на рисунке~\ref{fig:model}. 

\begin{figure}[H]
    \centering
    \includegraphics[width=\textwidth]{model}
    \caption{Feature модель}
    \label{fig:model}
\end{figure}

Функционал системы был разбит на четыре модуля: моделирование, сохранение отчета, загрузка данных, просмотр отчета.

\subsection{Разработка функционала}
За основу архитектуры было взята Hexagonal-архитектура.

Структура архитектурных уровней будет иметь вид, представленный на рисунке~\ref{fig:arch}.

\begin{figure}[H]
    \centering
    \includegraphics[width=0.5\textwidth]{arch}
    \caption{Модель архитектуры}
    \label{fig:arch}
\end{figure}

\subsection*{Выводы}
В ходе данной лабораторной работы было проведено знакомство с анализом FODA.

Было разработана модель представления доменной области <<моделирование логистических систем дистрибуции>> с помощью FeatureIDE, спроектирована структура функций, показана возможная архитектура системы. 

\end{document}
