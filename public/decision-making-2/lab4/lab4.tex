\usepackage{tikz}

\counterwithout{figure}{section}
\counterwithout{table}{section}
\counterwithout{equation}{section}

\titleformat{\subsection}[block]
  {\bfseries\filcenter}{#1}{0cm}{}
\titlespacing{\subsection}{0cm}{21pt}{21pt}

\DeclareCaptionLabelFormat{gosttable}{Таблица #2}

\usepackage{float}
\usepackage{pgfplots}
\usepackage{graphicx}
\usepackage{multirow}
\usepackage{amssymb,amsfonts,amsmath,amsthm}

\usepackage{listings}
\lstset{basicstyle=\footnotesize\ttfamily,breaklines=true}
\lstset{language=Matlab}


\newcommand{\labnumber}{4} % fourth lab
\usepackage{tikz}

\counterwithout{figure}{section}
\counterwithout{table}{section}
\counterwithout{equation}{section}

\titleformat{\subsection}[block]
  {\bfseries\filcenter}{#1}{0cm}{}
\titlespacing{\subsection}{0cm}{21pt}{21pt}

\DeclareCaptionLabelFormat{gosttable}{Таблица #2}

\newcommand{\khpistudentgroup}{2.КН201н.8а}
\newcommand{\khpistudentname}{Чепурний~А.~С.}

\newcommand{\khpidepartment}{Програмна інженерія та інформаційні технології управління}
\newcommand{\khpititlewhat}{
	Розрахунково-графічне завдання \\
	з предмету <<Фреймворки та платформи>>
}
\newcommand{\khpititlewho}{
	Виконав: \\
	\hspace*{\parindent} ст. групи \khpistudentgroup \\
	\hspace*{\parindent} \khpistudentname \\
	Перевірила: \\
	\hspace*{\parindent} к. т. н., вик. каф. ПІІТУ \\
	\hspace*{\parindent} Добряк~В.~С. \\
}


\usepackage{systeme}
\usepackage{longtable,tabu}
\usepackage{multirow}
\usepackage{array,multirow}
\usepackage{pdflscape}
\usepackage{afterpage}
\usepackage{tikz}
\usepackage{bm}

\graphicspath{{figures/}}

\begin{document}
\Russian

\begin{titlepage}

\begin{center}
	МІНІСТЕРСТВО ОСВІТИ І НАУКИ УКРАЇНИ \\
	НАЦІОНАЛЬНИЙ ТЕХНІЧНИЙ УНІВЕРСИТЕТ \\
	«ХАРКІВСЬКИЙ ПОЛІТЕХНІЧНИЙ ІНСТИТУТ» \\
	Кафедра <<\khpidepartment>> \\
\end{center}

\vspace{6cm}

\begin{center}
	\khpititlewhat
\end{center}

\vspace{3cm}

\begin{addmargin}[10cm]{0cm}
	\khpititlewho
\end{addmargin}

\vspace{\fill}

\begin{center}
	Харків \the\year
\end{center}

\end{titlepage}

\addtocounter{page}{1}

\textbf{Тема}: построение функций принадлежности на основе экспертных оценок.

\textbf{Цель}: 
\begin{itemize}
	\item изучить метод построения функций принадлежности на основе экспертных оценок;
	\item решить практическую задачу в соответствии с методом построения функций принадлежности на основе экспертных оценок;
	\item провести анализ полученных результатов и сделать выводы по работе.
\end{itemize}

\textbf{Задание}: 
\begin{itemize}
	\item построить функцию принадлежности нечеткого множества, соответствующего точечной оценке \textit{ПРИБЛИЗИТЕЛЬНО $N$};
	\item построить функцию принадлежности нечеткого множества, соответствующего интервальной оценке \textit{ПРИБЛИЗИТЕЛЬНО В ИНТЕРВАЛЕ ОТ $K$ ДО $L$};
	\item провести анализ полученных результатов.
\end{itemize}

\subsection{Построить функцию принадлежности нечеткого множества, соответствующего точечной оценке \textit{ПРИБЛИЗИТЕЛЬНО $N$}}
Имеем приближенную точечную экспертную оценку \textit{$X$ ПРИБЛИЗИТЕЛЬНО РАВЕН $10$}.
\[
	K = 10.
\]

Определяем значение переменных $q$, $r_q$, $r_{q+1}$ и $d$. Младшая значащая цифра числа $К$ стоит в разряде двоек, т.е. имеем $q=2$; $r_2$ = $1$ --- младшая значащая цифра числа $K$; $r_3 = 0$ --- цифра, имеющая порядок на единицу выше порядка младшей значащей цифры.

При делении числа $q$ на $3$ в остатке получаем $1$, т.е. число $К$ принадлежит к классу эквивалентности $M_1$ и переменная $d$ получает значение единицу. 

Тогда $z=r_q=1$, $\beta(K) = \beta(q)\cdot 10^{q-2} = 3.407 \cdot 10^0 = 3.407$

Функция имеет следующий вид:
\[
	y(x) = e^{-\alpha(k-x)^2}.
\]

Известно что функция принимает значение $y(10)=1$, $y(10-\beta(K)/2)=y(8.2965)=y(a)=0.5$, a $k = K$.

Найдем значение переменной $\alpha$:
\begin{align*}
	\ln y(a) &= -\alpha(k - a)^2, \\
	\ln 0.5 &= -\alpha(10 - 8.2965)^2, \\
	\ln 0.5 &= -\alpha \cdot 1.7035^2, \\
	\alpha = - \frac{\ln 0.5}{1.7035^2} &= - \frac{\ln 0.5}{2.9019} = 0.2388.
\end{align*}

Полученная функция принадлежности приведена на рисунке~\ref{fig:graph1}:

\begin{figure}[H]
	\centering
	  \includegraphics[width=0.8\textwidth]{graph1}
	\caption{Функция принадлежности нечеткого множества, соответствующего точечной оценке \textit{ПРИБЛИЗИТЕЛЬНО 10}}
	\label{fig:graph1}
  \end{figure}

\subsection{Построить функцию принадлежности нечеткого множества, соответствующего интервальной оценке \textit{ПРИБЛИЗИТЕЛЬНО В ИНТЕРВАЛЕ ОТ $K$ ДО $L$}}

Имеем оценку $X$ находится \textit{ПРИБЛИЗИТЕЛЬНО В ИНТЕРВАЛЕ ОТ $10$ ДО $15$}. 
На этом интервале функция принадлежности равна единице, а за его пределами будет повторять функции принадлежности, соответствующие точечным оценкам \textit{$X$ ПРИБЛИЗИТЕЛЬНО РАВЕН $10$} и \textit{$X$ ПРИБЛИЗИТЕЛЬНО РАВЕН $15$} слева и справа от интервала. Для построения функции принадлежности нечеткого множества, соответствующего интервальной оценке, необходимо дважды воспользоваться описанным выше методом.

Для \textit{$X$ ПРИБЛИЗИТЕЛЬНО РАВЕН $10$} формулу возьмем из предыдущих расчетов: 
\[
	y_1(x_1) = e^{-0.2388(10 - x_1)^2}.
\]

Для \textit{$X$ ПРИБЛИЗИТЕЛЬНО РАВЕН $15$} произведем расчет параметров $\alpha$ и $k$: 
\[
	y_2(x_2) = e^{-\alpha(k - x_2)^2}.
\]

Определяем значение переменных $q$, $r_q$, $r_{q+1}$ и $d$. Младшая значащая цифра числа $L$ стоит в разряде единиц, т.е. имеем $q=1$; $r_1$ = $5$ --- младшая значащая цифра числа $K$; $r_2 = 1$ --- цифра, имеющая порядок на единицу выше порядка младшей значащей цифры.

При делении числа $q$ на $3$ в остатке получаем $1$, т.е. число $К$ принадлежит к классу эквивалентности $M_1$ и переменная $d$ получает значение единицу. 

Тогда $z=r_2 \cdot 10 + r_1=15$, $\beta(L) = \beta(q)\cdot 10^{q-1} = 6.48$

Известно что функция принимает значение $y_2(15)=1$, $y_2(15-\beta(L)/2)=y_2(11.76)=y_2(a)=0.5$, a $k = 15$.

Найдем значение переменной $\alpha$:
\begin{align*}
	\ln y_(a) &= -\alpha(k - a)^2, \\
	\ln 0.5 &= -\alpha(15 - 11.76)^2, \\
	\ln 0.5 &= -\alpha \cdot 3.24^2, \\
	\alpha = - \frac{\ln 0.5}{3.24^2} &= - \frac{\ln 0.5}{10.4976} = 0.0660.
\end{align*}

Для \textit{$X$ ПРИБЛИЗИТЕЛЬНО РАВЕН $15$} формула будет иметь вид: 
\[
	y_2(x_2) = e^{-0.0660(15 - x_2)^2}.
\]

Окончательная формула имеет вид, график ее предствален на рисунке~\ref{fig:graph2}:
\[
	y(x) = 
	\begin{cases} 
		e^{-0.2388(10 - x)^2}, & \mbox{если } x \leq 10 \\ 
		e^{-0.0660(15 - x)^2}, & \mbox{если } x \geq 15 \\
		1.
	\end{cases}
\]

\begin{figure}[H]
	\centering
	  \includegraphics[width=0.8\textwidth]{graph2}
	\caption{Функция принадлежности нечеткого множества, соответствующего точечной оценке \textit{ПРИБЛИЗИТЕЛЬНО В ИНТЕРВАЛЕ ОТ $10$ ДО $15$}}
	\label{fig:graph2}
  \end{figure}

\subsection{Выводы}

В ходе лабораторной работы был изучен метод построения функций принадлежности на основе экспертных оценок, решена практическая задача в соответствии с методом построения функций принадлежности на основе экспертных оценок.

\end{document}
