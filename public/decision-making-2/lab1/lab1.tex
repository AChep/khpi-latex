% !TEX program = xelatex

\documentclass[a4paper,14pt,oneside,final]{extarticle}
\usepackage[top=2cm, bottom=2cm, left=3cm, right=1cm]{geometry}
\usepackage{scrextend}

\usepackage[T2A,T1]{fontenc}
\usepackage[ukrainian,russian,english]{babel}
\usepackage{tempora}
\usepackage{fontspec}
\setmainfont{tempora}

% Зачем: Отключает использование изменяемых межсловных пробелов.
% Почему: Так не принято делать в текстах на русском языке.
\frenchspacing

\usepackage{indentfirst}
\setlength{\parindent}{1.25cm}
\renewcommand{\baselinestretch}{1.5}

% Header
\usepackage{fancyhdr}
\pagestyle{fancy}
\fancyhead{}
\fancyfoot{}
\fancyhead[R]{\small \selectfont \thepage}
\renewcommand{\headrulewidth}{0pt}

% Captions
\usepackage{chngcntr}
\counterwithin{figure}{section}
\counterwithin{table}{section}
\usepackage[tableposition=top]{caption}
\usepackage{subcaption}
\DeclareCaptionLabelFormat{gostfigure}{Рисунок #2}
\DeclareCaptionLabelFormat{gosttable}{Таблиця #2}
\DeclareCaptionLabelSeparator{gost}{~---~}
\captionsetup{labelsep=gost}
\captionsetup[figure]{labelformat=gostfigure}
\captionsetup[table]{labelformat=gosttable}
\renewcommand{\thesubfigure}{\asbuk{subfigure}}

% Sections
\usepackage[explicit]{titlesec}
\newcommand{\sectionbreak}{\clearpage}

\titleformat{\section}
  {\centering}{\thesection \quad}{0pt}{\MakeUppercase{#1}}
\titleformat{\subsection}[block]
  {\bfseries}{\thesubsection \quad #1}{0cm}{}

\titlespacing{\section} {0cm}{0cm}{21pt}
\titlespacing{\subsection} {\parindent}{21pt}{0cm}
\titlespacing{\subsubsection} {\parindent}{0cm}{0cm}

% Lists
\usepackage{enumitem}
\renewcommand\labelitemi{--}
\setlist[itemize]{noitemsep, topsep=0pt, wide}
\setlist[enumerate]{noitemsep, topsep=0pt, wide, label=\arabic*}
\setlist[description]{labelsep=0pt, noitemsep, topsep=0pt, leftmargin=2\parindent, labelindent=\parindent, labelwidth=\parindent, font=\normalfont}

% Toc
\usepackage{tocloft}
\tocloftpagestyle{fancy}
\renewcommand{\cfttoctitlefont}{}
\setlength{\cftbeforesecskip}{0pt}
\renewcommand{\cftsecfont}{}
\renewcommand{\cftsecpagefont}{}
\renewcommand{\cftsecleader}{\cftdotfill{\cftdotsep}}

\usepackage{float}
\usepackage{pgfplots}
\usepackage{graphicx}
\usepackage{multirow}
\usepackage{amssymb,amsfonts,amsmath,amsthm}
\usepackage{csquotes}

\usepackage{listings}
\lstset{basicstyle=\footnotesize\ttfamily,breaklines=true}
\lstset{language=Matlab}

\usepackage[
	backend=biber,
	sorting=none,
	language=auto,
	autolang=other
]{biblatex}
\DeclareFieldFormat{labelnumberwidth}{#1}


\newcommand{\labnumber}{1} % first lab
\documentclass[a4paper,14pt,oneside,final]{extarticle}
\usepackage[top=2cm, bottom=2cm, left=3cm, right=1cm]{geometry}
\usepackage{scrextend}

\usepackage[T2A,T1]{fontenc}
\usepackage[ukrainian,russian,english]{babel}
\usepackage{tempora}
\usepackage{fontspec}
\setmainfont{tempora}

% Зачем: Отключает использование изменяемых межсловных пробелов.
% Почему: Так не принято делать в текстах на русском языке.
\frenchspacing

\usepackage{indentfirst}
\setlength{\parindent}{1.25cm}
\renewcommand{\baselinestretch}{1.5}

% Header
\usepackage{fancyhdr}
\pagestyle{fancy}
\fancyhead{}
\fancyfoot{}
\fancyhead[R]{\small \selectfont \thepage}
\renewcommand{\headrulewidth}{0pt}

% Captions
\usepackage{chngcntr}
\counterwithin{figure}{section}
\counterwithin{table}{section}
\usepackage[tableposition=top]{caption}
\usepackage{subcaption}
\DeclareCaptionLabelFormat{gostfigure}{Рисунок #2}
\DeclareCaptionLabelFormat{gosttable}{Таблиця #2}
\DeclareCaptionLabelSeparator{gost}{~---~}
\captionsetup{labelsep=gost}
\captionsetup[figure]{labelformat=gostfigure}
\captionsetup[table]{labelformat=gosttable}
\renewcommand{\thesubfigure}{\asbuk{subfigure}}

% Sections
\usepackage[explicit]{titlesec}
\newcommand{\sectionbreak}{\clearpage}

\titleformat{\section}
  {\centering}{\thesection \quad}{0pt}{\MakeUppercase{#1}}
\titleformat{\subsection}[block]
  {\bfseries}{\thesubsection \quad #1}{0cm}{}

\titlespacing{\section} {0cm}{0cm}{21pt}
\titlespacing{\subsection} {\parindent}{21pt}{0cm}
\titlespacing{\subsubsection} {\parindent}{0cm}{0cm}

% Lists
\usepackage{enumitem}
\renewcommand\labelitemi{--}
\setlist[itemize]{noitemsep, topsep=0pt, wide}
\setlist[enumerate]{noitemsep, topsep=0pt, wide, label=\arabic*}
\setlist[description]{labelsep=0pt, noitemsep, topsep=0pt, leftmargin=2\parindent, labelindent=\parindent, labelwidth=\parindent, font=\normalfont}

% Toc
\usepackage{tocloft}
\tocloftpagestyle{fancy}
\renewcommand{\cfttoctitlefont}{}
\setlength{\cftbeforesecskip}{0pt}
\renewcommand{\cftsecfont}{}
\renewcommand{\cftsecpagefont}{}
\renewcommand{\cftsecleader}{\cftdotfill{\cftdotsep}}

\newcommand{\khpistudentgroup}{КН-34г}
\newcommand{\khpistudentname}{Чепурний~А.~С.}

\newcommand{\khpidepartment}{Програмна інженерія та інформаційні технології управління}
\newcommand{\khpititlewhat}{
	Лабораторна робота №\labnumber \\
	з предмету <<Моделювання систем>>
}
\newcommand{\khpititlewho}{
	Виконав: \\
	\hspace*{\parindent} ст. групи \khpistudentgroup \\
	\hspace*{\parindent} \khpistudentname \\
	Перевірила: \\
	\hspace*{\parindent} ст. в. каф. ПІІТУ \\
	\hspace*{\parindent} Єршова~С.~І. \\
	\hspace*{\parindent} ас. каф. ПІІТУ \\
	\hspace*{\parindent} Литвинова~Ю.~С. \\
}



\usepackage{systeme}
\usepackage{longtable,tabu}
\usepackage{multirow}
\usepackage{array,multirow}
\usepackage{pdflscape}
\usepackage{afterpage}
\usepackage{bm}

\graphicspath{{../figures/}}

\begin{document}
\Russian

\begin{titlepage}

\begin{center}
	МІНІСТЕРСТВО ОСВІТИ І НАУКИ УКРАЇНИ \\
	НАЦІОНАЛЬНИЙ ТЕХНІЧНИЙ УНІВЕРСИТЕТ \\
	«ХАРКІВСЬКИЙ ПОЛІТЕХНІЧНИЙ ІНСТИТУТ» \\[0.5cm]
	Кафедра <<\khpidepartment>> \\
\end{center}

\vspace{6cm}

\begin{center}
	\khpititlewhat
\end{center}

\vspace{3cm}

\begin{addmargin}[10cm]{0cm}
	\khpititlewho
\end{addmargin}

\vspace{\fill}

\begin{center}
	Харків \the\year
\end{center}

\end{titlepage}

\addtocounter{page}{1}

\textbf{Тема}: решение задач формирования экспертной группы (ЭГ).

\textbf{Цель}: 
\begin{itemize}
	\item изучение процесса формирования ЭГ;
	\item решение основных задач формирования ЭГ.
\end{itemize}

\textbf{Задание}:
\begin{enumerate}[label={\arabic*)}]
	\item формализовать процесс формирования ЭГ;
	\item оценить компетентность потенциальных экспертов;
	\item определить количественный состав ЭГ;
	\item решить задачу выбора состава экспертной группы на основе теории прецедентов;
	\item сформировать эталонный вариант модели эксперта с использованием теории прецедентов.
\end{enumerate}

\textbf{Индивидуальное задание на лабораторную работу}:

Вариант №1
\begin{align*}
	k_{u_1}=0.68, k_{u_2}=0.83.
\end{align*}

{
\small
\tabulinesep=1.2mm
\begin{longtabu} to \textwidth {|X[12,l]|X[1,c]X[1,c]X[1,c]|X[1,c]X[1,c]X[1,c]|X[1,c]X[1,c]X[1,c]|X[1,c]X[1,c]X[1,c]|X[1,c]X[1,c]X[1,c]|}
	\caption{Самооценка экспертов}
	\label{tab:selfscore} \\
	\hline
	\multirow{3}{*}{Источник аргументации} & \multicolumn{15}{c|}{Уровень влияния источника на мнение эксперта} \\ \cline{2-16}
	& \multicolumn{3}{c|}{Эксперт 1} & \multicolumn{3}{c|}{Эксперт 2} & \multicolumn{3}{c|}{Эксперт 3} & \multicolumn{3}{c|}{Эксперт 4} & \multicolumn{3}{c|}{Эксперт 5} \\ \cline{2-16}
	& A & B & C & A & B & C & A & B & C & A & B & C & A & B & C \\ \hline
	\endfirsthead

	\caption*{Окончание таблицы \thetable{}}\\
	\hline
	\multirow{3}{*}{Источник аргументации} & \multicolumn{15}{c|}{Уровень влияния источника на мнение эксперта} \\ \cline{2-16}
	& \multicolumn{3}{c|}{Эксперт 1} & \multicolumn{3}{c|}{Эксперт 2} & \multicolumn{3}{c|}{Эксперт 3} & \multicolumn{3}{c|}{Эксперт 4} & \multicolumn{3}{c|}{Эксперт 5} \\ \cline{2-16}
	& A & B & C & A & B & C & A & B & C & A & B & C & A & B & C \\ \hline
	\endhead

	Проведенный экспертом теоретический анализ данной проблемы
	& & \checkmark & & & \checkmark & & & \checkmark & & \checkmark & & & & & \checkmark \\ \hline
	Производственный опыт эксперта, связанный с решаемой проблемой & \checkmark & & & & \checkmark & & & \checkmark & & & \checkmark & & & & \checkmark \\ \hline
	Участие в семинарах, совещаниях в своей стране по исследуемой проблеме & & & \checkmark & & & \checkmark & \checkmark & & & & \checkmark & & & \checkmark & \\ \hline
	Знакомство с работами зарубежных авторов по рассматриваемой проблеме & & & \checkmark & & & \checkmark & \checkmark & & & & & \checkmark & & \checkmark & \\ \hline
	Количество проектов, в подготовке, реализации и экспертизе которых эксперт принимал участие & & \checkmark & & & \checkmark & & & \checkmark & & \checkmark & & & & & \checkmark \\ \hline
	Влияние интуиции эксперта на принимаемые решения & & \checkmark & & \checkmark & & & & \checkmark & & & \checkmark & & \checkmark & & \\ \hline
\end{longtabu}
}

{
\small
\tabulinesep=1.2mm
\begin{longtabu} to \textwidth {|X[1,c]|X[1,c]|X[1,c]|X[1,c]|X[1,c]|X[1,c]|X[1,c]|}
	\caption{Параметры экспертов}
	\label{tab:score} \\
	\hline
	$k_i$  & $u$ & Эксперт 1 & Эксперт 2 & Эксперт 3 & Эксперт 4 & Эксперт 5 \\ \hline
	\endfirsthead

	\caption*{Окончание таблицы \thetable{}}\\
	\hline
	$k_i$  & $u$ & Эксперт 1 & Эксперт 2 & Эксперт 3 & Эксперт 4 & Эксперт 5 \\ \hline
	\endhead

	$0.0003$ & $u_{lt} $ & $34$ & $26$ & $52$ & $41$ & $60$ \\ \hline
	$0.0008$ & $u_{sm} $ & $2$ & $14$ & $15$ & $3$ & $16$ \\ \hline
	$0.0015$ & $u_{sd} $ & $16$ & $3$ & $20$ & $21$ & $12$ \\ \hline
	$0.0020$ & $u_{z3} $ & $12$ & $14$ & $2$ & $2$ & $32$ \\ \hline
	$0.0010$ & $u_{z5} $ & $16$ & $23$ & $24$ & $10$ & $39$ \\ \hline
	$0.0009$ & $u_{zsp} $ & $5$ & $2$ & $14$ & $8$ & $9$ \\ \hline
	$0.0010$ & $u_{zv} $ & $26$ & $7$ & $2$ & $20$ & $32$ \\ \hline
	$0.0015$ & $u_{vs} $ & $5$ & $12$ & $5$ & $2$ & $3$ \\ \hline
	$0.0020$ & $u_{vz} $ & $1$ & $2$ & $4$ & $7$ & $0$ \\ \hline
	$0.0033$ & $u_{vsp} $ & $18$ & $23$ & $7$ & $21$ & $5$ \\ \hline
	$0.0080$ & $u_{zdl} $ & $4$ & $3$ & $5$ & $5$ & $4$ \\ \hline
	$0.0070$ & $u_{dlz} $ & $4$ & $3$ & $4$ & $4$ & $4$ \\ \hline
	$0.0015$ & $u_{kn} $ & $1$ & $1$ & $1$ & $1$ & $1$ \\ \hline
	$0.0200$ & $u_{dn} $ & $0$ & $0$ & $1$ & $1$ & $1$ \\ \hline
	$0.0250$ & $u_{zd} $ & $1$ & $0$ & $1$ & $1$ & $1$ \\ \hline
	$0.0300$ & $u_{zn} $ & $1$ & $1$ & $1$ & $1$ & $1$ \\ \hline
	$0.0350$ & $u_{zpf} $ & $0$ & $0$ & $1$ & $0$ & $1$ \\ \hline
	$0.0015$ & $u_{pv} $ & $6$ & $4$ & $15$ & $3$ & $13$ \\ \hline
	$0.0018$ & $u_{pn} $ & $2$ & $4$ & $0$ & $6$ & $2$ \\ \hline
	$0.0020$ & $u_{ps} $ & $1$ & $0$ & $3$ & $2$ & $8$ \\ \hline
	$0.0250$ & $u_{sk} $ & $0.7$ & $0.65$ & $0.87$ & $0.93$ & $0.84$ \\ \hline
	$0.0015$ & $u_{skp} $ & $0.83$ & $0.84$ & $0.79$ & $0.85$ & $0.91$ \\ \hline
	$-0.0005$ & $u_{usn} $ & $0$ & $0$ & $0$ & $0$ & $0$ \\ \hline
	$0.0003$ & $u_{usi} $ & $0$ & $0$ & $0$ & $0$ & $0$ \\ \hline
	$0.0010$ & $u_{usj} $ & $0$ & $0$ & $0$ & $1$ & $0$ \\ \hline
	$0.0380$ & $u_{usv} $ & $1$ & $1$ & $1$ & $0$ & $1$ \\ \hline
	$0.0100$ & $u_{izp} $ & $5$ & $4$ & $5$ & $5$ & $4$ \\ \hline
	$0.0086$ & $u_{ikr} $ & $4$ & $5$ & $5$ & $5$ & $4$ \\ \hline
	$-0.0015$ & $u_{iuk} $ & $1$ & $1$ & $2$ & $1$ & $1$ \\ \hline
	$0.0100$ & $u_{ilz} $ & $1$ & $0$ & $0$ & $1$ & $1$ \\ \hline
	$0.0080$ & $u_{iak} $ & $5$ & $4$ & $3$ & $5$ & $5$ \\ \hline
\end{longtabu}
}

\subsection{Формализирование процесса формирования ЭГ}
{
	\itshape
	В качестве начальных данных взять 5 экспертов $Q$ и 10 функций $A$.

	Множество функций экспертов $F$, коэффициент резервирования $К$, затраты на привлечение к экспертизе специалистов $С$ --- задать самостоятельно.
	Решить задачу оптимизации затрат и выделить претендентов, которых необходимо включить в ЭГ.
}

Начальные параметры:
\begin{align*}
	Q & = \{ q_1, q_2, q_3, q_4, q_5 \},      \\
	A & = \{ a_1, a_2, \dots, a_9, a_{10} \},
\end{align*}
\begin{description}
	\item[где] $Q$ --- множество претендентов;
	\item $A$ --- множество функций.
\end{description}

Претенденты способны выполнять следующие функции:
\begin{align*}
	F_{q_1} & = \{ a_1, a_2, a_8, a_4, a_5 \},    \\
	F_{q_2} & = \{ a_1, a_6, a_3, a_9, a_5 \},    \\
	F_{q_3} & = \{ a_1, a_6, a_3, a_9, a_5 \},    \\
	F_{q_4} & = \{ a_1, a_2, a_7, a_4, a_{10} \}, \\
	F_{q_5} & = \{ a_8, a_2, a_7, a_4, a_{10} \}.
\end{align*}

Минимально необходимое количество экспертов, способных реализовать соответствующие функции:
\begin{align*}
	K & = \{ k_1, k_2, \dots, k_9, k_{10} \} = \\
	  & = \{ 1, 2, 1, 2, 1, 2, 1, 2, 1, 2 \}.
\end{align*}

Затраты на привлечение каждого из экспертов:
\begin{align*}
	C & = \{ c_1, c_2, c_3, c_4, c_5 \} = \\
	  & = \{ 1, 2, 4, 8, 16 \}.
\end{align*}

Задача оптимизации и ограничения:
\begin{align*}
	c_1d_1 + c_2d_2 + c_3d_3 + c_4d_4 + c_5d_5 \to \min,
\end{align*}
\begin{align*}
	\systeme{
		d_1 + d_2 + d_3 + d_4 \geq 1,
		d_1 + d_4 + d_5 \geq 2,
		d_2 + d_3 \geq 1,
		d_1 + d_4 + d_5 \geq 2,
		d_1 + d_2 + d_3 \geq 1,
		d_2 + d_3 \geq 2,
		d_4 + d_5 \geq 1,
		d_1 + d_5 \geq 2,
		d_2 + d_3 \geq 1,
		d_4 + d_5 \geq 2
	},
	\quad
	\systeme{
		0 \leq d_1 \leq 1,
		0 \leq d_2 \leq 1,
		0 \leq d_3 \leq 1,
		0 \leq d_4 \leq 1,
		0 \leq d_5 \leq 1
	}.
\end{align*}

Решив задачу, получаем $D = \{ 1, 1, 1, 1, 1 \}$ ($\min = 31$).
Это означает, что в экспертную группу необходимо включить всех претендентов.

\subsection{Оценивание компетентности потенциальных экспертов}
{
	\itshape
	Имеется 5 кандидатов в ЭГ.
	Оценить компетентность потенциальных экспертов согласно своему варианту.
}

Коэффициент компетентности $K$ на основе самооценки вычисляется по формуле:
\begin{align*}
	K = \frac{1}{2}(k_u+k_a),
\end{align*}
\begin{description}
	\item[где] $k_u$ --- коэффициент информированности по проблеме, на основе \\ самооценки эксперта по десятибалльной шкале, умноженный на $0.1$;
	\item $k_a$ --- коэффициент аргументации, определяемый в результате \\ суммирования баллов, полученных путем анкетирования экспертов по вопросу их оценки собственной компетентности в определенной предметной области.
\end{description}

Коэффициенты информированности для каждого эксперта имеют следующие значения:
\begin{align*}
	k_{u_1}=0.68, k_{u_2}=0.83, k_{u_3}=0.8, k_{u_4}=0.85, k_{u_5}=0.97.
\end{align*}

Показатели аргументации были рассчитаны согласно данным об \\ анкетировании, которые приведены в таблице~\ref{tab:selfscore}, и имеют следующие значения:
\begin{align*}
	k_{a_1} & = 0.2 + 0.3 + 0.02 + 0.06 + 0.25 + 0.02 = 0.85, \\
	k_{a_2} & = 0.2 + 0.25 + 0.02 + 0.06 + 0.25 + 0.02 = 0.8, \\
	k_{a_3} & = 0.2 + 0.25 + 0.02 + 0.06 + 0.25 + 0.02 = 0.8, \\
	k_{a_4} & = 0.3 + 0.25 + 0.02 + 0.06 + 0.3 + 0.02 = 0.95, \\
	k_{a_5} & = 0.1 + 0.1 + 0.02 + 0.06 + 0.2 + 0.02 = 0.5.
\end{align*}

Коэффициент компетентности для каждого эксперта составляет:
\begin{align*}
	K_1=0.765, K_2=0.815, K_3=0.8, K_4=0.9, K_5=0.735.
\end{align*}

\subsection{Определить количественный состав ЭГ}
{
	\itshape
	Имеется 5 кандидатов в ЭГ.
	Коэффициент компетенции кандидатов взять из прошлой задачи.
	Необходимо рассчитать максимальную численность группы.
}

Максимальная численность экспертной группы может быть определена на основе неравенства:
\begin{align*}
	N_{\max} \leq \frac{3 \cdot \sum_{i=1}^{n} K_i}{2 \cdot K_{\max}},
\end{align*}
\begin{description}
	\item[где] $N_{\max}$ --- максимальное количество экспертов в группе;
	\item $K_i$ --- компетентность $i$-го эксперта из шкалы компетентности;
	\item $K_{\max}$ --- максимально возможная компетентность эксперта из шкалы компетентности.
\end{description}

Максимальная численность экспертной группы составила 6 человек ($6 \leq 6.691$).

\subsection{Решить задачу выбора состава экспертной группы на основе теории прецедентов}
{
	\itshape
	Имеется 5 специалистов-кандидатов, которые имеют по 7 характеристик.
	Характеристики и их численные показатели задать самостоятельно \\ (характеристики экспертов могут совпадать или же быть различными).

	Решить задачу выбора кандидатов по критерию важности. Рассчитать меру близости прецедентов по основным метрикам близости: Эвклидово \\ расстояние, Манхэттенская метрика, Мера сходства Хэмминга.
}

Значения характеристик каждого из экспертов заданы в таблицах~\ref{tab:expert_char} и~\ref{tab:expert_char_w}.

\begin{table}[H]
	\caption{Значения характеристик экспертов}
	\label{tab:expert_char}
	\begin{tabular}{|c|c|c|c|c|c|c|c|}
		\hline
		\multirow{2}{*}{Эксперт, № } & \multicolumn{7}{c|}{Характеристика}                               \\ \cline{2-8}
		                             & 1                                   & 2  & 3  & 4  & 5  & 6  & 7  \\ \hline
		1                            & 15                                  & 24 & 27 & 4  & 17 & 2  & 43 \\ \hline
		2                            & 11                                  & 35 & 4  & 29 & 33 & 46 & 6  \\ \hline
		3                            & 6                                   & 2  & 26 & 11 & 36 & 36 & 17 \\ \hline
		4                            & 35                                  & 25 & 46 & 13 & 23 & 18 & 45 \\ \hline
		5                            & 4                                   & 14 & 18 & 43 & 44 & 4  & 43 \\ \hline
	\end{tabular}
\end{table}

\begin{table}[H]
	\caption{Веса характеристик}
	\label{tab:expert_char_w}
	\begin{tabular}{|c|c|c|c|c|c|c|c|}
		\hline
		\multirow{2}{*}{} & \multicolumn{7}{c|}{Характеристика}                                            \\ \cline{2-8}
		                  & 1                                   & 2    & 3    & 4   & 5     & 6     & 7    \\ \hline
		$a_i$             & 0.2                                 & 0.05 & 0.15 & 0.3 & 0.025 & 0.125 & 0.15 \\ \hline
	\end{tabular}
\end{table}

Значения функции полезности можно рассчитать по формуле:
\begin{align*}
	m_i[V_i(x)] = \frac{V_i(x)^{\max} - V_i(x)}{V_i(x)^{\max} - V_i(x)^{\min}},
\end{align*}
\begin{description}
	\item[где] $V_i(x)$ --- значение характеристики $i$ эксперта $x$.
\end{description}

\begin{table}[H]
	\caption{Значения функции полезности для характеристик экспертов}
	\label{tab:expert_char_m}
	\begin{tabular}{|c|c|c|c|c|c|c|c|}
		\hline
        \multirow{2}{*}{Эксперт, № } & \multicolumn{7}{c|}{Полезность} \\ \cline{2-8}
         & 1 & 2 & 3 & 4 & 5 & 6 & 7 \\ \hline
        2 & 0.6451 & 0.3333 & 0.4523 & 1 & 1 & 1 & 0.0512 \\ \hline
        3 & 0.77410 & 0 & 1 & 0.3589 & 0.4074 & 0 & 1 \\ \hline
        4 & 0.9354 & 1 & 0.4761 & 0.820 & 0.2962 & 0.2272 & 0.7179 \\ \hline
        5 & 0 & 0.3030 & 0 & 0.7692 & 0.7777 & 0.6363 & 0 \\ \hline
        6 & 1 & 0.6363 & 0.6666 & 0 & 0 &	0.9545 & 0.0512 \\ \hline
	\end{tabular}
\end{table}

Рассчитаем оценки значимости претендентов по формуле \\ $\Phi(x) = \sum^n_{i=1} a_i m_i[V_i(x)]$:
\begin{align*}
	\Phi(x_1) = 0.6712, \\
	\Phi(x_2) = 0.5727, \\
	\Phi(x_3) = 0.6982, \\
	\Phi(x_4) = 0.3449, \\
	\Phi(x_5) = 0.4588.
\end{align*}

Упорядочим оценки значимости претендентов по убыванию их оценки значимости и получим последовательность $<x_3, x_1, x_2, x_5, x_4>$.

Рассчитаем меру близости прецедентов по основным метрикам близости.

\textbf{Эвклидово расстояние}:
\begin{align*}
	d_{ik} = (\sum^N_{j=1}(x_{ij} - x_{kj})^2)^{\frac{1}{2}},
\end{align*}
\begin{description}
	\item[где] $d_{ik}$ --- мера близости $i$-го и $k$-го кандидатов в эксперты;
	\item $x_{ij}$ --- количественное значение $i$-го эксперта по $j$-му критерию;
	\item $x_{kj}$ --- количественное значение $k$-го эксперта по $j$-му критерию.
\end{description}
\begin{align*}
	d_{1,2} = 69.6562,
	d_{1,3} = 52.9905,
	d_{1,4} = 33.7490,
	d_{1,5} = 50.5568,
	d_{2,3} = 46.3896, \\
	d_{2,4} = 71.4212,
	d_{2,5} = 64.3117,
	d_{3,4} = 55.2358,
	d_{3,5} = 54.7722,
	d_{4,5} = 58.3695.
\end{align*}

\textbf{Манхэттенская метрика}:
\begin{align*}
	d^{(l)}_{ik} = \sum^N_{j=1}|x_{ij} - x_{kj}|,
\end{align*}
\begin{align*}
	d_{1,2} = 160,
	d_{1,3} = 118,
	d_{1,4} = 73,
	d_{1,5} = 98,
	d_{2,3} = 102, \\
	d_{2,4} = 169,
	d_{2,5} = 146,
	d_{3,4} = 133,
	d_{3,5} = 120,
	d_{4,5} = 137.
\end{align*}

\textbf{Мера сходства Хэмминга}:
\begin{align*}
	\mu^H_{ij} = \frac{n_{ik}}{N},
\end{align*}
\begin{description}
	\item[где] $n_{ik}$ --- число совпадающих признаков у образцов $X_i$ и $X_k$;
	\item $N$ --- общее число признаков.
\end{description}
\begin{align*}
	\mu_{1,2}^H = \frac{7}{7} = 1,
	\mu_{1,3}^H = \frac{7}{7} = 1,
	\mu_{1,4}^H = \frac{7}{7} = 1,
	\mu_{1,5}^H = \frac{7}{7} = 1,
	\mu_{2,3}^H = \frac{7}{7} = 1, \\
	\mu_{2,4}^H = \frac{7}{7} = 1,
	\mu_{2,5}^H = \frac{7}{7} = 1,
	\mu_{3,4}^H = \frac{7}{7} = 1,
	\mu_{3,5}^H = \frac{7}{7} = 1,
	\mu_{4,5}^H = \frac{7}{7} = 1.
\end{align*}

\subsection{Сформировать эталонный вариант модели эксперта с использованием теории прецедентов}
{
	\itshape
	Имеется 5 специалистов-кандидатов в ЭГ.
	Значения параметров и весовые коэффициенты этих параметров представлены в таблице~\ref{tab:score}.

	Рассчитать компетентность кандидатов в ЭГ.
}

Эталонный вариант модели эксперта для ЭГ представим как кортеж  для дальнейшего формирования критерия отбора по методу прецедентов:
\begin{align*}
	U^t = \langle \langle u^t_{lt}; u^t_{stg}; u^t_{pub}; u^t_{vst}; u^t_{dl}; u^t_{us}; u^t_{zv}; u^t_{pt} \rangle ; \langle u^t_{sam} \rangle ; \langle u^t_{usp} \rangle ; \langle u^t_{imn} \rangle \rangle, t = \overline{1,n}
\end{align*}
или как ассоциативную свертку:
\begin{align*}
	U^t = \langle k_1 u^t_{lt} & + k_2 u^t_{stg} + k_3 u^t_{pub} + k_4 u^t_{vst} + k_5 u^t_{dl} + k_6 u^t_{us} + k_7 u^t_{zv}+ \\ &+ k_8 u^t_{pt} + k_9 u^t_{sam} + k_{10} u^t_{usp} + k_{11} u^t_{imn} \rangle, t = \overline{1,n}
\end{align*}
\begin{description}
	\item[где] $u^t_{lt}$ --- возраст эксперта;
	\item $u^t_{stg}$ --- стаж работы в <<проблемной области>>;
	\item $u^t_{pub}$ --- количество публикаций по проблеме;
	\item $u^t_{vst}$ --- количество выступлений, связанных с проблематикой решения задачи;
	\item $u^t_{dl}$ --- занимаемая должность;
	\item $u^t_{us}$ --- ученая степень;
	\item $u^t_{zv}$ --- научное звание;
	\item $u^t_{pt}$ --- количество патентов, свидетельств (связанных с решаемой проблемой);
	\item $u^t_{sam}$ --- самооценка компетентности;
	\item $u^t_{usp}$ --- количество успешных реализованных проектов;
	\item $u^t_{imn}$ --- характеристика эксперта другими экспертами;
	\item $\vec{k}$ --- весовые коэффициенты;
	\item $n$ --- количество экспертов в базе.
\end{description}

Комплексная оценка представляет упрощенный вариант оценивания качеств отдельных экспертов для ЭО и их формирования для базы прецедентов. Опишем составляющие выражения:
\begin{align*}
	u^t_{lt} & = \langle u^t_{lt} \rangle, \\
	u^t_{lt} & = k_{1,1}u^t_{lt},
\end{align*}
\begin{align*}
	u^t_{stg} = \langle u^t_{sm} ; u^t_{sd} \rangle \quad \text{или} \quad u^t_{stg} = \langle k_{2,1} u^t_{sm} ; k_{2,2} u^t_{sd} \rangle,
\end{align*}
\begin{description}
	\item[где] $u^t_{sm}$ --- в смежной области;
	\item $u^t_{sd}$ --- в данной проблемной области.
\end{description}
\begin{align*}
	u^t_{pub} = \langle u^t_{z3} ; u^t_{z5} ; u^t_{zsp} ; u^t_{zv} \rangle \quad \text{или} \quad u^t_{pub} = \langle k_{3,1} u^t_{z3} ; k_{3,2} u^t_{z5} ; k_{3,3} u^t_{zsp} ; k_{3,4} u^t_{zv} \rangle,
\end{align*}
\begin{description}
	\item[где] $u_{z3}$ --- за последние три года;
	\item $u^t_{z5}$ --- за последние пять лет;
	\item $u^t_{zsp}$ --- публикаций в специализированных журналах;
	\item $u^t_{zv}$ --- всего публикаций по проблеме.
\end{description}
\begin{align*}
	u^t_{vst} = \langle u^t_{vs} ; u^t_{vz} ; u^t_{vsp} \rangle \quad \text{или} \quad u^t_{vst} = \langle k_{4,1} u^t_{vs} ; k_{4,2} u^t_{vz} ; k_{4,3} u^t_{vsp} \rangle,
\end{align*}
\begin{description}
	\item[где] $u_{vs}$ --- выступления по решаемой проблемной области в пределах страны;
	\item $u^t_{vz}$ --- выступления на международном уровне;
	\item $u^t_{vsp}$ --- выступление и участие в специализированных семинарах, конференциях, симпозиумах.
\end{description}
\begin{align*}
	u^t_{dl} = \langle u^t_{zdl} ; u^t_{dlz} \rangle \quad \text{или} \quad u^t_{dl} = \langle k_{5,1} u^t_{zdl} ; k_{5,2} u^t_{dlz} \rangle,
\end{align*}
\begin{description}
	\item[где] $u^t_{zdl}$ --- занимаемая должность;
	\item $u^t_{dlz}$ --- должность, которую эксперт занимал ранее.
\end{description}
\begin{align*}
	u^t_{us} = \langle u^t_{kn} ; u^t_{dn} \rangle \quad \text{или} \quad u^t_{us} = \langle k_{6,1} u^t_{kn} ; k_{6,2} u^t_{dn} \rangle,
\end{align*}
\begin{description}
	\item[где] $u^t_{kn}$ --- ученая степень кандидата наук;
	\item $u^t_{dn}$ --- степень доктора наук.
\end{description}
\begin{align*}
	u^t_{zv} = \langle u^t_{zd} ; u^t_{zn} ; u^t_{zpf} \rangle \quad \text{или} \quad u^t_{zv} = \langle k_{7,1} u^t_{zd} ; k_{7,2} u^t_{zn}  ; k_{7,3} u^t_{zpf} \rangle,
\end{align*}
\begin{description}
	\item[где] $u^t_{zd}$ --- аттестат доцента;
	\item $u^t_{zn}$ --- аттестат старшего научного сотрудника;
	\item $u^t_{zpf}$ --- аттестат профессора.
\end{description}
\begin{align*}
	u^t_{pt} = \langle u^t_{pv} ; u^t_{pn} ; u^t_{ps} \rangle \quad \text{или} \quad u^t_{pt} = \langle k_{8,1} u^t_{pv} ; k_{8,2} u^t_{pn}  ; k_{8,3} u^t_{ps} \rangle,
\end{align*}
\begin{description}
	\item[где] $u^t_{pv}$ --- количество патентов внутри страны;
	\item $u^t_{pn}$ --- количество патентов за пределами страны;
	\item $u^t_{ps}$ --- количество патентов, оформленных единолично.
\end{description}
\begin{align*}
	u^t_{sam} = \langle u^t_{sk} ; u^t_{skp} \rangle \quad \text{или} \quad u^t_{sam} = \langle k_{9,1} u^t_{sk} ; k_{9,2} u^t_{skp} \rangle,
\end{align*}
\begin{description}
	\item[где] $u^t_{sk}$ --- самооценка по проблемной области;
	\item $u^t_{skp}$ --- общая самооценка компетентности.
\end{description}
\begin{align*}
	u^t_{usp} = \langle u^t_{usn} ; u^t_{usj} ; u^t_{usi} ; u^t_{usv} \rangle \quad \text{или} \quad u^t_{usp} = \langle k_{10,1} u^t_{usn} ; k_{10,2} u^t_{usj} ; k_{10,3} u^t_{usi} ; k_{10,4} u^t_{usv} \rangle,
\end{align*}
\begin{description}
	\item[где] $u^t_{usn}$ --- успешно реализованных проектов нет;
	\item $u^t_{usj}$ --- количество успешных проектов 10\%–25\% от общего числа всех проектов;
	\item $u^t_{usi}$ --- количество успешных проектов 26\%–70\% от общего числа всех проектов;
	\item $u^t_{usv}$ --- количество успешных проектов 71\%–100\% от общего числа всех проектов.
\end{description}
\begin{align*}
	u^t_{imn} & = \langle u^t_{izp} ; u^t_{ikr} ; u^t_{iuk} ; u^t_{ilz} ; u^t_{iak} \rangle \quad \text{или} \\ \quad u^t_{imn} &= \langle k_{11,1} u^t_{izp} ; k_{11,2} u^t_{ikr} ; k_{11,3} u^t_{iuk} ; k_{11,4} u^t_{ilz} ; k_{11,4} u^t_{iak} \rangle,
\end{align*}
\begin{description}
	\item[где] $u^t_{izp}$ --- оценка уровня знания эксперта;
	\item $u^t_{ikr}$ --- способность к коллективной работе;
	\item $u^t_{iuk}$ --- уровень конфликтности;
	\item $u^t_{ilz}$ --- личное знакомство с экспертом;
	\item $u^t_{iak}$ --- характеристика его анкетных данных.
\end{description}

Расчет компетентности специалистов-кандидатов в эксперты с использованием теории прецедентов представлен в таблице\ref{tab:task5_result}.

{
\small
\tabulinesep=1.2mm
\begin{longtabu} to \textwidth {|X[1,c]|X[1,c]|X[1,c]|X[1,c]|X[1,c]|X[1,c]|}
	\caption{Компетентность специалистов-кандидатов в эксперты}
	\label{tab:task5_result} \\
	\hline
	& Эксперт 1 & Эксперт 2 & Эксперт 3 & Эксперт 4 & Эксперт 5 \\ \hline
	\endfirsthead

	\caption*{Окончание таблицы \thetable{}}\\
	\hline
	& Эксперт 1 & Эксперт 2 & Эксперт 3 & Эксперт 4 & Эксперт 5 \\ \hline
	\endhead

	$u_{lt}$ & $0.0102$ & $0.0078$ & $0.0156$ & $0.0123$ & $0.018$ \\ \hline
	$u_{stg}$ & $0.0256$ & $0.0157$ & $0.042$ & $0.0339$ & $0.0308$ \\ \hline
	$u_{pub}$ & $0.0705$ & $0.0598$ & $0.0426$ & $0.0412$ & $0.1431$ \\ \hline
	$u_{vst}$ & $0.0689$ & $0.0979$ & $0.0386$ & $0.0863$ & $0.021$ \\ \hline
	$u_{dl}$ & $0.06$ & $0.045$ & $0.068$ & $0.068$ & $0.06$ \\ \hline
	$u_{us}$ & $0.0015$ & $0.0015$ & $0.0215$ & $0.0215$ & $0.0215$ \\ \hline
	$u_{zv}$ & $0.055$ & $0.03$ & $0.09$ & $0.055$ & $0.09$ \\ \hline
	$u_{pt}$ & $0.0146$ & $0.0132$ & $0.0285$ & $0.0193$ & $0.0391$ \\ \hline
	$u_{sam}$ & $0.018745$ & $0.01751$ & $0.022935$ & $0.024525$ & $0.022365$ \\ \hline
	$u_{usp}$ & $0.038$ & $0.038$ & $0.038$ & $0.001$ & $0.038$ \\ \hline
	$u_{imn}$ & $0.1329$ & $0.1135$ & $0.114$ & $0.1415$ & $0.1229$ \\ \hline
	$U$ & $0.495945$ & $0.43991$ & $0.521735$ & $0.504525$ & $0.606765$ \\ \hline
\end{longtabu}
}

\subsection{Выводы}

В ходе лабораторной работы был изучен процесс формирования экспертной группы. 
Были выполнены задачи подбора ЭГ, такие как:
\begin{itemize}
    \item формализация процесса формирования ЭГ с помощью задачи линейного программирования; 
    \item оценка компетентности экспертов на основе их анкетирования; 
    \item определение количественного состава ЭГ, а именно её максимальной численности; 
    \item выбора состава экспертной группы на основе теории прецедентов; 
    \item расчет компетентности экспертов в ЭГ.
\end{itemize} 

\end{document}
