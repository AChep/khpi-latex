% 16.02.2019 

Основные этапы задачи подготовки и принятия коллективного решения:
\begin{enumerate}
    \item Подготовка коллективного экспертного оценивания (диагностика, анализ проблемы, прогноз развития проблемы, решение технических задач).
    \item Формирование экспертной группы:
    \begin{itemize}
        \item определение круга компетентных экспертов;
        \item формирование состава экспертной группы.
    \end{itemize}
    \item Генерация экспертной информации.
    \item Экспертиза.
    \item Агрегация экспертных суждений.
\end{enumerate}

Влияние на экспертов имеющих высокие административные и профессиональные статусы. 

% 02.03.2010

Для определения $\lambda_\max$ можно записать:
\begin{equation}
    \omega (A-\lambda I)  = 0
    \label{eq:first}
\end{equation}

\begin{equation}
	\omega_1 (a_{11} - \alpha) + \omega_2 a_{12} + \dots + \omega_n a_{1n} = 0 \\
	\omega_1 a_{21} + \omega_2 (a_{22} - \alpha) + \dots + \omega_n (a_{2n} = 0 \\
	\omega_1 (a_{n1} + \omega_2 (a_{n2} + \dots + \omega_n (a_{nn} - \alpha) = 0
    \label{eq:second}
\end{equation}

Относительно $\lambda$ решаем уравнение и получаем $n$ корней. 
После этого выбираем самое большое и подставляем.

Допустим имеется 3 альтернативы  $c_1$,  $c_2$,  $c_3$.
\begin{align*}
    A = 
    \begin{bmatrix}
        1 & 5 & 4 \\
        \frac{1}{5} & 1 & \frac{4}{5} \\
        \frac{1}{4} & \frac{5}{4} & 1
    \end{bmatrix}
\end{align*}

Матрица является согласованной...

Исходя из того, что для полностью согласованной матрицы $A$ $\lambda_\max=n$ а малое изменение $a_{ij}$ вызывает небольшие изменения $\lambda_\max$ то мерой согласованности является отклонение $\lambda_\max$ от $n$.

$CI = (\lambda_\max - n) / (n - 1)$

При этом нетрудно заметить что $i$-е уравнение этой системы имеет вид:
\[
    \sum^n_{j=1}a_{ij}\omega_j=\lambda_\max \omega_i
\]
\[
    \sum_{i=1}^n (\sum^n_{j=1}a_{ij}\omega_j)= \lambda_\max \sum_{i=1}^n \omega_i = \lambda_\max
\]

Для оыенки достаточности в степени согласованности Саати 
\[
    CR = \frac{CI}{CIS}
\]

Метод анализа иерархий Саати является систематической процедурой для иерархического представления элементов определяющих суть любой проблемы.
Данный метод состоит в декомпозиции пролемы на все более простые составляющие части...

Семья среднего достатка хочет купить дом. Есть восемь критериев: 
- размеры: емкость, хранилихе, размеры комнат, число комнат, общая площадь.
- удобство автобусных маршрутов.
- окрестости: интенсивность движения, безопасность, хороший вид, низкие налоги.
- возраст дома.
- характеристика двора дома.
- современное оборудование.
- финансовые условия

% 06.04.2010

Ограничение автоматически выполняется при условии что $\Delta b^i_l >= 0$...

Найти некоторый вектор $(x^*, y^* u^*)$ ($x$ --- вариант развития, $y$ --- дополнительный ресурсы, $u$ -- пространство) которой бы минимизировал целевую функцию $F(x^*, y^* u^*) = \min F(x, y, u) \in \check{D_j}, u \in D^d_j$. 
Алгоритмы рещения этой задачи за конечное число шагов:
\[ (x^*, u^*), \vec{F}(x^*, u^*) = \min \vec{F}(x,u), \]
\[ (x, u) \in \check{D_0} \cap \tilde{D_0^d} \neq \oslash \]
\[ \vec{F}(x, u) = F_0(u) + \sum_{i \in I}f_i(x) \]
\[ \tilde{D^d_0} = too hard \]


RPD алгоритм системной оптимизации
% vcpisa_2014_61_3.pdf
Решение задачи 4 известными классическими методами оптимизации практически неразрешимо ввиду большой размерности.

Рассмотри общую задачу выпуклого программирования:
\[ \max f(x), g_i(x) \geq 0, i = \overline{i, m}, x ] in S \]

$f(x), g_i(x)$ --- вогнутые функции $\vec{x}$.
Будет считать что число огранчений $g_i$ достаточно далеко, тогда разумная стратегия решения состоит в релаксации (временном отбрасывании некоторых ограничений) и решение задачи. Если же она разрешима и если полученное рещение удовлетворяет релаксированным ограничениям то такое решение оптималь и для исходной задачи. В противном случае необходимо ввести больше ограничений обратно.

Релаксация явялется эффективной стратегией если относительно малое число ограничений является существенным. 

Будем считать что $M = \{ 1, 2, \dots m \}$, а $R \subset M$.

Шаг 1.
Решаем основную задачу определения вектора $u^{s\lambda}$ такого что $F_0(u^{s\lambda}) = \min_u \{ \sum_{h \in H} c_h^0 u_h : u \in D^d_0 \}$, $S=\lambda=1$.

Шаг 2.
Прямой подстановкой решения $u^{s\lambda}$  в условие (1) (область $D_0$) определяем существенные ограничения. 
Множество таки ограничений для каждой $i$-й функциональной подсистемы обозначим как $L_i^s$.

Шаг 3.
Введем некоторые параметры $\beta_{il}^{s\lambda}$ значения которых определяются следющим образом $\beta_{il}^{s\lambda} = \sum_{h \in H_i^s} R_{lh}^i u_h^{s\lambda}, l \in L_i^s, i \in I^s$. 
После происходит декомпозиция исходной задачи (4) на ряд отдельных подзадач.
При этом осуществляется релаксация как множества отдельных подсистем. 
В результате формируется множество новых задач значительно меньшей размерности, которые имеют вид:
\[
    F_i^s(x^i,y^i, \beta^{})
\]

Задача 7-10 рещиется для все

На основе одного из методов негладкой оптимизации (обобщенный градиентный спуск, метод элипсоиды) определяется новое значение .
Исходя из физической постановки задачи.
