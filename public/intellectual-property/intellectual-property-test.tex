\usepackage{tikz}

\counterwithout{figure}{section}
\counterwithout{table}{section}
\counterwithout{equation}{section}

\titleformat{\subsection}[block]
  {\bfseries\filcenter}{#1}{0cm}{}
\titlespacing{\subsection}{0cm}{21pt}{21pt}

\DeclareCaptionLabelFormat{gosttable}{Таблица #2}


\usepackage{pifont}

\setlist[enumerate,1]{noitemsep, topsep=0pt, wide, labelindent=0pt}
\setlist[enumerate,2]{noitemsep, topsep=0pt, wide, labelindent=\parindent, leftmargin=\parindent}

\newcommand{\cmark}{\ding{51} }%

\lhead{\small \selectfont Чепурний А.С., 2.КН201н.8а}

\begin{document}
\Ukrainian

\section*{дисциплина “интеллектуальная собственность”}

\begin{enumerate}
    \item Что понимают под правом интеллектуальной собственности?
    \begin{enumerate}
        \item патенты, свидетельства и другие охранные документы;
        \item \cmark закрепленные законом права на результаты интеллектуальной, творческой деятельности человека;
        \item изобретения, идеи, произведения искусства.
    \end{enumerate}
    \item Составляющие права интеллектуальной собственности:
    \begin{enumerate}
        \item вещественные и имущественные;
        \item личные и имущественные;
        \item \cmark личные неимущественные и имущественные.
    \end{enumerate}
    \item Какие из перечисленных объектов относятся к объектам авторского права?
    \begin{enumerate}
        \item \cmark произведения литературы, науки, искусства;
        \item изобретения, полезные модели, промышленные образцы;
        \item компьютерные программы, коммерческая тайна, торговые марки.
    \end{enumerate}
    \item Какие из перечисленных объектов относятся к объектам промышленной собственности?
    \begin{enumerate}
        \item изобретения, музыкальные произведения, коммерческая тайна;
        \item торговые марки, промышленные образцы, произведения изобразительного искусства;
        \item \cmark изобретения, полезные модели, торговые марки.
    \end{enumerate}
    \item Срок охраны объектов авторского права в Украине в общем случае составляет:
    \begin{enumerate}
        \item \cmark всю жизнь автора плюс 70 лет после его смерти;
        \item 20 лет;
        \item 50 лет с возможностью продления срока охраны.
    \end{enumerate}
    \item Срок действия охранного документа на изобретение в Украине составляет:
    \begin{enumerate}
        \item 10 лет;
        \item 15 лет;
        \item \cmark 20 лет.
    \end{enumerate}
    \item Срок действия охранного документа на торговую марку составляет:
    \begin{enumerate}
        \item 10 лет:
        \item 15 лет;
        \item \cmark 10 лет с правом продления.
    \end{enumerate}
    \item Критерии охранноспособности  изобретения:
    \begin{enumerate}
        \item \cmark новизна, изобретательский уровень, промышленная применимость;
        \item новизна, промышленная применимость, экономический эффект;
        \item новизна, полезность, оригинальность.
    \end{enumerate}
    \item Критерии охранноспособности полезной модели:
    \begin{enumerate}
        \item уровень техники, полезность;
        \item \cmark новизна, промышленная применимость;
        \item новизна, изобретательский уровень, промышленная применимость.
    \end{enumerate}
    \item Критерии охранноспособности промышленного образца:
    \begin{enumerate}
        \item оригинальность, новизна, промышленная применимость;
        \item новизна, промышленная применимость;
        \item \cmark новизна.
    \end{enumerate}
    \item Критерии охранноспособности сорта растения:
    \begin{enumerate}
        \item оригинальность, новизна, промышленная применимость;
        \item \cmark новизна, отличительность, однородность и стабильность;
        \item новизна.
    \end{enumerate}
    \item Критерии охранноспособности топографии интегральной микросхемы:
    \begin{enumerate}
        \item оригинальность, новизна, промышленная применимость;
        \item новизна, отличительность, однородность и стабильность;
        \item \cmark оригинальность.
    \end{enumerate}
    \item Изобретение  признается новым, если информация о нем  не стала общедоступной в мире:
    \begin{enumerate}
        \item \cmark за год до даты подачи заявки; 
        \item за шесть месяцев до даты подачи заявки;
        \item за 2 года до даты подачи заявки.
    \end{enumerate}
    \item Промышленный образец признается новым, если совокупность его существенных признаков не стала общедоступной в мире:
    \begin{enumerate}
        \item \cmark за год до даты подачи заявки; 
        \item за шесть месяцев до даты подачи заявки;
        \item за 2 года до даты подачи заявки.
    \end{enumerate}
    \item Объектом изобретения являются:
    \begin{enumerate}
        \item компьютерная программа, способ, вещество;
        \item \cmark продукт, процесс, применение известного продукта или процесса по новому назначению;
        \item топография интегральных микросхем, штамм микроорганизмов, сорт растений.
    \end{enumerate}
    \item Какие права относятся к личным неимущественным:
    \begin{enumerate}
        \item право владеть, пользоваться и распоряжаться;
        \item признание своего авторства, выбор псевдонима, противодействие искажению;
        \item \cmark признание своего авторства, право изменять и продавать объект.
    \end{enumerate}
    \item Какие права относятся к имущественным:
    \begin{enumerate}
        \item \cmark право владеть, пользоваться и распоряжаться;
        \item признание своего авторства, выбор псевдонима, противодействие искажению;
        \item признание своего авторства, право изменять и продавать объект.
    \end{enumerate}
    \item Согласно Гражданскому Кодексу Украины имущественные права на служебные произведения принадлежат:
    \begin{enumerate}
        \item автору этого произведения;
        \item работодателю;
        \item \cmark автору и работодателю совместно.
    \end{enumerate}
    \item Права на изобретение удостоверяются:
    \begin{enumerate}
        \item \cmark патентом;
        \item свидетельством;
        \item лицензией.
    \end{enumerate}
    \item Права на сорт растений удостоверяются:
    \begin{enumerate}
        \item \cmark патентом;
        \item свидетельством;
        \item лицензией.
    \end{enumerate}
    \item Права на топографию интегральной микросхемы удостоверяются:
    \begin{enumerate}
        \item патентом;
        \item \cmark свидетельством;
        \item лицензией.
    \end{enumerate}
    \item Патент на изобретение удостоверяет:
    \begin{enumerate}
        \item \cmark приоритет, авторство и право собственности на объект интеллектуальной собственности;
        \item право авторства;
        \item право распоряжаться объектом интеллектуальной собственности.
    \end{enumerate}
    \item Право собственности на торговые марки удостоверяется:
    \begin{enumerate}
        \item \cmark свидетельством;
        \item авторским свидетельством;
        \item патентом.
    \end{enumerate}
    \item Какой из перечисленных объектов промышленной собственности имеет практически неограниченный срок действия охранного документа?
    \begin{enumerate}
        \item изобретение;
        \item \cmark торговая марка;
        \item полезная модель.
    \end{enumerate}
    \item Каков срок охраны личных неимущественных прав?
    \begin{enumerate}
        \item 20 лет;
        \item 70 лет; 
        \item \cmark охраняются бессрочно.
    \end{enumerate}
    \item Каков срок охраны имущественных прав?
    \begin{enumerate}
        \item 10 лет;
        \item 15 лет;
        \item \cmark срок зависит от конкретного объекта.
    \end{enumerate}
    \item После прекращения действия свидетельства на знак для товаров и услуг право на повторную регистрацию знака в течение трех может осуществить:
    \begin{enumerate}
        \item любое физическое лицо;
        \item любое юридическое  лицо;
        \item \cmark только бывший собственник свидетельства.
    \end{enumerate}
    \item Объем правовой охраны, которая предоставляется свидетельством на знак для товаров и услуг, определяется следующим:
    \begin{enumerate}
        \item \cmark приведенным в свидетельстве изображением знака и перечнем товаров и услуг;
        \item приведенными в свидетельстве перечнем товаров и услуг.
        \item описанием знака для товаров и услуг. 
    \end{enumerate}
    \item Субъектом права на торговую марку может быть только 
    \begin{enumerate}
        \item юридическое лицо;
        \item физическое лицо;
        \item \cmark юридическое и физическое лицо.
    \end{enumerate}
    \item Объем правовой охраны промышленного образца  определяется 
    \begin{enumerate}
        \item \cmark совокупностью существенных признаков промышленного образца, изображенных на фотографиях (изображении) изделия;
        \item описанием промышленного образца;
        \item внешним видом промышленного образца.
    \end{enumerate}
    \item В результате регистрации авторского права на произведение выдается следующий охранный документ:
    \begin{enumerate}
        \item патент;
        \item свидетельство;
        \item \cmark лицензия.
    \end{enumerate}
    \item Объем прав, которые вытекают из патента на изобретение, определяют:
    \begin{enumerate}
        \item \cmark формула изобретения;
        \item реферат изобретения;
        \item описание изобретения.
    \end{enumerate}
    \item Что из перечисленного не может быть объектом авторского права:
    \begin{enumerate}
        \item произведения науки, лекции, интервью;
        \item \cmark денежные знаки, расписания поездов, государственные символы;
        \item компьютерные программы, базы данных, произведения живописи.
    \end{enumerate}
    \item Документ об оплате сбора за подачу заявки должен поступить
    \begin{enumerate}
        \item \cmark обязвтельно вместе с заявкой;
        \item вместе с заявкой или на протяжении двух месяцев от даты подачи заявки;
        \item на протяжении шести месяцев от даты подачи заявки. 
    \end{enumerate}
    \item На какой  объект может быть продолжен  срок действия патента:
    \begin{enumerate}
        \item на секретное изобретение;
        \item \cmark на лекарственное средство;
        \item на любой объект изобретения.
    \end{enumerate}
    \item По заявке на изобретение после установления даты ее подачи проводится;
    \begin{enumerate}
        \item формальная экспертиза;
        \item квалификационная экспертиза; 
        \item \cmark формальная экспертиза и затем квалификационная экспертиза.
    \end{enumerate}
    По заявке на промышленный образец после установления даты ее подачи проводится;
    \begin{enumerate}
        \item формальная экспертиза;
        \item квалификационная экспертизa; 
        \item \cmark формальная экспертиза и затем квалификационная экспертиза.
    \end{enumerate}
    \item По заявке на торговую марку после установления даты ее подачи проводится;
    \begin{enumerate}
        \item формальная экспертиза;
        \item квалификационная экспертиза; 
        \item \cmark формальная экспертиза и затем квалификационная экспертиза.
    \end{enumerate}
    \item Согласно Гражданскому Кодексу лицензия на использование объекта права интеллектуальной собственности может быть:
    \begin{enumerate}
        \item \cmark исключительной, единичной, неисключительной;
        \item исключительной,  неисключительной;
        \item исключительной, всеобщей, неисключительной. 
    \end{enumerate}
    \item Передача права во временное использование объекта интеллектуальной собственности оформляется 
    \begin{enumerate}
        \item свидетельством;
        \item \cmark лицензионным договором;
        \item в устной форме.
    \end{enumerate}
    \item Сторонами лицензионного договора являются: 
    \begin{enumerate}
        \item покупатель и продавец;
        \item \cmark лицензиар и лицензиат;
        \item арендодатель и арендатор.
    \end{enumerate}
    \item Лицензиар это:
    \begin{enumerate}
        \item \cmark Владелец исключительных прав на объект интеллектуальной собственности;
        \item Получатель прав на временное использование объекта интеллектуальной собственности.
    \end{enumerate}
    \item Лицензиат это:
    \begin{enumerate}
        \item Владелец исключительных прав на объект интеллектуальной собственности; 
        \item \cmark Получатель прав на временное использование объекта интеллектуальной собственности.
    \end{enumerate}
    \item Выдать лицензию на использование объекта интеллектуальной собственности может: 
    \begin{enumerate}
        \item \cmark владелец охранного документа;
        \item получатель прав на временное использование объекта интеллектуальной собственности;
        \item автор изобретения.
    \end{enumerate}
    \item Исключительная лицензия:
    \begin{enumerate}
        \item предусматривает, что в пределах переданных по лицензии прав использовать объект может и лицензиат, и лицензиар, кроме того, владелец оставляет за собой право выдавать лицензии другим лицам; 
        \item \cmark выдается только одному лицензиату, в пределах переданных по лицензии прав не допускается выдача лицензий на этот же объект другим лицам и использование объекта самим владельцем; 
        \item владелец сохраняет возможность использовать объект в объеме переданных по лицензии прав, но выдача других лицензий не допускается.
    \end{enumerate}
    \item Неисключительная лицензия:
    \begin{enumerate}
        \item \cmark предусматривает, что в пределах переданных по лицензии прав использовать объект может и лицензиат, и лицензиар, кроме того, владелец оставляет за собой право выдавать лицензии другим лицам; 
        \item выдается только одному лицензиату, в пределах переданных по лицензии прав не допускается выдача лицензий на этот же объект другим лицам и использование объекта самим владельцем; 
        \item владелец сохраняет возможность использовать объект в объеме переданных по лицензии прав, но выдача других лицензий не допускается.
    \end{enumerate}
    \item Единичная лицензия:
    \begin{enumerate}
        \item предусматривает, что в пределах переданных по лицензии прав использовать объект может и лицензиат, и лицензиар, кроме того, владелец оставляет за собой право выдавать лицензии другим лицам; 
        \item выдается только одному лицензиату, в пределах переданных по лицензии прав не допускается выдача лицензий на этот же объект другим лицам и использование объекта самим владельцем; 
        \item \cmark владелец сохраняет возможность использовать объект в объеме переданных по лицензии прав, но выдача других лицензий не допускается.
    \end{enumerate}
    \item Если вид лицензии в договоре не указан, то считается, что лицензия
    \begin{enumerate}
        \item \cmark неисключительная;
        \item исключительная;
        \item единичная.
    \end{enumerate}
    \item Лицензионный платеж роялти подразумевает выплату вознаграждения:
    \begin{enumerate}
        \item \cmark равными частями на протяжении действия договора;
        \item одноразовую выплату до начала массового выпуска продукции;
        \item частичную выплату до начала производства и выплату части, что осталась, на протяжении действия договора.
    \end{enumerate}
    \item Лицензионный паушальный  платеж подразумевает выплату вознаграждения:
    \begin{enumerate}
        \item равными частями на протяжении действия договора;
        \item \cmark одноразовую выплату до начала массового выпуска продукции;\item частичную выплату до начала производства и выплату части, что осталась, на протяжении действия договора.
    \end{enumerate}
    \item Авторским правом охраняется:
    \begin{enumerate}
        \item авторская идея;
        \item \cmark выражение (форма) авторской идеи;
        \item авторская идея и выражение (форма) авторской идеи.
    \end{enumerate}
    \item Для возникновения и осуществления авторского права
    \begin{enumerate}
        \item \cmark требуется регистрация;
        \item не требуется регистрации;
        \item требуется  внесение в реестр.
    \end{enumerate}
    \item Какой знак может использовать обладатель исключительных авторских прав для оповещения о своих правах:
    \begin{enumerate}
        \item \cmark ©;
        \item ®;
        \item ™.
    \end{enumerate}
    \item Какой знак может использовать обладатель исключительных прав на торговую марку для оповещения о своих правах:
    \begin{enumerate}
        \item ©;
        \item ®;
        \item \cmark ™.
    \end{enumerate}
    \item Какой знак может использовать заявитель, подавший заявку на торговую марку для оповещения о своих правах:
    \begin{enumerate}
        \item ©;
        \item \cmark ®;
        \item ™.
    \end{enumerate}
    \item Классификация патентных документов проводится согласно:
    \begin{enumerate}
        \item МКПО;
        \item \cmark МПК;
        \item МКТУ.
    \end{enumerate}
    \item Классификация промышленных образцов проводится согласно:
    \begin{enumerate}
        \item \cmark МКПО;
        \item МПК;
        \item МКТУ.
    \end{enumerate}
    \item Классификация торговых марок проводится согласно:
    \begin{enumerate}
        \item МКПО;
        \item МПК;
        \item \cmark МКТУ.
    \end{enumerate}
    \item Где происходит официальная публикация о выданных охранных документах:
    \begin{enumerate}
        \item в журнале «Интеллектуальная собственность»;
        \item \cmark в Официальном бюллетене «Промышленная собственность»;
        \item в бюллетене «Авторское право».
    \end{enumerate}
    \item Каждый библиографический элемент, находящийся на первой странице патентного документа, идентифицируется:
    \begin{enumerate}
        \item \cmark двузначными цифровыми кодами ИНИД;
        \item двумя буквами;
        \item буквой и цифрой в круглых скобках.
    \end{enumerate}
\end{enumerate}

\end{document}
