\usepackage{tikz}

\counterwithout{figure}{section}
\counterwithout{table}{section}
\counterwithout{equation}{section}

\titleformat{\subsection}[block]
  {\bfseries\filcenter}{#1}{0cm}{}
\titlespacing{\subsection}{0cm}{21pt}{21pt}

\DeclareCaptionLabelFormat{gosttable}{Таблица #2}

\usepackage{float}
\usepackage{pgfplots}
\usepackage{graphicx}
\usepackage{multirow}
\usepackage{amssymb,amsfonts,amsmath,amsthm}

\usepackage{listings}
\lstset{basicstyle=\footnotesize\ttfamily,breaklines=true}
\lstset{language=Matlab}

\lstdefinelanguage{Python}{
  keywords={and, break, class, continue, def, yield, del, elif, else, except, exec, finally, for, from, global, if, import, in, lambda, not, or, pass, print, raise, return, try, while, assert, with},
  keywordstyle=\color{NavyBlue}\bfseries,
  ndkeywords={True, False},
  ndkeywordstyle=\color{BurntOrange}\bfseries,
  emph={as},
  emphstyle={\color{OrangeRed}},
  identifierstyle=\color{black},
  sensitive=true,
  commentstyle=\color{gray}\ttfamily,
  comment=[l]{\#},
  morecomment=[s]{/*}{*/},
  stringstyle=\color{ForestGreen}\ttfamily,
  morestring=[b]',
  morestring=[s]{"""*}{*"""},
}


\newcommand{\labnumber}{1} % first lab
\usepackage{tikz}

\counterwithout{figure}{section}
\counterwithout{table}{section}
\counterwithout{equation}{section}

\titleformat{\subsection}[block]
  {\bfseries\filcenter}{#1}{0cm}{}
\titlespacing{\subsection}{0cm}{21pt}{21pt}

\DeclareCaptionLabelFormat{gosttable}{Таблица #2}

\newcommand{\khpistudentgroup}{2.КН201н.8а}
\newcommand{\khpistudentname}{Чепурний~А.~С.}

\newcommand{\khpidepartment}{Програмна інженерія та інформаційні технології управління}
\newcommand{\khpititlewhat}{
	Розрахунково-графічне завдання \\
	з предмету <<Фреймворки та платформи>>
}
\newcommand{\khpititlewho}{
	Виконав: \\
	\hspace*{\parindent} ст. групи \khpistudentgroup \\
	\hspace*{\parindent} \khpistudentname \\
	Перевірила: \\
	\hspace*{\parindent} к. т. н., вик. каф. ПІІТУ \\
	\hspace*{\parindent} Добряк~В.~С. \\
}


\graphicspath{{figures/}}

\begin{document}
\Ukrainian

\begin{titlepage}

\begin{center}
	МІНІСТЕРСТВО ОСВІТИ І НАУКИ УКРАЇНИ \\
	НАЦІОНАЛЬНИЙ ТЕХНІЧНИЙ УНІВЕРСИТЕТ \\
	«ХАРКІВСЬКИЙ ПОЛІТЕХНІЧНИЙ ІНСТИТУТ» \\
	Кафедра <<\khpidepartment>> \\
\end{center}

\vspace{6cm}

\begin{center}
	\khpititlewhat
\end{center}

\vspace{3cm}

\begin{addmargin}[10cm]{0cm}
	\khpititlewho
\end{addmargin}

\vspace{\fill}

\begin{center}
	Харків \the\year
\end{center}

\end{titlepage}

\addtocounter{page}{1}

\section{Разработка интеллектуальной системы на языке SWI-Prolog}
\textbf{Цель работы}: разработка интеллектуальной системы на языке SWI-Prolog.

\textbf{Вихідні дані}: изучение технологии поиска решения задач с учетом интеллектуальных принципов, на базе основных механизмов, включая сопоставление с образцом, древовидного представления структур данных и автоматического перебора с возвратами, приобретение навыков их практической реализации на языке SWI-Prolog.

\textbf{Объект}: интеллектуальная система.

\textbf{Предмет}: изучение технологии поиска решения задач с учетом интеллектуальных принципов, на базе основных механизмов, включая сопоставление с образцом, древовидного представления структур данных и автоматического перебора с возвратами, приобретение навыков их практической реализации на языке SWI-Prolog.

\subsection{Индивидуальное задание} 
Разработайте интеллектуальную систему на языке SWI-Prolog моделирующую элемент \texttt{XOR} из элементов \texttt{AND}, \texttt{OR} и \texttt{NOT}, а затем проверьте его работу с помощью разработанной программы. 

Любая логическая цепь может быть представлена в SWI-Prolog при помощи предикатов, где они описывают соотношения между входными и выходными сигналами. 
Основные элементы логики должны быть описаны при помощи таблицы истинности значений. 
Основные элементы логики могут быть описаны с помощью не только внешних, но и внутренних связей.

\subsection{Ход работы}
Схема элемента представлена на рисунке~\ref{fig:xor}.

\begin{figure}[H]
    \centering
        \includegraphics[width=0.6\textwidth]{xor}
    \caption{\texttt{XOR}}
    \label{fig:xor}
\end{figure}

Для описания элемента необходимо три логических функции: \texttt{AND}, \texttt{OR} и \texttt{NOT}. Реализуем две первые с помощью правил:
\begin{lstlisting}
and(A,B) :- A,B.
or(A,B) :- A;B.
\end{lstlisting}

Функция \texttt{XOR} будет:
\begin{lstlisting}
xor(A,B) :- or(and(not(A), B), and(A, not(B))).
\end{lstlisting}

После упрощения функции \texttt{XOR} была получена такая функция:
\begin{lstlisting}
xor(A,B) :- or(A,B), not(and(A,B)).
\end{lstlisting}

Полученная функция была протестирована для всех возможных параметров и соответствует функции \texttt{XOR}:
\begin{lstlisting}
> xor(false, false).
false
> xor(true, false).
true
> xor(false, true).
true
> xor(true, true).
false
\end{lstlisting}


\subsection*{Выводы}
В процессе выполнения лабораторной работы были получены знания о программировании на языке Prolog c использованием среды SWI-Prolog и была создана программа для решения задачи согласно индивидуальному заданию.

\end{document}
