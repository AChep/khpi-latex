\section{Related work}
Clickbait Detection is a relatively new domain. Researchers have been exploring automatic approaches to perform Clickbait Detection. However, most of the attempts focused on news headlines.

An interesting model was proposed by Zhou~\cite{Chopra2017} for Clickbait Challenge 2017~\cite{Clickbait2016}. He employed an automatic approach to find clickbait in the tweet stream. Self-attentive neural network was employed for the first time in this article to examine each tweet’s probability of click baiting.

Another successful method~\cite{Joulin2016}, which was proposed in Clickbait Challenge 2017~\cite{Clickbait2016}, used an ensemble of Linear SVM models. They showed that how the clickbait can be detected using a small ensemble of linear models.

A machine learning based clickbait detection system was designed in~\cite{Mikolov2013}. They extracted six novel features for clickbait detection and they showed in their results that these novel features are the most effective ones for detecting clickbait news headlines. Totally, they extracted 331 features but to prevent overfitting, they just kept 180 features among them. They used all the fields in the dataset such as titles, passages, and keywords in their model for extracting
these features.