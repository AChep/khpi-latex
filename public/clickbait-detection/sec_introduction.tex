\section{Introduction}
With the bulk of information propagation and consumption moving to online platforms, the media world is suffering through a drastic shift. 
In the case of traditional, offline media sources, where a user preference would be static i.e. loyalty to a particular news source would be unwavering. However, now the Internet offers the readers a plethora of choices ranging from local to international, mainstream to niche, popular to alternative, editorials to blogs. 
This has forced traditional news outlets to change tack in order to stay in business. 
An added bonus is that online sources generally do not have a subscription charge, choosing to generate revenue through advertisements on the page. To stay relevant in the midst of such competition whilst staying afloat, online journalism has adopted a new technique to attract users --- clickbait.

Clickbaits employ the cognitive phenomenon known as Curiosity Gap, where the headlines provide forward referenced cues which generate sufficient curiosity compelling the reader to click the link and fill their curiosity gap. 
Clickbaits eventually cause disappointment, as they are not able to live up to the promises made in the headline. 
Due to their heavy use in online journalism, it is important to develop techniques that automatically detect and combat clickbaits.

The object of the paper is the clickbait in headlines.

The subject of the paper is the process of detecting click-bait headlines using a neural network. 

The goal of the paper is employing an automatic approach to detect clickbait articles in the news stream. We present the attempt of applying LSTM to evaluate each title's level of clickbaiting.