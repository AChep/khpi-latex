\documentclass[a4paper,14pt,oneside,final]{extarticle}
\usepackage[top=2cm, bottom=2cm, left=3cm, right=1cm]{geometry}
\usepackage{scrextend}

\usepackage[T2A,T1]{fontenc}
\usepackage[ukrainian,russian,english]{babel}
\usepackage{tempora}
\usepackage{fontspec}
\setmainfont{tempora}

% Зачем: Отключает использование изменяемых межсловных пробелов.
% Почему: Так не принято делать в текстах на русском языке.
\frenchspacing

\usepackage{indentfirst}
\setlength{\parindent}{1.25cm}
\renewcommand{\baselinestretch}{1.5}

% Header
\usepackage{fancyhdr}
\pagestyle{fancy}
\fancyhead{}
\fancyfoot{}
\fancyhead[R]{\small \selectfont \thepage}
\renewcommand{\headrulewidth}{0pt}

% Captions
\usepackage{chngcntr}
\counterwithin{figure}{section}
\counterwithin{table}{section}
\usepackage[tableposition=top]{caption}
\usepackage{subcaption}
\DeclareCaptionLabelFormat{gostfigure}{Рисунок #2}
\DeclareCaptionLabelFormat{gosttable}{Таблиця #2}
\DeclareCaptionLabelSeparator{gost}{~---~}
\captionsetup{labelsep=gost}
\captionsetup[figure]{labelformat=gostfigure}
\captionsetup[table]{labelformat=gosttable}
\renewcommand{\thesubfigure}{\asbuk{subfigure}}

% Sections
\usepackage[explicit]{titlesec}
\newcommand{\sectionbreak}{\clearpage}

\titleformat{\section}
  {\centering}{\thesection \quad}{0pt}{\MakeUppercase{#1}}
\titleformat{\subsection}[block]
  {\bfseries}{\thesubsection \quad #1}{0cm}{}

\titlespacing{\section} {0cm}{0cm}{21pt}
\titlespacing{\subsection} {\parindent}{21pt}{0cm}
\titlespacing{\subsubsection} {\parindent}{0cm}{0cm}

% Lists
\usepackage{enumitem}
\renewcommand\labelitemi{--}
\setlist[itemize]{noitemsep, topsep=0pt, wide}
\setlist[enumerate]{noitemsep, topsep=0pt, wide, label=\arabic*}
\setlist[description]{labelsep=0pt, noitemsep, topsep=0pt, leftmargin=2\parindent, labelindent=\parindent, labelwidth=\parindent, font=\normalfont}

% Toc
\usepackage{tocloft}
\tocloftpagestyle{fancy}
\renewcommand{\cfttoctitlefont}{}
\setlength{\cftbeforesecskip}{0pt}
\renewcommand{\cftsecfont}{}
\renewcommand{\cftsecpagefont}{}
\renewcommand{\cftsecleader}{\cftdotfill{\cftdotsep}}

\usepackage{float}
\usepackage{pgfplots}
\usepackage{graphicx}
\usepackage{multirow}
\usepackage{amssymb,amsfonts,amsmath,amsthm}
\usepackage{csquotes}

\usepackage{listings}
\lstset{basicstyle=\footnotesize\ttfamily,breaklines=true}
\lstset{language=Matlab}

\usepackage[
	backend=biber,
	sorting=none,
	language=auto,
	autolang=other
]{biblatex}
\DeclareFieldFormat{labelnumberwidth}{#1}

\lstdefinelanguage{Python}{
  keywords={and, break, class, continue, def, yield, del, elif, else, except, exec, finally, for, from, global, if, import, in, lambda, not, or, pass, print, raise, return, try, while, assert, with},
  keywordstyle=\color{NavyBlue}\bfseries,
  ndkeywords={True, False},
  ndkeywordstyle=\color{BurntOrange}\bfseries,
  emph={as},
  emphstyle={\color{OrangeRed}},
  identifierstyle=\color{black},
  sensitive=true,
  commentstyle=\color{gray}\ttfamily,
  comment=[l]{\#},
  morecomment=[s]{/*}{*/},
  stringstyle=\color{ForestGreen}\ttfamily,
  morestring=[b]',
  morestring=[s]{"""*}{*"""},
}


\newcommand{\labnumber}{2} % second lab
\documentclass[a4paper,14pt,oneside,final]{extarticle}
\usepackage[top=2cm, bottom=2cm, left=3cm, right=1cm]{geometry}
\usepackage{scrextend}

\usepackage[T2A,T1]{fontenc}
\usepackage[ukrainian,russian,english]{babel}
\usepackage{tempora}
\usepackage{fontspec}
\setmainfont{tempora}

% Зачем: Отключает использование изменяемых межсловных пробелов.
% Почему: Так не принято делать в текстах на русском языке.
\frenchspacing

\usepackage{indentfirst}
\setlength{\parindent}{1.25cm}
\renewcommand{\baselinestretch}{1.5}

% Header
\usepackage{fancyhdr}
\pagestyle{fancy}
\fancyhead{}
\fancyfoot{}
\fancyhead[R]{\small \selectfont \thepage}
\renewcommand{\headrulewidth}{0pt}

% Captions
\usepackage{chngcntr}
\counterwithin{figure}{section}
\counterwithin{table}{section}
\usepackage[tableposition=top]{caption}
\usepackage{subcaption}
\DeclareCaptionLabelFormat{gostfigure}{Рисунок #2}
\DeclareCaptionLabelFormat{gosttable}{Таблиця #2}
\DeclareCaptionLabelSeparator{gost}{~---~}
\captionsetup{labelsep=gost}
\captionsetup[figure]{labelformat=gostfigure}
\captionsetup[table]{labelformat=gosttable}
\renewcommand{\thesubfigure}{\asbuk{subfigure}}

% Sections
\usepackage[explicit]{titlesec}
\newcommand{\sectionbreak}{\clearpage}

\titleformat{\section}
  {\centering}{\thesection \quad}{0pt}{\MakeUppercase{#1}}
\titleformat{\subsection}[block]
  {\bfseries}{\thesubsection \quad #1}{0cm}{}

\titlespacing{\section} {0cm}{0cm}{21pt}
\titlespacing{\subsection} {\parindent}{21pt}{0cm}
\titlespacing{\subsubsection} {\parindent}{0cm}{0cm}

% Lists
\usepackage{enumitem}
\renewcommand\labelitemi{--}
\setlist[itemize]{noitemsep, topsep=0pt, wide}
\setlist[enumerate]{noitemsep, topsep=0pt, wide, label=\arabic*}
\setlist[description]{labelsep=0pt, noitemsep, topsep=0pt, leftmargin=2\parindent, labelindent=\parindent, labelwidth=\parindent, font=\normalfont}

% Toc
\usepackage{tocloft}
\tocloftpagestyle{fancy}
\renewcommand{\cfttoctitlefont}{}
\setlength{\cftbeforesecskip}{0pt}
\renewcommand{\cftsecfont}{}
\renewcommand{\cftsecpagefont}{}
\renewcommand{\cftsecleader}{\cftdotfill{\cftdotsep}}

\newcommand{\khpistudentgroup}{КН-34г}
\newcommand{\khpistudentname}{Чепурний~А.~С.}

\newcommand{\khpidepartment}{Програмна інженерія та інформаційні технології управління}
\newcommand{\khpititlewhat}{
	Лабораторна робота №\labnumber \\
	з предмету <<Моделювання систем>>
}
\newcommand{\khpititlewho}{
	Виконав: \\
	\hspace*{\parindent} ст. групи \khpistudentgroup \\
	\hspace*{\parindent} \khpistudentname \\
	Перевірила: \\
	\hspace*{\parindent} ст. в. каф. ПІІТУ \\
	\hspace*{\parindent} Єршова~С.~І. \\
	\hspace*{\parindent} ас. каф. ПІІТУ \\
	\hspace*{\parindent} Литвинова~Ю.~С. \\
}



\usepackage{longtable,tabu}

\graphicspath{{figures/}}

\begin{document}
\Ukrainian

\begin{titlepage}

\begin{center}
	МІНІСТЕРСТВО ОСВІТИ І НАУКИ УКРАЇНИ \\
	НАЦІОНАЛЬНИЙ ТЕХНІЧНИЙ УНІВЕРСИТЕТ \\
	«ХАРКІВСЬКИЙ ПОЛІТЕХНІЧНИЙ ІНСТИТУТ» \\[0.5cm]
	Кафедра <<\khpidepartment>> \\
\end{center}

\vspace{6cm}

\begin{center}
	\khpititlewhat
\end{center}

\vspace{3cm}

\begin{addmargin}[10cm]{0cm}
	\khpititlewho
\end{addmargin}

\vspace{\fill}

\begin{center}
	Харків \the\year
\end{center}

\end{titlepage}

\addtocounter{page}{1}

\section{Дистрибутивы и экосистема Hadoop}
\subsection*{Цель}
Проанализировать основные дистрибутивы Hadoop.
\subsection*{Задачи}
\begin{itemize}
    \item построить	отчет,	содержащий	сравнительную характеристику основных дистрибутивов Hadoop;
    \item проанализировать недостатки дистрибутивов, круг задач на которых дистрибутивы не применимы;
    \item определить оптимальный дистрибутив для решения задач.
\end{itemize}

\subsection{Сравнительная характеристика популярных дистрибутивов Hadoop}
\subsubsection{MapR}
MapR не имеет такого огромного сообщества разработчиков, как ранее рассмотренные проекты. Поэтому возможности проект в области внесения исправлений и определения стратегии развития Hadoop ограничены.

С точки зрения перспектив развития, MapR использует явно альтернативный подход к поддержке Hadoop. С самого начала проекта, было решено, что HDFS не подходит для корпоративного хранилища; вместо этого была разработана собственная распределенная файловая система, которая обладает достаточно интересными возможностями, такими как POSIX- согласованность (поддержка прямого доступа для записи и атомарных операций), высокая доступность, мониторинг NFS, зеркалирование данных, мгновенные снимки.

Некоторые из этих возможностей стали доступны в Hadoop 2, но MapR обладал этими возможностями с самого начала, что делает реализацию функций в MapR достаточно надежной. Следует отметить, что отдельные части стека технологий MapR, такие как файловая система или реализация HBase, являются закрытыми проектами и платными. Это оказывает влияние на возможности инженеров проекта по выявлению ошибок, их исправлению. В противоположность такому подходу, подходы компаний Cloudera и Hortonworks предполагают использование открытого исходного кода.

\subsubsection{Cloudera}
Cloudera является самым штатным дистрибутивом Hadoop, и на него приходится большое количество установок. Дуг Каттинг, который вместе с Майком Каферелла создали Hadoop, является главным архитектором на Cloudera. Это означает, что исправления и пожелания имеют больше шансов быть рассмотренными в Cloudera по сравнению с Hadoop. Кроме поддержки непосредственно Hadoop, Cloudera внедряет новшества в данной области путем разработки проектов, исправляющих слабые места Hadoop. Один из ярких проектов является проект Impala, который предлагает систему SQL-on- Hadoop, аналогичный Hive, но фокусирующий основное внимание на практически мгновенном учете пользовательских пожеланий и опыта разработчиков (в противоположность Hive). Существует большое количество различных проектов, поддерживаемых в рамках Cloudera: Flume (система распределения и хранения логов), Sqoop (перенос реляционных данных в систему Hadoop), Cloudera Search (предлагает механизм индексирования данных в реальном времени).

\subsubsection{Hortonworks}
Hortonworks также достаточно популярный проект. Как и Cloudera он предлагает большое количество преимуществ в области поддержки.
С точки зрения развития, Hortonworks немного отличается от Cloudera. Например, что касается Hive: если в Cloudera пошли по пути разработки полностью нового решения SQL-on-Hadoop, то в Hortonworks использовали исходный код Hive и удалили недостатки, связанные с высокой латентностью, добавив дополнительные возможности, связанные, например, с поддержкой ACID. Hortonworks является двигателем в области проектирования нового поколения платформы YARN, которая является ключевой технологией Hadoop. В Hortonworks используется Apache Ambari для администрирования. Данный проект уделяет основное внимание разработке и расширению экосистемы Apache, что несет огромные преимущества всему сообществу разработчиков, так как позволяет всем использовать инструментарий без необходимости заключения контрактов.

\subsection{Определение оптимального дистрибутива}
В таблице 1 приведено краткое сравнение вышеупомянутых дистрибутивов.

{
	\tabulinesep=1.2mm
	\begin{longtabu} to \textwidth {|X[1,l]|X[1,l]|X[1,l]|X[1,l]|}
		\caption{Cравнительная характеристика дистрибутивов MapR, Cloudera и Hortonworks}
		\label{tab:economy_total} \\
		\hline
		& Hortonworks & Cloudera & MapR \\
		\hline
		\endfirsthead
		\caption*{Закінчення таблиці \thetable{}}\\
		\hline
		& Hortonworks & Cloudera & MapR \\
		\hline
		\endhead

        Стратегия перезапуска &	Восстановление после отказа &	Восстановление после отказа	& Самовосстановление при множественных сбоях \\ \hline
        Стратегия восстановления задач MapReduce & Полный перезапуск задачи & Полный перезапуск задачи & Продолжение с момента остановки \\ \hline
        Обновление & Через запланированную блокировку & Параллельное обновление без ограничений доступа & Параллельное обновление без ограничений доступа \\ \hline
        Дубликация & Только данные & Только данные & Данные + метаданные \\ \hline
        Контрольные снимки & Только для закрытых файлов & Только для закрытых файлов & Для всех файлов согласно состоянию в конкретный момент времени \\ \hline
        Стратегия резервирования данных на случай аварий & Нет & Копирование файлов по расписанию & Зеркализация \\ \hline
	\end{longtabu}
}

Таким образом, все их перечисленных дистрибутивов могут быть в равной степени оценены как подходящие для предприятий, не имеющих критических ограничений к внедрению системы управления большими данными

\subsection*{Выводы}
В ходе выполнения лабораторной работы были рассмотрены особенности работы с технологией Hadoop в общем, а также рассмотрены дистрибутивы данной технологии, их положительные и отрицательные стороны, и характерные задачи, решаемые тем или иным дистрибутивом.

\end{document}
