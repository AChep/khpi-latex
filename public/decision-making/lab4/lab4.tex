\usepackage{tikz}

\counterwithout{figure}{section}
\counterwithout{table}{section}
\counterwithout{equation}{section}

\titleformat{\subsection}[block]
  {\bfseries\filcenter}{#1}{0cm}{}
\titlespacing{\subsection}{0cm}{21pt}{21pt}

\DeclareCaptionLabelFormat{gosttable}{Таблица #2}

\usepackage{float}
\usepackage{pgfplots}
\usepackage{graphicx}
\usepackage{multirow}
\usepackage{amssymb,amsfonts,amsmath,amsthm}

\usepackage{listings}
\lstset{basicstyle=\footnotesize\ttfamily,breaklines=true}
\lstset{language=Matlab}


\newcommand{\labnumber}{4} % third lab
\usepackage{tikz}

\counterwithout{figure}{section}
\counterwithout{table}{section}
\counterwithout{equation}{section}

\titleformat{\subsection}[block]
  {\bfseries\filcenter}{#1}{0cm}{}
\titlespacing{\subsection}{0cm}{21pt}{21pt}

\DeclareCaptionLabelFormat{gosttable}{Таблица #2}

\newcommand{\khpistudentgroup}{2.КН201н.8а}
\newcommand{\khpistudentname}{Чепурний~А.~С.}

\newcommand{\khpidepartment}{Програмна інженерія та інформаційні технології управління}
\newcommand{\khpititlewhat}{
	Розрахунково-графічне завдання \\
	з предмету <<Фреймворки та платформи>>
}
\newcommand{\khpititlewho}{
	Виконав: \\
	\hspace*{\parindent} ст. групи \khpistudentgroup \\
	\hspace*{\parindent} \khpistudentname \\
	Перевірила: \\
	\hspace*{\parindent} к. т. н., вик. каф. ПІІТУ \\
	\hspace*{\parindent} Добряк~В.~С. \\
}


\usepackage{systeme}
\usepackage{longtable,tabu}
\usepackage{multirow}
\usepackage{array,multirow}
\usepackage{pdflscape}
\usepackage{afterpage}
\usepackage{bm}

\graphicspath{{../figures/}}

\begin{document}
\Ukrainian

\begin{titlepage}

\begin{center}
	МІНІСТЕРСТВО ОСВІТИ І НАУКИ УКРАЇНИ \\
	НАЦІОНАЛЬНИЙ ТЕХНІЧНИЙ УНІВЕРСИТЕТ \\
	«ХАРКІВСЬКИЙ ПОЛІТЕХНІЧНИЙ ІНСТИТУТ» \\
	Кафедра <<\khpidepartment>> \\
\end{center}

\vspace{6cm}

\begin{center}
	\khpititlewhat
\end{center}

\vspace{3cm}

\begin{addmargin}[10cm]{0cm}
	\khpititlewho
\end{addmargin}

\vspace{\fill}

\begin{center}
	Харків \the\year
\end{center}

\end{titlepage}

\addtocounter{page}{1}

\textbf{Тема роботи}: Рішення багатокритеріальної задачі лінійного програмування методом обмежень.

\textbf{Завдання для виконання}
\begin{itemize}
	\item Вирішити таку багатокритеріальну задачу лінійного програмування методом обмежень одним способом, описаним в прикладі
	\item Для кожного рішення подати такі результати:
	\begin{enumerate}
		\item чисельне значення вектора вагових коефіцієнтів;
		\item чисельні значення перетворених критеріїв;
		\item добутку вектора вагових коефіцієнтів на відповідні чисельні значення перетворених критеріїв;
		\item чисельні значення змінних, в тому числі додатково введеної змінної;
		\item чисельні значення цільових функцій.
	\end{enumerate}
	\item Порівняти отримані результати з результатами, отриманими за допомогою теорем: Карліна, Гермейера і третьої теореми по знаходженню ефективних альтернатив (з відповідними значеннями вектора вагових коефіцієнтів). 
	Провести аналіз отриманих результатів.
\end{itemize}


\begin{align*} 
	f_1(\vec{x}) &= x_1 \to \max, \\
	f_2(\vec{x}) &= x_2 \to \max, \\
	f_3(\vec{x}) &= x_3 \to \max,
\end{align*}
% limits
\[
\systeme{
x_1 + x_2 + x_3 \leq 4,
3x_2 - x_3 \leq 6,
x_1 \geq 0,
x_2 \geq 0,
x_3 \geq 0
}
.
\]

\subsection{Формування багатокритерільної задачі лінійного програмування}

З використанням раніш визначеної багатокритеріальної задачі в загальному вигляді сформуємо багатокритеріальну задачу, що належить до класу задач лінійного програмування. 
У цьому випадку особливістю багатокритеріальної задачі буде вимога про дотримання властивості лінійності для множини функцій мети $\{ f_i(x), i \in I \}$
та множини функцій $\{ \phi_j(x), j \in J\}$, які визначають обмеження задачі.

З урахуванням вищевикладеного формально багатокритеріальна задача лінійного програмування може бути подана в наступному вигляді:

\begin{gather*}
	f_i(x) = \sum_{l \in L}{c_l^i x_l} \to \max, i \in I_1, \\
	f_i(x) = \sum_{l \in L}{c_l^i x_l} \to \min, i \in I_2, \\
	\sum_{l \in L}{a_{jl} x_l * b_j}, j \in J,
\end{gather*}

де * -знак типу $\leq, \geq або =$.

Для розв'язання цієї задачі застосуємо метод обмежень. 
Відповідно до методу обмежень необхідно провести перетворення вихідної множини функцій мети до безрозмірного вигляду одним із припустимих варіантів (1-2). 
Тоді компромісним розв'язанням задачі застосуємо метод обмежень. 
Відповідно до методу обмежень необхідно провести перетворення вихідної множини функцій мети до безрозмірного вигляду одним із припустимих варіантів (1-2). 

Тоді компромісним розв'язанням, одержання якого може забезпечувати метод обмежень, у рамках багатокритеріальної задачі, що розглядається, буде таке ефективне розв'язання, для якого зважені відносні втрати будуть одинакові та мінімальні, тобто:
$p_1 \omega_1 (f_1(x)) = p_2 \omega_2 (f_1(x)) = ... = p_m \omega_m (f_m(x)) = k_{0 (\min)}$,
де $M$ - потужність множини $I$, $\{p_i, i \in I\}$ - вектор вагових коефіцієнтів з компонентами 
$p_i > 0, \sum_{i \in I}{p_i} = 1$.

Відповідно до методу обмежень компромісний розв'язок, що шукається, може бути знайдений з розв'язання системи лінійних нерівностей:
\begin{gather*}
	f_i(x) \geq f_{i}^0 - \frac{k_0}{p_i}(f_{i}^0 - f_{i(\min)}) , i \in I_1, \\
	f_i(x) \leq f_{i}^0 + \frac{k_0}{p_i}(f_{i(\max)} - f_{i}^0 ), i \in I_2, \\
	\sum_{l \in L}{a_{jl} x_l * b_j}, j \in J,
\end{gather*}

Для мінімального значення $k_0$, при якому ця система ще є спільною. Розв'язання системи(20) еквівалентне розв'язанню наступної задачі лінійного програмування:

\begin{gather*}
	k_0 \to \min, \\
\end{gather*}
при обмеженнях 
\begin{gather*}
	d_{11} x_1 	+ d_{12} x_2 + ... + d_{1N} x_N + d_{1 N+1} K_0 + d_1 \geq 0, \\
	d_{i1} x_1 	+ d_{i2} x_2 + ... + d_{iN} x_N + d_{i N+1} K_0 + d_i \geq 0, \\
	d_{M1} x_1 	+ d_{M2} x_2 + ... + d_{MN} x_N + d_{M N+1} K_0 + d_M \geq 0, \\
	\sum_{l \in L}{a_{jl} x_l * b_j}, j \in J \\
\end{gather*}

де $N$ - потужність множини $L$ та
\begin{gather*}
	
	d_{il} = \begin{cases}
		p_i c_{i} ^ i, l \in L, i \in I_1, \\
		-p_i c_{i} ^ i, ; \in L, i \in I_2, 
	\end{cases} 
	
	d_{i, N+1} = \begin{cases}
		f_{i}^0 - f_{i(\min)}, i \in I_1, \\
		f_{i(\max)} - f_{i}^0 , i \in I_2,
	\end{cases} \\

	d_{i} = \begin{cases}
		- p_i f_{i} ^ 0, i \in I_1, \\
		p_i f_{i} ^ 0, i \in I_2, 
	\end{cases} \\

\end{gather*}

У методі обмежень спочатку відшукується мінімально можливе значення параметра $k_0$, при якому система обмежень є спільною. 
Якщо рішення не єдине, тобто альтернативи еквівалентні з точністю до $\varepsilon$ за значенням параметра $k_0$, то вибір 
компромісної альтернативи здійснюється за допомогою критерію.

Особливістю методу обмежень є те, що він не залежить від вигляду функціональної залежності $\{ f_i(x), i \in I \}$ та множини припустимих варіантів альтернатив 
$A$. Потрібно тільки для кожної конкретної задачі мати ефективні способи перевірки системи нерівностей.  

\end{document}
