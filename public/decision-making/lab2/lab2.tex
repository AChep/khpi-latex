\documentclass[a4paper,14pt,oneside,final]{extarticle}
\usepackage[top=2cm, bottom=2cm, left=3cm, right=1cm]{geometry}
\usepackage{scrextend}

\usepackage[T2A,T1]{fontenc}
\usepackage[ukrainian,russian,english]{babel}
\usepackage{tempora}
\usepackage{fontspec}
\setmainfont{tempora}

% Зачем: Отключает использование изменяемых межсловных пробелов.
% Почему: Так не принято делать в текстах на русском языке.
\frenchspacing

\usepackage{indentfirst}
\setlength{\parindent}{1.25cm}
\renewcommand{\baselinestretch}{1.5}

% Header
\usepackage{fancyhdr}
\pagestyle{fancy}
\fancyhead{}
\fancyfoot{}
\fancyhead[R]{\small \selectfont \thepage}
\renewcommand{\headrulewidth}{0pt}

% Captions
\usepackage{chngcntr}
\counterwithin{figure}{section}
\counterwithin{table}{section}
\usepackage[tableposition=top]{caption}
\usepackage{subcaption}
\DeclareCaptionLabelFormat{gostfigure}{Рисунок #2}
\DeclareCaptionLabelFormat{gosttable}{Таблиця #2}
\DeclareCaptionLabelSeparator{gost}{~---~}
\captionsetup{labelsep=gost}
\captionsetup[figure]{labelformat=gostfigure}
\captionsetup[table]{labelformat=gosttable}
\renewcommand{\thesubfigure}{\asbuk{subfigure}}

% Sections
\usepackage[explicit]{titlesec}
\newcommand{\sectionbreak}{\clearpage}

\titleformat{\section}
  {\centering}{\thesection \quad}{0pt}{\MakeUppercase{#1}}
\titleformat{\subsection}[block]
  {\bfseries}{\thesubsection \quad #1}{0cm}{}

\titlespacing{\section} {0cm}{0cm}{21pt}
\titlespacing{\subsection} {\parindent}{21pt}{0cm}
\titlespacing{\subsubsection} {\parindent}{0cm}{0cm}

% Lists
\usepackage{enumitem}
\renewcommand\labelitemi{--}
\setlist[itemize]{noitemsep, topsep=0pt, wide}
\setlist[enumerate]{noitemsep, topsep=0pt, wide, label=\arabic*}
\setlist[description]{labelsep=0pt, noitemsep, topsep=0pt, leftmargin=2\parindent, labelindent=\parindent, labelwidth=\parindent, font=\normalfont}

% Toc
\usepackage{tocloft}
\tocloftpagestyle{fancy}
\renewcommand{\cfttoctitlefont}{}
\setlength{\cftbeforesecskip}{0pt}
\renewcommand{\cftsecfont}{}
\renewcommand{\cftsecpagefont}{}
\renewcommand{\cftsecleader}{\cftdotfill{\cftdotsep}}

\usepackage{float}
\usepackage{pgfplots}
\usepackage{graphicx}
\usepackage{multirow}
\usepackage{amssymb,amsfonts,amsmath,amsthm}
\usepackage{csquotes}

\usepackage{listings}
\lstset{basicstyle=\footnotesize\ttfamily,breaklines=true}
\lstset{language=Matlab}

\usepackage[
	backend=biber,
	sorting=none,
	language=auto,
	autolang=other
]{biblatex}
\DeclareFieldFormat{labelnumberwidth}{#1}


\newcommand{\labnumber}{2} % first lab
\documentclass[a4paper,14pt,oneside,final]{extarticle}
\usepackage[top=2cm, bottom=2cm, left=3cm, right=1cm]{geometry}
\usepackage{scrextend}

\usepackage[T2A,T1]{fontenc}
\usepackage[ukrainian,russian,english]{babel}
\usepackage{tempora}
\usepackage{fontspec}
\setmainfont{tempora}

% Зачем: Отключает использование изменяемых межсловных пробелов.
% Почему: Так не принято делать в текстах на русском языке.
\frenchspacing

\usepackage{indentfirst}
\setlength{\parindent}{1.25cm}
\renewcommand{\baselinestretch}{1.5}

% Header
\usepackage{fancyhdr}
\pagestyle{fancy}
\fancyhead{}
\fancyfoot{}
\fancyhead[R]{\small \selectfont \thepage}
\renewcommand{\headrulewidth}{0pt}

% Captions
\usepackage{chngcntr}
\counterwithin{figure}{section}
\counterwithin{table}{section}
\usepackage[tableposition=top]{caption}
\usepackage{subcaption}
\DeclareCaptionLabelFormat{gostfigure}{Рисунок #2}
\DeclareCaptionLabelFormat{gosttable}{Таблиця #2}
\DeclareCaptionLabelSeparator{gost}{~---~}
\captionsetup{labelsep=gost}
\captionsetup[figure]{labelformat=gostfigure}
\captionsetup[table]{labelformat=gosttable}
\renewcommand{\thesubfigure}{\asbuk{subfigure}}

% Sections
\usepackage[explicit]{titlesec}
\newcommand{\sectionbreak}{\clearpage}

\titleformat{\section}
  {\centering}{\thesection \quad}{0pt}{\MakeUppercase{#1}}
\titleformat{\subsection}[block]
  {\bfseries}{\thesubsection \quad #1}{0cm}{}

\titlespacing{\section} {0cm}{0cm}{21pt}
\titlespacing{\subsection} {\parindent}{21pt}{0cm}
\titlespacing{\subsubsection} {\parindent}{0cm}{0cm}

% Lists
\usepackage{enumitem}
\renewcommand\labelitemi{--}
\setlist[itemize]{noitemsep, topsep=0pt, wide}
\setlist[enumerate]{noitemsep, topsep=0pt, wide, label=\arabic*}
\setlist[description]{labelsep=0pt, noitemsep, topsep=0pt, leftmargin=2\parindent, labelindent=\parindent, labelwidth=\parindent, font=\normalfont}

% Toc
\usepackage{tocloft}
\tocloftpagestyle{fancy}
\renewcommand{\cfttoctitlefont}{}
\setlength{\cftbeforesecskip}{0pt}
\renewcommand{\cftsecfont}{}
\renewcommand{\cftsecpagefont}{}
\renewcommand{\cftsecleader}{\cftdotfill{\cftdotsep}}

\newcommand{\khpistudentgroup}{КН-34г}
\newcommand{\khpistudentname}{Чепурний~А.~С.}

\newcommand{\khpidepartment}{Програмна інженерія та інформаційні технології управління}
\newcommand{\khpititlewhat}{
	Лабораторна робота №\labnumber \\
	з предмету <<Моделювання систем>>
}
\newcommand{\khpititlewho}{
	Виконав: \\
	\hspace*{\parindent} ст. групи \khpistudentgroup \\
	\hspace*{\parindent} \khpistudentname \\
	Перевірила: \\
	\hspace*{\parindent} ст. в. каф. ПІІТУ \\
	\hspace*{\parindent} Єршова~С.~І. \\
	\hspace*{\parindent} ас. каф. ПІІТУ \\
	\hspace*{\parindent} Литвинова~Ю.~С. \\
}



\usepackage{systeme}
\usepackage{longtable,tabu}
\usepackage{multirow}
\usepackage{array,multirow}
\usepackage{pdflscape}
\usepackage{afterpage}
\usepackage{bm}

\graphicspath{{figures/}}

\begin{document}
\Ukrainian

\begin{titlepage}

\begin{center}
	МІНІСТЕРСТВО ОСВІТИ І НАУКИ УКРАЇНИ \\
	НАЦІОНАЛЬНИЙ ТЕХНІЧНИЙ УНІВЕРСИТЕТ \\
	«ХАРКІВСЬКИЙ ПОЛІТЕХНІЧНИЙ ІНСТИТУТ» \\[0.5cm]
	Кафедра <<\khpidepartment>> \\
\end{center}

\vspace{6cm}

\begin{center}
	\khpititlewhat
\end{center}

\vspace{3cm}

\begin{addmargin}[10cm]{0cm}
	\khpititlewho
\end{addmargin}

\vspace{\fill}

\begin{center}
	Харків \the\year
\end{center}

\end{titlepage}

\addtocounter{page}{1}

\textbf{Тема роботи}: розв'язання багатокритеріальної задачі щодо знаходження ефективних альтернатив за допомогою теореми Гермейєра.

\textbf{Завдання для виконання}: вирішити наступну задачу багатокритеріальної оптимізації:
Вариант №1

\begin{align*}
k_{u_1}=0.68, k_{u_2}=0.83.
\end{align*}

{
	\small
	\tabulinesep=1.2mm
	\begin{longtabu} to \textwidth {|X[12,l]|X[1,c]X[1,c]X[1,c]|X[1,c]X[1,c]X[1,c]|X[1,c]X[1,c]X[1,c]|X[1,c]X[1,c]X[1,c]|X[1,c]X[1,c]X[1,c]|}  
		\caption{Самооценка экспертов}
		\label{tab:selfscore} \\
		\hline
		\multirow{3}{*}{Источник аргументации} & \multicolumn{15}{c|}{Уровень влияния источника на мнение эксперта} \\ \cline{2-16}
		& \multicolumn{3}{c|}{Эксперт 1} & \multicolumn{3}{c|}{Эксперт 2} & \multicolumn{3}{c|}{Эксперт 3} & \multicolumn{3}{c|}{Эксперт 4} & \multicolumn{3}{c|}{Эксперт 5} \\ \cline{2-16}
		& A & B & C & A & B & C & A & B & C & A & B & C & A & B & C \\ \hline
		\endfirsthead

		\caption*{Окончание таблицы \thetable{}}\\
	    \hline
		\multirow{3}{*}{Источник аргументации} & \multicolumn{15}{c|}{Уровень влияния источника на мнение эксперта} \\ \cline{2-16}
		& \multicolumn{3}{c|}{Эксперт 1} & \multicolumn{3}{c|}{Эксперт 2} & \multicolumn{3}{c|}{Эксперт 3} & \multicolumn{3}{c|}{Эксперт 4} & \multicolumn{3}{c|}{Эксперт 5} \\ \cline{2-16}
		& A & B & C & A & B & C & A & B & C & A & B & C & A & B & C \\ \hline
		\endhead

	 	Проведенный экспертом теоретический анализ данной проблемы 
		& & \checkmark & & & \checkmark & & & \checkmark & & \checkmark & & & & & \checkmark \\ \hline
		Производственный опыт эксперта, связанный с решаемой проблемой & \checkmark & & & & \checkmark & & & \checkmark & & & \checkmark & & & & \checkmark \\ \hline
		Участие в семинарах, совещаниях в своей стране по исследуемой проблеме & & & \checkmark & & & \checkmark & \checkmark & & & & \checkmark & & & \checkmark & \\ \hline
		Знакомство с работами зарубежных авторов по рассматриваемой проблеме & & & \checkmark & & & \checkmark & \checkmark & & & & & \checkmark & & \checkmark & \\ \hline
		Количество проектов, в подготовке, реализации и экспертизе которых эксперт принимал участие & & \checkmark & & & \checkmark & & & \checkmark & & \checkmark & & & & & \checkmark \\ \hline
		Влияние интуиции эксперта на принимаемые решения & & \checkmark & & \checkmark & & & & \checkmark & & & \checkmark & & \checkmark & & \\ \hline
	\end{longtabu}
}

{
	\small
	\tabulinesep=1.2mm
	\begin{longtabu} to \textwidth {|X[1,c]|X[1,c]|X[1,c]|X[1,c]|X[1,c]|X[1,c]|X[1,c]|}  
		\caption{Параметры экспертов}
		\label{tab:score} \\
		\hline
		$k_i$  & $u$ & Эксперт 1 & Эксперт 2 & Эксперт 3 & Эксперт 4 & Эксперт 5 \\ \hline
		\endfirsthead

		\caption*{Окончание таблицы \thetable{}}\\
	    \hline
		$k_i$  & $u$ & Эксперт 1 & Эксперт 2 & Эксперт 3 & Эксперт 4 & Эксперт 5 \\ \hline
		\endhead

	 	$0.0003$ & $u_{lt} $ & $34$ & $26$ & $52$ & $41$ & $60$ \\ \hline
	 	$0.0008$ & $u_{sm} $ & $2$ & $14$ & $15$ & $3$ & $16$ \\ \hline
	 	$0.0015$ & $u_{sd} $ & $16$ & $3$ & $20$ & $21$ & $12$ \\ \hline
	 	$0.0020$ & $u_{z3} $ & $12$ & $14$ & $2$ & $2$ & $32$ \\ \hline
	 	$0.0010$ & $u_{z5} $ & $16$ & $23$ & $24$ & $10$ & $39$ \\ \hline
	 	$0.0009$ & $u_{zsp} $ & $5$ & $2$ & $14$ & $8$ & $9$ \\ \hline
	 	$0.0010$ & $u_{zv} $ & $26$ & $7$ & $2$ & $20$ & $32$ \\ \hline
	 	$0.0015$ & $u_{vs} $ & $5$ & $12$ & $5$ & $2$ & $3$ \\ \hline
	 	$0.0020$ & $u_{vz} $ & $1$ & $2$ & $4$ & $7$ & $0$ \\ \hline
	 	$0.0033$ & $u_{vsp} $ & $18$ & $23$ & $7$ & $21$ & $5$ \\ \hline
	 	$0.0080$ & $u_{zdl} $ & $4$ & $3$ & $5$ & $5$ & $4$ \\ \hline
	 	$0.0070$ & $u_{dlz} $ & $4$ & $3$ & $4$ & $4$ & $4$ \\ \hline
	 	$0.0015$ & $u_{kn} $ & $1$ & $1$ & $1$ & $1$ & $1$ \\ \hline
	 	$0.0200$ & $u_{dn} $ & $0$ & $0$ & $1$ & $1$ & $1$ \\ \hline
	 	$0.0250$ & $u_{zd} $ & $1$ & $0$ & $1$ & $1$ & $1$ \\ \hline
	 	$0.0300$ & $u_{zn} $ & $1$ & $1$ & $1$ & $1$ & $1$ \\ \hline
	 	$0.0350$ & $u_{zpf} $ & $0$ & $0$ & $1$ & $0$ & $1$ \\ \hline
	 	$0.0015$ & $u_{pv} $ & $6$ & $4$ & $15$ & $3$ & $13$ \\ \hline
	 	$0.0018$ & $u_{pn} $ & $2$ & $4$ & $0$ & $6$ & $2$ \\ \hline
	 	$0.0020$ & $u_{ps} $ & $1$ & $0$ & $3$ & $2$ & $8$ \\ \hline
		$0.0250$ & $u_{sk} $ & $0.7$ & $0.65$ & $0.87$ & $0.93$ & $0.84$ \\ \hline
	 	$0.0015$ & $u_{skp} $ & $0.83$ & $0.84$ & $0.79$ & $0.85$ & $0.91$ \\ \hline
	 	$-0.0005$ & $u_{usn} $ & $0$ & $0$ & $0$ & $0$ & $0$ \\ \hline
	 	$0.0003$ & $u_{usi} $ & $0$ & $0$ & $0$ & $0$ & $0$ \\ \hline
	 	$0.0010$ & $u_{usj} $ & $0$ & $0$ & $0$ & $1$ & $0$ \\ \hline
	 	$0.0380$ & $u_{usv} $ & $1$ & $1$ & $1$ & $0$ & $1$ \\ \hline
	 	$0.0100$ & $u_{izp} $ & $5$ & $4$ & $5$ & $5$ & $4$ \\ \hline
	 	$0.0086$ & $u_{ikr} $ & $4$ & $5$ & $5$ & $5$ & $4$ \\ \hline
	 	$-0.0015$ & $u_{iuk} $ & $1$ & $1$ & $2$ & $1$ & $1$ \\ \hline
	 	$0.0100$ & $u_{ilz} $ & $1$ & $0$ & $0$ & $1$ & $1$ \\ \hline
	 	$0.0080$ & $u_{iak} $ & $5$ & $4$ & $3$ & $5$ & $5$ \\ \hline
	\end{longtabu}
}


\subsection{Математична постановка задачі багатокритеріальної оптимізації в загальному вигляді}

У загальному випадку формально задача багатокритеріальної оптимізації, ключовою особливістю якої є суперечливість множини функцій мети (критеріїв), може бути подана в наступному вигляді:
\begin{gather*} 
	f_i(\vec{x}) \to \max, i \in I_1, \\
	f_i(\vec{x}) \to \min, i \in I_2, \\
	\varphi_j(\vec{x}) \leq 0, j \in J.
\end{gather*}
\begin{description}
	\item[де] $I_1$ та $I_2$ --- множини індексів функцій мети $f_i(\vec{x})$, які відповідно максимізуються та мінімізуються, причому $I=I_1 \cup I_2$;
	\item $J$ --- множина індексів функцій $\varphi_j(\vec{x})$, що визначають обмеження задачі та формують множину припустимих варіантів альтернатив $A = \{ \varphi_j(\vec{x}) \leq 0, j \in J \}$;
	\item $\vec{x}$ --- вектор змінних задачі багатокритеріальної оптимізації, з яким пов’яжемо поняття альтернативи --- варіанта розв’язку, що задовольняє обмеження задачі і є способом досягнення поставлених цілей.
\end{description}

\subsection{Математична постановка однокритеріального еквіваленту вихідної багатокритеріальної задачі відповідно до теореми Гермейєра в загальному вигляді}

Основні положення теореми Гермейєра формулюються не для первісно заданої множини функцій мети $\{f_i(\vec{x}),i \in I\}$, а для множини функцій $\{\omega_i(\vec{x})=\omega_i(f_i(\vec{x})), i \in I\}$, що складається з монотонних перетворень окремих функцій мети $f_i(\vec{x})$, які приводять їх до безрозмірного вигляду.

За останні можна взяти одну з монотонних функцій такого вигляду:
\begin{equation}\label{omega1}
\omega^1_i(f_i(\vec{x})) = \systeme[][:]{
\cfrac{f_i^0-f_i(\vec{x})}{f_i^0 - f_{i(\min)}},i \in I_1
:
\cfrac{f_i(\vec{x})-f_i^0}{f_{i(\max)} - f_i^0},i \in I_2
}
,
\end{equation}
\begin{equation}\label{omega2}
\omega^2_i(f_i(\vec{x})) = \systeme[][:]{
\cfrac{f_i^0-f_i(\vec{x})}{f_i^0},i \in I_1
:
\cfrac{f_i(\vec{x})-f_i^0}{f_i^0},i \in I_2
}
,
\end{equation}
\begin{equation}\label{omega3}
\omega^3_i(f_i(\vec{x}))=\omega^j_i(f_i(\vec{x}))^\mu, i \in I, j \in \{1,2\}
.
\end{equation}
\begin{description}
	\item[де] $f_{i(\min)}$, $f_{i(\max)}$ --- найменші і найбільші значення функцій мети, які відповідно максимізуються і мінімізуються на множині припустимих варіантів альтернатив;
	\item $f_i^0$ --- оптимальне значення $i$-ї функції мети на множині припустимих варіантів альтернатив;
	\item $\mu$ --- число, що визначає степінь, на яку підноситься перетворення~\eqref{omega1}~або~\eqref{omega2}.
\end{description}

Нехай $x^*$ --- ефективна альтернатива множини функцій мети $\{\omega_i(f_i(\vec{x})), i \in I\}$, причому нехай $\{\omega_i(f_i(\vec{x}))>0, i \in I\}$. 
Тоді існує вектор $\vec{\rho}$ з компонентами $\rho_i > 0, \sum_{i \in I}\rho_i=1$ такий, що критерій 
\[
F(\vec{x})=\max_{i \in I} \rho_i \omega_i(f_i(\vec{x}))
\]
досягає мінімуму на множині припустимих варіантів альтернатів $A$, при $x=x^*$.

Особливістю даної теореми є той факт, що ніякі умови на вигляд функцій $\omega_i(f_i(\vec{x}))$ і обмежень, що описують множину припустимих варіантів альтернитив $A$, не накладаються.

Таким чином, множина ефективних альтернатив може бути знайдена з використанням теореми Гермейєра шляхом розв'язання наступної задачі:
\[
F(\vec{x})=\max_{i \in I} \rho_i\omega_i(f_i(\vec{x})) \to \min
\]
\begin{description}
    \item[де] $\omega_i(f_i(\vec{x}))$ --- монотонне перетворення $i$-ї функції мети $f_i(x)$, побудоване на основі одного із припустимих варіантів~\eqref{omega1},~\eqref{omega2}~та~\eqref{omega3}.
\end{description}

\subsection{Математична постановка задачі багатокритеріальної оптимізації згідно з виданим завданням}

Згідно виданого завдання задача багатокритеріальної оптимізації прийме наступний вигляд:
Вариант №1

\begin{align*}
k_{u_1}=0.68, k_{u_2}=0.83.
\end{align*}

{
	\small
	\tabulinesep=1.2mm
	\begin{longtabu} to \textwidth {|X[12,l]|X[1,c]X[1,c]X[1,c]|X[1,c]X[1,c]X[1,c]|X[1,c]X[1,c]X[1,c]|X[1,c]X[1,c]X[1,c]|X[1,c]X[1,c]X[1,c]|}  
		\caption{Самооценка экспертов}
		\label{tab:selfscore} \\
		\hline
		\multirow{3}{*}{Источник аргументации} & \multicolumn{15}{c|}{Уровень влияния источника на мнение эксперта} \\ \cline{2-16}
		& \multicolumn{3}{c|}{Эксперт 1} & \multicolumn{3}{c|}{Эксперт 2} & \multicolumn{3}{c|}{Эксперт 3} & \multicolumn{3}{c|}{Эксперт 4} & \multicolumn{3}{c|}{Эксперт 5} \\ \cline{2-16}
		& A & B & C & A & B & C & A & B & C & A & B & C & A & B & C \\ \hline
		\endfirsthead

		\caption*{Окончание таблицы \thetable{}}\\
	    \hline
		\multirow{3}{*}{Источник аргументации} & \multicolumn{15}{c|}{Уровень влияния источника на мнение эксперта} \\ \cline{2-16}
		& \multicolumn{3}{c|}{Эксперт 1} & \multicolumn{3}{c|}{Эксперт 2} & \multicolumn{3}{c|}{Эксперт 3} & \multicolumn{3}{c|}{Эксперт 4} & \multicolumn{3}{c|}{Эксперт 5} \\ \cline{2-16}
		& A & B & C & A & B & C & A & B & C & A & B & C & A & B & C \\ \hline
		\endhead

	 	Проведенный экспертом теоретический анализ данной проблемы 
		& & \checkmark & & & \checkmark & & & \checkmark & & \checkmark & & & & & \checkmark \\ \hline
		Производственный опыт эксперта, связанный с решаемой проблемой & \checkmark & & & & \checkmark & & & \checkmark & & & \checkmark & & & & \checkmark \\ \hline
		Участие в семинарах, совещаниях в своей стране по исследуемой проблеме & & & \checkmark & & & \checkmark & \checkmark & & & & \checkmark & & & \checkmark & \\ \hline
		Знакомство с работами зарубежных авторов по рассматриваемой проблеме & & & \checkmark & & & \checkmark & \checkmark & & & & & \checkmark & & \checkmark & \\ \hline
		Количество проектов, в подготовке, реализации и экспертизе которых эксперт принимал участие & & \checkmark & & & \checkmark & & & \checkmark & & \checkmark & & & & & \checkmark \\ \hline
		Влияние интуиции эксперта на принимаемые решения & & \checkmark & & \checkmark & & & & \checkmark & & & \checkmark & & \checkmark & & \\ \hline
	\end{longtabu}
}

{
	\small
	\tabulinesep=1.2mm
	\begin{longtabu} to \textwidth {|X[1,c]|X[1,c]|X[1,c]|X[1,c]|X[1,c]|X[1,c]|X[1,c]|}  
		\caption{Параметры экспертов}
		\label{tab:score} \\
		\hline
		$k_i$  & $u$ & Эксперт 1 & Эксперт 2 & Эксперт 3 & Эксперт 4 & Эксперт 5 \\ \hline
		\endfirsthead

		\caption*{Окончание таблицы \thetable{}}\\
	    \hline
		$k_i$  & $u$ & Эксперт 1 & Эксперт 2 & Эксперт 3 & Эксперт 4 & Эксперт 5 \\ \hline
		\endhead

	 	$0.0003$ & $u_{lt} $ & $34$ & $26$ & $52$ & $41$ & $60$ \\ \hline
	 	$0.0008$ & $u_{sm} $ & $2$ & $14$ & $15$ & $3$ & $16$ \\ \hline
	 	$0.0015$ & $u_{sd} $ & $16$ & $3$ & $20$ & $21$ & $12$ \\ \hline
	 	$0.0020$ & $u_{z3} $ & $12$ & $14$ & $2$ & $2$ & $32$ \\ \hline
	 	$0.0010$ & $u_{z5} $ & $16$ & $23$ & $24$ & $10$ & $39$ \\ \hline
	 	$0.0009$ & $u_{zsp} $ & $5$ & $2$ & $14$ & $8$ & $9$ \\ \hline
	 	$0.0010$ & $u_{zv} $ & $26$ & $7$ & $2$ & $20$ & $32$ \\ \hline
	 	$0.0015$ & $u_{vs} $ & $5$ & $12$ & $5$ & $2$ & $3$ \\ \hline
	 	$0.0020$ & $u_{vz} $ & $1$ & $2$ & $4$ & $7$ & $0$ \\ \hline
	 	$0.0033$ & $u_{vsp} $ & $18$ & $23$ & $7$ & $21$ & $5$ \\ \hline
	 	$0.0080$ & $u_{zdl} $ & $4$ & $3$ & $5$ & $5$ & $4$ \\ \hline
	 	$0.0070$ & $u_{dlz} $ & $4$ & $3$ & $4$ & $4$ & $4$ \\ \hline
	 	$0.0015$ & $u_{kn} $ & $1$ & $1$ & $1$ & $1$ & $1$ \\ \hline
	 	$0.0200$ & $u_{dn} $ & $0$ & $0$ & $1$ & $1$ & $1$ \\ \hline
	 	$0.0250$ & $u_{zd} $ & $1$ & $0$ & $1$ & $1$ & $1$ \\ \hline
	 	$0.0300$ & $u_{zn} $ & $1$ & $1$ & $1$ & $1$ & $1$ \\ \hline
	 	$0.0350$ & $u_{zpf} $ & $0$ & $0$ & $1$ & $0$ & $1$ \\ \hline
	 	$0.0015$ & $u_{pv} $ & $6$ & $4$ & $15$ & $3$ & $13$ \\ \hline
	 	$0.0018$ & $u_{pn} $ & $2$ & $4$ & $0$ & $6$ & $2$ \\ \hline
	 	$0.0020$ & $u_{ps} $ & $1$ & $0$ & $3$ & $2$ & $8$ \\ \hline
		$0.0250$ & $u_{sk} $ & $0.7$ & $0.65$ & $0.87$ & $0.93$ & $0.84$ \\ \hline
	 	$0.0015$ & $u_{skp} $ & $0.83$ & $0.84$ & $0.79$ & $0.85$ & $0.91$ \\ \hline
	 	$-0.0005$ & $u_{usn} $ & $0$ & $0$ & $0$ & $0$ & $0$ \\ \hline
	 	$0.0003$ & $u_{usi} $ & $0$ & $0$ & $0$ & $0$ & $0$ \\ \hline
	 	$0.0010$ & $u_{usj} $ & $0$ & $0$ & $0$ & $1$ & $0$ \\ \hline
	 	$0.0380$ & $u_{usv} $ & $1$ & $1$ & $1$ & $0$ & $1$ \\ \hline
	 	$0.0100$ & $u_{izp} $ & $5$ & $4$ & $5$ & $5$ & $4$ \\ \hline
	 	$0.0086$ & $u_{ikr} $ & $4$ & $5$ & $5$ & $5$ & $4$ \\ \hline
	 	$-0.0015$ & $u_{iuk} $ & $1$ & $1$ & $2$ & $1$ & $1$ \\ \hline
	 	$0.0100$ & $u_{ilz} $ & $1$ & $0$ & $0$ & $1$ & $1$ \\ \hline
	 	$0.0080$ & $u_{iak} $ & $5$ & $4$ & $3$ & $5$ & $5$ \\ \hline
	\end{longtabu}
}


\subsection{Математична постановка однокритеріального еквіваленту вихідної багатокритеріальної задачі відповідно до теореми Гермейєра згідно до виданого завдання}

Згідно теореми Гермейєра для виконання перетворень~\eqref{omega1},~\eqref{omega2},~\eqref{omega3} необхідно знайти мінімальне та максимальне значення окремо для кожної функції мети на допустимій множині альтернатив:
\begin{align*}
	f_{1(\min)}&=0,	&	f_{2(\min)}&=0,	&	f_{3(\min)}&=0, \\
	f_{1(\max)}&=f_1^0=4,	&	f_{2(\max)}&=f_2^0=2.5,	&	f_{3(\max)}&=f_3^0=4.
\end{align*}

Так як мінімальне значення для всіх $f_i$ дорівнюють $0$, то перетворення~\eqref{omega2} буде аналогічне~\eqref{omega1}. Як наслідок, задачі багатокритеріальної оптимізації будуть однакові.

Перетворення~\eqref{omega1},~\eqref{omega2} приймуть наступний вигляд:
\begin{align*}
\omega^{(1,2)}_1(f_1(\vec{x})) &= 
\cfrac{f_1^0-f_1(\vec{x})}{f_1^0 - f_{1(\min)}} =
\cfrac{4 - x_1}{4}, \\
\omega^{(1,2)}_2(f_2(\vec{x})) &= 
\cfrac{f_2^0-f_2(\vec{x})}{f_2^0 - f_{2(\min)}} =
\cfrac{2.5 - x_2}{2.5}, \\
\omega^{(1,2)}_3(f_3(\vec{x})) &= 
\cfrac{f_3^0-f_3(\vec{x})}{f_3^0 - f_{3(\min)}} =
\cfrac{4 - x_3}{4}.
\end{align*}

При використанні перетворень~\eqref{omega1} та~\eqref{omega2} задача матиме вигляд:
\[
F(\vec{x})=\max\{ \rho_1\cfrac{4 - x_1}{4},\rho_2\cfrac{2.5 - x_2}{2.5},\rho_3\cfrac{4 - x_3}{4} \} \to \min.
\]

При використанні перетвореня~\eqref{omega3}~($\mu = 2$) задача матиме вигляд:
\[
F(\vec{x})=\max\{ \rho_1\left(\frac{4 - x_1}{4}\right)^2,\rho_2\left(\frac{2.5 - x_2}{2.5}\right)^2,\rho_3\left(\frac{4 - x_3}{4}\right)^2 \} \to \min.
\]

Результати розрахунків для 3 різних наборів значень вагових коефіцієнтів $\vec{p} = (0.7, 0.1, 0.2)$, $\vec{p} = (0.3, 0.6, 0.1)$ та $\vec{p} = (0.3, 0.3, 0.4)$ були занесені до таблиці~\ref{tab:result}.

    \begin{landscape}% Landscape page
        \begin{table}[H]        
    	    \caption{Результати розрахунків}
	        \label{tab:result}
	        \small
        \begin{tabular}{c|c c c|c c c|c c c|c c c|c c c|c}
        	% head
        	% p
        	$j$
            & $p_1$   & $p_2$   & $p_3$ 
        	% x
            & $x_1^*$ & $x_2^*$ & $x_3^*$
        	% fx
            & $f_1(\vec{x^*})$ & $f_2(\vec{x^*})$ & $f_3(\vec{x^*})$
        	% wx
            & $\omega^j_1(\vec{x^*})$ & $\omega^j_2(\vec{x^*})$ & $\omega^j_3(\vec{x^*})$
        	% pwx
            & $p_1\omega^j_1(\vec{x^*})$ & $p_2\omega^j_2(\vec{x^*})$ & $p_3\omega^j_3(\vec{x^*})$
        	% fx
            & $F(\vec{x^*})$ \\
            \hline

            % 1 first
            \multirow{3}{*}{1}
        	% p
            & $0.7$ & $0.1$ & $0.2$	
        	% x
            & $3.1111$ & $0.0$ & $0.8888$
        	% fx
            & $3.1111$ & $0.0$ & $0.8888$
        	% wx
            & $0.2222$ & $1.0$ & $0.7777$
        	% pwx
            & $0.1555$ & $0.1$ & $0.1555$
        	% fx
            & $0.1555$ \\

            % 1 two
        	% p
            & $0.3$ & $0.6$ & $0.1$	
        	% x
            & $2.0952$ & $1.9047$ & $0.0$
        	% fx
            & $2.0952$ & $1.9047$ & $0.0$
        	% wx
            & $0.4761$ & $0.2380$ & $1.0$
        	% pwx
            & $0.1428$ & $0.1428$ & $0.1$
        	% fx
            & $0.1428$ \\

            % 1 three
        	% p
            & $0.3$ & $0.3$ & $0.4$	
        	% x
            & $0.0$ & $0.0$ & $4.0$
        	% fx
            & $0.0$ & $0.0$ & $4.0$
        	% wx
            & $1.0$ & $1.0$ & $0.0$
        	% pwx
            & $0.3$ & $0.3$ & $0.0$
        	% fx
            & $0.3$ \\
            \hline

            % 2 first
            \multirow{3}{*}{2}
            % p
            & $0.7$ & $0.1$ & $0.2$ 
            % x
            & $3.1111$ & $0.0$ & $0.8888$
            % fx
            & $3.1111$ & $0.0$ & $0.8888$
            % wx
            & $0.2222$ & $1.0$ & $0.7777$
            % pwx
            & $0.1555$ & $0.1$ & $0.1555$
            % fx
            & $0.1555$ \\

            % 2 two
            % p
            & $0.3$ & $0.6$ & $0.1$ 
            % x
            & $2.0952$ & $1.9047$ & $0.0$
            % fx
            & $2.0952$ & $1.9047$ & $0.0$
            % wx
            & $0.4761$ & $0.2380$ & $1.0$
            % pwx
            & $0.1428$ & $0.1428$ & $0.1$
            % fx
            & $0.1428$ \\

            % 2 three
        	% p
            & $0.3$ & $0.3$ & $0.4$ 
            % x
            & $0.0$ & $0.0$ & $4.0$
            % fx
            & $0.0$ & $0.0$ & $4.0$
            % wx
            & $1.0$ & $1.0$ & $0.0$
            % pwx
            & $0.3$ & $0.3$ & $0.0$
            % fx
            & $0.3$ \\
            \hline

            % 3 one
            \multirow{3}{*}{3}
        	% p
            & $0.7$ & $0.1$ & $0.2$	
        	% x
            & $2.5633$ & $0.1243$ & $1.3123$
        	% fx
            & $2.5633$ & $0.1243$ & $1.3123$
        	% wx
            & $0.1299$ & $0.9029$ & $0.4515$
        	% pwx
            & $0.0903$ & $0.0903$ & $0.0903$
        	% fx
            & $0.0903$ \\

            % 3 two
        	% p
            & $0.3$ & $0.6$ & $0.1$	
        	% x
            & $1.7838$ & $2.054$ & $0.162$
        	% fx
            & $1.7838$ & $2.054$ & $0.162$
        	% wx
            & $0.3069$ & $0.0318$ & $0.9206$
        	% pwx
            & $0.092$ & $0.019$ & $0.092$
        	% fx
            & $0.092$ \\

            % 3 three
        	% p
            & $0.3$ & $0.3$ & $0.4$ 
            % x
            & $0.0$ & $0.0$ & $4.0$
            % fx
            & $0.0$ & $0.0$ & $4.0$
            % wx
            & $1.0$ & $1.0$ & $0.0$
            % pwx
            & $0.3$ & $0.3$ & $0.0$
            % fx
            & $0.3$ \\
        \end{tabular}
        \end{table}
        \begin{description}
        	\item[де] $\vec{x^*}$ --- ефективна альтернатива.
        \end{description}
    \end{landscape}

\subsection{Висновки}

На даній лабораторній роботі було вивчено загальні положення задач багатокритеріальної оптимізації та теорему Гермейєра про знаходження ефективних альтернатив для багатокритеріальних задач лінійного (нелінійного) програмування. 
Було вирішено задачу багатокритеріальної оптимізації на основі виданого завдання за допомогою  теореми Гермейєра.

Проаналізуємо отримане рішення задачі однокритеріальної оптимізації при використанні перетворення~\eqref{omega1} та вагових коефіцієнтів $\vec{p_1} = (0.7, 0.1, 0.2)$:
була отримана ефективна альтернатива $\vec{x^*} = (3.1111,0.0,0.8888)$, яка забезпечує досягнення оптимального значення для  функції $f_1(\vec{x^*})$, так як відповідне перетворення дорівнює $0.1555$, що свідчить про близькість до оптимуму $f_1(\vec{x^*})$. Отримане рішення відображає наступну закономірність:
\[
p_1 \geq p_3 \geq p_2 \Longrightarrow \omega_1(x^*) \leq \omega_3(x^*) \leq \omega_2(x^*) 
\]

\end{document}
