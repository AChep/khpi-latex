\usepackage{tikz}

\counterwithout{figure}{section}
\counterwithout{table}{section}
\counterwithout{equation}{section}

\titleformat{\subsection}[block]
  {\bfseries\filcenter}{#1}{0cm}{}
\titlespacing{\subsection}{0cm}{21pt}{21pt}

\DeclareCaptionLabelFormat{gosttable}{Таблица #2}

\usepackage{float}
\usepackage{pgfplots}
\usepackage{graphicx}
\usepackage{multirow}
\usepackage{amssymb,amsfonts,amsmath,amsthm}

\usepackage{listings}
\lstset{basicstyle=\footnotesize\ttfamily,breaklines=true}
\lstset{language=Matlab}


\newcommand{\labnumber}{2} % first lab
\usepackage{tikz}

\counterwithout{figure}{section}
\counterwithout{table}{section}
\counterwithout{equation}{section}

\titleformat{\subsection}[block]
  {\bfseries\filcenter}{#1}{0cm}{}
\titlespacing{\subsection}{0cm}{21pt}{21pt}

\DeclareCaptionLabelFormat{gosttable}{Таблица #2}

\newcommand{\khpistudentgroup}{2.КН201н.8а}
\newcommand{\khpistudentname}{Чепурний~А.~С.}

\newcommand{\khpidepartment}{Програмна інженерія та інформаційні технології управління}
\newcommand{\khpititlewhat}{
	Розрахунково-графічне завдання \\
	з предмету <<Фреймворки та платформи>>
}
\newcommand{\khpititlewho}{
	Виконав: \\
	\hspace*{\parindent} ст. групи \khpistudentgroup \\
	\hspace*{\parindent} \khpistudentname \\
	Перевірила: \\
	\hspace*{\parindent} к. т. н., вик. каф. ПІІТУ \\
	\hspace*{\parindent} Добряк~В.~С. \\
}


\usepackage{systeme}
\usepackage{longtable,tabu}
\usepackage{multirow}
\usepackage{array,multirow}
\usepackage{pdflscape}
\usepackage{afterpage}
\usepackage{bm}

\graphicspath{{figures/}}

\begin{document}
\Ukrainian

\begin{titlepage}

\begin{center}
	МІНІСТЕРСТВО ОСВІТИ І НАУКИ УКРАЇНИ \\
	НАЦІОНАЛЬНИЙ ТЕХНІЧНИЙ УНІВЕРСИТЕТ \\
	«ХАРКІВСЬКИЙ ПОЛІТЕХНІЧНИЙ ІНСТИТУТ» \\
	Кафедра <<\khpidepartment>> \\
\end{center}

\vspace{6cm}

\begin{center}
	\khpititlewhat
\end{center}

\vspace{3cm}

\begin{addmargin}[10cm]{0cm}
	\khpititlewho
\end{addmargin}

\vspace{\fill}

\begin{center}
	Харків \the\year
\end{center}

\end{titlepage}

\addtocounter{page}{1}

\textbf{Тема роботи}: розв'язання багатокритеріальної задачі щодо знаходження ефективних альтернатив за допомогою теореми Гермейєра.

\textbf{Завдання для виконання}: вирішити наступну задачу багатокритеріальної оптимізації:
\begin{align*} 
	f_1(\vec{x}) &= x_1 \to \max, \\
	f_2(\vec{x}) &= x_2 \to \max, \\
	f_3(\vec{x}) &= x_3 \to \max,
\end{align*}
% limits
\[
\systeme{
x_1 + x_2 + x_3 \leq 4,
3x_2 - x_3 \leq 6,
x_1 \geq 0,
x_2 \geq 0,
x_3 \geq 0
}
.
\]

\subsection{Математична постановка задачі багатокритеріальної оптимізації в загальному вигляді}

У загальному випадку формально задача багатокритеріальної оптимізації, ключовою особливістю якої є суперечливість множини функцій мети (критеріїв), може бути подана в наступному вигляді:
\begin{gather*} 
	f_i(\vec{x}) \to \max, i \in I_1, \\
	f_i(\vec{x}) \to \min, i \in I_2, \\
	\varphi_j(\vec{x}) \leq 0, j \in J.
\end{gather*}
\begin{description}
	\item[де] $I_1$ та $I_2$ --- множини індексів функцій мети $f_i(\vec{x})$, які відповідно максимізуються та мінімізуються, причому $I=I_1 \cup I_2$;
	\item $J$ --- множина індексів функцій $\varphi_j(\vec{x})$, що визначають обмеження задачі та формують множину припустимих варіантів альтернатив $A = \{ \varphi_j(\vec{x}) \leq 0, j \in J \}$;
	\item $\vec{x}$ --- вектор змінних задачі багатокритеріальної оптимізації, з яким пов’яжемо поняття альтернативи --- варіанта розв’язку, що задовольняє обмеження задачі і є способом досягнення поставлених цілей.
\end{description}

\subsection{Математична постановка однокритеріального еквіваленту вихідної багатокритеріальної задачі відповідно до теореми Гермейєра в загальному вигляді}

Основні положення теореми Гермейєра формулюються не для первісно заданої множини функцій мети $\{f_i(\vec{x}),i \in I\}$, а для множини функцій $\{\omega_i(\vec{x})=\omega_i(f_i(\vec{x})), i \in I\}$, що складається з монотонних перетворень окремих функцій мети $f_i(\vec{x})$, які приводять їх до безрозмірного вигляду.

За останні можна взяти одну з монотонних функцій такого вигляду:
\begin{equation}\label{omega1}
\omega^1_i(f_i(\vec{x})) = \systeme[][:]{
\cfrac{f_i^0-f_i(\vec{x})}{f_i^0 - f_{i(\min)}},i \in I_1
:
\cfrac{f_i(\vec{x})-f_i^0}{f_{i(\max)} - f_i^0},i \in I_2
}
,
\end{equation}
\begin{equation}\label{omega2}
\omega^2_i(f_i(\vec{x})) = \systeme[][:]{
\cfrac{f_i^0-f_i(\vec{x})}{f_i^0},i \in I_1
:
\cfrac{f_i(\vec{x})-f_i^0}{f_i^0},i \in I_2
}
,
\end{equation}
\begin{equation}\label{omega3}
\omega^3_i(f_i(\vec{x}))=\omega^j_i(f_i(\vec{x}))^\mu, i \in I, j \in \{1,2\}
.
\end{equation}
\begin{description}
	\item[де] $f_{i(\min)}$, $f_{i(\max)}$ --- найменші і найбільші значення функцій мети, які відповідно максимізуються і мінімізуються на множині припустимих варіантів альтернатив;
	\item $f_i^0$ --- оптимальне значення $i$-ї функції мети на множині припустимих варіантів альтернатив;
	\item $\mu$ --- число, що визначає степінь, на яку підноситься перетворення~\eqref{omega1}~або~\eqref{omega2}.
\end{description}

Нехай $x^*$ --- ефективна альтернатива множини функцій мети $\{\omega_i(f_i(\vec{x})), i \in I\}$, причому нехай $\{\omega_i(f_i(\vec{x}))>0, i \in I\}$. 
Тоді існує вектор $\vec{\rho}$ з компонентами $\rho_i > 0, \sum_{i \in I}\rho_i=1$ такий, що критерій 
\[
F(\vec{x})=\max_{i \in I} \rho_i \omega_i(f_i(\vec{x}))
\]
досягає мінімуму на множині припустимих варіантів альтернатів $A$, при $x=x^*$.

Особливістю даної теореми є той факт, що ніякі умови на вигляд функцій $\omega_i(f_i(\vec{x}))$ і обмежень, що описують множину припустимих варіантів альтернитив $A$, не накладаються.

Таким чином, множина ефективних альтернатив може бути знайдена з використанням теореми Гермейєра шляхом розв'язання наступної задачі:
\[
F(\vec{x})=\max_{i \in I} \rho_i\omega_i(f_i(\vec{x})) \to \min
\]
\begin{description}
    \item[де] $\omega_i(f_i(\vec{x}))$ --- монотонне перетворення $i$-ї функції мети $f_i(x)$, побудоване на основі одного із припустимих варіантів~\eqref{omega1},~\eqref{omega2}~та~\eqref{omega3}.
\end{description}

\subsection{Математична постановка задачі багатокритеріальної оптимізації згідно з виданим завданням}

Згідно виданого завдання задача багатокритеріальної оптимізації прийме наступний вигляд:
\begin{align*} 
	f_1(\vec{x}) &= x_1 \to \max, \\
	f_2(\vec{x}) &= x_2 \to \max, \\
	f_3(\vec{x}) &= x_3 \to \max,
\end{align*}
% limits
\[
\systeme{
x_1 + x_2 + x_3 \leq 4,
3x_2 - x_3 \leq 6,
x_1 \geq 0,
x_2 \geq 0,
x_3 \geq 0
}
.
\]

\subsection{Математична постановка однокритеріального еквіваленту вихідної багатокритеріальної задачі відповідно до теореми Гермейєра згідно до виданого завдання}

Згідно теореми Гермейєра для виконання перетворень~\eqref{omega1},~\eqref{omega2},~\eqref{omega3} необхідно знайти мінімальне та максимальне значення окремо для кожної функції мети на допустимій множині альтернатив:
\begin{align*}
	f_{1(\min)}&=0,	&	f_{2(\min)}&=0,	&	f_{3(\min)}&=0, \\
	f_{1(\max)}&=f_1^0=4,	&	f_{2(\max)}&=f_2^0=2.5,	&	f_{3(\max)}&=f_3^0=4.
\end{align*}

Так як мінімальне значення для всіх $f_i$ дорівнюють $0$, то перетворення~\eqref{omega2} буде аналогічне~\eqref{omega1}. Як наслідок, задачі багатокритеріальної оптимізації будуть однакові.

Перетворення~\eqref{omega1},~\eqref{omega2} приймуть наступний вигляд:
\begin{align*}
\omega^{(1,2)}_1(f_1(\vec{x})) &= 
\cfrac{f_1^0-f_1(\vec{x})}{f_1^0 - f_{1(\min)}} =
\cfrac{4 - x_1}{4}, \\
\omega^{(1,2)}_2(f_2(\vec{x})) &= 
\cfrac{f_2^0-f_2(\vec{x})}{f_2^0 - f_{2(\min)}} =
\cfrac{2.5 - x_2}{2.5}, \\
\omega^{(1,2)}_3(f_3(\vec{x})) &= 
\cfrac{f_3^0-f_3(\vec{x})}{f_3^0 - f_{3(\min)}} =
\cfrac{4 - x_3}{4}.
\end{align*}

При використанні перетворень~\eqref{omega1} та~\eqref{omega2} задача матиме вигляд:
\[
F(\vec{x})=\max\{ \rho_1\cfrac{4 - x_1}{4},\rho_2\cfrac{2.5 - x_2}{2.5},\rho_3\cfrac{4 - x_3}{4} \} \to \min.
\]

При використанні перетвореня~\eqref{omega3}~($\mu = 2$) задача матиме вигляд:
\[
F(\vec{x})=\max\{ \rho_1\left(\frac{4 - x_1}{4}\right)^2,\rho_2\left(\frac{2.5 - x_2}{2.5}\right)^2,\rho_3\left(\frac{4 - x_3}{4}\right)^2 \} \to \min.
\]

Результати розрахунків для 3 різних наборів значень вагових коефіцієнтів $\vec{p} = (0.7, 0.1, 0.2)$, $\vec{p} = (0.3, 0.6, 0.1)$ та $\vec{p} = (0.3, 0.3, 0.4)$ були занесені до таблиці~\ref{tab:result}.

    \begin{landscape}% Landscape page
        \begin{table}[H]        
    	    \caption{Результати розрахунків}
	        \label{tab:result}
	        \small
        \begin{tabular}{c|c c c|c c c|c c c|c c c|c c c|c}
        	% head
        	% p
        	$j$
            & $p_1$   & $p_2$   & $p_3$ 
        	% x
            & $x_1^*$ & $x_2^*$ & $x_3^*$
        	% fx
            & $f_1(\vec{x^*})$ & $f_2(\vec{x^*})$ & $f_3(\vec{x^*})$
        	% wx
            & $\omega^j_1(\vec{x^*})$ & $\omega^j_2(\vec{x^*})$ & $\omega^j_3(\vec{x^*})$
        	% pwx
            & $p_1\omega^j_1(\vec{x^*})$ & $p_2\omega^j_2(\vec{x^*})$ & $p_3\omega^j_3(\vec{x^*})$
        	% fx
            & $F(\vec{x^*})$ \\
            \hline

            % 1 first
            \multirow{3}{*}{1}
        	% p
            & $0.7$ & $0.1$ & $0.2$	
        	% x
            & $3.1111$ & $0.0$ & $0.8888$
        	% fx
            & $3.1111$ & $0.0$ & $0.8888$
        	% wx
            & $0.2222$ & $1.0$ & $0.7777$
        	% pwx
            & $0.1555$ & $0.1$ & $0.1555$
        	% fx
            & $0.1555$ \\

            % 1 two
        	% p
            & $0.3$ & $0.6$ & $0.1$	
        	% x
            & $2.0952$ & $1.9047$ & $0.0$
        	% fx
            & $2.0952$ & $1.9047$ & $0.0$
        	% wx
            & $0.4761$ & $0.2380$ & $1.0$
        	% pwx
            & $0.1428$ & $0.1428$ & $0.1$
        	% fx
            & $0.1428$ \\

            % 1 three
        	% p
            & $0.3$ & $0.3$ & $0.4$	
        	% x
            & $0.0$ & $0.0$ & $4.0$
        	% fx
            & $0.0$ & $0.0$ & $4.0$
        	% wx
            & $1.0$ & $1.0$ & $0.0$
        	% pwx
            & $0.3$ & $0.3$ & $0.0$
        	% fx
            & $0.3$ \\
            \hline

            % 2 first
            \multirow{3}{*}{2}
            % p
            & $0.7$ & $0.1$ & $0.2$ 
            % x
            & $3.1111$ & $0.0$ & $0.8888$
            % fx
            & $3.1111$ & $0.0$ & $0.8888$
            % wx
            & $0.2222$ & $1.0$ & $0.7777$
            % pwx
            & $0.1555$ & $0.1$ & $0.1555$
            % fx
            & $0.1555$ \\

            % 2 two
            % p
            & $0.3$ & $0.6$ & $0.1$ 
            % x
            & $2.0952$ & $1.9047$ & $0.0$
            % fx
            & $2.0952$ & $1.9047$ & $0.0$
            % wx
            & $0.4761$ & $0.2380$ & $1.0$
            % pwx
            & $0.1428$ & $0.1428$ & $0.1$
            % fx
            & $0.1428$ \\

            % 2 three
        	% p
            & $0.3$ & $0.3$ & $0.4$ 
            % x
            & $0.0$ & $0.0$ & $4.0$
            % fx
            & $0.0$ & $0.0$ & $4.0$
            % wx
            & $1.0$ & $1.0$ & $0.0$
            % pwx
            & $0.3$ & $0.3$ & $0.0$
            % fx
            & $0.3$ \\
            \hline

            % 3 one
            \multirow{3}{*}{3}
        	% p
            & $0.7$ & $0.1$ & $0.2$	
        	% x
            & $2.5633$ & $0.1243$ & $1.3123$
        	% fx
            & $2.5633$ & $0.1243$ & $1.3123$
        	% wx
            & $0.1299$ & $0.9029$ & $0.4515$
        	% pwx
            & $0.0903$ & $0.0903$ & $0.0903$
        	% fx
            & $0.0903$ \\

            % 3 two
        	% p
            & $0.3$ & $0.6$ & $0.1$	
        	% x
            & $1.7838$ & $2.054$ & $0.162$
        	% fx
            & $1.7838$ & $2.054$ & $0.162$
        	% wx
            & $0.3069$ & $0.0318$ & $0.9206$
        	% pwx
            & $0.092$ & $0.019$ & $0.092$
        	% fx
            & $0.092$ \\

            % 3 three
        	% p
            & $0.3$ & $0.3$ & $0.4$ 
            % x
            & $0.0$ & $0.0$ & $4.0$
            % fx
            & $0.0$ & $0.0$ & $4.0$
            % wx
            & $1.0$ & $1.0$ & $0.0$
            % pwx
            & $0.3$ & $0.3$ & $0.0$
            % fx
            & $0.3$ \\
        \end{tabular}
        \end{table}
        \begin{description}
        	\item[де] $\vec{x^*}$ --- ефективна альтернатива.
        \end{description}
    \end{landscape}

\subsection{Висновки}

На даній лабораторній роботі було вивчено загальні положення задач багатокритеріальної оптимізації та теорему Гермейєра про знаходження ефективних альтернатив для багатокритеріальних задач лінійного (нелінійного) програмування. 
Було вирішено задачу багатокритеріальної оптимізації на основі виданого завдання за допомогою  теореми Гермейєра.

Проаналізуємо отримане рішення задачі однокритеріальної оптимізації при використанні перетворення~\eqref{omega1} та вагових коефіцієнтів $\vec{p_1} = (0.7, 0.1, 0.2)$:
була отримана ефективна альтернатива $\vec{x^*} = (3.1111,0.0,0.8888)$, яка забезпечує досягнення оптимального значення для  функції $f_1(\vec{x^*})$, так як відповідне перетворення дорівнює $0.1555$, що свідчить про близькість до оптимуму $f_1(\vec{x^*})$. Отримане рішення відображає наступну закономірність:
\[
p_1 \geq p_3 \geq p_2 \Longrightarrow \omega_1(x^*) \leq \omega_3(x^*) \leq \omega_2(x^*) 
\]

\end{document}
