\usepackage{tikz}

\counterwithout{figure}{section}
\counterwithout{table}{section}
\counterwithout{equation}{section}

\titleformat{\subsection}[block]
  {\bfseries\filcenter}{#1}{0cm}{}
\titlespacing{\subsection}{0cm}{21pt}{21pt}

\DeclareCaptionLabelFormat{gosttable}{Таблица #2}

\usepackage{float}
\usepackage{pgfplots}
\usepackage{graphicx}
\usepackage{multirow}
\usepackage{amssymb,amsfonts,amsmath,amsthm}

\usepackage{listings}
\lstset{basicstyle=\footnotesize\ttfamily,breaklines=true}
\lstset{language=Matlab}


\newcommand{\labnumber}{3} % third lab
\usepackage{tikz}

\counterwithout{figure}{section}
\counterwithout{table}{section}
\counterwithout{equation}{section}

\titleformat{\subsection}[block]
  {\bfseries\filcenter}{#1}{0cm}{}
\titlespacing{\subsection}{0cm}{21pt}{21pt}

\DeclareCaptionLabelFormat{gosttable}{Таблица #2}

\newcommand{\khpistudentgroup}{2.КН201н.8а}
\newcommand{\khpistudentname}{Чепурний~А.~С.}

\newcommand{\khpidepartment}{Програмна інженерія та інформаційні технології управління}
\newcommand{\khpititlewhat}{
	Розрахунково-графічне завдання \\
	з предмету <<Фреймворки та платформи>>
}
\newcommand{\khpititlewho}{
	Виконав: \\
	\hspace*{\parindent} ст. групи \khpistudentgroup \\
	\hspace*{\parindent} \khpistudentname \\
	Перевірила: \\
	\hspace*{\parindent} к. т. н., вик. каф. ПІІТУ \\
	\hspace*{\parindent} Добряк~В.~С. \\
}


\usepackage{systeme}
\usepackage{longtable,tabu}
\usepackage{multirow}
\usepackage{array,multirow}
\usepackage{pdflscape}
\usepackage{afterpage}
\usepackage{bm}

\graphicspath{{../figures/}}

\begin{document}
\Ukrainian

\begin{titlepage}

\begin{center}
	МІНІСТЕРСТВО ОСВІТИ І НАУКИ УКРАЇНИ \\
	НАЦІОНАЛЬНИЙ ТЕХНІЧНИЙ УНІВЕРСИТЕТ \\
	«ХАРКІВСЬКИЙ ПОЛІТЕХНІЧНИЙ ІНСТИТУТ» \\
	Кафедра <<\khpidepartment>> \\
\end{center}

\vspace{6cm}

\begin{center}
	\khpititlewhat
\end{center}

\vspace{3cm}

\begin{addmargin}[10cm]{0cm}
	\khpititlewho
\end{addmargin}

\vspace{\fill}

\begin{center}
	Харків \the\year
\end{center}

\end{titlepage}

\addtocounter{page}{1}

\textbf{Тема роботи}: Рішення багатокритеріальної задачі лінійного програмування по знаходженню ефективних альтернатив за допомогою третьої теореми по знаходженню ефективних альтернатив.

\textbf{Завдання для виконання}: вирішити наступну задачу багатокритеріальної оптимізації:
\begin{align*} 
	f_1(\vec{x}) &= x_1 \to \max, \\
	f_2(\vec{x}) &= x_2 \to \max, \\
	f_3(\vec{x}) &= x_3 \to \max,
\end{align*}
% limits
\[
\systeme{
x_1 + x_2 + x_3 \leq 4,
3x_2 - x_3 \leq 6,
x_1 \geq 0,
x_2 \geq 0,
x_3 \geq 0
}
.
\]

\subsubsection{Математична постановка задачі багатокритеріальної оптимізації в загальному вигляді}
Особливістю третьої теореми в порівнянні з теоремою Карліна та теоремою Гермейєра є те, що її основні положення формулюються для первісно заданої множини функцій мети ${f_i(x), i \in I}$, що не вимагає виконання додаткових перетворень, що приводять функції мети до безрозмірного вигляду. 

З урахуванням вищевикладеного матеріалу формулюється третя теорема зі знахадження еффективних альтернатив.
\textbf{Теорема 3.} 
Якщо $x_0$ - эффетивна альтернатива множини функції цілі $f$, то для кожної $l \in I_1$:

\begin{gather*} 
    f_l (x^*) = \max {f_l(x)}, \\
    f_i (x) >= f_i(x^*), \forall i \in I_1, i \not = l, \\
    f_i (x) <= f_i(x^*), \forall i \in I_2 \\
    x \in A, \\
\end{gather*}

Або для кожного $l \in I_2$

\begin{gather*} 
    f_l (x^*) = \min{f_l(x)}, \\
    f_i (x) >= f_i(x^*), \forall i \in I_1 \\
    f_i (x) <= f_i(x^*), \forall i \in I_2, i \not = l, \\
    x \in A, \\
\end{gather*}

Таким чином, множина ефективних альтернатив для множини функцій мети ${f_i(x), i \in I}$ може бути знайдена при вирішенні задачі параметричного програмування щодо параметрів $z \in Z^(M-1)$, якщо за головний критерій обрано критерій, що максимізується 

$f_i(x) \to \max$

при наявності обмежень

\begin{gather*} 
    f_i (x) >= z, \forall i \in I_1, i \not = l,\\
    f_i (x) <= z, \forall i \in I_2,  \\
    x \in A, \\
\end{gather*}

де під $Z^(M-1)$ розуміють:

\begin{gather*}
    \cap_{i \in I_1, i \not = l} ^ {} [f_{i(min)}, f_i^0] \times \cap_{i \in I_2} ^ {} [f_i^0, f_{i(max)}], \\
\end{gather*}

де:

\begin{itemize}
    \item $f_i^0$ - оптимальні значення функцій мети;
    \item f_{i(min)} - найменші значення функцій мети, якщо вони максимізуються;
    \item f_{i(max) - найбільші значення функцій мети, якщо вони мінімізуються.
    \item $M$ - множина індексів функцій мети, при чому $I_1={1, ... , m}, I_2 = {m + 1, ..., M}, i \not = l$ - множина індексів відповідно для максимізованих й мінімізованих функцій мети.
\end{itemize}

У випадку, якщо за головний критерій обрано критерій, що мінімізується, то множина ефективних альтернатив може бути знайдена таким чином: 

$f_i(x) \to \min$

при наявності обмежень

\begin{gather*} 
    f_i (x) >= z, \forall i \in I_1, \\
    f_i (x) <= z, \forall i \in I_2, i \not = l, \\
    x \in A, \\
\end{gather*}

де під $Z^(M-1)$ розуміють:

\begin{gather*}
    \cap_{i \in I_1} ^ {} [f_{i(min)}, f_i^0] \times \cap_{i \in I_2, i \not = l} ^ {} [f_i^0, f_{i(max)}], \\
\end{gather*}

де $M$ - потужніть множини I.

\subsubsection{Математична постановка задачі багатокритеріальної оптимізації відповідно до виданого завданням}

Відповідно до виданого завдання множина ефективних альтернатив для множини функцій мети буде знайдена при вирішенні задачі параметричного програмування щодо параметрів $z \in Z^(M-1)$, де кожен критерій максимізується.
Дані про максимальні та мінімальні значення функцій мети обрані із матерілу минулої лабораторної роботи.
\begin{align*}
    f_{1(\min)}&=0, &   f_{2(\min)}&=0, &   f_{3(\min)}&=0, \\
    f_{1(\max)}&=f_1^0=4,   &   f_{2(\max)}&=f_2^0=2.5, &   f_{3(\max)}&=f_3^0=4.
\end{align*} 

Так для першого крітерію задача параметричного програмування буде мати вигляд:

\begin{gather*} 
    l = 1, \\
    Z^{m-1} \in [0; 2.5] \cap [0; 4] = [0; 2.5] \\
    f_1 (x) = x_1 \to \max, \\
    f_2 (x) = x_2 \geqslant z,  \\
    f_3 (x) = x_3 \geqslant z,  \\
    x_1 + x_2 + x_3 \leqslant 4, \\
    3 x_2 - x_3 \leqslant 6 \\
    \vec{x} \geqslant 0
\end{gather*}

Якщо обирати другий критерій за головний, задача параметричного програмування буде мати вигляд:
\begin{gather*} 
    l = 2, \\
    Z^{m-1} \in [0; 4] \cap [0; 4] = [0; 4] \\
    f_2 (x) = x_2 \to \max \\
    f_1 (x) = x_1 \geqslant z,  \\
    f_3 (x) = x_3 \geqslant z,  \\
    x_1 + x_2 + x_3 \leqslant 4, \\
    3 x_2 - x_3 \leqslant 6 \\
    \vec{x} \geqslant 0
\end{gather*}

Для варіанту коли третій критерій буде головним, задача параметричного програмування буде мати вигляд:

\begin{gather*}
    l = 3, \\ 
    Z^{m-1} \in [0; 2.5] \cap [0; 4] = [0; 2.5] \\
    f_3 (x) = x_3 \to \max \\
    f_1 (x) = x_1 \geqslant z,  \\
    f_2 (x) = x_2 \geqslant z,  \\
    x_1 + x_2 + x_3 \leqslant 4, \\
    3 x_2 - x_3 \leqslant 6 \\
    \vec{x} \geqslant 0
\end{gather*}

\subsection{Результати досліджень}



\end{document}

