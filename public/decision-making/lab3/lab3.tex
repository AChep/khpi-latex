\usepackage{tikz}

\counterwithout{figure}{section}
\counterwithout{table}{section}
\counterwithout{equation}{section}

\titleformat{\subsection}[block]
  {\bfseries\filcenter}{#1}{0cm}{}
\titlespacing{\subsection}{0cm}{21pt}{21pt}

\DeclareCaptionLabelFormat{gosttable}{Таблица #2}

\usepackage{float}
\usepackage{pgfplots}
\usepackage{graphicx}
\usepackage{multirow}
\usepackage{amssymb,amsfonts,amsmath,amsthm}

\usepackage{listings}
\lstset{basicstyle=\footnotesize\ttfamily,breaklines=true}
\lstset{language=Matlab}


\newcommand{\labnumber}{3} % third lab
\usepackage{tikz}

\counterwithout{figure}{section}
\counterwithout{table}{section}
\counterwithout{equation}{section}

\titleformat{\subsection}[block]
  {\bfseries\filcenter}{#1}{0cm}{}
\titlespacing{\subsection}{0cm}{21pt}{21pt}

\DeclareCaptionLabelFormat{gosttable}{Таблица #2}

\newcommand{\khpistudentgroup}{2.КН201н.8а}
\newcommand{\khpistudentname}{Чепурний~А.~С.}

\newcommand{\khpidepartment}{Програмна інженерія та інформаційні технології управління}
\newcommand{\khpititlewhat}{
	Розрахунково-графічне завдання \\
	з предмету <<Фреймворки та платформи>>
}
\newcommand{\khpititlewho}{
	Виконав: \\
	\hspace*{\parindent} ст. групи \khpistudentgroup \\
	\hspace*{\parindent} \khpistudentname \\
	Перевірила: \\
	\hspace*{\parindent} к. т. н., вик. каф. ПІІТУ \\
	\hspace*{\parindent} Добряк~В.~С. \\
}


\usepackage{systeme}
\usepackage{longtable,tabu}
\usepackage{multirow}
\usepackage{array,multirow}
\usepackage{pdflscape}
\usepackage{afterpage}
\usepackage{bm}

\graphicspath{{../figures/}}

\begin{document}
\Ukrainian

\begin{titlepage}

\begin{center}
	МІНІСТЕРСТВО ОСВІТИ І НАУКИ УКРАЇНИ \\
	НАЦІОНАЛЬНИЙ ТЕХНІЧНИЙ УНІВЕРСИТЕТ \\
	«ХАРКІВСЬКИЙ ПОЛІТЕХНІЧНИЙ ІНСТИТУТ» \\
	Кафедра <<\khpidepartment>> \\
\end{center}

\vspace{6cm}

\begin{center}
	\khpititlewhat
\end{center}

\vspace{3cm}

\begin{addmargin}[10cm]{0cm}
	\khpititlewho
\end{addmargin}

\vspace{\fill}

\begin{center}
	Харків \the\year
\end{center}

\end{titlepage}

\addtocounter{page}{1}

\textbf{Тема роботи}: Рішення багатокритеріальної задачі лінійного програмування по знаходженню ефективних альтернатив за допомогою третьої теореми по знаходженню ефективних альтернатив.

\textbf{Завдання для виконання}: вирішити наступну задачу багатокритеріальної оптимізації:
\begin{align*} 
	f_1(\vec{x}) &= x_1 \to \max, \\
	f_2(\vec{x}) &= x_2 \to \max, \\
	f_3(\vec{x}) &= x_3 \to \max,
\end{align*}
% limits
\[
\systeme{
x_1 + x_2 + x_3 \leq 4,
3x_2 - x_3 \leq 6,
x_1 \geq 0,
x_2 \geq 0,
x_3 \geq 0
}
.
\]

\subsection{Математична постановка задачі багатокритеріальної оптимізації в загальному вигляді}
Особливістю третьої теореми в порівнянні з теоремою Карліна та теоремою Гермейєра є те, що її основні положення формулюються для первісно заданої множини функцій мети ${f_i(\vec{x}), i \in I}$, що не вимагає виконання додаткових перетворень, що приводять функції мети до безрозмірного вигляду. 

З урахуванням вищевикладеного матеріалу формулюється третя теорема зі знаходження ефективних альтернатив. \\[7em]

Якщо $\vec{x_0}$ --- ефективна альтернатива множини функції цілі $f(\vec{x})$, то для кожного $l \in I_1$:
\begin{gather*} 
    f_l (\vec{x^*}) = \max {f_l(\vec{x})}, \\
    f_i (\vec{x}) \geq f_i(\vec{x^*}), \forall i \in I_1, i \not = l, \\
    f_i (\vec{x}) \leq f_i(\vec{x^*}), \forall i \in I_2, \\
    \vec{x} \in A,
\end{gather*}
для кожного $l \in I_2$:
\begin{gather*} 
    f_l (\vec{x^*}) = \min{f_l(\vec{x})}, \\
    f_i (\vec{x}) >= f_i(\vec{x^*}), \forall i \in I_1 \\
    f_i (\vec{x}) <= f_i(\vec{x^*}), \forall i \in I_2, i \not = l, \\
    \vec{x} \in A.
\end{gather*}

Таким чином, множина ефективних альтернатив для множини функцій мети $\{f_i(\vec{x}), i \in I\}$ може бути знайдена при вирішенні задачі параметричного програмування щодо параметрів $z \in Z^{(M-1)}$, якщо за головний критерій обрано критерій, що максимізується при наявності обмежень
\begin{gather*} 
    f_i (\vec{x}) >= z, \forall i \in I_1, i \not = l,\\
    f_i (\vec{x}) <= z, \forall i \in I_2,  \\
    \vec{x} \in A,
\end{gather*}

де під $Z^{M-1}$ розуміють:
\begin{gather*}
    \cap_{i \in I_1, i \not = l} ^ {} [f_{i(min)}, f_i^0] \times \cap_{i \in I_2} ^ {} [f_i^0, f_{i(max)}],
\end{gather*}
\begin{description}
    \item[де] $f_i^0$ --- оптимальні значення функцій мети;
    \item $f_{i(\min)}$ --- найменші значення функцій мети, якщо вони максимізуються;
    \item $f_{i(\max)}$ --- найбільші значення функцій мети, якщо вони мінімізуються.
    \item $M$ --- множина індексів функцій мети, при чому $I_1=\{1, \ldots, m\}$, $I_2 = \{m + 1, \ldots, M\}$, $i \not = l$ --- множина індексів відповідно для максимізованих й мінімізованих функцій мети.
\end{description}

У випадку, якщо за головний критерій обрано критерій, що мінімізується, то множина ефективних альтернатив може бути знайдена таким чином: 
\begin{gather*} 
    f_i (\vec{x}) >= z, \forall i \in I_1, \\
    f_i (\vec{x}) <= z, \forall i \in I_2, i \not = l, \\
    \vec{x} \in A,
\end{gather*}

де під $Z^{M-1}$ розуміють:
\begin{gather*}
    \cap_{i \in I_1} ^ {} [f_{i(min)}, f_i^0] \times \cap_{i \in I_2, i \not = l} ^ {} [f_i^0, f_{i(max)}],
\end{gather*}
\begin{description}
    \item[де] $M$ - потужність множини I.
\end{description}

\subsection{Математична постановка задачі багатокритеріальної оптимізації відповідно до виданого завданням}

Відповідно до виданого завдання множина ефективних альтернатив для множини функцій мети буде знайдена при вирішенні задачі параметричного програмування щодо параметрів $z \in Z^{M-1}$, де кожен критерій максимізується.
Дані про максимальні та мінімальні значення функцій мети обрані із матеріалу минулої лабораторної роботи.
\begin{align*}
    f_{1(\min)}&=0, &   f_{2(\min)}&=0, &   f_{3(\min)}&=0, \\
    f_{1(\max)}&=f_1^0=4,   &   f_{2(\max)}&=f_2^0=2.5, &   f_{3(\max)}&=f_3^0=4.
\end{align*} 
\clearpage
Так для першого критерію задача параметричного програмування буде мати вигляд:
\begin{gather*} 
    l = 1, \\
    Z^{m-1} \in [0; 2.5] \cap [0; 4] = [0; 2.5], \\
    f_1 (\vec{x}) = x_1 \to \max, \\
    f_2 (\vec{x}) = x_2 \geqslant z,  \\
    f_3 (\vec{x}) = x_3 \geqslant z,  \\
    x_1 + x_2 + x_3 \leqslant 4, \\
    3 x_2 - x_3 \leqslant 6, \\
    \vec{x} \geqslant 0.
\end{gather*}

Якщо обирати другий критерій за головний, задача параметричного програмування буде мати вигляд:
\begin{gather*} 
    l = 2, \\
    Z^{m-1} \in [0; 4] \cap [0; 4] = [0; 4], \\
    f_2 (\vec{x}) = x_2 \to \max, \\
    f_1 (\vec{x}) = x_1 \geqslant z,  \\
    f_3 (\vec{x}) = x_3 \geqslant z,  \\
    x_1 + x_2 + x_3 \leqslant 4, \\
    3 x_2 - x_3 \leqslant 6, \\
    \vec{x} \geqslant 0.
\end{gather*}
\clearpage
Для варіанту коли третій критерій буде головним, задача параметричного програмування буде мати вигляд:
\begin{gather*}
    l = 3, \\ 
    Z^{m-1} \in [0; 2.5] \cap [0; 4] = [0; 2.5], \\
    f_3 (\vec{x}) = x_3 \to \max, \\
    f_1 (\vec{x}) = x_1 \geqslant z,  \\
    f_2 (\vec{x}) = x_2 \geqslant z,  \\
    x_1 + x_2 + x_3 \leqslant 4, \\
    3 x_2 - x_3 \leqslant 6, \\
    \vec{x} \geqslant 0.
\end{gather*}

Результати розрахунків були занесені до
таблиці~\ref{tab:result}.

        \begin{table}[H]        
            \caption{Результати розрахунків}
            \label{tab:result}
            \small
        \begin{tabular}{c|c|c|c c c|c c c}
            % head
            Критерій & Область & $Z$ & $x^*_1$ & $x^*_2$ & $x^*_3$ & $f_1(\vec{x^*})$ & $f_2(\vec{x^*})$ & $f_3(\vec{x^*})$ \\
            \hline

            \multirow{11}{*}{$f_1$} & \multirow{11}{*}{$[0;2.5]$} &
            0 & 4 & 0 & 0 & 4 & 0 & 0 \\
            & & 0.25 & 3.5 & 0.25 & 0.25 & 3.5 & 0.25 & 0.25 \\
            & & 0.5 & 3 & 0.5 & 0.5 & 3 & 0.5 & 0.5 \\
            & & 0.75 & 2.5 & 0.75 & 0.75 & 2.5 & 0.75 & 0.75 \\
            & & 1 & 2 & 1 & 1 & 2 & 1 & 1 \\
            & & 1.25 & 1.5 & 1.25 & 1.25 & 1.5 & 1.25 & 1.25 \\
            & & 1.5 & 1 & 1.5 & 1.5 & 1 & 1.5 & 1.5 \\
            & & 1.75 & 0.5 & 1.75 & 1.75 & 0.5 & 1.75 & 1.75 \\
            & & 2 & 0 & 2 & 2 & 0 & 2 & 2 \\
            & & 2.25 & 0 & 2.5 & 1.5 & 0 & 2.5 & 1.5 \\
            & & 2.5 & 0 & 2.5 & 1.5 & 0 & 2.5 & 1.5 \\
            \hline

            \multirow{11}{*}{$f_2$} & \multirow{11}{*}{$[0;4]$} &
            0 & 0 &  2.5& 1.5 &0 &  2.5 &1.5 \\
            & & 0.4& 0.4& 2.4& 1.2& 0.4& 2.4& 1.2 \\
            & & 0.8 &0.8& 2.3& 0.9& 0.8& 2.3& 0.9 \\
            & & 1.2 &1.2& 1.6& 1.2& 1.2& 1.6& 1.2 \\
            & & 1.6& 1.6& 0.8& 1.6& 1.6& 0.8 1.6 \\ 
            & & 2  & 2  & 0 &  2  & 2  & 0  & 2 \\ 
            & & 2.4& 2.4& 0 &  1.6& 2.4& 0 &  1.6 \\
            & & 2.8& 2.8& 0 &  1.2& 2.8& 0&   1.2 \\
            & & 3.2& 3.2& 0 &  0.8& 3.2& 0 &  0.8 \\
            & & 3.6& 3.6& 0 &  0.4& 3.6& 0 &  0.4 \\
            & & 4  & 4  & 0 &  0  & 4  & 0 &  0 \\
            \hline 

            \multirow{11}{*}{$f_3$} & \multirow{11}{*}{$[0;2.5]$} &
            0  & 0 &  0  & 4 &  0 &  0 &  4 \\
            & & 0.25 &   0.25 &   0.25  &  3.5& 0.25  &  0.25 &   3.5 \\
            & & 0.5 &0.5& 0.5& 3 &  0.5& 0.5 &3 \\
            & & 0.75 &   0.75  &  0.75 &   2.5& 0.75  &  0.75&    2.5\\
            & & 1  & 1  & 1 &  2   1  & 1 &  2 \\
            & & 1.25  &  1.25 &   1.25  &  1.5 &1.25  &  1.25  &  1.5 \\
            & & 1.5& 1.5& 1.5& 1 &  1.5& 1.5 &1 \\
            & & 1.75 &   1.75 &   1.75 &   0.5 &1.75 &   1.75  &  0.5 \\
            & & 2 &  2  & 2  & 0  & 2 &  2  & 0 \\
            & & 2.25  &  2.25 &   1.75 &   0  & 2.25 &   1.75 &   0 \\
            & & 2.5 &2.5& 1.5& 0  & 2.5 &1.5& 0

        \end{tabular}
        \end{table}

\subsection{Висновки}
В ході виконання лабораторної роботи було вивчено загальні положення третьої теореми про знаходження ефективних альтернатив для багатокритеріальних задач лінійного (нелінійного) програмування. 
Було вирішено задачу багатокритеріальної оптимізації за допомоги третьої теореми з почерговим вибором кожного критерію вихідної задачі як головного та інших як додаткових обмежень.

Обчислення для кожного критерію виконувались з різноманітними значеннями параметру $z$, які належали області $Z^{M-1}$. 
Було помічено, що меншим значенням z відповідали альтернативи, які більшою мірою задовольняли головному критерію, та навпаки. 
Так, для $z = 0$ та $z = 1.75$ відповідними значеннями головного критерію є
\begin{align*}
f_1(&\vec{x^*})=4, \\
&\vec{x^*}=(4, 0, 0),
\end{align*}
та 
\begin{align*}
f_1(&\vec{x^*})=0.5, \\
&\vec{x^*}=(0.5, 1.75, 1.75).
\end{align*}

\end{document}

