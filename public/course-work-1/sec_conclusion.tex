\section*{Висновки}
\addcontentsline{toc}{section}{Висновки}
Сучасні інструменти імітаційного моделювання дозволяють ефективно застосовувати його не тільки в наукових дослідженнях, а й як засоби для побудови систем підтримки прийняття рішень у бізнесі. 

Агентне моделювання дозволяє змоделювати систему максимально наближену до реальності, зробити значний крок у розумінні та управлінні сукупністю складних процесів.

Основним завданням даної роботи був аналіз предметної області та опис моделей та алгоритмів для управління розподільчими логістичними системами.

В ході написання роботи дану задачу було розкрито повною мірою.
Були описані методи та алгоритми для управління логістичними системами, розглянуті проблеми моделювання логистичних систем.
